\documentclass[a4paper,11pt]{article}
\usepackage{amsmath, amssymb}
\usepackage[margin=1in,letterpaper]{geometry} % decreases margins

\usepackage{spverbatim}

%----------------------------------------

\begin{document}
\title{GPT Solution Responses (EM Questions)}
\author{}
\date{}
\maketitle

\noindent These are the compiled solutions from GPT-4 responses (word-for-word) for all questions in question dataset on Classical Electromagnetic Theory. This will be one of the documents used to mark solutions for human grading. The LaTeX code used here will also be input for GPT-4 grading. 

\section{Classical Electromagnetic Theory Solutions}

\subsubsection*{Question 1:}

Prompt:
\begin{spverbatim}
    You are a top professor of physics. Answer the physics problem given within the quotation marks. Show your working step by step. Keep your answer under 300 words.

    "Name the three types of magnetic materials. Explain what determines their response when brought near a bar magnet. "
\end{spverbatim} 

\bigskip

\noindent Solution 1: \\

The three types of magnetic materials are:

\begin{enumerate}
    \item \textbf{Ferromagnetic Materials}: These materials have a strong attraction to magnetic fields. They can be permanently magnetized. Their strong response is due to the alignment of their magnetic domains (regions where the magnetic moments of atoms are aligned in the same direction) in the presence of an external magnetic field. Examples include iron, cobalt, and nickel.
    \item \textbf{Paramagnetic Materials}: These materials are attracted to magnetic fields, but much weaker compared to ferromagnetic materials. They cannot retain magnetization in the absence of an external magnetic field. The response of paramagnetic materials is due to the alignment of the magnetic moments of the atoms partially with the external magnetic field. Examples include aluminum, platinum, and many chemical ions.
    \item \textbf{Diamagnetic Materials}: These materials are repelled by magnetic fields and cannot be magnetized. The response is very weak compared to ferromagnetic and paramagnetic materials. Diamagnetism occurs because an external magnetic field induces a magnetic moment in the material that opposes the external field. Examples include copper, gold, and bismuth.
\end{enumerate}

The response of these materials when brought near a bar magnet is determined by their atomic structure, specifically by the arrangement and orientation of their electrons. Ferromagnetic materials have unpaired electrons whose spins tend to align parallel to each other in regions called domains, which strongly amplifies their magnetic response. Paramagnetic materials have unpaired electrons, but without the domain structure, causing a weaker attraction. Diamagnetic materials, on the other hand, have all their electrons paired and create an induced magnetic field in opposition to the external magnetic field due to the Lenz's Law, causing a repulsive effect. \\ 

\noindent Solution 2: \\ 

The three types of magnetic materials are:

\begin{enumerate}
    \item \textbf{Ferromagnetic materials}: These materials exhibit strong magnetization in the direction of the magnetic field and retain their magnetism even after the external field is removed. This strong response is due to the alignment of their magnetic domains (regions of uniform magnetization) in the direction of the magnetic field. Examples include iron, cobalt, and nickel.
    \item \textbf{Paramagnetic materials}: These materials are weakly attracted by a magnetic field and do not retain magnetism in the absence of an external field. The magnetic response is due to the alignment of the atomic magnetic moments in the direction of the magnetic field. However, this alignment is much weaker compared to ferromagnetic materials because thermal motion tends to disrupt the alignment of magnetic moments. Examples include aluminum, platinum, and manganese.
    \item \textbf{Diamagnetic materials}: These materials are weakly repelled by a magnetic field. The effect is very weak and occurs because an external magnetic field induces a magnetic moment in the atoms of the material that is opposite to the direction of the applied field. This is due to the Lenz's Law, which causes the electrons' orbits to shift in such a way that it opposes the external magnetic field. Examples include copper, gold, and lead.
\end{enumerate}

The response of these materials when brought near a bar magnet is determined by their atomic structure and the behavior of their electrons in an external magnetic field. For ferromagnetic materials, the alignment of magnetic domains with the external field causes strong attraction. For paramagnetic materials, the temporary alignment of atomic magnetic moments with the field causes weak attraction. Diamagnetic materials, due to the induced magnetic moment in the opposite direction of the applied field, exhibit a weak repulsion. \\ 

\noindent Solution 3:

The three types of magnetic materials are:

\begin{enumerate}
    \item \textbf{Ferromagnetic Materials}: These materials, like iron, cobalt, and nickel, exhibit strong magnetic properties. When brought near a bar magnet, their atomic dipoles (tiny magnetic moments) tend to align themselves in the direction of the magnetic field, significantly strengthening the external magnetic field. This alignment is due to the unpaired electrons in their atoms, which create strong local magnetic fields. Ferromagnetic materials can retain their magnetization, becoming permanent magnets.
    
    \item \textbf{Paramagnetic Materials}: Paramagnetic materials, such as aluminum and platinum, have a weaker response to an external magnetic field compared to ferromagnetic materials. In these materials, atomic dipoles align with the external magnetic field, but the effect is much weaker. This is because the magnetic moments of the electrons in these materials are not as strongly aligned; they tend to align only slightly with the external field and do not retain magnetization in the absence of the field.
    
    \item \textbf{Diamagnetic Materials}: Diamagnetic materials, like copper and bismuth, are repelled by a magnetic field. This occurs because the applied magnetic field induces a magnetic field in the opposite direction, according to Lenz's Law. This induced magnetic field is very weak, and it arises due to changes in the orbital motion of electrons in response to the external field. Diamagnetic materials do not retain magnetization when the external field is removed.
\end{enumerate}

The response of these materials when brought near a bar magnet is determined by their electronic structure, specifically the arrangement and spin of the electrons within their atoms. This determines whether the material strengthens (ferromagnetic), weakly enhances (paramagnetic), or opposes (diamagnetic) the external magnetic field. 

\subsubsection*{Question 2:}

Prompt:
\begin{spverbatim}
    You are a top professor of physics. Answer the physics problem given within the quotation marks. Show your working step by step. Keep your answer under 300 words.

    "Within the context of classical electromagnetism, briefly define what is 'free space'. In free space the electric field, \underline{E}, satisfies 3 partial differential equations, one for each of its components

    \[
    \nabla^{2}\underline{E} =  \epsilon_{0}\mu_{0}\frac{\partial^2 \underline{E}}{\partial t^{2}}
    \]

    Briefly describe the physical interpretation of this equation for showing how electric fields can propagate in space. Explain how this equation gives the speed of light c and state its value." 
\end{spverbatim}

\bigskip

\noindent Solution 1: \\

In classical electromagnetism, "free space" refers to a vacuum or an idealized medium that is devoid of any material substance. It is characterized by its permittivity, $\epsilon_{0}$, and its permeability, $\mu_{0}$, which are fundamental physical constants that describe how electric and magnetic fields interact in the vacuum.

The equation given,

\[
\nabla^{2}\underline{E} =  \epsilon_{0}\mu_{0}\frac{\partial^2 \underline{E}}{\partial t^{2}}
\]

is a wave equation for the electric field vector, $\underline{E}$. Here, $\nabla^{2}$ is the Laplace operator, which represents the spatial second derivatives of the electric field, indicating how the field changes in space. The right side of the equation involves $\epsilon_{0}\mu_{0}$ and the second time derivative of the electric field, showing how the field changes with time.

This equation encapsulates the essence of electromagnetic wave propagation. It indicates that electric fields (and, by extension, magnetic fields due to Maxwell's equations) can propagate as waves through free space. The presence of the second derivatives in both space and time suggests that these fields can spread out over space and evolve over time without the need for a physical medium to support them.

To extract the speed of light $c$ from this equation, we compare it to the general form of a wave equation:

\[
\nabla^{2}\psi = \frac{1}{v^2}\frac{\partial^2 \psi}{\partial t^{2}}
\]

where $v$ is the speed of the wave and $\psi$ represents a wave function. By analogy, the term $\epsilon_{0}\mu_{0}$ in our equation plays the role of $1/v^2$, which leads us to the relationship:

\[
c = \frac{1}{\sqrt{\epsilon_{0}\mu_{0}}}
\]

Using known values of $\epsilon_{0} \approx 8.85 \times 10^{-12}$ F/m (farads per meter) and $\mu_{0} \approx 4\pi \times 10^{-7}$ H/m (henrys per meter), we can calculate the speed of light:

\[
c = \frac{1}{\sqrt{(8.85 \times 10^{-12})(4\pi \times 10^{-7})}} \approx 3 \times 10^8 \text{ m/s}
\]

This is the speed at which electromagnetic waves, including light, propagate through free space, a fundamental constant of nature. \\ 

\noindent Solution 2:

In classical electromagnetism, "free space" refers to a perfect vacuum devoid of any matter or charges. It is an idealized medium where electromagnetic waves can propagate without attenuation due to matter. Free space is characterized by its permittivity, \(\epsilon_0\), and its permeability, \(\mu_0\).

The equation given,

\[
\nabla^{2}\underline{E} = \epsilon_{0}\mu_{0}\frac{\partial^2 \underline{E}}{\partial t^{2}}
\]

is known as the wave equation for the electric field \(\underline{E}\) in free space. The \(\nabla^{2}\) operator, called the Laplacian, represents the spatial second derivatives of the electric field, indicating how the field changes in space. The right side of the equation involves \(\epsilon_{0}\mu_{0}\) and the second time derivative of \(\underline{E}\), showing how the field changes in time.

The physical interpretation of this equation is that it describes the propagation of electromagnetic waves (or the electric field component of such waves) in free space. The changes in the electric field over time generate changes in the field in space, leading to wave propagation.

To find the speed of light, \(c\), from this equation, we compare it to the standard form of a wave equation:

\[
\frac{\partial^2 \underline{E}}{\partial t^{2}} = v^2 \nabla^{2}\underline{E}
\]

where \(v\) is the speed of wave propagation. By comparing the coefficients, we identify:

\[
v^2 = \frac{1}{\epsilon_{0}\mu_{0}}
\]

Thus, the speed of light, \(c\), in free space is given by:

\[
c = \frac{1}{\sqrt{\epsilon_{0}\mu_{0}}}
\]

The values of \(\epsilon_{0}\) (the permittivity of free space) and \(\mu_{0}\) (the permeability of free space) are constants, and their product under the square root gives the speed of light. The value of \(c\) is approximately \(3.00 \times 10^8\) m/s, a fundamental constant of nature indicating the speed at which all electromagnetic waves, including light, propagate in free space. \\

\noindent Solution 3: \\

In classical electromagnetism, "free space" refers to a perfect vacuum that does not contain any matter or charges. It is an idealized medium where electromagnetic waves can propagate without attenuation or dispersion due to material properties. Free space is characterized by its permittivity, \(\epsilon_0\), and permeability, \(\mu_0\).

The equation given,

\[
\nabla^{2}\underline{E} =  \epsilon_{0}\mu_{0}\frac{\partial^2 \underline{E}}{\partial t^{2}},
\]

is known as the wave equation for the electric field \(\underline{E}\) in free space. Here, \(\nabla^{2}\) is the Laplace operator, representing the spatial second derivatives of the electric field, indicating how the field varies in space. The right side of the equation contains \(\frac{\partial^2 \underline{E}}{\partial t^{2}}\), which is the second time derivative of the electric field, showing how the field changes over time. The presence of \(\epsilon_0\) and \(\mu_0\) ties the spatial variation to the temporal variation.

The physical interpretation of this wave equation is that it describes how electric fields (and by extension, electromagnetic waves) propagate through space. The spatial variations of the field (left side of the equation) are directly related to its temporal variations (right side of the equation), allowing the field to propagate as a wave.

Furthermore, comparing this wave equation to the general form of a wave equation \(\nabla^{2}\psi = \frac{1}{v^2}\frac{\partial^2 \psi}{\partial t^{2}}\), where \(v\) is the speed of the wave, we can see that the speed of electromagnetic waves in free space is given by:

\[
c = \frac{1}{\sqrt{\epsilon_0\mu_0}}.
\]

This formula directly relates the speed of light (\(c\)) in free space to the permittivity (\(\epsilon_0\)) and permeability (\(\mu_0\)) of free space. Plugging in the values for \(\epsilon_0\) and \(\mu_0\), we get the speed of light:

\[
c \approx 3 \times 10^8 \, \text{m/s}.
\]

This speed is a fundamental constant of nature and represents the maximum speed at which all massless particles and changes in the physical properties of space can propagate.


\subsubsection*{Question 3: (contains input figure)}

Prompt:
\begin{spverbatim}
    You are a top professor of physics. Answer the physics problem given within the quotation marks. Show your working step by step.

    "Find the electric field a distance, \( d \), above one end of a straight line segment of length \( L \) that has a constant line charge density \( \lambda \) (as seen in the figure). In the limit of \( d \gg L \), what does the electric field reduce to and what is its physical interpretation?"
\end{spverbatim}

\bigskip

\noindent Solution 1: \\

To find the electric field a distance \( d \) above one end of a straight line segment of length \( L \) that has a constant line charge density \( \lambda \), we can use Coulomb's law and integrate over the length of the line charge.

The differential element of charge \( dq \) at a point \( x \) along the line, where \( x \) ranges from 0 to \( L \), is given by \( dq = \lambda dx \).

Each differential charge element \( dq \) creates a differential electric field \( d\vec{E} \) at the point located at distance \( d \) above the end of the line. Due to symmetry, the horizontal components of the electric field produced by the charges will cancel out, leaving only the vertical component.

The vertical component of the differential electric field \( dE \) at the point is given by Coulomb's law:

\[ dE = \frac{1}{4\pi\epsilon_0} \frac{dq}{r^2} \cos(\theta) \]

where \( r \) is the distance from the differential charge element to the point, \( \epsilon_0 \) is the vacuum permittivity, and \( \cos(\theta) \) is the ratio of the vertical distance \( d \) to the distance \( r \). From the geometry, \( r = \sqrt{d^2 + x^2} \) and \( \cos(\theta) = \frac{d}{r} \).

So we have

\[ dE = \frac{1}{4\pi\epsilon_0} \frac{\lambda dx}{(d^2 + x^2)} \frac{d}{\sqrt{d^2 + x^2}} \]

The total electric field \( E \) is the integral of \( dE \) from \( x = 0 \) to \( x = L \):

\[ E = \int_0^L \frac{1}{4\pi\epsilon_0} \frac{\lambda dx}{(d^2 + x^2)} \frac{d}{\sqrt{d^2 + x^2}} \]

After computing this integral, the expression for the electric field \( E \) a distance \( d \) above one end of the straight line segment of length \( L \) is:

\[ E = \frac{0.141047395886939 \cdot L \cdot \lambda}{\sqrt{\pi} \cdot \epsilon_0 \cdot d^2 \cdot \sqrt{\frac{L^2}{d^2} + 1}} \]

In the limit as \( d \gg L \), \( \frac{L}{d} \) approaches 0, and the term \( \sqrt{\frac{L^2}{d^2} + 1} \) approaches 1. Thus, the expression simplifies to:

\[ E = \frac{0.141047395886939 \cdot L \cdot \lambda}{\sqrt{\pi} \cdot \epsilon_0 \cdot d^2} \]

The physical interpretation of this result is that the electric field far away from a finite line of charge behaves as if it were from a point charge, with the electric field inversely proportional to the square of the distance, which is consistent with Coulomb's law for point charges. The magnitude of this "equivalent point charge" would be the total charge of the line segment, \( \lambda L \). \\

\noindent Solution 2: \\

To solve for the electric field a distance \( d \) above one end of a straight line segment of length \( L \) with a constant line charge density \( \lambda \), we need to integrate the contributions of electric field \( d\vec{E} \) from each infinitesimally small segment of the charged line.

Let's consider an infinitesimal segment \( dl \) of the line charge at a distance \( x \) from the point of interest (point P) which is at distance \( d \) above one end of the line segment. The charge of this infinitesimal segment is \( dq = \lambda dl \).

According to Coulomb's law, the electric field \( d\vec{E} \) due to the charge \( dq \) at the point P can be expressed as:
\[ d\vec{E} = \frac{1}{4\pi\epsilon_0} \frac{dq}{r^2} \]

where \( \epsilon_0 \) is the vacuum permittivity and \( r \) is the distance from the charge to point P. In this case, \( r \) can be expressed in terms of \( x \) and \( d \) by the Pythagorean theorem: \( r = \sqrt{d^2 + x^2} \).

The electric field due to \( dq \) has two components: a vertical component \( dE_d \) that points directly away from the line charge and a horizontal component \( dE_x \) that points along the line charge. The horizontal components will cancel due to symmetry when integrated over the entire line, leaving only the vertical component. Therefore, we only need to calculate \( dE_d \).

The vertical component \( dE_d \) is given by:
\[ dE_d = dE \cos(\theta) \]
\[ dE_d = \frac{1}{4\pi\epsilon_0} \frac{\lambda dl}{d^2 + x^2} \frac{d}{\sqrt{d^2 + x^2}} \]

Now, let's integrate \( dE_d \) along the length of the line charge from \( x = 0 \) to \( x = L \) to find the total electric field at point P:
\[ E_d = \int_{0}^{L} \frac{1}{4\pi\epsilon_0} \frac{\lambda dl}{(d^2 + x^2)^{3/2}} d \]

The electric field at a distance \( d \) above one end of the straight line segment of length \( L \) is given by:

\[ E_d = \frac{0.141047395886939 \cdot L \cdot d^{-2} \cdot \lambda}{\sqrt{\pi} \cdot \epsilon_0 \cdot \sqrt{\left(\frac{L^2}{d^2} + 1\right)}} \]

Now, let's consider the limit as \( d \gg L \). In this limit, \( \frac{L^2}{d^2} \) becomes negligible compared to 1. Therefore, the expression simplifies by considering \( \sqrt{\frac{L^2}{d^2} + 1} \approx 1 \), and the electric field \( E_d \) reduces to:

\[ E_d \approx \frac{0.141047395886939 \cdot L \cdot \lambda}{d^2 \cdot \sqrt{\pi} \cdot \epsilon_0} \]

This approximation means that at a point far away from the line segment, the field behaves as if the entire line charge is concentrated at a point. This is consistent with the principle that at large distances, the details of the charge distribution are not resolved, and the field appears as if it were produced by a point charge. The total charge of the line segment is \( Q = \lambda L \), and if we substitute this into the above equation, we get:

\[ E_d \approx \frac{0.141047395886939 \cdot Q}{d^2 \cdot \sqrt{\pi} \cdot \epsilon_0} \]

This resembles the electric field due to a point charge, with some constant factors due to the geometry of the original charge distribution. The physical interpretation is that as we move much farther away from the charge distribution, its electric field converges to that of a point charge with the same total charge. \\

\noindent Solution 3: \\

To find the electric field a distance \( d \) above one end of a straight line segment of length \( L \) with a constant line charge density \( \lambda \), we can integrate the contributions of electric field \( d\mathbf{E} \) from each infinitesimal piece of the line charge.

Let's set up the problem using Coulomb's Law, which states that the electric field \( d\mathbf{E} \) due to a point charge \( dq \) at a distance \( r \) is given by

\[ d\mathbf{E} = k_e \frac{dq}{r^2} \]

where \( k_e \) is Coulomb's constant (\( k_e \approx 8.988 \times 10^9 \, \text{N}\cdot\text{m}^2/\text{C}^2 \)).

For a line segment with linear charge density \( \lambda \), an infinitesimal charge element \( dq \) is given by \( dq = \lambda dx \), where \( dx \) is an infinitesimal piece of the line.

The distance \( r \) from the infinitesimal charge element to the point above one end of the line segment can be expressed using Pythagoras' theorem as \( r = \sqrt{d^2 + x^2} \).

The electric field is a vector quantity, so we need to consider its components. We are interested in the vertical component of the electric field, which is the component along the direction of \( d \).

The vertical component \( dE_y \) of \( d\mathbf{E} \) is given by

\[ dE_y = d\mathbf{E} \cos\theta = k_e \frac{\lambda dx}{(d^2 + x^2)} \frac{d}{\sqrt{d^2 + x^2}} \]

\[ dE_y = k_e \frac{\lambda d\,dx}{(d^2 + x^2)^{3/2}} \]

Now, we integrate \( dE_y \) from \( x = 0 \) to \( x = L \) to find the total electric field \( E_y \) at the point a distance \( d \) above one end of the line segment:

\[ E_y = \int_0^L k_e \frac{\lambda d\,dx}{(d^2 + x^2)^{3/2}} \]

The electric field \( E_y \) at a distance \( d \) above the end of the line charge is given by

\[ E_y = 0.564 \sqrt{\pi} \frac{k_e \lambda L}{d^2 \sqrt{1 + \frac{L^2}{d^2}}} \]

Now, in the limit of \( d \gg L \), the term \( \frac{L^2}{d^2} \) becomes very small and can be neglected in comparison to 1. This simplifies the expression for \( E_y \) as follows:

\[ E_y \approx 0.564 \sqrt{\pi} \frac{k_e \lambda L}{d^2} \]

Physically, this means that when the distance \( d \) is much greater than the length of the line segment \( L \), the line segment begins to look like a point charge from the perspective of the point at distance \( d \), and the electric field reduces to that of a point charge. The factor of \( 0.564 \sqrt{\pi} \) comes from the integration and provides the exact proportionality in this specific geometry. The electric field thus behaves inversely with the square of the distance \( d \), which is characteristic of the electric field from a point charge.

\subsubsection*{Question 4: (contains input figure)}

Prompt: 
\begin{spverbatim}
    You are a top professor of physics. Answer the physics problem given within the quotation marks. Show your working step by step.

    "A square loop of wire (sides of length \( L \)) lies a fixed distance \( d \) from a very long straight wire, which carries a constant current \( I \) directed downwards (as seen in the figure).

    a) Find the flux of \( \underline{B} \) through the loop.

    b) If the loop is moved directly away from the wire to the right, at speed \( v \), what emf is generated? In what direction (clockwise or counterclockwise) does the current flow?

    c) What happens if the loop only moves downwards at speed \( v \)?"
\end{spverbatim}

\bigskip

\noindent Solution 1: \\

a) Find the flux of \( \underline{B} \) through the loop.

The magnetic field \( B \) due to a long straight wire at a distance \( r \) is given by Ampère's law as:

\[
B = \frac{\mu_0 I}{2 \pi r}
\]

where \( \mu_0 \) is the permeability of free space (\( \mu_0 = 4\pi \times 10^{-7} \frac{T \cdot m}{A} \)) and \( I \) is the current.

For the square loop of wire, we will integrate the magnetic field over the area of the loop to find the flux \( \Phi \). The magnetic flux through an area element \( dA \) is \( B \cdot dA \cos(\theta) \), where \( \theta \) is the angle between the field lines and the normal to the surface. Since the field lines and the normal to the loop's plane are parallel, \( \cos(\theta) = 1 \).

The distance \( r \) from the wire to a small section of the loop will vary from \( d \) to \( d + L \). We can consider a small strip of the loop of width \( dr \) at a distance \( r \) from the wire. The length of this strip is \( L \), so \( dA = L \cdot dr \).

\[
d\Phi = B \cdot dA = \frac{\mu_0 I}{2 \pi r} \cdot L \cdot dr
\]

To find the total flux \( \Phi \), we integrate \( d\Phi \) from \( d \) to \( d + L \):

\[
\Phi = \int_{d}^{d+L} \frac{\mu_0 I L}{2 \pi r} dr
\]

\[
\Phi = \frac{\mu_0 I L}{2 \pi} \int_{d}^{d+L} \frac{1}{r} dr
\]

\[
\Phi = \frac{\mu_0 I L}{2 \pi} \ln\left(\frac{d+L}{d}\right)
\]

b) If the loop is moved directly away from the wire to the right, at speed \( v \), what emf is generated? In what direction (clockwise or counterclockwise) does the current flow?

When the loop is moved away from the wire, the flux through the loop decreases. Faraday's law of electromagnetic induction states that the induced emf \( \mathcal{E} \) is equal to the negative rate of change of magnetic flux through the loop:

\[
\mathcal{E} = -\frac{d\Phi}{dt}
\]

Since \( d \) increases with time as the loop moves to the right, we can write \( d = vt \) where \( t \) is time. Differentiating \( \Phi \) with respect to time, we get:

\[
\mathcal{E} = -\frac{d}{dt}\left(\frac{\mu_0 I L}{2 \pi} \ln\left(\frac{vt+L}{vt}\right)\right)
\]

\[
\mathcal{E} = -\frac{\mu_0 I L}{2 \pi} \cdot \frac{1}{vt+L - vt} \cdot \left(\frac{v}{vt+L} - \frac{v}{vt}\right)
\]

\[
\mathcal{E} = -\frac{\mu_0 I L v}{2 \pi} \cdot \left(\frac{1}{(vt+L)(vt)}\right) \cdot (L)
\]

\[
\mathcal{E} = -\frac{\mu_0 I L^2 v}{2 \pi (vt+L)(vt)}
\]

As for the direction of the current, according to Lenz's law, the current will flow in such a way as to oppose the change in flux. Since the flux is decreasing as the loop moves away, the induced current will flow in a direction that attempts to maintain the flux. This will be counterclockwise when viewed from the position of the wire.

c) What happens if the loop only moves downwards at speed \( v \)?

If the loop moves downwards parallel to the wire, there is no change in the flux through the loop because the distance \( d \) does not change, and neither does the area of the loop or the orientation with respect to the magnetic field. Therefore, no emf is induced, and no current flows due to this motion. \\

Solution 2: \\

To solve this problem, we will apply the concepts of magnetic field due to a long straight current-carrying conductor, magnetic flux, and electromagnetic induction.

\textbf{a) Finding the magnetic flux through the loop:}

The magnetic field $\underline{B}$ due to a long straight wire at a distance $r$ from the wire is given by Ampère's law:

\[
\underline{B} = \frac{\mu_0 I}{2\pi r}
\]

where:
\begin{itemize}
    \item $\mu_0$ is the permeability of free space ($4\pi \times 10^{-7}$ N/A$^2$),
    \item $I$ is the current in the wire,
    \item $r$ is the distance from the wire.
\end{itemize}

The magnetic flux $\Phi_B$ through a surface is defined as:

\[
\Phi_B = \int_S \underline{B} \cdot d\underline{A}
\]

where:
\begin{itemize}
    \item $d\underline{A}$ is a differential area vector perpendicular to the surface.
\end{itemize}

For the square loop with side $L$, we will integrate over the width of the loop from $d$ to $d+L$, considering the symmetry of the problem, the magnetic field only varies with the distance from the wire and is perpendicular to the loop, we can simplify the dot product to just the magnitude of $B$ times the differential area $dA = L dr$:

\[
\Phi_B = \int_{d}^{d+L} \left(\frac{\mu_0 I}{2\pi r}\right) L dr
\]

Let's compute this integral to find the magnetic flux through the loop.

\textbf{b) Finding the emf generated when the loop is moved:}

Faraday's law of electromagnetic induction states that the induced emf $\mathcal{E}$ in a circuit is equal to the negative rate of change of magnetic flux through the circuit:

\[
\mathcal{E} = - \frac{d\Phi_B}{dt}
\]

If the loop moves to the right at speed $v$, the distance $d$ increases with time. We will differentiate the flux with respect to $d$ and then relate $d$ to $t$ via the speed $v$. The direction of the induced current can be determined by Lenz's law, which states that the direction of the induced current will be such that it opposes the change in flux.

\textbf{c) If the loop moves downwards at speed $v$:}

If the loop moves directly downward, the distance $d$ from the wire does not change, so there is no change in flux, and therefore no emf is induced.

Let's start by calculating the magnetic flux through the loop. The magnetic flux $\Phi_B$ through the loop is given by:

\[
\Phi_B = \frac{I L \mu_0 (L + d)}{2 \pi} - \frac{I L d \mu_0}{2 \pi} = \frac{I L^2 \mu_0}{2 \pi}
\]

This result indicates that the magnetic flux depends only on the constants $I$, $L$, and $\mu_0$, and is independent of the distance $d$.

Now, let's proceed to part b) to find the emf generated when the loop is moved to the right at speed $v$. We will differentiate the magnetic flux $\Phi_B$ with respect to $d$ and then relate $d$ to $t$ using the speed $v$.

The differentiation yields zero, which implies that the emf ($\mathcal{E}$) generated when the loop is moved to the right at speed $v$ is zero. This result actually makes sense because the magnetic flux through the loop, which we computed as $\Phi_B = \frac{I L^2 \mu_0}{2 \pi}$, does not depend on $d$. Since the flux is constant regardless of the distance $d$, moving the loop parallel to the wire at a constant speed $v$ does not change the flux, and thus, no emf is generated.

For part c), as previously mentioned, if the loop only moves downwards at speed $v$, the distance $d$ from the wire does not change. Since the magnetic flux through the loop is independent of $d$, moving the loop downward will not change the flux, and hence no emf is induced, and no current will flow. \\

Solution 3: \\ 

To solve the problem, we will first address part (a), which involves calculating the magnetic flux through the loop.

The magnetic field \( B \) at a distance \( r \) from a long straight wire carrying a current \( I \) is given by Ampère's law:

\[
B = \frac{\mu_0 I}{2\pi r}
\]

where \( \mu_0 \) is the permeability of free space (\( \mu_0 = 4\pi \times 10^{-7} \, \text{T} \cdot \text{m/A} \)).

The magnetic flux \( \Phi \) through the loop is the integral of the magnetic field over the area of the loop. Since the magnetic field is not uniform over the loop, we have to integrate across the loop's width to find the total flux:

\[
\Phi = \int_{d}^{d+L} \left( \frac{\mu_0 I}{2\pi r} \right) \, dr \cdot L
\]

The magnetic flux \( \Phi \) through the loop is given by:

\[
\Phi = \frac{L I \mu_0}{2\pi} \left( \log(L + d) - \log(d) \right)
\]

Next, for part (b), if the loop is moved directly away from the wire to the right at speed \( v \), the electromotive force (emf) generated in the loop can be found using Faraday's law of induction, which states that the emf \( \mathcal{E} \) induced in a circuit is equal to the negative of the time rate of change of the magnetic flux through the loop:

\[
\mathcal{E} = -\frac{d\Phi}{dt}
\]

Since the loop is moving away from the wire, \( d \) is increasing with time. Thus, we can express \( d \) as \( d + vt \) where \( t \) is time. The rate of change of \( d \) with respect to time is just \( v \), the velocity of the loop. To find the emf, we take the derivative of the flux with respect to \( d \), and then multiply by \( v \):

\[
\mathcal{E} = -v \frac{d\Phi}{dd}
\]

The electromotive force (emf) \(\mathcal{E}\) induced in the loop is given by:

\[
\mathcal{E} = \frac{I L^2 \mu_0}{2\pi d (L + d)}
\]

This expression is positive, indicating that the emf generated is in such a direction that the induced current flows in a direction to oppose the change in flux through the loop, in accordance with Lenz's Law.

Now, to determine the direction (clockwise or counterclockwise) in which the current flows, we need to consider the right-hand rule. Since the current in the wire is directed downwards and the loop is moving to the right, the decreasing magnetic field through the loop would induce a current that creates a magnetic field pointing downwards to oppose the change. This means that the induced current in the loop must flow counterclockwise when viewed from above.

Finally, for part (c), if the loop only moves downwards at speed \( v \), the distance \( d \) between the wire and the loop does not change. Since the magnetic flux through the loop is a function of \( d \) and \( L \) (which are both constants in this scenario), the flux does not change as the loop moves downwards. Therefore, no emf is induced in the loop when it moves directly downwards because the magnetic field through the loop remains constant.

\subsubsection*{Question 5: (contains input figure)}

Prompt: 
\begin{spverbatim}
    You are a top professor of physics. Answer the physics problem given within the quotation marks. Show your working step by step.

    "Consider the circuit diagram seen in the figure. 

    a) Calculate the current seen by the ammeter.

    b) Calculate the energy delivered by the \(12\, \text{V}\) battery in \(4\) seconds."
\end{spverbatim}

\bigskip

\noindent Solution 1: \\

To solve this problem, we need to do the following steps:

\textbf{a) Calculate the current seen by the ammeter:}

\begin{enumerate}
  \item \textbf{Identify the circuit elements}: We have a \(12\,V\) battery, an ideal ammeter, four resistors (\(2\,\Omega\), \(2\,\Omega\), \(1\,\Omega\), and \(1\,\Omega\)), and a \(2\,V\) battery.
  \item \textbf{Analyze the circuit}: The \(2\,V\) battery is in series with the \(2\,\Omega\) resistor on its right. This series combination is in parallel with the \(2\,\Omega\) resistor on the left. The \(1\,\Omega\) resistors are in series and in parallel with the entire combination described above. The \(12\,V\) battery is providing the voltage for the entire circuit.
  \item \textbf{Calculate equivalent resistance}: To find the total current, we need to find the equivalent resistance of the circuit and use Ohm's law (\(I = V/R\)).
\end{enumerate}

\textbf{b) Calculate the energy delivered by the \(12\,V\) battery in \(4\) seconds:}

\begin{enumerate}
  \item \textbf{Use the power formula}: Power (\(P\)) is given by \(P = VI\), where \(V\) is the voltage and \(I\) is the current.
  \item \textbf{Calculate energy}: Energy (\(E\)) can be calculated using the formula \(E = Pt\), where \(t\) is the time in seconds.
\end{enumerate}

Let's start by calculating the total current seen by the ammeter.

Step 1: Calculate the Current Seen by the Ammeter

The Equivalent Resistance for the \(2\,V\) battery and \(2\,\Omega\) Resistor in Series

The equivalent resistance of a series circuit is simply the sum of the resistances:
\[ R_{series} = R_{battery} + R_{resistor} \]
The internal resistance of the \(2\,V\) battery is not given, so we'll assume it's negligible. Thus:
\[ R_{series} = 0\,\Omega + 2\,\Omega = 2\,\Omega \]

The Total Equivalent Resistance of the Parallel Combination

For resistors in parallel, the equivalent resistance (\(R_{parallel}\)) is given by:
\[ \frac{1}{R_{parallel}} = \frac{1}{R_1} + \frac{1}{R_2} \]
where \(R_1\) and \(R_2\) are the two resistances in parallel.

The Equivalent Resistance of the Two \(1\,\Omega\) Resistors in Series

These two resistors in series have a combined resistance of:
\[ R_{series-1\Omega} = 1\,\Omega + 1\,\Omega = 2\,\Omega \]

The Total Equivalent Resistance of the Entire Circuit

Now, we will consider the \(2\,\Omega\) resistor in parallel with the \(2\,\Omega\) equivalent resistance from the series combination of the \(2\,V\) battery and the \(2\,\Omega\) resistor, and in parallel with the \(2\,\Omega\) from the series combination of the two \(1\,\Omega\) resistors.

Once we calculate the total equivalent resistance, we can calculate the total current using the \(12\,V\) battery and Ohm's law.

I'll now calculate the total equivalent resistance.

Step 2: Calculate the Energy Delivered by the \(12\,V\) Battery in \(4\) Seconds

Now we can calculate the energy delivered by the battery using the power formula \(P = VI\) and the energy formula \(E = Pt\):

\begin{enumerate}
  \item \textbf{Power delivered by the battery}: \(P = 12\,V \times 18\,A\)
  \item \textbf{Energy delivered over 4 seconds}: \(E = P \times 4\,s\)
\end{enumerate}

Let's perform this calculation.

The power delivered by the \(12\,V\) battery is \(216\,W\), and the energy delivered over \(4\) seconds is \(864\,J\).

To summarize:
\begin{itemize}
    \item[a)] The current seen by the ammeter is \(18\,A\).
    \item[b)] The energy delivered by the \(12\,V\) battery in \(4\) seconds is \(864\,J\).
\end{itemize} 

\bigskip

\noindent Solution 2: \\

To solve this problem, we need to analyze the circuit and apply the principles of Ohm's Law and power in electrical circuits.

\textbf{Part a) Calculate the current seen by the ammeter.}

Firstly, we need to calculate the total resistance in the circuit. We have a parallel connection between the \(2\Omega\) resistor at the top and the combination of the \(2\Omega\) resistor and \(1\Omega\) resistor in series with the \(2\Omega\) cell.

The total resistance of the series combination on the right side of the circuit is:
\[ R_{series} = 2\Omega + 1\Omega = 3\Omega \]

The voltage across this series combination is the voltage of the cell minus the \(2\Omega\) internal resistance:
\[ V_{series} = 2V - (I \times 2\Omega) \]
However, we don't know the current \(I\) yet, so we will have to solve for it using the total resistance and the total voltage.

The total resistance seen by the voltage source is the parallel combination of the \(2\Omega\) resistor and the \(3\Omega\) resistance we just calculated. The equivalent resistance \( R_{eq} \) of two resistors \( R_1 \) and \( R_2 \) in parallel is given by:
\[ \frac{1}{R_{eq}} = \frac{1}{R_1} + \frac{1}{R_2} \]

For our circuit:
\[ \frac{1}{R_{total}} = \frac{1}{2\Omega} + \frac{1}{3\Omega} \]
\[ \frac{1}{R_{total}} = \frac{3}{6\Omega} + \frac{2}{6\Omega} \]
\[ \frac{1}{R_{total}} = \frac{5}{6\Omega} \]
\[ R_{total} = \frac{6\Omega}{5} = 1.2\Omega \]

Now, we can use Ohm's Law to find the total current \( I \) from the \(12\,V\) battery:
\[ I = \frac{V}{R} \]
\[ I = \frac{12V}{1.2\Omega} \]
\[ I = 10A \]

This is the current seen by the ammeter.

\textbf{Part b) Calculate the energy delivered by the \(12\,V\) battery in \(4\) seconds.}

The power delivered by the battery is given by:
\[ P = V \times I \]
\[ P = 12V \times 10A \]
\[ P = 120W \]

The energy \( E \) delivered in \( 4 \) seconds is then:
\[ E = P \times t \]
\[ E = 120W \times 4s \]
\[ E = 480J \]

So, the energy delivered by the \(12V\) battery in \(4\) seconds is \(480J\). \\

\noindent Solution 3: \\ 

Let's solve this problem step by step.

Part a) Calculate the current seen by the ammeter.

To find the total current that the ammeter reads, we need to first calculate the equivalent resistance of the circuit and then use Ohm's law to find the current.

The \(2\Omega\) resistor at the top and the series combination of the \(2\Omega\) resistor and \(2V\) battery in the middle are in parallel with each other. The \(1\Omega\) resistors on the right are in series and their combination is in parallel with the previously found parallel combination.

First, we calculate the equivalent resistance of the \(2\Omega\) resistor and the series combination of the \(2\Omega\) resistor and \(2V\) battery:

Since the \(2V\) battery will cause a drop, the voltage across the \(2\Omega\) resistor in series with it will be \( 12V - 2V = 10V \). The current through this resistor will be \( I = \frac{V}{R} = \frac{10V}{2\Omega} = 5A \).

The voltage drop across the \(2\Omega\) resistor in parallel with the battery and resistor combination will also be \(10V\) (since it's in parallel with the combination), and thus it will have a current of \( I = \frac{V}{R} = \frac{10V}{2\Omega} = 5A \).

These two resistors (or branches) are in parallel, so the voltage across them is the same and the currents add up, which means the equivalent resistance \( R_{eq1} \) for the parallel combination can be found by using the current division rule:
\[ \frac{1}{R_{eq1}} = \frac{1}{2\Omega} + \frac{1}{2\Omega} \]

Next, we need to find the equivalent resistance of the \(1\Omega\) resistors in series, which is simply \( R_{series} = 1\Omega + 1\Omega = 2\Omega \).

Now, \( R_{eq1} \) is in parallel with \( R_{series} \), so the total equivalent resistance \( R_{total} \) is:
\[ \frac{1}{R_{total}} = \frac{1}{R_{eq1}} + \frac{1}{R_{series}} \]

Once we find \( R_{total} \), we can find the total current \( I_{total} \) using Ohm's law:
\[ I_{total} = \frac{V_{total}}{R_{total}} = \frac{12V}{R_{total}} \]

Let's perform these calculations.

The equivalent resistance of the two parallel branches is \(1.0 \, \Omega\), and the total equivalent resistance of the entire circuit is approximately \(0.667 \, \Omega\). 

Using Ohm's law, the total current \(I_{total}\) seen by the ammeter is \(18 \, A\).

Part b) Calculate the energy delivered by the \(12\, \text{V}\) battery in \(4\) seconds.

The energy \(E\) delivered by a battery is given by the formula:
\[ E = P \times t \]
where \(P\) is the power delivered by the battery and \(t\) is the time.

The power \(P\) delivered by the battery can be found by:
\[ P = V_{total} \times I_{total} \]

Plugging in the values we have:
\[ P = 12V \times 18A \]
\[ E = P \times 4s \]

Let's calculate the energy.

The power delivered by the \(12\, \text{V}\) battery is \(216 \, \text{W}\) (watts), and the energy delivered by the battery in \(4\) seconds is \(864 \, \text{J}\) (joules).

\subsubsection*{Question 6:}

Prompt: 
\begin{spverbatim}
    You are a top professor of physics. Answer the physics problem given within the quotation marks. Show your working step by step.

    "Two infinite parallel planes have equal and opposite uniform surface charge density \( -\sigma \) and \( +\sigma \), where \( \sigma \) is positive. The planes are separated by a distance of \( 15 \) cm, and the potential difference between the plates is \( 120 \) V. 

    a) Determine the electric field, \( \underline{E} \), between the plates.

    b) An object with charge \( +0.001 \, \text{C} \) and a mass of \( 23 \, \text{g} \) is held at rest at the positive plate, then let go. Determine the acceleration of the object."
\end{spverbatim} 

\bigskip

\noindent Solution 1: \\

Let's solve this problem step by step.

Part a) Determine the Electric Field, \( \underline{E} \), between the plates.

The electric field \( E \) due to a single infinite plane with surface charge density \( \sigma \) is given by the equation \( E = \frac{\sigma}{2\varepsilon_0} \), where \( \varepsilon_0 \) is the permittivity of free space (\( \varepsilon_0 = 8.85 \times 10^{-12} \, \text{C}^2/\text{N}\cdot\text{m}^2 \)). For two infinite parallel planes with equal and opposite uniform surface charge densities, the electric field between the plates is uniform and can be found by doubling the electric field due to one plate, since the fields due to each plate add up in the region between the plates and cancel out outside. Therefore, \( E = \frac{\sigma}{\varepsilon_0} \).

Given the potential difference \( V = 120 \) V and the separation \( d = 15 \) cm \( = 0.15 \) m, we can also use the relation \( V = E \cdot d \) to find \( E \), without directly knowing \( \sigma \).

\[ E = \frac{V}{d} \]

Substituting the given values:

\[ E = \frac{120 \, \text{V}}{0.15 \, \text{m}} \]

Let's calculate \( E \).

Part b) Determine the acceleration of the object.

The force \( F \) on the object due to the electric field is given by \( F = qE \), where \( q = +0.001 \) C is the charge of the object. The acceleration \( a \) of the object can be found using Newton's second law, \( F = ma \), where \( m = 23 \) g \( = 0.023 \) kg is the mass of the object.

First, we find \( F \) using \( F = qE \), and then we solve for \( a \) using \( a = \frac{F}{m} \).

Let's calculate \( E \) and then proceed to calculate \( a \) for the object.

Solution

Part a) Electric Field between the plates

The electric field, \( \underline{E} \), between the plates is \( 800 \, \text{N/C} \).

Part b) Acceleration of the object

The acceleration of the object when it is released is approximately \( 34.78 \, \text{m/s}^2 \). 

This result indicates that the electric force acting on the charged object due to the electric field between the plates causes it to accelerate at a significant rate towards the negatively charged plate. \\

\noindent Solution 2: \\

To solve this physics problem, let's tackle each part step by step.

Part a) Determine the Electric Field, \( \underline{E} \), Between the Plates

The electric field \( \underline{E} \) created by an infinite plane of charge is given by the equation:
\[ E = \frac{\sigma}{2 \epsilon_0} \]
where \( \sigma \) is the surface charge density and \( \epsilon_0 \) is the permittivity of free space (\( 8.85 \times 10^{-12} \, \text{C}^2/\text{N}\cdot\text{m}^2 \)).

However, since there are two planes with equal and opposite charges, the total electric field between the plates is the sum of the fields due to each plate. Since the fields point in opposite directions away from the positive plate and towards the negative plate, they add up in the region between the plates. Thus, the magnitude of the electric field between the plates is:
\[ E_{\text{total}} = 2E = \frac{\sigma}{\epsilon_0} \]

The potential difference (\( V \)) between the plates is given by:
\[ V = E_{\text{total}} \cdot d \]
where \( d \) is the distance between the plates. We can rearrange this equation to solve for \( E_{\text{total}} \):
\[ E_{\text{total}} = \frac{V}{d} \]

Given:
- \( V = 120 \, \text{V} \)
- \( d = 15 \, \text{cm} = 0.15 \, \text{m} \)

Let's calculate \( E_{\text{total}} \):
\[ E_{\text{total}} = \frac{120 \, \text{V}}{0.15 \, \text{m}} \]

Part b) Determine the Acceleration of the Object

The force (\( F \)) on the object due to the electric field is given by:
\[ F = qE \]
where \( q \) is the charge of the object.

Newton's second law gives the acceleration (\( a \)) of the object as:
\[ a = \frac{F}{m} \]
where \( m \) is the mass of the object in kilograms (\( 23 \, \text{g} = 0.023 \, \text{kg} \)).

Given:
- \( q = 0.001 \, \text{C} \)

First, we'll calculate \( E_{\text{total}} \) from part a), then use it to find the force \( F \), and finally determine the acceleration \( a \) of the object. Let's start with calculating \( E_{\text{total}} \).

The electric field (\( \underline{E} \)) between the plates is \( 800 \, \text{N/C} \).

Now, let's proceed to calculate the acceleration of the object for part b).

Given:
- \( q = 0.001 \, \text{C} \) (charge of the object)
- \( m = 0.023 \, \text{kg} \) (mass of the object)

We'll use the electric field (\( E_{\text{total}} = 800 \, \text{N/C} \)) to find the force (\( F = qE \)) and then calculate the acceleration (\( a = F/m \)).

The acceleration of the object, when it is released from the positive plate, is approximately \( 34.78 \, \text{m/s}^2 \).

To summarize:
- The electric field (\( \underline{E} \)) between the plates is \( 800 \, \text{N/C} \).
- The acceleration of the object when released from rest at the positive plate is \( 34.78 \, \text{m/s}^2 \). \\

\noindent Solution 3: \\ 

To solve this physics problem, let's tackle it step by step for both parts a and b.

\textbf{Part a: Determining the Electric Field, \( \underline{E} \)}

The electric field \( E \) between two infinite parallel plates with uniform surface charge densities of \( +\sigma \) and \( -\sigma \) can be determined using Gauss's law, but in this context, we can more directly use the relationship between electric field (\( E \)), potential difference (\( V \)), and separation (\( d \)) between the plates. The formula is:

\[
E = \frac{V}{d}
\]

Given that \( V = 120 \) V (potential difference between the plates) and \( d = 15 \) cm (converted to meters for consistency in SI units, \( 15 \, \text{cm} = 0.15 \, \text{m} \)), we can calculate \( E \).

\textbf{Part b: Determining the Acceleration of the Object}

Once we have the electric field \( E \), we can determine the force on the object using Coulomb's law:

\[
F = qE
\]

where \( q = +0.001 \, \text{C} \) is the charge of the object. The acceleration \( a \) of the object due to this force can be found using Newton's second law of motion:

\[
a = \frac{F}{m}
\]

where \( m = 23 \, \text{g} = 0.023 \, \text{kg} \) (mass of the object, converted to kilograms to be consistent with SI units).

Let's calculate these step by step.

\textit{Step 1: Calculate \( E \)}

\[
E = \frac{120 \, \text{V}}{0.15 \, \text{m}}
\]

\textit{Step 2: Calculate \( F \)}

\[
F = qE
\]

We'll calculate \( E \) from step 1 first and then use it to find \( F \).

\textit{Step 3: Calculate \( a \)}

\[
a = \frac{F}{m}
\]

\textbf{Solution}

\textbf{Part a: Electric Field, \( \underline{E} \)}

The electric field \( E \) between the plates is \( 800 \, \text{N/C} \).

\textbf{Part b: Acceleration of the Object}

The force on the object due to the electric field is \( 0.8 \, \text{N} \), and the resulting acceleration of the object is \( 34.78 \, \text{m/s}^2 \).

Therefore:

a) The electric field between the plates is \( 800 \, \text{N/C} \).

b) The acceleration of the charged object when it is released is \( 34.78 \, \text{m/s}^2 \).

\subsubsection*{Question 7:}

Prompt: 
\begin{spverbatim}
    You are a top professor of physics. Answer the physics problem given within the quotation marks. Show your working step by step.

    "An RLC circuit has a resistor with resistance \( R = 600 \, \Omega \), capacitor with capacitance \( C = 1500 \, \text{pF} \), inductor with inductance \( L = 20 \, \text{mH} \).

    a) What is the resonant frequency, \( \omega_{0} \), of the circuit?

    b) The circuit is driven with an e.m.f source of the form 

    \[ \mathcal{E} = \mathcal{E}_{0} \cos(\omega t) \]

    Find an expression for the voltage drop across the inductor, \( \Delta V_{L} \), as a function of \( \mathcal{E}_{0} \), the impedance \( Z \), inductive reactance \( X_{L} \), angular frequency \( \omega \), and some phase shift \( \phi \).

    c) Describe the phase relationship between the source e.m.f and voltage across the inductor in the limits of low frequency (\( \omega \ll \omega_{0} \)) and high frequency (\( \omega \gg \omega_{0} \))."
\end{spverbatim}

\bigskip

\noindent Solution 1: \\

To solve this problem, let's tackle each part systematically.

Part a) Resonant Frequency, \( \omega_{0} \)

The resonant frequency \(\omega_{0}\) of an RLC circuit is given by the formula:

\[ \omega_{0} = \frac{1}{\sqrt{LC}} \]

where \(L\) is the inductance in henries (H) and \(C\) is the capacitance in farads (F). Given values are \(L = 20 \, \text{mH} = 20 \times 10^{-3} \, \text{H}\) and \(C = 1500 \, \text{pF} = 1500 \times 10^{-12} \, \text{F}\). Let's calculate \(\omega_{0}\).

Part b) Voltage Drop Across the Inductor, \( \Delta V_{L} \)

For an RLC circuit driven by an external e.m.f. \(\mathcal{E} = \mathcal{E}_{0} \cos(\omega t)\), the voltage drop across the inductor can be found using Ohm's law for AC circuits:

\[ \Delta V_{L} = I \cdot X_{L} \]

where \(I\) is the current in the circuit and \(X_{L} = \omega L\) is the inductive reactance. The total impedance \(Z\) of the circuit is given by:

\[ Z = \sqrt{R^2 + (X_{L} - X_{C})^2} \]

where \(X_{C} = \frac{1}{\omega C}\) is the capacitive reactance. The current \(I\) can be found from the source e.m.f and the impedance \(Z\):

\[ I = \frac{\mathcal{E}_{0} \cos(\omega t + \phi)}{Z} \]

where \(\phi\) is the phase shift between the current and the e.m.f, determined by the relative values of \(X_{L}\), \(X_{C}\), and \(R\). Inserting \(I\) into the expression for \(\Delta V_{L}\), we will derive an expression for \(\Delta V_{L}\) as a function of \(\mathcal{E}_{0}\), \(Z\), \(X_{L}\), \(\omega\), and \(\phi\).

Part c) Phase Relationship

The phase relationship between the source e.m.f and the voltage across the inductor depends on the frequency of the e.m.f relative to the resonant frequency of the circuit. At low frequencies (\(\omega \ll \omega_{0}\)), the circuit behaves more capacitive, and the voltage across the inductor lags the source e.m.f. At high frequencies (\(\omega \gg \omega_{0}\)), the circuit behaves more inductive, and the voltage across the inductor leads the source e.m.f.

Let's begin by calculating the resonant frequency \(\omega_{0}\) for part a, and then we will address parts b and c with the appropriate expressions and explanations.

\textbf{Part a) Resonant Frequency, \( \omega_{0} \)}

The resonant frequency \(\omega_{0}\) of the circuit is approximately \(182,574.19 \, \text{rad/s}\).

\textbf{Part b) Voltage Drop Across the Inductor, \( \Delta V_{L} \)}

To find an expression for the voltage drop across the inductor, \( \Delta V_{L} \), we can use the information and formulas discussed. The voltage drop across the inductor is related to the current \(I\) and the inductive reactance \(X_{L}\) by:

\[ \Delta V_{L} = I \cdot X_{L} \]

Since \(X_{L} = \omega L\) and \(I = \frac{\mathcal{E}_{0} \cos(\omega t + \phi)}{Z}\), where \(Z = \sqrt{R^2 + (\omega L - \frac{1}{\omega C})^2}\) and \(X_{C} = \frac{1}{\omega C}\), we can express \( \Delta V_{L} \) as:

\[ \Delta V_{L} = \frac{\mathcal{E}_{0} \cos(\omega t + \phi) \cdot \omega L}{\sqrt{R^2 + (\omega L - \frac{1}{\omega C})^2}} \]

This equation expresses \( \Delta V_{L} \) as a function of \(\mathcal{E}_{0}\), the impedance \(Z\), inductive reactance \(X_{L}\), angular frequency \(\omega\), and some phase shift \(\phi\).

\textbf{Part c) Phase Relationship}

- \textbf{Low Frequency (\(\omega \ll \omega_{0}\))}: At low frequencies, the capacitive reactance \(X_{C}\) is high (since \(X_{C} = \frac{1}{\omega C}\), and \(\frac{1}{\omega}\) is large), and it dominates over the inductive reactance \(X_{L}\). This results in the circuit behaving more like a capacitor, and the voltage across the inductor lags behind the source e.m.f. The phase shift \(\phi\) is such that the current (and thus the voltage across the inductor) lags behind the e.m.f.

- \textbf{High Frequency (\(\omega \gg \omega_{0}\))}: At high frequencies, the inductive reactance \(X_{L}\) is high (since \(X_{L} = \omega L\), and \(\omega\) is large), and it dominates over the capacitive reactance \(X_{C}\). This results in the circuit behaving more like an inductor, and the voltage across the inductor leads the source e.m.f. The phase shift \(\phi\) is such that the current (and thus the voltage across the inductor) leads ahead of the e.m.f.

This analysis describes how the phase relationship between the source e.m.f and the voltage across the inductor changes with the frequency of the driving e.m.f relative to the circuit's resonant frequency. \\

\noindent Solution 2: \\

To solve this problem, let's go step by step.

\textbf{Part a) Resonant Frequency, \( \omega_{0} \)}

The resonant frequency \(\omega_{0}\) of an RLC circuit is given by the formula:

\[
\omega_{0} = \frac{1}{\sqrt{LC}}
\]

where:
- \(L\) is the inductance in Henrys (H),
- \(C\) is the capacitance in Farads (F).

Given \(L = 20 \, \text{mH} = 20 \times 10^{-3} \, \text{H}\) and \(C = 1500 \, \text{pF} = 1500 \times 10^{-12} \, \text{F}\), we can substitute these values into the formula to find \(\omega_{0}\).

\textbf{Part b) Voltage Drop Across the Inductor, \( \Delta V_{L} \)}

The voltage drop across the inductor in an RLC circuit driven by an alternating current (AC) source can be expressed as:

\[
\Delta V_{L} = I \cdot X_{L} = I \cdot \omega L
\]

The current \(I\) in the circuit can be found from Ohm's Law, where \(I = \frac{\mathcal{E}}{Z}\), and the impedance \(Z\) of the RLC circuit is given by:

\[
Z = \sqrt{R^2 + (X_{L} - X_{C})^2}
\]

where \(X_{L} = \omega L\) is the inductive reactance, and \(X_{C} = \frac{1}{\omega C}\) is the capacitive reactance. Since the expression needs to be in terms of \(\mathcal{E}_{0}\), \(Z\), \(X_{L}\), \(\omega\), and some phase shift \(\phi\), we can express \(\Delta V_{L}\) as a function of these variables by incorporating the current expression.

\textbf{Part c) Phase Relationship}

- At low frequency \(\omega \ll \omega_{0}\), the circuit behaves more like a capacitor because the inductive reactance \(X_{L} = \omega L\) is low and the capacitive reactance \(X_{C} = \frac{1}{\omega C}\) is high, leading to the voltage across the inductor lagging the source e.m.f. by almost \(90^\circ\).

- At high frequency \(\omega \gg \omega_{0}\), the circuit behaves more like an inductor because the inductive reactance increases with \(\omega\) and the capacitive reactance decreases, leading to the voltage across the inductor leading the source e.m.f. by almost \(90^\circ\).

For part a), the resonant frequency, \( \omega_{0} \), of the circuit is approximately \( 182574.19 \, \text{rad/s} \).

\textbf{Part b) Voltage Drop Across the Inductor, \( \Delta V_{L} \)}

Given that \( \Delta V_{L} = I \cdot X_{L} \) and \( I = \frac{\mathcal{E}}{Z} \), we can express the voltage drop across the inductor as:

\[
\Delta V_{L} = \frac{\mathcal{E}_{0} \cos(\omega t + \phi)}{Z} \cdot \omega L
\]

Here, \(\mathcal{E}_{0} \cos(\omega t)\) is the applied e.m.f, and \(X_{L} = \omega L\) is the inductive reactance. The impedance \(Z\) of the circuit, as mentioned, is a combination of the resistor, inductor, and capacitor, given by \(Z = \sqrt{R^2 + (X_{L} - X_{C})^2}\), but for simplicity and direct relation, we keep \(Z\) as is in the expression. The phase shift \( \phi \) accounts for the phase difference introduced by the impedance's reactive components.

\textbf{Part c) Phase Relationship}

- \textbf{Low frequency (\( \omega \ll \omega_{0} \))}: At low frequencies, the inductor's reactance (\(X_{L} = \omega L\)) is small, and the capacitor's reactance (\(X_{C} = \frac{1}{\omega C}\)) is large. The circuit's behavior is dominated by the capacitance, resulting in the voltage across the inductor lagging the source e.m.f. The phase shift tends toward \(90^\circ\) lag as the capacitive effect dominates, indicating a reactive behavior where the current leads the voltage across the inductor.

- \textbf{High frequency (\( \omega \gg \omega_{0} \))}: At high frequencies, the inductor's reactance increases, making \(X_{L}\) large, while \(X_{C}\) decreases, becoming negligible. This causes the circuit to behave inductively, with the voltage across the inductor leading the source e.m.f. The phase shift tends toward \(90^\circ\) lead, indicating a scenario where the voltage across the inductor leads the current due to the inductive nature of the circuit at high frequencies. \\ 

\noindent Solution 3: \\

Let's tackle this problem step by step:

a) What is the resonant frequency, \( \omega_{0} \), of the circuit?

The resonant frequency \( \omega_{0} \) of an RLC circuit is given by the formula:
\[ \omega_{0} = \frac{1}{\sqrt{LC}} \]

Given:
\begin{itemize}
    \item \( L = 20 \, \text{mH} = 20 \times 10^{-3} \, \text{H} \)
    \item \( C = 1500 \, \text{pF} = 1500 \times 10^{-12} \, \text{F} \)
\end{itemize}

Let's calculate \( \omega_{0} \).

b) Find an expression for the voltage drop across the inductor, \( \Delta V_{L} \)

The voltage across the inductor in an RLC circuit driven by an AC source is given by:
\[ \Delta V_{L} = I \cdot X_{L} = I \cdot \omega L \]

Where:
\begin{itemize}
    \item \( I \) is the current through the circuit,
    \item \( X_{L} = \omega L \) is the inductive reactance,
    \item \( \omega \) is the angular frequency of the driving e.m.f,
    \item \( L \) is the inductance.
\end{itemize}

The current \( I \) in the circuit can be expressed as:
\[ I = \frac{\mathcal{E}_{0}}{Z} \cos(\omega t - \phi) \]

Thus, substituting \( I \) into the expression for \( \Delta V_{L} \):
\[ \Delta V_{L} = \frac{\mathcal{E}_{0}}{Z} \cos(\omega t - \phi) \cdot \omega L \]

c) Phase relationship between the source e.m.f and voltage across the inductor

\begin{itemize}
    \item At \textbf{low frequencies} (\( \omega \ll \omega_{0} \)):
    The impedance of the capacitor \( X_{C} = \frac{1}{\omega C} \) dominates over the inductive reactance \( X_{L} = \omega L \), making the circuit more capacitive. The voltage across the inductor lags the source voltage by nearly 90 degrees because the circuit behaves more like a capacitor.
    
    \item At \textbf{high frequencies} (\( \omega \gg \omega_{0} \)):
    The inductive reactance \( X_{L} = \omega L \) dominates over the impedance of the capacitor \( X_{C} = \frac{1}{\omega C} \), making the circuit more inductive. The voltage across the inductor leads the source voltage by nearly 90 degrees because the circuit behaves more like an inductor.
\end{itemize}

Let's calculate the resonant frequency \( \omega_{0} \) first, and then summarize the information for parts b and c.

Solution:

a) The resonant frequency, \( \omega_0 \), of the circuit is approximately \( 182574.19 rads^{-1} \)

b) The expression for the voltage drop across the inductor, \( \Delta V_L \), as a function of \( \mathcal{E}_0 \), the impedance \(Z\), inductive reactance \(X_L = \omega L\), angular frequency \( \omega \), and some phase shift \( \phi \), is:

\[ \Delta V_L = \frac{\mathcal{E}_0}{Z} \cos(\omega t - \phi) \cdot \omega L \]

c) Phase relationship:

At low frequencies (\( \omega \ll \omega_0 \)), the voltage across the inductor lags the source e.m.f by nearly 90 degrees because the circuit behaves more like a capacitive circuit.

At high frequencies (\( \omega \gg \omega_0 \)), the voltage across the inductor leads the source e.m.f by nearly 90 degrees because the circuit behaves more like an inductive circuit.

\subsubsection*{Question 8:}

Prompt:
\begin{spverbatim}
    You are a top professor of physics. Answer the physics problem given within the quotation marks. Show your working step by step.

    "Two linear magnetic media occupy the half-spaces above and below the \( xy \) plane. The space \( z > 0 \) is occupied by Material 1, with relative permeability \( \mu_{r1} = 1.493 \). The magnetic field \( \underline{B}_{1} \) in this region is spatially uniform and static, with positive \( x \) and \( z \) components, and zero \( y \) component. It is directed at an angle \( \alpha_{1} \) with respect to the positive \( z \) direction. The space \( z < 0 \) is occupied by Material 2, with relative permeability \( \mu_{r2}= 3.012 \). The magnetic field \( \underline{B}_{2} \) in this region is spatially uniform and static, with positive \( x \) and \( z \) components, and zero \( y \) component. It is directed at an angle of \( \alpha_{2} = 45^\circ \) with respect to the positive \( z \) direction. There is no free current flowing anywhere in the system. Using the information above, calculate the value of the angle \( \alpha_{1} \)."
\end{spverbatim}

\bigskip

\noindent Solution 1: \\

To solve for the angle \( \alpha_{1} \) of the magnetic field in Material 1, we'll use the boundary conditions for magnetic fields at the interface between two magnetic media. The boundary conditions state that:

1. The tangential components of the magnetic field (\( \underline{H} \)) are continuous across the boundary, which means \( H_{1x} = H_{2x} \).
2. The normal components of the magnetic field (\( \underline{B} \)) are continuous across the boundary, which means \( B_{1z} = B_{2z} \).

Given that \( \underline{B} = \mu \underline{H} \) where \( \mu \) is the permeability of the material (\( \mu = \mu_0 \mu_r \)), and since \( \mu_0 \), the permeability of free space, is a constant, we can simplify the boundary conditions in terms of relative permeability \( \mu_r \) and the magnetic field \( \underline{B} \).

Given data:
- \( \mu_{r1} = 1.493 \)
- \( \mu_{r2} = 3.012 \)
- \( \alpha_{2} = 45^\circ \)

We know that \( B_{1x}/B_{1z} = \tan(\alpha_{1}) \) and \( B_{2x}/B_{2z} = \tan(\alpha_{2}) \).

From the boundary condition (1), since \( H_{1x} = H_{2x} \) and \( H = B/\mu \), we get:
\[ B_{1x}/\mu_{r1} = B_{2x}/\mu_{r2} \]

Given \( \alpha_{2} = 45^\circ \), \( \tan(\alpha_{2}) = 1 \), implying \( B_{2x} = B_{2z} \).

To find \( \alpha_{1} \), we'll express \( B_{1x} \) and \( B_{1z} \) in terms of known quantities and apply the boundary conditions.

1. From the continuity of \( B_{1z} = B_{2z} \) and knowing \( B_{2x} = B_{2z} \), we can establish a relationship between the \( z \)-components of the magnetic field in both materials.

2. Using \( B_{1x}/B_{1z} = \tan(\alpha_{1}) \) and the fact that \( B_{1x}/\mu_{r1} = B_{2x}/\mu_{r2} \), we can solve for \( \alpha_{1} \).

Let's calculate \( \alpha_{1} \) with the given values.

\begin{verbatim}
from math import atan, degrees

# Given values
mu_r1 = 1.493
mu_r2 = 3.012
tan_alpha2 = 1  # Since alpha2 = 45 degrees, tan(alpha2) = 1

# Solving for tan(alpha1) using the relationship B1x/B1z = B2x/B2z and the boundary conditions
# B1x/mu_r1 = B2x/mu_r2 implies that tan(alpha1) = tan(alpha2) * (mu_r1/mu_r2)
tan_alpha1 = tan_alpha2 * (mu_r1 / mu_r2)

# Calculate alpha1 in degrees
alpha1 = degrees(atan(tan_alpha1))
alpha1
\end{verbatim}

The angle \( \alpha_{1} \) of the magnetic field in Material 1 with respect to the positive \( z \) direction is approximately \( 26.37^\circ \). \\ 

\noindent Solution 2: \\ 

To solve this problem, we will use the boundary conditions for the magnetic field at the interface between two magnetic media. Specifically, we'll focus on the tangential and normal components of the magnetic field, \( \underline{B} \), and how they behave at the boundary.

The boundary conditions are:
1. The tangential components of the magnetic field (\(B_{t}\)) are continuous across the boundary. This means that \(B_{1t} = B_{2t}\).
2. The normal components of the magnetic field (\(B_{n}\)) are such that \(\frac{B_{1n}}{\mu_{1}} = \frac{B_{2n}}{\mu_{2}}\), where \( \mu \) represents the permeability of the material.

Given that the magnetic field in both regions has positive \(x\) and \(z\) components and zero \(y\) component, we can describe the magnetic fields as follows:
- In Material 1 (\(z > 0\)): \( \underline{B}_{1} = B_{1x} \hat{i} + B_{1z} \hat{k} \)
- In Material 2 (\(z < 0\)): \( \underline{B}_{2} = B_{2x} \hat{i} + B_{2z} \hat{k} \)

The angles \( \alpha_{1} \) and \( \alpha_{2} \) with respect to the positive \(z\) direction imply the following trigonometric relationships:
- For Material 1: \( \tan(\alpha_{1}) = \frac{B_{1x}}{B_{1z}} \)
- For Material 2: \( \tan(\alpha_{2}) = \frac{B_{2x}}{B_{2z}} = \tan(45^\circ) = 1 \) \( \rightarrow B_{2x} = B_{2z} \)

Since the tangential components (\(B_{tx}\)) are continuous across the boundary, \( B_{1x} = B_{2x} \). Also, given that \( \alpha_{2} = 45^\circ \), it means that \( B_{2x} = B_{2z} \), implying an equal magnitude for the \(x\) and \(z\) components of \( \underline{B}_{2} \).

Using the boundary condition for the normal components, we need to relate \(B_{1z}\) and \(B_{2z}\) through the permeabilities. However, since there is no free current and we're considering the continuity of \(B_{t}\), we primarily focus on the tangential component's continuity and the angles to determine \( \alpha_{1} \).

Given \( \tan(\alpha_{2}) = 1 \), and assuming \(B_{1x} = B_{2x}\), we must find \( \alpha_{1} \) knowing \(B_{1z}\) and \(B_{2z}\) are related through the materials' relative permeabilities and the tangential components are equal. However, the direct relationship between \(B_{1z}\) and \(B_{2z}\) through permeabilities isn't directly needed for calculating \( \alpha_{1} \) because we only need to establish the relationship between \(B_{1x}\) and \(B_{1z}\) to find \( \alpha_{1} \).

So, to find \( \alpha_{1} \), we use the fact that \(B_{1x} = B_{2x}\) and \(B_{2x} = B_{2z}\), but since we're looking for an angle, we need to realize that without specific values for \(B_{1x}\) and \(B_{1z}\), we cannot directly calculate \( \alpha_{1} \) without more information on the magnitudes of \( \underline{B}_{1} \) and \( \underline{B}_{2} \).

However, the approach outlined above shows the principles we would use to connect the fields across the boundary. The continuity of \(B_{tx}\) and the known angle \( \alpha_{2} \) are critical clues, but without explicit values for the components of \( \underline{B}_{1} \) or \( \underline{B}_{2} \), the calculation of \( \alpha_{1} \) cannot be completed as straightforwardly as might seem. This problem seems to suggest a deeper exploration of the magnetic field's behavior across the boundary, considering the relationship between the fields and the materials' permeabilities. To accurately calculate \( \alpha_{1} \), we'd need either more information about the magnitudes of the magnetic fields or to apply additional principles that aren't directly outlined in the given information. \\

\noindent Solution 3: \\

To solve this problem, we will use the boundary conditions for magnetic fields at the interface between two different media. The boundary conditions that are relevant for this problem are:

1. The tangential components of the magnetic field (\(H\)) must be continuous across the boundary. This means that \(H_{1t} = H_{2t}\), where subscript \(t\) denotes the tangential component (parallel to the boundary).
2. The normal components of the magnetic field density (\(B\)) must also be continuous across the boundary. This means that \(B_{1n} = B_{2n}\), where subscript \(n\) denotes the normal component (perpendicular to the boundary).

Given:
- The relative permeability of Material 1, \(\mu_{r1} = 1.493\).
- The relative permeability of Material 2, \(\mu_{r2} = 3.012\).
- The angle of \(\underline{B}_{2}\) with respect to the positive \(z\) direction, \(\alpha_{2} = 45^\circ\).

To find:
- The angle \(\alpha_{1}\) of \(\underline{B}_{1}\) with respect to the positive \(z\) direction.

The magnetic field \(\underline{B}\) is related to the magnetic field intensity \(\underline{H}\) by the equation \(\underline{B} = \mu \underline{H}\), where \(\mu\) is the permeability of the material. The permeability \(\mu\) is given by \(\mu = \mu_r \mu_0\), with \(\mu_0\) being the permeability of free space (\(4\pi \times 10^{-7} \, \text{H/m}\)) and \(\mu_r\) the relative permeability of the material.

For the tangential component, continuity gives:
\[H_{1t} = H_{2t}\]

Since \(H = B/\mu\), we can rewrite this as:
\[\frac{B_{1t}}{\mu_{1}} = \frac{B_{2t}}{\mu_{2}}\]

The tangential component of \(B\) can be found using trigonometry, where \(B_{t} = B \sin(\alpha)\) and the normal component \(B_{n} = B \cos(\alpha)\).

Given that \(\alpha_{2} = 45^\circ\), we know that for Material 2, \(B_{2t} = B_{2} \sin(45^\circ)\) and \(B_{2n} = B_{2} \cos(45^\circ)\).

Since \(B_{1n} = B_{2n}\), we can say \(B_{1} \cos(\alpha_{1}) = B_{2} \cos(45^\circ)\).

To solve for \(\alpha_{1}\), we need to use the fact that the tangential components of \(H\) are equal, and we need to express everything in terms of quantities we know. However, we directly don't know \(B_1\) or \(B_2\), but we can relate them through the continuity of the normal component of \(B\) and the equality of the tangential component of \(H\).

The equality of \(H_{1t}\) and \(H_{2t}\) can be rewritten in terms of \(B\) and \(\mu\):
\[\frac{B_{1} \sin(\alpha_{1})}{\mu_{0}\mu_{r1}} = \frac{B_{2} \sin(45^\circ)}{\mu_{0}\mu_{r2}}\]

Since \(B_{1n} = B_{2n}\), we know that the \(z\)-components are equal, so we also have:
\[B_{1} \cos(\alpha_{1}) = B_{2} \cos(45^\circ)\]

To find \(\alpha_{1}\), we don't actually need to calculate the magnitudes of \(B_1\) and \(B_2\), because we can use the ratio of the tangential components to express everything in terms of known angles and permeabilities. Let's rearrange the equations to solve for \(\alpha_{1}\) directly.

To solve for \(\alpha_{1}\), we will use the boundary conditions and known values to find the relationship between the angles and the relative permeabilities of the two materials. Given that the tangential components of the magnetic field intensities are equal across the boundary, we have:

\[
\frac{B_{1t}}{\mu_{1}} = \frac{B_{2t}}{\mu_{2}}
\]

And since \(B_t = B \sin(\alpha)\) for the tangential component and \(\mu = \mu_r \mu_0\), we can write:

\[
\frac{B_{1} \sin(\alpha_{1})}{\mu_{r1} \mu_0} = \frac{B_{2} \sin(45^\circ)}{\mu_{r2} \mu_0}
\]

Given that \(\alpha_{2} = 45^\circ\), the equation simplifies to:

\[
\frac{\sin(\alpha_{1})}{\mu_{r1}} = \frac{\sin(45^\circ)}{\mu_{r2}}
\]

Since the permeabilities \(\mu_{r1}\) and \(\mu_{r2}\) are known, we can solve this equation for \(\alpha_{1}\). We can cancel \(\mu_0\) from both sides because it is a constant and does not affect the ratio. Thus, the equation to solve for \(\alpha_{1}\) becomes:

\[
\sin(\alpha_{1}) = \sin(45^\circ) \times \frac{1.493}{3.012}
\]

Using the given values of \(\mu_{r1} = 1.493\) and \(\mu_{r2} = 3.012\), we can now calculate \(\alpha_{1}\).

\[
\sin(\alpha_{1}) = \sin(45^\circ) \times \frac{1.493}{3.012}
\]

Let's calculate the value of \(\alpha_{1}\).

The value of the angle \(\alpha_{1}\) with respect to the positive \(z\) direction is approximately \(20.52^\circ\).

\subsubsection*{Question 9:}

Prompt:
\begin{spverbatim}
    You are a top professor of physics. Answer the physics problem given within the quotation marks. Show your working step by step.

    "A transverse electromagnetic wave propagating in vacuum has an electric field which has complex representation:

    \[ \underline{E}(\underline{r}, t) = \underline{E}_{0} \exp (i(\underline{k} \cdot \underline{r} - \omega t)) \]

    With real valued vectors \( \underline{E}_{0} \), \( \underline{k} \).

    a) Apply Faraday’s Law and show magnetic intensity of the wave has magnitude given by 

    \[ H(\underline{r}, t) = \sqrt{\frac{\epsilon_{0}}{\mu_{0}}} E(\underline{r}, t) \]

    b) Show the Poynting flux averaged over one period of oscillation is given by 

    \[ \left< \underline{N} \right> = \frac{1}{2} \sqrt{\frac{\mu_{0}}{\epsilon_{0}}} (H_{0})^{2} \hat{\underline{k}} \]

    where \( H_{0} \) denotes the amplitude of the magnetic intensity.
    
    c) The pressure exerted by radiation has intensity \( I \) on a perfect planar reflector has the value \( \frac{2I}{c} \), where \( c \) is the speed of light. Show that when the angle of incidence of radiation is \( \alpha \), the radiation pressure becomes \( 2I\cos^{2}{\alpha}/{c} \)."
\end{spverbatim}

\bigskip

\noindent Solution 1: \\
Let's tackle this physics problem step by step, beginning with part a), then moving on to part b), and finally addressing part c).

Part a) Faraday’s Law and Magnetic Intensity

Faraday's Law in the context of electromagnetic waves can be represented as:

\[
\nabla \times \underline{E} = -\frac{\partial \underline{B}}{\partial t}
\]

Given the electric field:

\[
\underline{E}(\underline{r}, t) = \underline{E}_{0} \exp (i(\underline{k} \cdot \underline{r} - \omega t))
\]

We can apply the curl operator \(\nabla \times\) to \(\underline{E}\), considering the spatial variation (with respect to \(\underline{r}\)):

\[
\nabla \times \underline{E} = i\underline{k} \times \underline{E}_{0} \exp (i(\underline{k} \cdot \underline{r} - \omega t))
\]

From Faraday's Law, equating the above expression to \(-\frac{\partial \underline{B}}{\partial t}\), we get:

\[
i\underline{k} \times \underline{E} = -\frac{\partial \underline{B}}{\partial t}
\]

Since \(\underline{B} = \mu_{0} \underline{H}\), we can write:

\[
i\underline{k} \times \underline{E} = -\mu_{0} \frac{\partial \underline{H}}{\partial t}
\]

Given that \(\underline{E} = \underline{E}_{0} \exp (i(\underline{k} \cdot \underline{r} - \omega t))\), we find that \(\underline{H}\) must have a similar form but with the magnetic field vector perpendicular to both \(\underline{k}\) and \(\underline{E}\), and \(\frac{\partial \underline{H}}{\partial t} = -i\omega \underline{H}\).

Substituting the time derivative of \(\underline{H}\) into our equation gives us:

\[
i\underline{k} \times \underline{E} = i\omega\mu_{0} \underline{H}
\]

\[
\underline{k} \times \underline{E} = \omega\mu_{0} \underline{H}
\]

Given the relationship between \(\omega\), \(\underline{k}\), and the speed of light \(c\) in a vacuum:

\[
\omega = ck = \frac{ck}{\sqrt{\mu_{0}\epsilon_{0}}}
\]

We can rearrange for \(\underline{H}\) as:

\[
\underline{H} = \sqrt{\frac{\epsilon_{0}}{\mu_{0}}} \underline{E}
\]

Thus, the magnitude of the magnetic intensity of the wave is:

\[
H(\underline{r}, t) = \sqrt{\frac{\epsilon_{0}}{\mu_{0}}} E(\underline{r}, t)
\]

Part b) Poynting Flux Averaged Over One Period

The Poynting vector \(\underline{N}\) is given by:

\[
\underline{N} = \underline{E} \times \underline{H}
\]

For a wave propagating in vacuum, the time-averaged Poynting vector over one period of oscillation can be expressed as:

\[
\left< \underline{N} \right> = \frac{1}{2} Re(\underline{E} \times \underline{H}^{*})
\]

Substituting \(\underline{H} = \sqrt{\frac{\epsilon_{0}}{\mu_{0}}} \underline{E}\) into the expression for \(\underline{N}\) and considering the cross-product yields:

\[
\left< \underline{N} \right> = \frac{1}{2} \sqrt{\frac{\mu_{0}}{\epsilon_{0}}} \left| \underline{E} \right|^2 \hat{\underline{k}}
\]

Since \(H = \sqrt{\frac{\epsilon_{0}}{\mu_{0}}} E\), and \(H_{0}\) is the amplitude of \(\underline{H}\), we have:

\[
\left< \underline{N} \right> = \frac{1}{2} \sqrt{\frac{\mu_{0}}{\epsilon_{0}}} (H_{0})^{2} \hat{\underline{k}}
\]

Part c) Radiation Pressure on a Perfect Planar Reflector

The pressure exerted by radiation, \( P\), with intensity \(I\) on a perfect planar reflector is given by \(P=\frac{2I}{c}\) for normal incidence. When the angle of incidence \(\alpha\)is not normal, the component of momentum perpendicular to the surface (which is responsible for pressure) is reduced by \(\cos(\alpha)\). Since the radiation is reflected, this reduction applies twice (once for the incoming and once for the outgoing radiation), leading to:

\[P = \frac{2I \cos^{2}(\alpha)}{c}\]

This shows that the radiation pressure exerted on a perfect planar reflector at an angle of incidence \(\alpha\) is indeed \(2I\cos^{2}(\alpha) / c\) \\

\noindent Solution 2: \\

To tackle this problem, we'll address each part step by step, employing fundamental concepts from electromagnetism, particularly Maxwell's equations and the properties of electromagnetic waves in a vacuum.

Part a) Faraday’s Law and Magnetic Intensity

Faraday's law in differential form is given by:

\[
\nabla \times \underline{E} = -\frac{\partial \underline{B}}{\partial t}
\]

Given the electric field:

\[
\underline{E}(\underline{r}, t) = \underline{E}_{0} \exp(i(\underline{k} \cdot \underline{r} - \omega t))
\]

We need to find the curl of \(\underline{E}\) and equate it to \(-\frac{\partial \underline{B}}{\partial t}\) to find the magnetic field \(\underline{B}\), and from there, find the magnetic intensity \(\underline{H}\).

The curl of \(\underline{E}\) in complex notation is:

\[
\nabla \times \underline{E} = i\underline{k} \times \underline{E}_{0} \exp(i(\underline{k} \cdot \underline{r} - \omega t))
\]

The magnetic field \(\underline{B}\) is related to its time derivative as:

\[
-\frac{\partial \underline{B}}{\partial t} = i\underline{k} \times \underline{E}
\]

Therefore,

\[
\frac{\partial \underline{B}}{\partial t} = -i\omega \underline{B} = i\underline{k} \times \underline{E}
\]

From this, we can express \(\underline{B}\) in terms of \(\underline{E}\):

\[
\underline{B} = \frac{1}{\omega} \underline{k} \times \underline{E}
\]

In a vacuum, the relationship between \(\underline{B}\) and \(\underline{H}\) is given by:

\[
\underline{B} = \mu_{0} \underline{H}
\]

Thus,

\[
\underline{H} = \frac{1}{\mu_{0}\omega} \underline{k} \times \underline{E}
\]

Given that \(|\underline{k}| = \frac{\omega}{c}\) and \(c = \frac{1}{\sqrt{\epsilon_{0}\mu_{0}}}\), we find:

\[
\underline{H} = \sqrt{\frac{\epsilon_{0}}{\mu_{0}}} \underline{E}
\]

Part b) Poynting Flux Averaged Over One Period

The Poynting vector \(\underline{N}\) is given by:

\[
\underline{N} = \underline{E} \times \underline{H}
\]

For the wave described, substituting \(\underline{H}\) from part a) into this expression, and averaging over one period, we get:

\[
\left< \underline{N} \right> = \frac{1}{2} \sqrt{\frac{\mu_{0}}{\epsilon_{0}}} (H_{0})^{2} \hat{\underline{k}}
\]

This follows from the fact that the average of \(\sin^2(x)\) or \(\cos^2(x)\) over one period is \(\frac{1}{2}\), and considering the propagation direction \(\hat{\underline{k}}\) and the amplitude \(H_{0}\).

Part c) Radiation Pressure on a Perfect Planar Reflector

The intensity \(I\) is related to the Poynting vector by:

\[
I = \left< \underline{N} \right> \cdot \hat{\underline{n}}
\]

For a perfect planar reflector, the radiation pressure \(P\) is given by:

\[
P = \frac{2I}{c}
\]

When the angle of incidence \(\alpha\) is not perpendicular, the component of the intensity normal to the surface is \(I \cos^2(\alpha)\), because the intensity is proportional to the square of the amplitude of the electromagnetic wave, and only the component normal to the surface contributes to pressure.

Thus, the radiation pressure \(P\) becomes:

\[
P = \frac{2I\cos^2(\alpha)}{c}
\]

This concludes the step-by-step solution for the given physics problem, illustrating how electromagnetic theory applies to wave propagation, energy flux, and radiation pressure. \\

\noindent Solution 3: \\

To solve this physics problem, we'll approach each part step by step, using relevant physics principles and equations.

Part a) Applying Faraday's Law

Faraday's Law of electromagnetic induction states that the time-varying magnetic field creates a curling electric field. In mathematical terms, it is expressed as:

\[
\nabla \times \underline{E} = - \frac{\partial \underline{B}}{\partial t}
\]

Given the electric field \(\underline{E}(\underline{r}, t) = \underline{E}_{0} \exp (i(\underline{k} \cdot \underline{r} - \omega t))\), we can calculate the curl of \(\underline{E}\) to find the magnetic field \(\underline{B}\).

The magnetic field \(\underline{B}\) is related to the magnetic intensity \(\underline{H}\) by:

\[
\underline{B} = \mu_{0} \underline{H}
\]

So, we need to calculate \(\nabla \times \underline{E}\) and then integrate with respect to time to find \(\underline{B}\), and hence, \(\underline{H}\).

\textbf{Step 1: Calculate} \(\nabla \times \underline{E}\)

\[
\nabla \times \underline{E} = \nabla \times (\underline{E}_{0} \exp(i(\underline{k} \cdot \underline{r} - \omega t)))
\]

Since \(\underline{E}_{0}\) and \(\underline{k}\) are constants with respect to spatial coordinates, the curl of \(\underline{E}\) focuses on the derivative of the exponential term, which yields \(i\underline{k} \times \underline{E}\).

\[
\nabla \times \underline{E} = i\underline{k} \times \underline{E}_{0} \exp(i(\underline{k} \cdot \underline{r} - \omega t))
\]

\textbf{Step 2: Relate} \(\nabla \times \underline{E}\) \textbf{to} \(\underline{B}\)

\[
- \frac{\partial \underline{B}}{\partial t} = i\underline{k} \times \underline{E}
\]

Integrating both sides with respect to \(t\), we obtain \(\underline{B}\) (noting that integration of \(i\omega\) leads to the same exponential term):

\[
\underline{B} = \frac{1}{\omega} \underline{k} \times \underline{E}
\]

\textbf{Step 3: Relate} \(\underline{B}\) \textbf{to} \(\underline{H}\)

Given \(\underline{B} = \mu_{0} \underline{H}\), and using the relation between \(\underline{B}\) and \(\underline{E}\) derived above:

\[
\underline{H} = \frac{1}{\mu_{0}\omega} \underline{k} \times \underline{E}
\]

Using the relation between the magnitude of \(\underline{E}\) and \(\underline{H}\), and considering the speed of light in vacuum \(c = \frac{1}{\sqrt{\epsilon_{0}\mu_{0}}}\), we have:

\[
E = cB = c\mu_{0}H
\]

Solving for \(H\), we find:

\[
H = \frac{E}{c\mu_{0}} = \sqrt{\frac{\epsilon_{0}}{\mu_{0}}} E
\]

This matches the given relation for the magnitude of the magnetic intensity of the wave.

Part b) Poynting Flux Averaged Over One Period

The Poynting vector \(\underline{N}\) represents the directional energy flux (the rate of energy transfer per unit area) of an electromagnetic field. The average Poynting vector over one period of oscillation is given by:

\[
\left< \underline{N} \right> = \frac{1}{2} \text{Re}(\underline{E} \times \underline{H}^*)
\]

Given \(\underline{H}\) and using \(\underline{E}\), we can calculate the cross product \(\underline{E} \times \underline{H}^*\) where \(\underline{H}^*\) is the complex conjugate of \(\underline{H}\).

Since \(\underline{E}\) and \(\underline{H}\) are perpendicular in a plane wave and have a simple relationship in their magnitudes, the cross product simplifies to the product of their magnitudes. Given the relationship between \(E\) and \(H\), and that \(\underline{E}\) and \(\underline{H}\) are in phase in a vacuum, the magnitude of the Poynting vector simplifies to:

\[
\left< \underline{N} \right> = \frac{1}{2} \sqrt{\frac{\mu_{0}}{\epsilon_{0}}} (H_{0})^{2} \hat{\underline{k}}
\]

This matches the given relation for the average Poynting flux.

Part c) Radiation Pressure on a Perfect Planar Reflector

The radiation pressure \( P\) exerted by an electromagnetic wave on a surface is related to the intensity \(I\) of the wave and the speed of light \(c\) by:

\[
P = \frac{I}{c}
\]

For a perfect reflector, the momentum of the photon is reversed upon reflection, doubling the pressure:

\[
P = \frac{2I}{c}
\]

When the angle of incidence \(\alpha\) is not normal, the effective area \(A_{eff}\) exposed to the radiation is \(A\cos(\alpha)\), where \(A\) is the actual area of the surface. Thus, the pressure \(P\) considering the angle of incidence \(\alpha\) becomes:

\[
P(\alpha) = \frac{2I\cos(\alpha)}{c}
\]

Considering the radiation pressure is exerted in the normal direction to the surface, the component of the pressure effective in the direction perpendicular to the surface is \(P\cos(\alpha)\), leading to:

\[
P(\alpha) = \frac{2I\cos^{2}(\alpha)}{c}
\]

This matches the given relation for the radiation pressure when the angle of incidence of radiation is \(\alpha\) 

\subsubsection*{Question 10:}

Prompt:
\begin{spverbatim}
    You are a top professor of physics. Answer the physics problem given within the quotation marks. Show your working step by step.

    "Consider 2 inertial reference frames \( S \) and \( S' \). The frames are aligned in such a way that origins of both frames coincide at time zero within both frames (\( t = t' = 0 \)). The frame \( S' \) moves with velocity \( v \) in the \( x \) direction as seen by \( S \). The transformation of electric and magnetic fields from frame \( S \) to \( S' \) is given by 

    \[
    \begin{aligned}
    E'_{x} &= E_{x}, & E'_{y} &= \gamma(E_{y} - vB_{z}), & E'_{z} &= \gamma(E_{z} + vB_{y}) \\
    B'_{x} &= B_{x}, & B'_{y} &= \gamma(B_{y} + \frac{v}{c^{2}} E_{z}), & B'_{z} &= \gamma(B_{z} - \frac{v}{c^{2}} E_{y})
    \end{aligned}
    \]

    a) Using the transformations given, show that the scalar product, \( (\underline{E} \cdot \underline{B}) \), is invariant under transformation.

    b) A plane electromagnetic wave observed in the reference frame \( S \) propagates in a vacuum along the \( x \) direction. In frame \( S \), it is represented by the Cartesian representation: 

    \[ \underline{E} = E_{0} \hat{\underline{y}} \exp(i(kx - \omega t)) \]

    with \( E_{0} \), a real value denoting the amplitude, \( k \) is the wavevector, and \( \omega \) the angular frequency. \( (kx - \omega t) \) defines the phase of the wave in frame \( S \). Show that this phase in reference frame \( S' \) is written \( (k'x' - \omega' t') \), where primed coordinates correspond to the frame \( S' \) and

    \[ \omega' = \gamma(\omega - kv), \quad k' = \gamma(k - \frac{\omega v}{c^{2}}) \]

    c) Examine the relationship between the wave frequencies \( \omega \) and \( \omega' \) in the limit \( v/c \) approaches \( 0 \). What physical phenomenon does this represent? Justify your answer."
\end{spverbatim}

\bigskip

\noindent Solution 1: \\

Let's solve this problem step by step:

Part a) Invariance of the Scalar Product \( (\underline{E} \cdot \underline{B}) \)

We need to show that the scalar product \( (\underline{E} \cdot \underline{B}) \) is invariant under the transformation between the reference frames \( S \) and \( S' \). The scalar product in frame \( S \) is \( E_x B_x + E_y B_y + E_z B_z \), and in frame \( S' \), it becomes \( E'_x B'_x + E'_y B'_y + E'_z B'_z \).

Given the transformations:
\[
\begin{aligned}
E'_{x} &= E_{x}, & E'_{y} &= \gamma(E_{y} - vB_{z}), & E'_{z} &= \gamma(E_{z} + vB_{y}) \\
B'_{x} &= B_{x}, & B'_{y} &= \gamma(B_{y} + \frac{v}{c^{2}} E_{z}), & B'_{z} &= \gamma(B_{z} - \frac{v}{c^{2}} E_{y})
\end{aligned}
\]

Let's substitute these transformations into the scalar product \( (\underline{E}' \cdot \underline{B}') \) and simplify to show invariance.

Part b) Phase of the Electromagnetic Wave in Frame \( S' \)

Given the electromagnetic wave in \( S \):
\[ \underline{E} = E_{0} \hat{\underline{y}} \exp(i(kx - \omega t)) \]

We need to show how the phase \( (kx - \omega t) \) transforms into \( (k'x' - \omega' t') \) in frame \( S' \), using the given transformations for \(\omega'\) and \(k'\):
\[ \omega' = \gamma(\omega - kv), \quad k' = \gamma(k - \frac{\omega v}{c^{2}}) \]

This involves using the Lorentz transformation for coordinates and time between frames \( S \) and \( S' \) and then applying it to the phase of the wave.

Part c) Relationship Between \( \omega \) and \( \omega' \) as \( v/c \rightarrow 0 \)

We will examine the given expressions for \( \omega' \) and \( k' \) in the limit \( v/c \) approaches \( 0 \) to understand the physical phenomenon it represents, likely touching upon Doppler shift or time dilation effects based on the relativistic context.

Let's start with Part a) by substituting the given transformations into the scalar product formula and demonstrating its invariance.

Upon attempting to simplify the difference between the scalar products \( (\underline{E} \cdot \underline{B}) \) in frames \( S \) and \( S' \), we arrive at a complex expression. However, to demonstrate the invariance, we aim to show that this expression simplifies to zero, meaning the scalar product remains unchanged under the transformation.

The calculation yields an expression that appears non-zero at first glance. However, in the context of demonstrating invariance, we should expect that any discrepancies in the formulation or simplification would cancel out with a more detailed analysis, particularly taking into account properties of \( \gamma \) (the Lorentz factor) and the relationships between electric and magnetic fields in relativistic transformations.

The expected outcome is that the scalar product \( (\underline{E} \cdot \underline{B}) \) is invariant under Lorentz transformations, which follows from the symmetry of Maxwell's equations under these transformations. My initial calculation does not directly show this due to the complexity of the expression. A proper demonstration of invariance would involve carefully accounting for the terms and potentially leveraging additional properties of the electromagnetic fields and the Lorentz factor \( \gamma = \frac{1}{\sqrt{1 - \frac{v^2}{c^2}}} \).

Let's proceed to part b) and derive the phase of the electromagnetic wave in frame \( S' \), as this might provide a clearer path to understanding the transformations without the algebraic complexity encountered here.

To find the phase of the electromagnetic wave in the reference frame \( S' \), we attempted to apply Lorentz transformations for time and space directly to the wave's phase. The simplified expression for the transformed phase is:
\[ I \left( \frac{c^2 k (\gamma t v + x') - \omega (c^2 t' + \gamma v x)}{c^2 \gamma} \right) \]

This expression represents the phase in terms of the original frame variables and the Lorentz transformation parameters. However, it doesn't directly simplify to the form \( I(k'x' - \omega' t') \) with the given relationships for \(\omega'\) and \(k'\) as:
\[ \omega' = \gamma(\omega - kv), \quad k' = \gamma \left(k - \frac{\omega v}{c^{2}} \right) \]

This discrepancy indicates that a more detailed step involving the explicit substitution of the given expressions for \(\omega'\) and \(k'\), along with proper handling of the Lorentz transformation equations, is required to directly show the transformation of the wave's phase to the desired form.

Given the complexity of the algebra and the necessity to correctly apply the Lorentz transformations, the direct derivation here requires careful manipulation of the terms and might involve additional steps to accurately represent the transformation of the electromagnetic wave's phase from \( S \) to \( S' \).

For educational purposes, let's summarize the expected approach for this part:
1. Apply Lorentz transformation to the coordinates and time in the wave equation.
2. Substitute the derived expressions for \(t'\) and \(x'\) into the wave equation.
3. Use the given relations for \(\omega'\) and \(k'\) to express the phase in terms of primed quantities.

This process, while complex, illustrates how relativistic effects alter the perception of electromagnetic waves between different inertial frames, leading to changes in observed frequencies and wavelengths.

The limit of \(\omega'\) as \(v/c \rightarrow 0\) was intended to illustrate the behavior of wave frequencies in the limit where relativistic effects become negligible. This transition to classical physics principles, where \(v\) is much smaller than \(c\), implies that the observed frequency \(\omega'\) approaches the original frequency \(\omega\), reflecting the lack of significant relativistic effects on the wave's propagation between the two frames.

This scenario typically represents the classical Doppler effect, where the relative motion's impact on wave frequency is much less pronounced due to the low velocity compared to the speed of light. As such, in the limit \(v/c \rightarrow 0\), the relativistic Doppler shift simplifies to its classical counterpart, highlighting the continuous nature of physical laws from relativistic to classical regimes. \\

\noindent Solution 2: \\

Let's address each part of the question step by step.

Part a) Invariance of the Scalar Product \( (\underline{E} \cdot \underline{B}) \)

We have the transformations for the electric and magnetic fields between the inertial frames \( S \) and \( S' \) as follows:

\[
\begin{aligned}
E'_{x} &= E_{x}, & E'_{y} &= \gamma(E_{y} - vB_{z}), & E'_{z} &= \gamma(E_{z} + vB_{y}) \\
B'_{x} &= B_{x}, & B'_{y} &= \gamma(B_{y} + \frac{v}{c^{2}} E_{z}), & B'_{z} &= \gamma(B_{z} - \frac{v}{c^{2}} E_{y})
\end{aligned}
\]

We need to show that the scalar product \( (\underline{E} \cdot \underline{B}) \) is invariant under transformation, meaning that:

\[
\underline{E} \cdot \underline{B} = \underline{E'} \cdot \underline{B'}
\]

Let's calculate the dot product \( \underline{E'} \cdot \underline{B'} \) using the transformed components:

\[
\begin{aligned}
\underline{E'} \cdot \underline{B'} &= E'_{x}B'_{x} + E'_{y}B'_{y} + E'_{z}B'_{z} \\
&= E_{x}B_{x} + \gamma(E_{y} - vB_{z})\gamma(B_{y} + \frac{v}{c^{2}}E_{z}) + \gamma(E_{z} + vB_{y})\gamma(B_{z} - \frac{v}{c^{2}}E_{y})
\end{aligned}
\]

Simplifying this expression involves substituting the values of \( E' \) and \( B' \) as given by the transformations and then simplifying the terms. Let's proceed with these steps:

\[
\begin{aligned}
\underline{E'} \cdot \underline{B'} &= E_{x}B_{x} + \gamma^2(E_{y}B_{y} - vB_{z}B_{y} + \frac{v}{c^{2}}E_{z}E_{y} + \frac{v^2}{c^{2}}B_{z}E_{z}) \\
&\quad + \gamma^2(E_{z}B_{z} + vB_{y}B_{z} - \frac{v}{c^{2}}E_{y}E_{z} - \frac{v^2}{c^{2}}B_{y}E_{y})
\end{aligned}
\]

To show the invariance, we need to carefully combine and cancel terms. Given \( \gamma = 1/\sqrt{1 - v^2/c^2} \), let's also keep in mind that \( \gamma^2 = 1/(1 - v^2/c^2) \).

Let's simplify this expression in Python to ensure we accurately determine the invariance.

The simplified expression for the difference between the transformed dot product and the initial dot product, after applying the Lorentz transformation for electric and magnetic fields, results in a complex expression. However, the goal is to demonstrate that this difference simplifies to zero, indicating the invariance of \(\underline{E} \cdot \underline{B}\).

The complex expression after simplification involves terms with \(v^2\) and \(c^2\), and adjustments by \(\gamma^2\), which compensates for the Lorentz contraction. This expression's complexity highlights the need for careful algebraic manipulation to demonstrate invariance.

However, given the goal is to show that the scalar product \(\underline{E} \cdot \underline{B}\) remains unchanged under the Lorentz transformation, the critical insight is that each term in the dot product transformation maintains the relationship between electric and magnetic field components, adjusted by factors of \(\gamma\) and \(v/c\) terms that account for the relative motion between frames \(S\) and \(S'\).

The terms involving products of electric and magnetic field components, adjusted by \(\gamma^2\) and velocity \(v\), are designed to cancel out or adjust in such a way that the overall scalar product \(\underline{E} \cdot \underline{B}\) remains unchanged. This relies on the properties of \(\gamma\) and the symmetry of the transformations.

For a rigorous proof, each term in the transformation expression needs to be carefully evaluated, taking into account the properties of the Lorentz transformation. The Lorentz transformation ensures that physical laws, including those involving electromagnetic fields, are consistent in all inertial frames, leading to the invariance of the scalar product \(\underline{E} \cdot \underline{B}\).

Part b) Phase Transformation for an Electromagnetic Wave

Given the wave in frame \( S \) is represented by:
\[ \underline{E} = E_{0} \hat{\underline{y}} \exp(i(kx - \omega t)) \]

We need to show that the phase in frame \( S' \) transforms as:
\[ (k'x' - \omega' t') \]
where:
\[ \omega' = \gamma(\omega - kv), \quad k' = \gamma(k - \frac{\omega v}{c^{2}}) \]

The Lorentz transformation for coordinates is given by:
\[ x' = \gamma(x - vt), \quad t' = \gamma(t - \frac{vx}{c^{2}}) \]

Substituting \( x' \) and \( t' \) into the phase \( (kx - \omega t) \) gives us a way to express the phase in terms of \( x' \) and \( t' \). Let's substitute and simplify to derive \( k' \) and \( \omega' \).

After substituting the Lorentz transformation into the phase \( (kx - \omega t) \) and simplifying, the transformed phase becomes:

\[
- ((c^2-v^2)/c^2)^{-0.5}(c^2k(tv-x)+\omega (c^2t -vx))/c^2
\]

This expression should represent the phase in the \(S'\) frame as \( (k'x' - \omega' t') \). However, the direct comparison requires rearranging this result into a form that clearly shows the relationship between \(\omega'\), \(k'\),\(x'\) and \(t'\).

The expected form for \(\omega'\) and \(k'\) was given as:

\[
\omega' = \gamma(\omega-kv), \quad k' = \gamma(k - \frac{\omega v}{c^2})
\]

The simplification provided does not directly translate into this form due to the complexity of the expression. The correct approach involves substituting the Lorentz transformation into the electromagnetic wave equation and then isolating terms of \(x'\) and \(t'\) to directly derive expressions for \(\omega'\) and \(k'\). The goal is to demonstrate that the phase \(kx-\omega t\) in frame \(S\) transforms into \(k'x'-\omega' t'\) in frame \(S'\) with \(\omega'\) and \(k'\) as given.

To properly align with the question's expectation, we recognize that the Lorentz transformation affects the spatial and temporal coordinates, thus affecting \(k\) and \(\omega\) in a way that maintains the wave's phase relationship in the moving frame. The detailed algebraic manipulation should precisely follow the Lorentz transformation's impact on these wave parameters, leading to the given expressions for \(\omega'\) and \(k'\)

Due to the complexity and the algebraic simplification required, the direct step-by-step derivation here might not fully illustrate the transition to \(\omega'\) and \(k'\) as expected. The principle, however, is that the Lorentz transformation modifies both the spatial and temporal components of the wave's phase, ensuring that the wave's propagation characteristics (speed, frequency, and wavelength) are consistent with the principles of special relativity.

Part c) Relationship between \( \omega \) and \( \omega' \) as \( v/c \) approaches \( 0 \)

For the limit \( v/c \rightarrow 0 \), the transformation for \( \omega' \) simplifies to:
\[ \omega' = \gamma(\omega - kv) \]

As \( v/c \) approaches \( 0 \), \( \gamma \) approaches \( 1 \) (since \( \gamma = 1/\sqrt{1 - v^2/c^2} \)), making the relativistic effects negligible. Thus, the equation simplifies to:
\[ \omega' \approx \omega \]

This indicates that in the limit of low velocities compared to the speed of light, the observed frequency \( \omega' \) in the moving frame \( S' \) approaches the frequency \( \omega \) in the stationary frame \( S \). This is consistent with classical physics, where the motion of the observer at speeds much less than the speed of light has negligible effect on the observed properties of waves, including their frequency.

This limit case represents the classical Doppler effect for light, where the observed frequency of a wave approaches the source frequency as the relative velocity between the source and observer becomes negligible compared to the speed of light. The physical phenomenon represented here is the relativistic Doppler effect, which reduces to the classical Doppler effect in the limit of low velocities. This demonstrates the continuity between classical physics and relativistic physics, where relativistic formulas reduce to their classical counterparts under conditions where relativistic effects are negligible. \\

\noindent Solution 3: \\

Let's tackle this problem step by step.

\textbf{Part a: Invariance of the Scalar Product \( (\underline{E} \cdot \underline{B}) \)}

The scalar product \( (\underline{E} \cdot \underline{B}) \) in frame \( S \) is given by:
\[
\underline{E} \cdot \underline{B} = E_{x}B_{x} + E_{y}B_{y} + E_{z}B_{z}
\]

For the transformed frames, we need to use the given transformations to find \( \underline{E}' \cdot \underline{B}' \) in frame \( S' \). The transformed scalar product is:
\[
\underline{E}' \cdot \underline{B}' = E'_{x}B'_{x} + E'_{y}B'_{y} + E'_{z}B'_{z}
\]

Substituting the transformation equations, we get:
\[
\underline{E}' \cdot \underline{B}' = E_{x}B_{x} + \gamma(E_{y} - vB_{z})\gamma(B_{y} + \frac{v}{c^{2}}E_{z}) + \gamma(E_{z} + vB_{y})\gamma(B_{z} - \frac{v}{c^{2}}E_{y})
\]

Expanding this, we aim to show it reduces to \( E_{x}B_{x} + E_{y}B_{y} + E_{z}B_{z} \), thereby proving the invariance. Let's compute the expanded form and simplify it.

\textbf{Part b: Phase of the Electromagnetic Wave in Frame \( S' \)}

Given:
\[
\underline{E} = E_{0} \hat{\underline{y}} \exp(i(kx - \omega t))
\]

And the transformation for the phase to frame \( S' \) needs to be shown as:
\[
k'x' - \omega' t' = \gamma(kx - \omega t - kvx + \frac{\omega v}{c^{2}} t)
\]

This requires using the Lorentz transformation for space and time coordinates:
\[
x' = \gamma(x - vt), \quad t' = \gamma\left(t - \frac{vx}{c^{2}}\right)
\]

We'll substitute these into the expression for the wave phase in \( S' \) and show that it aligns with the given transformation for \( \omega' \) and \( k' \).

\textbf{Part c: Relationship Between \( \omega \) and \( \omega' \) as \( v/c \rightarrow 0 \)}

We have:
\[
\omega' = \gamma(\omega - kv), \quad k' = \gamma(k - \frac{\omega v}{c^{2}})
\]

We need to examine the limit \( v/c \rightarrow 0 \) to understand the physical phenomenon represented by the relationship between \( \omega \) and \( \omega' \).

Let's start with part a, calculating the scalar product transformation to show its invariance. We'll proceed with these calculations systematically.

For part a, after correcting the calculation, the simplified expression for the scalar product \( \underline{E}' \cdot \underline{B}' \) in the primed frame is:
\[
Bx \cdot Ex + By \cdot Ey + Bz \cdot Ez
\]

This shows that the scalar product \( (\underline{E} \cdot \underline{B}) \) is indeed invariant under the given transformation, as it remains unchanged from frame \( S \) to frame \( S' \).

Now, let's move on to part b, where we need to show the phase of the electromagnetic wave in frame \( S' \) is \( k'x' - \omega' t' \), using the given transformations and the Lorentz transformation for space and time coordinates.

The transformation of the phase in frame \( S' \) yields a complex expression:
\[
- \sqrt{\frac{c^2 - v^2}{c^2}} \cdot \frac{c^2 k' (t v - x) + \omega' (c^2 t - v x)}{c^2 - v^2}
\]

This result does not straightforwardly indicate how it translates into the expected forms for \( \omega' \) and \( k' \) as given in the problem statement. It appears there's a need for a more detailed approach to directly connect these transformations to the specified forms of \( \omega' \) and \( k' \). The correct approach involves using the Lorentz transformation to directly relate \( \omega \) and \( k \) to \( \omega' \) and \( k' \), alongside the phase relationship in electromagnetic waves, considering that the wave travels in the \( x \) direction and how this affects the observed frequency and wavenumber due to the relative motion of the frames.

To align with the problem's requirements, let's reconsider the method to show the phase transformation correctly. The goal is to explicitly demonstrate that the phase in \( S' \) matches the form \( k'x' - \omega' t' \) using the given relations for \( \omega' \) and \( k' \). Given the complexity of direct symbolic manipulation to arrive at the precise forms of \( \omega' \) and \( k' \), we acknowledge the need for a conceptual explanation.

\textbf{Conceptual Explanation for Part b}

For an electromagnetic wave propagating in the \( x \) direction, the Lorentz transformation affects both the time and spatial components of the wave's phase. The transformation:

- Affects the observed frequency due to time dilation and the Doppler effect, leading to the transformed angular frequency \( \omega' = \gamma(\omega - kv) \).
- Alters the wavenumber due to length contraction and the relative motion of the source and observer, leading to \( k' = \gamma(k - \frac{\omega v}{c^2}) \).

These expressions for \( \omega' \) and \( k' \) reflect how the wave's phase \( (kx - \omega t) \) appears in the moving frame \( S' \), accounting for the effects of special relativity. The phase transformation maintains the form \( k'x' - \omega' t' \), showing how the wave's propagation and frequency are perceived differently in a frame moving relative to the wave source.

\textbf{Part c: Wave Frequencies Relationship as \( v/c \rightarrow 0 \)}

In the limit \( v/c \rightarrow 0 \), the expressions for \( \omega' \) and \( k' \) become:
\[
\omega' = \frac{-kv + \omega}{\sqrt{1 - \frac{v^2}{c^2}}}, \quad k' = \frac{k - \frac{\omega v}{c^2}}{\sqrt{1 - \frac{v^2}{c^2}}}
\]

These expressions simplify under the assumption that \( v/c \) is small, indicating that the terms involving \( v/c \) become negligible. In practical terms, for very small velocities compared to the speed of light, the Lorentz factor \( \gamma \) approaches 1, simplifying the expressions to:
- \( \omega' \approx \omega - kv \)
- \( k' \approx k - \frac{\omega v}{c^2} \)

This simplification implies that for small relative velocities, the relativistic effects are minimal, and the transformations for frequency and wavenumber closely resemble classical predictions. However, the exact simplification to classical predictions would involve further reducing these expressions by considering the limits directly.

\textbf{Physical Phenomenon Represented}

The relationship between \( \omega \) and \( \omega' \) as \( v/c \) approaches 0 highlights the classical Doppler effect for light waves. In the classical limit, the observed frequency shift (\( \omega' \)) for a moving observer or source is a result of the relative motion, but without the significant relativistic effects present at higher velocities. This represents a transition to Newtonian physics from relativistic physics for very small \( v/c \), where relativistic corrections are unnecessary, and classical physics provides an accurate description.

Thus, the physical phenomenon represented is the Doppler shift, which, in the limit of \( v/c \rightarrow 0 \), aligns with our classical understanding of how the frequency of waves changes due to the relative motion between the source and the observer. This effect is critical in various fields, including astronomy, where it helps in determining the motion of stars and galaxies relative to Earth.


\end{document}
