\documentclass[a4paper,11pt]{article}
\setlength\parindent{0pt}
\usepackage{amsmath, amssymb}
\usepackage[margin=1in,letterpaper]
{geometry} % decreases margins

\usepackage{setspace}
\setstretch{1.2}
\usepackage{spverbatim}

\begin{document}
\title{GPT-4 Self-Graded Responses (With Markscheme)}
\author{Group 3A - Ryan Mok, Faraaz Akhtar, Banita Bamrah, \\ Fahim M A, Richang Xie, Kashan Ali }
\date{}
\maketitle

This document compiles the GPT-4 responses when prompted to grade the solutions which have been generated. The markscheme is given to GPT-4 in the prompt. This is the grading for the Classical Electromagnetic Theory (EM) questions. 

\section{EM}

\subsection*{Question 1:}

\textbf{Response 1:}

"Solution 1 Evaluation

Naming 3 types of magnetic materials: 1 mark (Correctly named all three types)
Mentioning differing magnetic susceptibility/magnetic dipole moment values determining response: 0 marks (The response is attributed to atomic structure and electron arrangement/orientation rather than explicitly mentioning magnetic susceptibility or dipole moments)

Explanation for paramagnetism: 1 mark (Mentions weaker attraction and alignment of magnetic moments but lacks specific mention of positive magnetic susceptibility and its value relative to 1)

Explanation for diamagnetism: 1 mark (Mentions repulsion and induced magnetic moment opposing the field but lacks specific mention of negative magnetic susceptibility)

Explanation for ferromagnetism: 2 marks (Correctly explains strong attraction due to alignment of magnetic domains and implies large magnetic susceptibility through the description of domain alignment and permanent magnetization)
Total: 5 marks

Feedback: The explanation correctly identifies the three types of magnetic materials and provides a good overview of their behaviors and examples. However, it lacks specificity in discussing the role of magnetic susceptibility and magnetic dipole moments in determining their response to a magnetic field. For improved clarity and completeness, explicitly mentioning these properties and their values would align more closely with the mark scheme's criteria.

Solution 2 Evaluation:

Naming 3 types of magnetic materials: 1 mark (Correctly named all three types)
Mentioning differing magnetic susceptibility/magnetic dipole moment values determining response: 0 marks (Focuses on the behavior of electrons and magnetic domains without specifically mentioning magnetic susceptibility or dipole moments)

Explanation for paramagnetism: 1 mark (Identifies weak attraction and alignment of atomic magnetic moments but lacks mention of positive magnetic susceptibility and its quantitative value)

Explanation for diamagnetism: 1 mark (Mentions induced magnetic moment opposing the field and very weak effect but does not explicitly state negative magnetic susceptibility)

Explanation for ferromagnetism: 2 marks (Accurately describes strong magnetization and retention of magnetism due to domain alignment, implying significant magnetic susceptibility)
Total: 5 marks

Feedback: Your response effectively outlines the characteristics and examples of the three types of magnetic materials, emphasizing the role of magnetic domains and electron behavior. However, to fully meet the criteria, it's essential to explicitly address the concept of magnetic susceptibility and how it quantitatively influences the material's magnetic response.

Solution 3 Evaluation

Naming 3 types of magnetic materials: 1 mark (Correctly named all three types)
Mentioning differing magnetic susceptibility/magnetic dipole moment values determining response: 0 marks (Discusses atomic dipoles and electron spins without directly referencing magnetic susceptibility or magnetic dipole moments)

Explanation for paramagnetism: 1 mark (Mentions weaker response and alignment with the external field but does not address positive magnetic susceptibility and its magnitude)

Explanation for diamagnetism: 1 mark (Identifies repulsion due to an induced magnetic field opposing the external field but does not mention negative magnetic susceptibility)

Explanation for ferromagnetism: 2 marks (Describes strong magnetic properties and retention of magnetization through unpaired electrons and domain alignment, suggesting large magnetic susceptibility)
Total: 5 marks

Feedback: Your explanation provides a clear distinction between the types of magnetic materials and their responses to a magnetic field, focusing on electronic structure and domain alignment. To enhance the answer according to the mark scheme, it would be beneficial to explicitly mention magnetic susceptibility and how its values differ among these materials, thereby affecting their magnetic response."

\textbf{Response 2:}

"Solution 1:

Grade: 6/8

Feedback: This solution successfully identifies the three types of magnetic materials and provides detailed explanations for their responses when brought near a bar magnet. It correctly names ferromagnetic, paramagnetic, and diamagnetic materials, earning 1 mark. The explanations for the behavior of these materials are mostly aligned with the criteria, particularly highlighting the role of atomic structure and electron arrangements. However, it lacks explicit mention of magnetic susceptibility or magnetic dipole moment values as the determining factors for their responses, missing out on 1 mark. The descriptions for each type of material align with the expected explanations, but the clarity on magnetic susceptibility's role and its values for each type could be improved to fully meet the mark scheme requirements.

Solution 2:

Grade: 6/8

Feedback: This solution accurately identifies the three types of magnetic materials and offers a comprehensive explanation of their magnetic responses. It earns 1 mark for correctly naming the materials. The response effectively explains the behavior of these materials in the presence of a magnetic field, with a focus on atomic and electronic structures. However, like Solution 1, it misses explicit discussion on magnetic susceptibility or magnetic dipole moment values as key determinants of their magnetic responses, resulting in a missed mark. The explanations provided for each material type are well-aligned with the mark scheme but could be enhanced by directly addressing the specifics of magnetic susceptibility and dipole moments to fully satisfy the grading criteria.

Solution 3:

Grade: 6/8

Feedback: Solution 3 does a commendable job of listing the three magnetic materials and explaining the underlying reasons for their magnetic behaviors. It earns 1 mark for the identification of ferromagnetic, paramagnetic, and diamagnetic materials. The explanations delve into the atomic level, discussing the role of unpaired electrons and their alignments in response to an external magnetic field. However, this solution, like the others, does not explicitly mention magnetic susceptibility or magnetic dipole moments as the primary factors determining their responses, which is a missed opportunity to earn an additional mark. While the descriptions for each material's reaction to a magnetic field are in line with expectations, mentioning magnetic susceptibility explicitly and its respective values for each material would align the response more closely with the mark scheme."

\textbf{Response 3:}

"Grading and Feedback

Solution 1:

Naming 3 types of magnetic materials: 1 mark.

Mention of differing magnetic susceptibility/magnetic dipole moment values: 0 marks. (The solution talks about atomic structure and electron arrangement but doesn't explicitly mention magnetic susceptibility or magnetic dipole moment values.)

Explanation for paramagnetism: 1 mark. (It mentions weak attraction but lacks specific mention of magnetic susceptibility being positive and much less than 1.)

Explanation for diamagnetism: 1 mark. (It correctly identifies that diamagnetic materials are repelled and mentions induced magnetic moment but does not explicitly state magnetic susceptibility is negative and magnitude much less than 1.)

Explanation for ferromagnetism: 2 marks. (It provides a detailed explanation of strong attraction due to alignment of magnetic domains.)

Total: 5/8

General Feedback:
Your explanation covers the basic distinctions between ferromagnetic, paramagnetic, and diamagnetic materials well, focusing on their physical properties and responses to a magnetic field. However, the response could be improved by explicitly mentioning magnetic susceptibility or magnetic dipole moment values to directly answer the question about what determines their response when brought near a bar magnet. Including these specifics would provide a clearer, more complete explanation of the materials' behaviors.

Solution 2:

Naming 3 types of magnetic materials: 1 mark.

Mention of differing magnetic susceptibility/magnetic dipole moment values: 0 marks. (Similar to Solution 1, it focuses on atomic structure and behavior without mentioning magnetic susceptibility or dipole moments.)

Explanation for paramagnetism: 1 mark. (Mentions weak attraction but does not clearly state magnetic susceptibility is positive and much less than 1.)

Explanation for diamagnetism: 1 mark. (Correctly identifies repulsion and the role of induced magnetic moments but lacks explicit reference to negative magnetic susceptibility.)

Explanation for ferromagnetism: 2 marks. (Provides a good explanation of strong magnetization and retention of magnetism.)

Total: 5/8

General Feedback:
Your descriptions of the types of magnetic materials and their responses to magnetic fields are accurate and well-explained, particularly with regard to ferromagnetism. However, to fully address the question, it would be beneficial to explicitly mention the concepts of magnetic susceptibility and magnetic dipole moments. This would directly tie the materials' behaviors to the properties that determine their response to a magnetic field, providing a more thorough understanding.

Solution 3:

Naming 3 types of magnetic materials: 1 mark.

Mention of differing magnetic susceptibility/magnetic dipole moment values: 0 marks. (Though it discusses the electronic structure and magnetic properties, it doesn't explicitly mention magnetic susceptibility or dipole moments.)

Explanation for paramagnetism: 1 mark. (Identifies weaker attraction but lacks the specific detail about magnetic susceptibility being positive and much less than 1.)

Explanation for diamagnetism: 1 mark. (Correctly mentions repulsion due to an induced magnetic field but does not specify negative magnetic susceptibility.)

Explanation for ferromagnetism: 2 marks. (Effectively explains strong magnetic properties and retention of magnetism.)
Total: 5/8

General Feedback:
You have provided a comprehensive overview of the different types of magnetic materials and their reactions to external magnetic fields. While the descriptions are generally accurate, incorporating specific details about magnetic susceptibility and magnetic dipole moments would significantly enhance the response. These details are crucial for a complete understanding of what determines the materials' responses to a magnetic field, as per the question's requirements."

\subsection*{Question 2:}

\textbf{Response 1:}

"Solution 1 Grading

\begin{itemize}
    \item \textbf{Free Space Definition:} The definition is accurate and includes necessary characteristics such as permittivity and permeability. \textbf{[1 mark]}
    \item \textbf{Wave Equation Explanation:} Provides a clear explanation of the wave equation, including spatial and temporal changes. \textbf{[1 mark]}
    \item \textbf{Speed of Light Calculation:} Accurately explains how the speed of light \(c\) is derived from the equation, including a comparison to the general form of a wave equation and calculating \(c\) using the values of \(\epsilon_{0}\) and \(\mu_{0}\). \textbf{[3 marks]}
\end{itemize}

\textbf{Total: 5 marks}

\textbf{Feedback:} The solution effectively describes free space, explains the wave equation's physical interpretation, and accurately derives the speed of light. It provides a thorough explanation, correctly applies the relevant equations, and calculates the speed of light, adhering closely to the mark scheme.

Solution 2 Grading

\begin{itemize}
    \item \textbf{Free Space Definition:} Provides an accurate definition of free space as a perfect vacuum, mentioning permittivity and permeability. \textbf{[1 mark]}
    \item \textbf{Wave Equation Explanation:} Explains the wave equation and its significance for the propagation of electromagnetic waves in free space. \textbf{[1 mark]}
    \item \textbf{Speed of Light Calculation:} Correctly identifies the method to derive the speed of light from the given equation and confirms the value of \(c\) is approximately \(3.00 \times 10^8\) m/s. \textbf{[3 marks]}
\end{itemize}

\textbf{Total: 5 marks}

\textbf{Feedback:} The response accurately describes free space, effectively interprets the wave equation, and correctly derives the speed of light. It maintains a clear and logical explanation throughout, aligning well with the mark scheme.

Solution 3 Grading

\begin{itemize}
    \item \textbf{Free Space Definition:} Correctly defines free space and mentions its characteristics, including permittivity and permeability. \textbf{[1 mark]}
    \item \textbf{Wave Equation Explanation:} Provides a solid interpretation of the wave equation and its implications for electromagnetic wave propagation. \textbf{[1 mark]}
    \item \textbf{Speed of Light Calculation:} Successfully explains how the speed of light is determined from the equation and confirms the speed of light as a fundamental constant. \textbf{[3 marks]}
\end{itemize}

\textbf{Total: 5 marks}

\textbf{Feedback:} The solution offers a comprehensive explanation of free space, a detailed interpretation of the wave equation, and accurately calculates the speed of light. It aligns with the mark scheme by providing a correct and sufficient argument throughout."

\textbf{Response 2:}

"Solution 1 Feedback
Grade: 5/5

This solution excellently addresses all parts of the question. It provides a clear and accurate definition of "free space" in the context of classical electromagnetism, correctly identifies the given equation as a wave equation for electric fields, and thoroughly explains how this equation demonstrates the propagation of electromagnetic waves in space. Furthermore, the solution correctly derives the speed of light from the equation, matching the value given in the mark scheme. The explanation is detailed, including the process of comparing the given wave equation to the general form of a wave equation to extract the speed of light, and concludes with the correct value for the speed of light, \(c = 3 \times 10^8 ms^{-1}\), using appropriate units and precision.

Solution 2 Feedback
Grade: 5/5

This submission is also of high quality, mirroring the first solution in its thoroughness and accuracy. It correctly defines "free space," accurately interprets the wave equation for electric fields, and properly explains the mechanism of electromagnetic wave propagation in free space. Additionally, it successfully derives the speed of light from the equation, presenting the calculation clearly and arriving at the correct value with appropriate units and precision. The explanation of comparing the coefficients to find the speed of light is correctly applied, fulfilling the requirements of the mark scheme.

Solution 3 Feedback
Grade: 5/5

Similar to the first two solutions, this one effectively covers all the requirements of the question. It provides a precise definition of "free space," interprets the given equation as a wave equation for the electric field, and explains how this supports the propagation of electromagnetic waves in free space. The solution correctly derives the speed of light from the equation, showing a good understanding of how to compare it to the general form of a wave equation to deduce the speed of light, \(c\), with the final value correctly calculated and presented. The use of appropriate units and precision is noted, aligning with the mark scheme.

Solution 3 Feedback

Grade: 5/5

Similar to the first two solutions, this one effectively covers all the requirements of the question. It provides a precise definition of "free space," interprets the given equation as a wave equation for the electric field, and explains how this supports the propagation of electromagnetic waves in free space. The solution correctly derives the speed of light from the equation, showing a good understanding of how to compare it to the general form of a wave equation to deduce the speed of light, \(c\), with the final value correctly calculated and presented. The use of appropriate units and precision is noted, aligning with the mark scheme.

General Feedback
All three solutions exhibit a strong understanding of classical electromagnetism, particularly in explaining the propagation of electric fields as waves in free space and deriving the speed of light from the given wave equation. Each solution effectively communicated complex ideas clearly and accurately, using appropriate terminology and mathematical expressions. The derivations for the speed of light are correctly performed, and the importance of \(\epsilon_0\) and \(\mu_0\) in determining this speed is well articulated. The precision in the final values for the speed of light and the use of appropriate units are consistent with the mark scheme's requirements."

\textbf{Response 3:}

"Solution 1 Grading:

Free Space Definition: 1 mark. The student accurately defines "free space" with appropriate emphasis on its physical properties.

Wave Equation Interpretation: 1 mark. The explanation clearly describes the wave equation for the electric field and its implications for wave propagation in space.

Speed of Light Derivation and Value: 3 marks. The student successfully derives the speed of light from the given equation and correctly states its value as \( 3 \times 10^8 ms^{-1}\).

Total: 5 marks

General Feedback: The response thoroughly covers all aspects of the question, providing a clear explanation of "free space," the physical interpretation of the wave equation, and how it leads to the speed of light. The derivation is correctly related to the fundamental constants of nature, and the value of the speed of light is accurately stated.

Solution 2 Grading:

Free Space Definition: 1 mark. The definition of "free space" is correct and concisely matches the criteria.

Wave Equation Interpretation: 1 mark. The explanation effectively addresses the wave equation's significance for electric field propagation.

Speed of Light Derivation and Value: 3 marks. The student accurately derives the speed of light and provides its value, fulfilling all criteria in the mark scheme.

Total: 5 marks

General Feedback: This solution is precise and clear, correctly defining "free space," explaining the wave equation's relevance to electromagnetic wave propagation, and deriving the speed of light in a manner that is consistent with the given requirements. The explanation is concise and aligns well with the expected outcomes.

Solution 3 Grading:

Free Space Definition: 1 mark. The student provides an accurate definition of "free space" as required.

Wave Equation Interpretation: 1 mark. The explanation of the wave equation and its implications for the propagation of electric fields is correct and informative.

Speed of Light Derivation and Value: 3 marks. The derivation of the speed of light from the equation and the stated value are both correct and complete as per the mark scheme.

Total: 5 marks

General Feedback: The student's response comprehensively addresses the question, providing an accurate definition of "free space," a clear interpretation of the wave equation, and a correct derivation of the speed of light. The explanations are detailed, correctly relating the physical concepts to the mathematical representation, and culminate in the accurate calculation of the speed of light."

\subsection*{Question 3:}

\textbf{Response 1:}

"\textbf{Grading Solution 1:}

Use of Coulomb’s law to integrate and find electric field: [1/1 mark] - The solution correctly uses Coulomb’s law as a basis for integration over the line charge.

Correct integral setup to evaluate \( E_x \): [0/3 marks] - The solution does not provide the integral setup for the x-component of the electric field.

Correct integral setup to evaluate \( E_z \): [3/3 marks] - The integral setup for the z-component of the electric field is provided and is correct.

Correct evaluation of integral to get \( E_x \): [0/2 marks] - The solution does not show the evaluation of the integral for the x-component.

Correct evaluation of integral to get \( E_z \): [0/2 marks] - While the correct form of \( E_z \) is not explicitly provided, the final expression given seems to imply correct evaluation, but full workings are not shown.

Total expression of \underline{E}: [0/1 mark] - The solution does not provide the total expression for \underline{E} explicitly.

Correct simplification in \underline{E} for \( d \gg L \): [0/2 marks] - The solution attempts to simplify for \( d \gg L \), but the provided expression does not match the expected result from the mark scheme, and the numerical factor included is incorrect for the simplified case.

Total Marks: 4/14

Feedback:
The student has shown an understanding of using Coulomb's law for integration over a charge distribution, which is commendable. However, the solution is incomplete as it lacks the integral setup and evaluation for the x-component of the electric field. The final expression provided does not match the mark scheme, and there is an unnecessary numerical factor included. The student should focus on setting up the problem correctly and ensuring that all components of the electric field are accounted for in the final expression.

\textbf{Grading Solution 2:}

Use of Coulomb’s law to integrate and find electric field: [1/1 mark] - The solution correctly applies Coulomb's law to set up the integration.

Correct integral setup to evaluate \( E_x \): [0/3 marks] - The solution does not explicitly set up the integral for the x-component.

Correct integral setup to evaluate \( E_z \): [3/3 marks] - The integral setup for the z-component is provided and is correct.

Correct evaluation of integral to get \( E_x \): [0/2 marks] - The solution does not show the evaluation of the integral for the x-component.

Correct evaluation of integral to get \( E_z \): [0/2 marks] - Although the final form for \( E_z \) is not shown, the final expression seems to imply correct evaluation, but the complete process is not demonstrated.

Total expression of \underline{E}: [0/1 mark] - The solution does not provide the total vector expression for the electric field.

Correct simplification in \underline{E} for \( d \gg L \): [0/2 marks] - The solution does attempt to simplify for \( d \gg L \), but the simplification includes a numerical factor that does not match the mark scheme, and the physical interpretation is slightly off.

Total Marks: 4/14

Feedback:
The student has a basic understanding of how to approach the problem using integration of Coulomb's law. However, the solution is missing the integral setup and evaluation for the x-component, and there is no total vector expression for the electric field. The simplification for the case when \( d \gg L \) is also incorrect due to the inclusion of an unnecessary numerical factor. The student should ensure to include all necessary components for the electric field and follow the mark scheme more closely for full credit.

\textbf{Grading Solution 3:}

Use of Coulomb’s law to integrate and find electric field: [1/1 mark] - The solution correctly uses Coulomb's law as the basis for setting up the integration.

Correct integral setup to evaluate \( E_x \): [0/3 marks] - The solution does not show the integral setup for the x-component of the electric field.

Correct integral setup to evaluate \( E_z \): [3/3 marks] - The integral setup for the z-component is correct.

Correct evaluation of integral to get \( E_x \): [0/2 marks] - The solution does not evaluate the integral for the x-component.

Correct evaluation of integral to get \( E_z \): [0/2 marks] - No complete workings are shown for the evaluation of \( E_z \), although the final expression suggests a correct evaluation.

Total expression of \underline{E}: [0/1 mark] - The total vector expression for the electric field is not given.

Correct simplification in \(\underline{E}\) for \(d \gg L\) : [0/2 marks] - The simplification is incorrect as it does not match the expected result from the mark scheme, and the inclusion of the numerical factor is not correct for the simplified case.

Total Marks: 4/14

Feedback:
The student demonstrates an understanding of the use of Coulomb's law in the context of an extended charge distribution. However, the solution is incomplete, lacking the integral setup and evaluation for the x-component, and it does not provide a total vector expression for the electric field. Furthermore, the simplification for \(d \gg L\)  is incorrect, with an inappropriate numerical factor included. The student should present a more comprehensive solution that includes all components of the electric field and follows the mark scheme more precisely, especially when simplifying for \(d \gg L\)."

\textbf{Response 2:}

"Grading each solution:

\textbf{Solution 1:}

\begin{itemize}
  \item Use of Coulomb's law to integrate and find the electric field: 1 mark.
  \item Correct integral setup to evaluate \( E_x \): 0 marks (They didn't calculate \( E_x \)).
  \item Correct integral setup to evaluate \( E_z \): 3 marks (They correctly set up the integral for \( E_z \)).
  \item Correct evaluation of the integral to get \( E_x \): 0 marks (They didn't evaluate \( E_x \)).
  \item Correct evaluation of the integral to get \( E_z \): 0 marks (They gave a numerical answer without showing the integration steps).
  \item Total expression of \( \vec{E} \): 0 marks (They didn't provide the correct vector form).
  \item Correct simplification in \( \vec{E} \) for \( d \gg L \): 1 mark (They mentioned that the line charge looks like a point charge, which is correct).
\end{itemize}

Total: 5 marks

Feedback: The solution correctly applied Coulomb's law and the integral for \( E_z \), but it did not evaluate \( E_x \) and \( E_z \) properly. The final expression for \( \vec{E} \) was not provided, and the numerical answer does not follow the mark scheme. The physical interpretation in the limit of \( d \gg L \) was correct.

\textbf{Solution 2:}

\begin{itemize}
  \item Use of Coulomb's law to integrate and find the electric field: 1 mark.
  \item Correct integral setup to evaluate \( E_x \): 0 marks (They didn't calculate \( E_x \)).
  \item Correct integral setup to evaluate \( E_z \): 3 marks (They correctly set up the integral for \( E_z \)).
  \item Correct evaluation of the integral to get \( E_x \): 0 marks (They didn't evaluate \( E_x \)).
  \item Correct evaluation of the integral to get \( E_z \): 0 marks (They gave a numerical answer without showing the integration steps).
  \item Total expression of \( \vec{E} \): 0 marks (They didn't provide the correct vector form).
  \item Correct simplification in \( \vec{E} \) for \( d \gg L \): 1 mark (They mentioned that the line charge looks like a point charge, which is correct).
\end{itemize}

Total: 5 marks

Feedback: The solution correctly applied Coulomb's law and the integral for \( E_z \), but it did not evaluate \( E_x \) and \( E_z \) properly. The final expression for \( \vec{E} \) was not provided, and the numerical answer does not follow the mark scheme. The physical interpretation in the limit of \( d \gg L \) was correct.

\textbf{Solution 3:}

\begin{itemize}
  \item Use of Coulomb's law to integrate and find the electric field: 1 mark.
  \item Correct integral setup to evaluate \( E_x \): 0 marks (They used \( dE_y \) instead of \( dE_x \)).
  \item Correct integral setup to evaluate \( E_z \): 3 marks (Although they used \( E_y \), their setup matches the requirement for \( E_z \)).
  \item Correct evaluation of the integral to get \( E_x \): 0 marks (They did not evaluate \( E_x \)).
  \item Correct evaluation of the integral to get \( E_z \): 0 marks (They provided a numerical approximation without proper integration).
  \item Total expression of \( \vec{E} \): 0 marks (They did not provide the correct vector form).
  \item Correct simplification in \( \vec{E} \) for \( d \gg L \): 1 mark (They mentioned the line charge appearing as a point charge).
\end{itemize}

Total: 5 marks

Feedback: The solution correctly applied Coulomb's law and the integral for \( E_z \), but it did not evaluate \( E_x \) and \( E_z \) properly. They provided a numerical approximation without proper integration steps. The final expression for \( \vec{E} \) was not provided in vector form. The physical interpretation in the limit of \( d \gg L \) was correct but lacked the correct factor.

Each solution has missed key parts of the evaluation of the integral and expression of \( \vec{E} \), which are crucial for full marks. The final expressions and their simplifications are also not fully correct or consistent with the mark scheme. The physical interpretations provided at the end are appropriate, recognizing that the line charge appears as a point charge at a distance much greater than the length of the line charge. However, the numerical values and lack of vector expressions have significantly impacted the marks awarded. It is crucial to follow the integral evaluation and expression of results as outlined in the mark scheme for full credit."

\textbf{Response 3:}

"Solution 1:

Marks Awarded:
\begin{itemize}
  \item Use of Coulomb's law to integrate and find electric field: 1 mark
  \item Correct integral setup to evaluate \( E_z \): 3 marks
  \item Correct evaluation of integral to get \( E_z \): 2 marks
  \item Total expression of \( E \): 1 mark
  \item Correct simplification in \( E \) for \( d \gg L \): 2 marks
\end{itemize}

The student has correctly used Coulomb's law and set up the integral for \( E_z \) as per the requirements, as well as the total expression for \( E \). However, the integral is only partially evaluated, and there's no explicit integral setup for \( E_x \), nor is there any evaluation of \( E_x \). The final simplification for \( d \gg L \) is correct, and they have included the physical interpretation of the field looking like a point charge. There might be an issue with the precision of the constants used in the final expression.

Total Marks: 9/14

Feedback:
The solution correctly applies Coulomb's law and the physical interpretation for \( d \gg L \). However, it lacks the correct integral setup for \( E_x \) and does not include the evaluation for \( E_x \). The constants used in the final expression should be reviewed for appropriate units and precision. 

Solution 2:

Marks Awarded:
\begin{itemize}
  \item Use of Coulomb's law to integrate and find electric field: 1 mark
  \item Correct integral setup to evaluate \( E_z \): 3 marks
  \item Correct evaluation of integral to get \( E_z \): 2 marks
  \item Total expression of \( E \): 1 mark
  \item Correct simplification in \( E \) for \( d \gg L \): 2 marks
\end{itemize}

Like Solution 1, this student also set up the integral correctly for \( E_z \), evaluated it, and simplified it for the case \( d \gg L \). They also provided the correct physical interpretation. However, they did not address the \( E_x \) component nor did they evaluate the \( E_x \) integral. The constants used in the final expression are unusual and not in standard form.

Total Marks: 9/14

Feedback:
The solution has a good approach to integrating the electric field and understanding the physical significance of the limit \( d \gg L \). It fails to include the \( E_x \) component and its evaluation. The constants in the final expression should be presented in the standard form used in physics for clarity and standardization.

Solution 3:

Marks Awarded:
\begin{itemize}
  \item Use of Coulomb's law to integrate and find electric field: 1 mark
  \item Correct integral setup to evaluate \( E_z \): 3 marks
  \item Evaluation of \( E_z \): 0 marks (since the evaluation provided is incorrect)
  \item Total expression of \( E \): 0 marks (since it is not provided in the correct form)
  \item Correct simplification in \( E \) for \( d \gg L \): 2 marks
\end{itemize}

This student correctly uses Coulomb's law and integrates for the electric field. The setup for the integral of \( E_z \) is correct, and the simplification for \( d \gg L \) is also correct. However, the student doesn't provide the correct evaluation of \( E_z \), nor do they provide the total expression of \( E \) in the correct form. There's no mention of the \( E_x \) component or its evaluation.

Total Marks: 6/14

Feedback:
The solution correctly sets up the integral for \( E_z \) and addresses the simplification for \( d \gg L \). However, the evaluation of the integral is incorrect, and the total electric field expression is not provided. There is no attempt to evaluate or even set up the \( E_x \) component, which is a significant omission. The precision of constants and their presentation should be consistent with standard physics notation."

\subsection*{Question 4:}

\textbf{Response 1:}

"Given the problem and the mark scheme provided, let's evaluate each of the three student solutions.

Solution 1 Evaluation:

a) Magnetic flux calculation:

Correct use of the definition of flux and the setup of the integral: 4 marks (They didn't explicitly mention the surface integral definition, but the setup and evaluation are correct)
Correct evaluation of the integral to get flux: 1 mark

b) EMF generated and current direction:

Correct use of emf definition: 1 mark
The evaluation of the derivative to get the emf is incorrect, as there is a mistake in the differentiation: 0 marks
Correct direction of current by Lenz's Law: 2 marks

c) Loop moving downwards:

Correct conclusion, no emf generated since flux does not change: 1 mark
Total:

Part a: 5 marks

Part b: 3 marks

Part c: 1 mark

Overall: 9 marks

General Feedback:
The student provided a correct integral for the magnetic flux and evaluated it properly. However, there is an error in the differentiation of the flux with respect to time in part b, leading to an incorrect emf expression. The direction of the induced current is correctly reasoned.

Solution 2 Evaluation:

a) Magnetic flux calculation:

The student incorrectly states that the magnetic flux is independent of the distance \(d\): 0 marks

b) EMF generated and current direction:

Incorrect claim that differentiation yields zero, leading to an incorrect conclusion that no emf is generated when the loop is moved to the right: 0 marks
Correct identification of the direction of current flow is not justified as no emf is claimed to be generated: 0 marks

c) Loop moving downwards:

Correct conclusion, no emf generated since flux does not change: 1 mark
Total:

Part a: 0 marks

Part b: 0 marks

Part c: 1 mark

Overall: 1 mark

General Feedback:
The student has a fundamental misunderstanding of the magnetic flux calculation, incorrectly stating that it is independent of the distance \(d\). This error propagates through to part b, leading to the incorrect conclusion that no emf is generated when the loop is moved to the right.

Solution 3 Evaluation:

a) Magnetic flux calculation:

Correct expression for magnetic field and setup of flux integral: 4 marks (The integral is not explicitly stated as a surface integral, but the setup is correct)

Correct evaluation of the integral to get flux: 1 mark

b) EMF generated and current direction:

No explicit use of the emf definition as a time derivative: 0 marks

Incorrect expression for emf, and the final expression does not match the required form: 0 marks

Correct reasoning for the direction of the current flow, but not justified by calculation as the emf is incorrect: 1 mark

c) Loop moving downwards:

Correct conclusion, no emf generated since flux does not change: 1 mark
Total:

Part a: 5 marks

Part b: 1 mark

Part c: 1 mark

Overall: 7 marks

General Feedback:
Solution 3 correctly calculates the magnetic flux through the loop but fails to properly define the emf as the time derivative of flux. The emf calculation is incorrect, and while the reasoning for the current's direction is correct, it is not justified by the calculation. The student correctly understands the concept of no emf being induced when the loop moves downwards."

\textbf{Response 2:}

"Given the mark scheme, let's evaluate each solution.

Solution 1:

a) This solution correctly uses the definition of magnetic flux and sets up the integral properly using the magnetic field for an infinitely long wire. The integral is correctly evaluated to give the flux.

Mark: 5/5

b) The student uses the correct definition of emf and correctly differentiates the flux with respect to time. However, there is a mistake in the differentiation; the student didn't properly apply the chain rule when differentiating the natural logarithm function. The final expression for the emf is incorrect because the student did not simplify the expression properly and left it in a form that is not equivalent to the correct expression. The explanation of the direction of the induced current using Lenz's law is correct.

Mark: 2/5

c) The student correctly states that no emf is generated when the loop moves downwards at speed \(v\) because the flux does not change.

Mark: 1/1

General Feedback for Solution 1: The calculation of flux is well done, but the differentiation in part b) needs to be corrected. Ensure that the chain rule is applied correctly when differentiating functions.

Solution 2:

a) The solution starts well with the correct expression for the magnetic field and the definition of flux. However, the solution makes a critical error in assuming the magnetic flux through the loop is independent of the distance \(d\), which is not true. The expression for the magnetic flux is incorrect as it doesn't include the logarithmic term that arises from the integration.

Mark: 2/5

b) The differentiation of the magnetic flux with respect to time is incorrect as the student has concluded that the differentiation yields zero, which is not true. The expression for emf is incorrect, and the student does not provide any justification for the direction of the induced current.

Mark: 0/5

c) The student correctly states that no emf is generated when the loop moves downwards at speed \(v\) because the flux does not change.

Mark: 1/1

General Feedback for Solution 2: The approach to finding the magnetic flux in part a) is incorrect, and the calculation of the emf in part b) is also incorrect. Ensure that the integration in part a) is done correctly and that the differentiation in part b) follows the proper rules.

Solution 3:

a) The student correctly writes the expression for the magnetic field and correctly sets up the integral for the magnetic flux. The final expression for the flux is given in a logarithmic form, which is correct.

Mark: 5/5

b) The student states Faraday's law correctly but then makes an error in the application by not using the derivative of \(d\) with respect to \(t\). The expression given for the emf is not correct. The student also did not properly justify the direction of the induced current.

Mark: 1/5

c) The student correctly states that no emf is generated when the loop moves downwards at speed \(v\) because the flux does not change.

Mark: 1/1

General Feedback for Solution 3: The calculation of flux in part a) is correct, but the application of Faraday's law in part b) is not correctly executed. Ensure that the differentiation is performed correctly and that the direction of the induced current is properly justified."

\textbf{Response 3:}

"Grading each student's solution according to the provided mark scheme:

Solution 1:

a) The student correctly uses the definition of flux and sets up the integral properly, using the magnetic field for an infinite long wire and integrates it correctly.

Definition of flux as surface integral: 1 mark

Correct setup of integral for flux: 3 marks

Correct evaluation of integral: 1 mark

Total for part a): 5 marks

b) The student uses the definition of emf correctly, but there is a mistake in the differentiation which leads to an incorrect expression for the emf.

Use of emf definition: 1 mark

Incorrect evaluation of the derivative to get the emf: 0 marks

Correct justification of the current's direction: 2 marks

Total for part b): 3 marks

c) The student correctly states that no emf is generated when the loop moves downwards.

Correct conclusion about emf when moving downwards: 1 mark

Total for part c): 1 mark

Total marks for Solution 1: 9 marks

Feedback for Solution 1:
The student has demonstrated a good understanding of magnetic flux and the integral setup for part a). However, in part b), the differentiation with respect to time was not carried out correctly, resulting in an incorrect expression for emf. The understanding of Lenz's Law and the induced current's direction is correct. Part c) was correctly answered.

Solution 2:

a) The student states the correct formula for the magnetic field due to a long wire but fails to actually perform the integral to find the flux.

No use of flux as surface integral shown: 0 marks

Incorrect setup of integral for flux (no actual integral performed): 0 marks

No evaluation of integral: 0 marks

Total for part a): 0 marks

b) The student incorrectly concludes that the emf generated is zero due to a misunderstanding of the flux's dependence on distance \(d\).

No correct use of emf definition as they did not find the correct change in flux: 0 marks

Incorrect evaluation of the derivative to get the emf: 0 marks

No justification for the direction of current, as the emf was incorrectly found to be zero: 0 marks

Total for part b): 0 marks

c) The student correctly states that no emf is generated when the loop moves downwards.

Correct conclusion about emf when moving downwards: 1 mark

Total for part c): 1 mark

Total marks for Solution 2: 1 mark

Feedback for Solution 2:

The student has not demonstrated a correct approach to calculating the magnetic flux and has made a critical error in assuming the flux does not depend on distance \(d\). This misunderstanding carried through to part b), leading to an incorrect conclusion that no emf is generated when the loop moves to the right. The student did correctly answer part c).

Solution 3:

a) The student correctly writes down the formula for the magnetic field but does not present it as an integral over the loop's area; instead, they jump to the conclusion.

No use of flux as surface integral shown: 0 marks

Incorrect setup of integral for flux (no actual integral shown): 0 marks

Correct final expression for flux, but process not shown: 1 mark (given for the correct final answer, although process is missing)

Total for part a): 1 mark

b) The student presents an incorrect final expression for the emf. They also made errors in their calculation, which led to the wrong conclusion.

Use of emf definition: 1 mark (for stating Faraday's law)

Incorrect evaluation of the derivative to get the emf: 0 marks

Incorrect justification for the direction of current: 0 marks

Total for part b): 1 mark

c) The student correctly states that no emf is generated when the loop moves downwards.

Correct conclusion about emf when moving downwards: 1 mark

Total for part c): 1 mark

Total marks for Solution 3: 3 marks

Feedback for Solution 3:
The student correctly identifies the final expression for magnetic flux but does not show the integral calculation as required by the mark scheme. The approach to calculating the emf when the loop moves to the right is flawed, resulting in an incorrect expression. However, the student correctly understands that no emf is induced when the loop moves downwards."

\subsection*{Question 5:}

\textbf{Response 1:}

"Let's grade each of the solutions based on the provided mark scheme.

Solution 1:

a) Calculate the current seen by the ammeter:

No mention of Kirchhoff’s Circuit Laws: 0 marks

Did not split current into three branches: 0 marks

No equations using Kirchhoff’s voltage law: 0 marks

Incorrect current through the ammeter: 0 marks

Subtotal for part a): 0 marks

b) Calculate the energy delivered by the 
\(12 V\) battery in \(4\) seconds:

Incorrect use of power and energy formulas (used assumed current of 18A which is incorrect): 0 marks

Incorrect energy calculation: 0 marks

Subtotal for part b): 0 marks

Total for Solution 1: 0 marks

Feedback: The student did not apply Kirchhoff's Laws to solve the circuit and incorrectly assumed a series circuit for current calculation. The energy delivered was also incorrectly calculated based on the erroneous current. It is important to analyze the circuit correctly and apply Kirchhoff's laws to find the current through each branch and use the correct values to find the power and energy.

Solution 2:

a) Calculate the current seen by the ammeter:

No mention of Kirchhoff’s Circuit Laws: 0 marks

Did not split current into three branches: 0 marks

No equations using Kirchhoff’s voltage law: 0 marks

Incorrect current through the ammeter: 0 marks

Subtotal for part a): 0 marks

b) Calculate the energy delivered by the 
\(12 V\) battery in \(4\) seconds:

Incorrect power calculation due to wrong current: 0 marks

Incorrect energy calculation: 0 marks

Subtotal for part b): 0 marks

Total for Solution 2: 0 marks

Feedback: The student did not use Kirchhoff's Laws to analyze the circuit. The total resistance and the current were incorrectly calculated, leading to incorrect power and energy calculations. The student needs to apply the correct principles for circuit analysis and ensure their understanding of parallel and series resistances.

Solution 3:

a) Calculate the current seen by the ammeter:

No mention of Kirchhoff’s Circuit Laws: 0 marks

Did not split current into three branches: 0 marks

No equations using Kirchhoff’s voltage law: 0 marks

Incorrect current through the ammeter: 0 marks

Subtotal for part a): 0 marks

b) Calculate the energy delivered by the 
\(12 V\) in \(4\) seconds:

Incorrect power calculation due to wrong current: 0 marks

Incorrect energy calculation: 0 marks

Subtotal for part b): 0 marks

Total for Solution 3: 0 marks

Feedback: Like the previous solutions, this student also did not apply Kirchhoff's Laws. The calculation of equivalent resistance is incorrect, which affected the subsequent calculations for current, power, and energy. The student must revisit the circuit analysis using Kirchhoff's laws and the correct method for calculating equivalent resistance in complex circuits.

None of the solutions provided by the students meet the criteria set out in the mark scheme. Each student should review the basic principles of circuit analysis, particularly Kirchhoff's laws, and practice solving for currents in complex circuits before attempting to calculate the power and energy delivered by the battery."

\textbf{Response 2:}

"Let's review and grade each solution according to the given mark scheme.

Solution 1:

Part a)

Identifies circuit elements but does not mention Kirchhoff's Laws explicitly (0 marks).

Does not split the current into branches or use Kirchhoff's current law (0 marks).

Does not use Kirchhoff's voltage law to obtain equations for the currents (0 marks).

Incorrect current through the ammeter (0 marks).

Total for part a): 0 marks

Part b)

Attempts to use the power formula but does not calculate the power generated by the 12V battery correctly (0 marks).

Incorrect calculation of energy delivered in 4 seconds (0 marks).

Total for part b): 0 marks

General Feedback for Solution 1:
The approach of calculating equivalent resistance without using Kirchhoff's laws is not correct for this circuit due to the presence of two voltage sources. The solution lacks the necessary steps of setting up and solving simultaneous equations which are required to find the current through the ammeter. The calculation of the total current and energy is incorrect.

Overall Grade: 0 marks out of 11

Solution 2:

Part a)

Does not mention Kirchhoff's Laws explicitly (0 marks).

Incorrectly calculates total resistance; no splitting into branches or use of Kirchhoff's current law (0 marks).

Does not use Kirchhoff's voltage law correctly; the provided equations are not valid for this circuit (0 marks).

Incorrect current through the ammeter (0 marks).

Total for part a): 0 marks

Part b)

Correct use of the power formula, but the current value used is incorrect (1 mark for the correct formula but wrong values).

Correct approach to calculate energy delivered but based on incorrect power calculation (0 marks for correct formula but wrong calculation).

Total for part b): 1 mark

General Feedback for Solution 2:
The solution shows an understanding of Ohm's law and power calculations but fails to apply Kirchhoff's laws correctly to this complex circuit with multiple branches and voltage sources. The total resistance and current calculations are incorrect, leading to the wrong power and energy values.

Overall Grade: 1 mark out of 11

Solution 3:

Part a)

Does not mention Kirchhoff's Laws explicitly (0 marks).

Incorrect method of calculating equivalent resistance; no splitting into branches or use of Kirchhoff's current law (0 marks).

Does not use Kirchhoff's voltage law correctly; the equations provided do not apply to this circuit (0 marks).

Incorrect current through the ammeter (0 marks).

Total for part a): 0 marks

Part b)

Correct formula for power, but uses incorrect current value (1 mark for the correct formula but wrong values).

Correct approach to calculate energy delivered but based on incorrect power calculation (0 marks for correct formula but wrong calculation).

Total for part b): 1 mark

General Feedback for Solution 3:
This solution also fails to use Kirchhoff's laws, which are essential for solving this circuit. The method of combining resistances is incorrect for a circuit with multiple voltage sources. The power and energy calculations are based on incorrect assumptions and values.

Overall Grade: 1 mark out of 11

In summary, all solutions have significant issues and do not follow the required method of using Kirchhoff's laws to solve for the currents and subsequently the power and energy. There is a lack of correct application of Kirchhoff's current and voltage laws, leading to incorrect solutions."

\textbf{Response 3:}

"Solution 1 Grading

a) Calculate the current seen by the ammeter:

Mention use of Kirchhoff Circuit Laws: 0 marks

Split current into 3 branches, use of Kirchhoff’s current law: 0 marks

Use of Kirchhoff’s voltage law to obtain equations: 0 marks

Solve simultaneous equations, obtain 4/3 A: 0 marks

Total for part a): 0 marks

b) Calculate the energy delivered by the 12V battery in 4 seconds:

Power generated by the 12V battery is the sum of power delivered to each component: 0 marks

Energy delivered in 4 seconds: 0 marks

Total for part b): 0 marks

General Feedback:
The solution provided does not correctly apply Kirchhoff's laws, does not correctly split the current into branches or solve for them, and does not arrive at the correct current through the ammeter. Additionally, the power and energy calculations are incorrect, failing to account for the correct circuit analysis.

Solution 2 Grading

a) Calculate the current seen by the ammeter:

Mention use of Kirchhoff Circuit Laws: 0 marks

Split current into 3 branches, use of Kirchhoff’s current law: 0 marks

Use of Kirchhoff’s voltage law to obtain equations: 0 marks

Solve simultaneous equations, obtain 4/3 A: 0 marks

Total for part a): 0 marks

b) Calculate the energy delivered by the 12V battery in 4 seconds:

Power generated by the 12V battery is the sum of power delivered to each component: 1 mark (for using P=VI, though the calculation is incorrect)

Energy delivered in 4 seconds: 1 mark (for using E=P×t, though the calculation is incorrect)

Total for part b): 2 marks

General Feedback:
The approach does not correctly apply Kirchhoff's laws or solve the circuit correctly. The total resistance and current calculations are incorrect, and while the formulas for power and energy are correct, the values used in calculations are incorrect due to the faulty analysis.

Solution 3 Grading

a) Calculate the current seen by the ammeter:

Mention use of Kirchhoff Circuit Laws: 0 marks

Split current into 3 branches, use of Kirchhoff’s current law: 0 marks

Use of Kirchhoff’s voltage law to obtain equations: 0 marks

Solve simultaneous equations, obtain 4/3 A: 0 marks

Total for part a): 0 marks

b) Calculate the energy delivered by the 12V battery in 4 seconds:

Power generated by the 12V battery is the sum of power delivered to each component: 0 marks

Energy delivered in 4 seconds: 0 marks

Total for part b): 0 marks

General Feedback:
The student's approach does not apply Kirchhoff's laws, and the current through the ammeter and the power/energy calculations are incorrect. The student did not correctly find the equivalent resistance or the total current, and the use of the 2V battery in the calculations is incorrect."

\subsection*{Question 6:}

\textbf{Response 1:}

"To grade each solution according to the provided markscheme, let's break down the evaluation criteria for each part of the problem and then apply these criteria to the provided solutions.

Solution 1:

\textbf{Part a)}:
\begin{itemize}
    \item The student correctly identifies the relationship $E = \frac{V}{d}$ and calculates $E = 800 \, \text{N/C}$ (2 marks).
    \item The explanation is clear but lacks the specific mention of the direction of the electric field, which is a minor oversight since the calculation is focused on magnitude. However, they implicitly reference the direction by discussing the addition of fields due to each plate (0.5 marks rounded to the nearest whole number as per instructions).
\end{itemize}

\textbf{Part b)}:
\begin{itemize}
    \item Correct use of $F = qE$ and $a = \frac{qE}{m}$ to calculate acceleration accurately as $34.78 \, \text{m/s}^2$ (2 marks).
\end{itemize}

\textbf{Total}: 4.5 rounded to 5 marks

\textbf{Feedback}: Solution 1 demonstrates a strong understanding of the problem's concepts, correctly calculates the electric field and the object's acceleration, and includes thorough explanations. Minor deduction for not explicitly mentioning the direction of the electric field in part a. Excellent work.

Solution 2:

\textbf{Part a)}:
\begin{itemize}
    \item The explanation begins with an unnecessary discussion on electric field due to a single plate but correctly converges to $E = \frac{V}{d}$ for the calculation, correctly arriving at $E = 800 \, \text{N/C}$ (2 marks).
    \item There's no explicit mention of the direction of the electric field, which is part of the requirement for the additional mark in part a. Similar to Solution 1, the direction is implied through the explanation (0.5 marks rounded to the nearest whole number).
\end{itemize}

\textbf{Part b)}:
\begin{itemize}
    \item Accurately uses $F = qE$ and $a = \frac{qE}{m}$ to find acceleration, $34.78 \, \text{m/s}^2$ (2 marks).
\end{itemize}

\textbf{Total}: 4.5 rounded to 5 marks

\textbf{Feedback}: Solution 2 provides a comprehensive explanation and accurate calculations for both parts of the problem. There's a minor oversight in not explicitly mentioning the electric field's direction. However, the overall approach is methodically sound and demonstrates a good understanding of the principles involved.

Solution 3:

\textbf{Part a)}:
\begin{itemize}
    \item Directly uses $E = \frac{V}{d}$ to calculate the electric field as $800 \, \text{N/C}$ (2 marks).
    \item Like the previous solutions, it doesn't explicitly state the direction of the electric field, which affects the full mark allocation for part a (0.5 marks rounded to the nearest whole number).
\end{itemize}

\textbf{Part b)}:
\begin{itemize}
    \item Correctly applies $F = qE$ and $a = \frac{qE}{m}$ to calculate the acceleration as $34.78 \, \text{m/s}^2$ (2 marks).
\end{itemize}

\textbf{Total}: 4.5 rounded to 5 marks

\textbf{Feedback}: Solution 3 efficiently addresses both parts of the problem with accurate calculations and clear explanations. The direct approach to solving for the electric field and acceleration is commendable. A minor deduction for not explicitly discussing the electric field's direction in part a. Overall, a well-presented solution.

Summary:

All three solutions demonstrate a high level of understanding and accuracy in their calculations. The main area for improvement across the solutions is the explicit mention of the electric field's direction, which was a requirement for full marks in part a."

\textbf{Response 2:}

"Grading Solutions

Solution 1:

\begin{itemize}
    \item \textbf{Part a)} This solution correctly identifies the method to calculate the electric field between the plates using the potential difference and the distance, resulting in \( E = 800 \, \text{N/C} \). The explanation starts with an unnecessary derivation of \( E \) from \( \sigma \) and \( \varepsilon_0 \), which is not needed for the solution but doesn't detract from the correctness. \textbf{[3 marks]} for correctly calculating \( E \) and mentioning the direction of the electric field.
    \item \textbf{Part b)} Correctly calculates the acceleration using \( a = \frac{qE}{m} \) and arrives at the correct value, \textbf{[2 marks]} for correct application of Newton's second law and calculation of acceleration.
\end{itemize}

\textbf{Total: 5 marks}

\textbf{Feedback:} The solution correctly identifies and applies the necessary formulas for both parts. The initial derivation of \( E \) from \( \sigma \) and \( \varepsilon_0 \) in part a) is not needed but shows a good understanding of the concepts. Ensure calculations directly relate to the question's requirements.

Solution 2:

\begin{itemize}
    \item \textbf{Part a)} Similar to Solution 1, this solution correctly calculates the electric field using the potential difference and distance, despite an initial unnecessary focus on the surface charge density and permittivity of free space. \textbf{[3 marks]} for correct calculation of \( E \) and mentioning the electric field's direction.
    \item \textbf{Part b)} Correct application of Newton's second law and the correct calculation of the acceleration, \textbf{[2 marks]} for correctly finding the acceleration using \( a = \frac{qE}{m} \).
\end{itemize}

\textbf{Total: 5 marks}

\textbf{Feedback:} The solution is thorough, correctly calculating the electric field and acceleration. Like Solution 1, the focus on \( \sigma \) and \( \varepsilon_0 \) is unnecessary for this particular problem but demonstrates a comprehensive understanding. Ensure concise solutions that directly address the question's demands.

Solution 3:

\begin{itemize}
    \item \textbf{Part a)} Directly applies the correct formula for calculating the electric field from the potential difference and distance, \textbf{[3 marks]} for correct calculation and mention of direction.
    \item \textbf{Part b)} Correctly uses Newton's second law to find the acceleration, \textbf{[2 marks]} for accurate calculation.
\end{itemize}

\textbf{Total: 5 marks}

\textbf{Feedback:} Solution 3 is concise and directly addresses the question without unnecessary derivations. It correctly calculates both the electric field and the object's acceleration. This solution demonstrates an effective approach to problem-solving by directly applying relevant formulas and understanding their physical meaning."

\textbf{Response 3:}

"Solution 1 Grading
Part a)
Calculation of \(E = \frac{V}{d} = 800 N/C \): 2 marks. The student correctly calculates the magnitude of the electric field between the plates using the potential difference and distance.
Mention of direction of \(\underline{E}\): 1 mark. The student implicitly considers the direction by discussing the addition of electric fields due to each plate but does not explicitly mention the unit vector direction. However, the explanation provided implies a correct understanding of the direction.

Part b) 
Application of Newton's 2nd Law and electric force to find \(a\): 2 marks. The student correctly applies the formula \( a = \frac{qE}{m}\) and arrives at the correct acceleration.

Total: 5/5 marks

Feedback:
The student provides a comprehensive solution, correctly calculating the electric field and the acceleration of the object. The explanation is clear, and the approach directly addresses the requirements of the problem. The explicit mention of the direction of \(\underline{E}\) could enhance clarity, but overall, the solution demonstrates a solid understanding of the concepts involved.

Solution 2 Grading
Part a)
Derivation and calculation of \(E\): 2 marks. The student correctly calculates 
\(E\) using the potential difference and distance, aligning with the markscheme. Direction of \(\underline{E}\): 0 marks. The student does not explicitly mention the direction of the electric field, which is required for the full mark.

Part b)
Application of Newton’s 2nd Law and calculation of \(a\): 2 marks. The student correctly uses the electric force to find the acceleration, following the correct process.

Total: 4/5 marks

Feedback:
The student accurately calculates the electric field and the object's acceleration. However, the solution lacks an explicit mention of the electric field's direction, which is necessary for a complete understanding. Despite this, the methodological approach and the calculations are correctly executed, showing good comprehension of the physical principles.

Solution 3 Grading
Part a)
Calculation of \(E\): 2 marks. The student correctly identifies and calculates the electric field as \(800 N/C\) using the given potential difference and distance.

Mention of direction: 1 mark. The feedback and explanation do not specify this, but assuming the approach is similar to Solution 1, where the direction is implied through the explanation of the electric field's behavior between plates.

Part b)
Use of Newton's 2nd Law and electric force: 2 marks. The student follows the correct procedure to calculate the acceleration, using the given charge and mass.

Total: 5/5 marks

Feedback:
The solution effectively addresses both parts of the problem, with correct calculations for the electric field and acceleration. The student's explanations indicate a clear understanding of the principles and equations involved. Including explicit mention of the electric field's direction relative to the charge plates would be beneficial for clarity, but overall, the student demonstrates solid comprehension and application of relevant physics concepts."

\subsection*{Question 7:}

\textbf{Response 1:}

"Grading Solution 1
Part a) Resonant Frequency
Correct formula and calculation. [1 mark] The student correctly identifies the formula for the resonant frequency and calculates it accurately with appropriate unit conversions.

Part b) Voltage Drop Across the Inductor
Incorrect impedance expression and current formula. The student attempts to derive the expression for \(\Delta V_L\) but does not accurately express the impedance \(Z\) or the correct form of the current \(I\). No marks for incorrect formula derivation. [0 marks]

Part c) Phase Relationship
General description of phase relationship. The student provides a qualitative description of the phase relationship at low and high frequencies but does not accurately describe the phase as in phase or antiphase according to the mark scheme. [1 mark]

Total: 2/6

Feedback: The student correctly calculated the resonant frequency but failed to accurately derive the voltage drop across the inductor with the correct expressions for impedance and current. The description of the phase relationship, while qualitatively correct, did not meet the criteria specified in the mark scheme. It's crucial to ensure mathematical expressions and phase relationships are correctly understood and described in terms of the given parameters.

Grading Solution 2

Part a) Resonant Frequency
Correct formula and calculation. [1 mark] The student correctly uses the resonant frequency formula and properly converts units.

Part b) Voltage Drop Across the Inductor
Partial correct expression for \(\Delta V_L\). The student correctly identifies components of the voltage drop calculation but does not provide the correct final expression as per the mark scheme. [1 mark]

Part c) Phase Relationship
General description of phase relationship. The student explains the phase relationship at different frequencies but does not explicitly state the relationship as in phase or antiphase as required. [1 mark]

Total: 3/6

Feedback: The student has a good grasp of the resonant frequency calculation and makes a reasonable attempt at explaining the voltage drop and phase relationships. However, the precision in mathematical expressions and the explicit description of phase relationships as per the mark scheme needs improvement. Accuracy in representing the impedance and the phase shift according to the frequency is essential for a complete understanding.

Grading Solution 3

Part a) Resonant Frequency
Correct formula and calculation. [1 mark] The student correctly calculates the resonant frequency with the proper formula and unit conversions.

Part b) Voltage Drop Across the Inductor
Correct approach but incomplete expression. The student outlines the process for deriving \(\Delta V_L\) but does not arrive at the correct expression as detailed in the mark scheme. [1 mark]

Part c) Phase Relationship
Accurate qualitative description. The student provides a qualitative description of the phase relationship at low and high frequencies but fails to explicitly state the phase as in phase or antiphase, missing the specific criteria. [1 mark]

Total: 3/6

Feedback: The calculation of the resonant frequency is accurate, and the approach to solving for the voltage drop across the inductor is on the right track. However, achieving the precise expression and explicitly stating the phase relationship as in phase or antiphase is crucial for full marks. A deeper understanding of the impedance in the circuit and its effect on the phase relationship between the source e.m.f and the voltage across the inductor is necessary for a comprehensive analysis."

\textbf{Response 2:}

"Grading and Feedback for Each Solution

Solution 1

Part a): Correct formula and calculation for the resonant frequency. The value is accurate and matches the mark scheme. 1/1 mark

Part b): The explanation for deriving \(\Delta V_L\) is somewhat convoluted but essentially correct. The student correctly identifies the components involved (\(Z\), \(X\), \(L\), \(\omega\) and \(\phi\)) but does not provide a clear, final expression matching exactly the mark scheme. However, the method is fundamentally correct. 2/3 marks

Part c): The student's explanation of the phase relationship is qualitatively correct but lacks precision. The description matches the expected behavior at high and low frequencies but doesn't explicitly state "in phase" or "antiphase," which are key terms from the mark scheme. 1/2 marks
Total: 4/6 marks

Feedback: Your solution demonstrates a good understanding of the RLC circuit's behavior. However, for part b, aim to present your final expressions more clearly and concisely. In part c, using specific terms like "in phase" and "antiphase" would make your explanation more precise.

Solution 2

Part a): Correct formula and calculation for the resonant frequency. The value provided matches the mark scheme. 1/1 mark

Part b): The student provides a correct approach to finding the voltage drop across the inductor but does not simplify their expression to a form that closely matches the mark scheme. The inclusion of phase shift \(\phi\) in the final expression is correct, but the presentation lacks clarity. 2/3 marks

Part c): The explanation for phase relationship at low and high frequencies is qualitatively correct but does not use the terms "in phase" or "antiphase." The description of the behavior is appropriate, but the lack of these terms results in a less precise answer. 1/2 marks

Total: 4/6 marks

Feedback: Your approach to the problem is methodical and shows good insight. For part b, ensure your final expressions are as clear and simplified as possible. In part c, using precise terms like "in phase" and "antiphase" would enhance the accuracy of your explanation.

Solution 3

Part a): Correct application of the formula and calculation for the resonant frequency. 1/1 mark

Part b): The expression for \( \Delta V_L\) is derived correctly, incorporating all necessary components. However, the expression could be more directly matched to the mark scheme's expectations. The core elements are present but lack a bit of refinement in presentation. 2/3 marks

Part c): The explanation of phase relationships is correct and demonstrates an understanding of the circuit's behavior at different frequencies. However, it does not explicitly state "in phase" or "antiphase." 1/2 marks

Total: 4/6 marks

Feedback: You have shown a solid understanding of how the RLC circuit operates. For part b, aim for more precise expressions that align closely with expected formulas. In part c, including specific terminology like "in phase" and "antiphase" would make your answers more precise and aligned with the mark scheme."

\textbf{Response 3:}

"To grade each solution according to the provided mark scheme, we'll evaluate them part by part:

Solution 1 Grading:

Part a)

\begin{itemize}
    \item Correct formula and calculation for resonant frequency. \textbf{1 mark}
\end{itemize}

Part b)

\begin{itemize}
    \item Provides an expression for the voltage drop across the inductor in terms of \( \mathcal{E}_{0} \), \( Z \), \( X_{L} \), \(\omega\), and \(\phi\), but incorrectly incorporates \(\cos(\omega t + \phi)\) instead of \(\sin(\omega t - \phi)\) or an equivalent expression with correct phase relationship. \textbf{2 marks} (for effort in integrating components despite the trigonometric function error).
\end{itemize}

Part c)

\begin{itemize}
    \item Describes the phase relationship qualitatively without clear indication of in-phase or antiphase relationships for low and high frequencies as per the mark scheme. \textbf{1 mark}
\end{itemize}

\textbf{Total for Solution 1: 4 marks}

General Feedback for Solution 1: The solution accurately calculates the resonant frequency and attempts to derive an expression for the voltage drop across the inductor. However, there's a minor mistake in the trigonometric function used for current, and the phase relationship explanation lacks the specificity required by the mark scheme. Ensure the use of correct trigonometric functions and clarify the phase relationship in future responses.

Solution 2 Grading:

Part a)

\begin{itemize}
    \item Correct formula and calculation for resonant frequency. \textbf{1 mark}
\end{itemize}

Part b)

\begin{itemize}
    \item Accurately provides a process for finding the voltage drop across the inductor but falls short of explicitly stating the final expression as required. \textbf{2 marks} (for correctly identifying components and attempting the derivation).
\end{itemize}

Part c)

\begin{itemize}
    \item Provides a detailed explanation of the phase relationship at low and high frequencies, but it incorrectly states the phase shifts (leading/lagging by 90 degrees) instead of specifying in-phase or antiphase. \textbf{1 mark}
\end{itemize}

\textbf{Total for Solution 2: 4 marks}

General Feedback for Solution 2: Your approach to calculating the resonant frequency and attempting to derive the voltage drop across the inductor is commendable. However, clarity in expressing the final formula for part b and accuracy in describing the phase relationships in part c according to the mark scheme are needed. Focus on providing explicit expressions and accurate descriptions of phase relationships.

Solution 3 Grading:

Part a)

\begin{itemize}
    \item Correct formula and calculation for resonant frequency. \textbf{1 mark}
\end{itemize}

Part b)

\begin{itemize}
    \item Provides an expression for the voltage drop across the inductor, incorporating all required components. \textbf{3 marks}
\end{itemize}

Part c)

\begin{itemize}
    \item Attempts to describe the phase relationship at low and high frequencies, but does not use the terms "in phase" or "antiphase" as specified in the mark scheme. \textbf{1 mark}
\end{itemize}

\textbf{Total for Solution 3: 5 marks}

General Feedback for Solution 3: Well done on accurately calculating the resonant frequency and providing a comprehensive expression for the voltage drop across the inductor. However, the description of the phase relationship could be improved by explicitly stating "in phase" and "antiphase" as described in the mark scheme. Aim for precision in terminology to enhance the clarity of your explanations."

\subsection*{Question 8:}

\textbf{Response 1:}

"To grade each solution according to the given mark scheme, we'll assess how well they adhere to the outlined criteria, specifically focusing on their application of boundary conditions, the use of \( B = \mu_0 \mu_r H \), and their manipulation of equations to obtain the final expression for \( \alpha_1 \).

Solution 1

\begin{itemize}
    \item \textbf{[3 marks]} The solution correctly applies the boundary conditions for \( H \) and \( B \) by equating the tangential components of \( H \) across the boundary and the normal components of \( B \). It successfully transitions these conditions into equations involving the magnetic field \( \underline{B} \) and its components, directly leading to the correct formulation for \( \alpha_1 \).
    \item \textbf{[1 mark]} It correctly uses the relationship \( B = \mu_0 \mu_r H \) in deriving the equations, specifically acknowledging the role of relative permeability \( \mu_r \).
    \item \textbf{[2 marks]} The solution manipulates these equations to arrive at the correct formula for \( \alpha_1 \), explicitly solving for \( \tan(\alpha_1) \) and using it to calculate \( \alpha_1 \) using the inverse tangent function, correctly resulting in \( \alpha_1 \approx 26.37^\circ \), which is in line with the expected answer.
\end{itemize}

\textbf{Total: 6/6}

\textbf{Feedback:} This solution is well-structured and rigorously applies the boundary conditions, correctly utilizes the relationship between \( B \) and \( H \), and accurately manipulates the equations to arrive at the correct value for \( \alpha_1 \). It is a comprehensive and correct approach to solving the problem.

Solution 2

\begin{itemize}
    \item \textbf{[1 mark]} This solution attempts to use the boundary conditions for the magnetic field at the interface but does so incorrectly by not applying them in a way that leads to a solvable equation for \( \alpha_1 \).
    \item \textbf{[0 marks]} It does not explicitly use \( B = \mu_0 \mu_r H \) to relate \( B \) and \( H \) or to derive the final expression for \( \alpha_1 \).
    \item \textbf{[0 marks]} The solution fails to manipulate the equations correctly to arrive at the expression for \( \alpha_1 \) as required. It does not provide a clear path to solving for \( \alpha_1 \) and does not reach the correct formula or value.
\end{itemize}

\textbf{Total: 1/6}

\textbf{Feedback:} The solution demonstrates an understanding of the importance of boundary conditions but does not apply them correctly to solve for \( \alpha_1 \). It lacks the explicit use of \( B = \mu_0 \mu_r H \) and fails to correctly manipulate the equations to obtain the required expression for \( \alpha_1 \). More attention to the mathematical treatment of the boundary conditions and their implications for \( \alpha_1 \) is needed.

Solution 3

\begin{itemize}
    \item \textbf{[2 marks]} This solution mentions the correct boundary conditions for \( H \) and \( B \) but does not apply them correctly to derive equations that would lead directly to solving for \( \alpha_1 \). The approach is partially correct but lacks the necessary manipulation to directly solve for \( \alpha_1 \).
    \item \textbf{[1 mark]} The relationship \( B = \mu_0 \mu_r H \) is mentioned, but the solution does not effectively use this relationship to facilitate the calculation of \( \alpha_1 \).
    \item \textbf{[0 marks]} The solution fails to correctly manipulate the equations to arrive at the required expression for \( \alpha_1 \) and erroneously calculates \( \alpha_1 \) without correctly applying the tangent relationship and its inverse to find the angle.
\end{itemize}

\textbf{Total: 3/6}

\textbf{Feedback:} This solution acknowledges the boundary conditions and the relationship between \( B \) and \( H \), but it does not apply these principles correctly to solve for \( \alpha_1 \). The manipulation of equations to derive \( \alpha_1 \) is incorrect, leading to an inaccurate result. A clearer understanding and application of the boundary conditions and their mathematical implications are necessary to solve the problem accurately."

\textbf{Response 2:}

"To grade each solution based on the provided mark scheme, I will evaluate them against the three key criteria outlined, namely the application of correct boundary conditions, the use of \( B = \mu_0 \mu_r H \), and the manipulation of equations to obtain the specified result.

Solution 1

\begin{itemize}
    \item \textbf{Application of Correct Boundary Conditions}: The student correctly applies the boundary conditions for the tangential and normal components of \(H\) and \(B\). They use the relationships \(H_1 \sin(\alpha_1) = H_2 \sin(\alpha_2)\) and \(B_1 \cos(\alpha_1) = B_2 \cos(\alpha_2)\) implicitly by converting them into terms of \(B\) and \(\mu_r\), which is a valid approach. \textbf{[3 marks]}
    \item \textbf{Use of \( B = \mu_0 \mu_r H \)}: The student correctly uses the relationship between \(B\), \(\mu\), and \(H\), incorporating the concept of relative permeability (\(\mu_r\)) into their solution. \textbf{[1 mark]}
    \item \textbf{Manipulation of Equations}: The student correctly manipulates the equations to express \(\tan(\alpha_1)\) in terms of \(\tan(\alpha_2)\) and the relative permeabilities, leading to the correct calculation of \(\alpha_1\). They obtain a value of approximately \(26.37^\circ\), which is consistent with the mark scheme's expected result. \textbf{[2 marks]}
\end{itemize}

\textbf{Total: 6 marks}

\textbf{Feedback}: Solution 1 effectively uses the given data and boundary conditions to calculate the angle \( \alpha_{1} \). The approach is methodical and correctly applies physical principles and mathematical relationships, leading to the accurate result.

Solution 2

\begin{itemize}
    \item \textbf{Application of Correct Boundary Conditions}: The student discusses the boundary conditions but fails to apply them correctly in a mathematical form that leads to the calculation of \( \alpha_{1} \). They do not provide a clear path from the boundary conditions to the solution. \textbf{[1 mark]}
    \item \textbf{Use of \( B = \mu_0 \mu_r H \)}: This concept is mentioned but not effectively used to derive a solution. \textbf{[0 marks]}
    \item \textbf{Manipulation of Equations}: The student does not reach the manipulation step that results in the final expression or the calculation of \( \alpha_{1} \) as expected. \textbf{[0 marks]}
\end{itemize}

\textbf{Total: 1 mark}

\textbf{Feedback}: Solution 2 discusses the relevant concepts but lacks a coherent strategy to apply the boundary conditions and relative permeabilities to solve for \( \alpha_{1} \). The solution would benefit from a more rigorous mathematical approach and clear steps leading to the calculation of the angle.

Solution 3

\begin{itemize}
    \item \textbf{Application of Correct Boundary Conditions}: The student begins to apply the boundary conditions but mistakenly focuses on solving for \(\sin(\alpha_{1})\) without correctly relating it to the tangential components or using the correct equation for manipulation to solve for \( \alpha_{1} \). \textbf{[1 mark]}
    \item \textbf{Use of \( B = \mu_0 \mu_r H \)}: The concept is mentioned, but the student misapplies it in the context of solving for \( \alpha_{1} \). \textbf{[0 marks]}
    \item \textbf{Manipulation of Equations}: The manipulation does not lead to the correct form or the expected result. The student's approach results in an incorrect expression and, consequently, an incorrect value for \( \alpha_{1} \). \textbf{[0 marks]}
\end{itemize}

\textbf{Total: 1 mark}

\textbf{Feedback}: Solution 3 shows an attempt to apply the relevant physical principles but deviates significantly from the correct approach. The solution would benefit from correctly applying the boundary conditions to derive the relationship between \(\alpha_{1}\) and the relative permeabilities, leading to accurate manipulation and calculation of the angle."

\textbf{Response 3:}

"Grading and Feedback for Each Solution

\textbf{Solution 1:}
\begin{itemize}
    \item \textbf{Correct Boundary Conditions and Use of Equations:} 3/3 \\
    The student correctly applies the boundary conditions for the parallel and perpendicular components of \(H\) and \(B\) and uses the relationship \(B = \mu_0 \mu_r H\) correctly.
    \item \textbf{Use of \(B = \mu_0 \mu_r H\):} 1/1 \\
    The student accurately incorporates the relationship between \(B\), \(\mu_0\), \(\mu_r\), and \(H\) into their solution.
    \item \textbf{Manipulation of Equations to Obtain Final Answer:} 2/2 \\
    The student correctly manipulates the equations to derive the formula for \(\alpha_1\) and calculates the angle correctly as \(26.37^\circ\), which matches the expected answer.
\end{itemize}

\textbf{General Feedback:}
Excellent application of magnetic field boundary conditions and mathematical manipulation to arrive at the correct angle \(\alpha_1\). The solution is well-structured and demonstrates a thorough understanding of the concepts involved.

\textbf{Grade:} 6/6

\hrulefill

\textbf{Solution 2:}
\begin{itemize}
    \item \textbf{Correct Boundary Conditions and Use of Equations:} 2/3 \\
    The student discusses the boundary conditions but does not apply them correctly to derive a relationship between the angles and the relative permeabilities.
    \item \textbf{Use of \(B = \mu_0 \mu_r H\):} 0/1 \\
    The relationship between \(B\), \(\mu_0\), \(\mu_r\), and \(H\) is mentioned but not effectively used to progress towards the solution.
    \item \textbf{Manipulation of Equations to Obtain Final Answer:} 0/2 \\
    The student does not manipulate the equations to find a final expression for \(\alpha_1\), and no calculation of \(\alpha_1\) is provided.
\end{itemize}

\textbf{General Feedback:}
The approach indicates an understanding of the boundary conditions conceptually but lacks the mathematical application needed to solve for \(\alpha_1\). To improve, focus on applying the given conditions to establish a direct relationship between the angles and the materials' relative permeabilities, leading to the calculation of \(\alpha_1\).

\textbf{Grade:} 2/6

\hrulefill

\textbf{Solution 3:}
\begin{itemize}
    \item \textbf{Correct Boundary Conditions and Use of Equations:} 1/3 \\
    The student correctly identifies the boundary conditions but incorrectly applies them, leading to an inaccurate method for calculating \(\alpha_1\).
    \item \textbf{Use of \(B = \mu_0 \mu_r H\):} 1/1 \\
    The use of the relationship between \(B\) and \(H\) is correctly mentioned, but the application in the context of solving for \(\alpha_1\) is flawed.
    \item \textbf{Manipulation of Equations to Obtain Final Answer:} 0/2 \\
    The equations are not correctly manipulated to derive the correct formula for \(\alpha_1\), resulting in an incorrect final answer.
\end{itemize}

\textbf{General Feedback:}
The student shows an initial correct understanding of the problem's requirements but fails to correctly apply the boundary conditions and the relationship between \(B\) and \(H\) to solve for \(\alpha_1\). The error in the manipulation of equations leads to an incorrect conclusion. A more careful review of how the tangential and normal components relate across the boundary and their impact on calculating \(\alpha_1\) is needed.

\textbf{Grade:} 2/6"

\subsection*{Question 9:}

\textbf{Response 1:}

"Let's grade each solution based on the provided mark scheme.

### Solution 1 Grading

#### Part a) (Total: 7 marks)
- **Assumption of \(\underline{H}\) solution**: Mentioned implicitly but not explicitly assumed as per the mark scheme. (0/1)
- **Applying Faraday's law and relation between \(\underline{B}\) and \(\underline{H}\)**: Correctly applies Faraday's law but does not explicitly show the relationship as per mark scheme requirements. (1/2)
- **Noting \(\nabla \times \underline{E}\) and its relation to \(\underline{H}\)**: Correctly identifies the relationship but lacks clear step-by-step alignment with the mark scheme's expectations. (1/2)
- **Taking magnitude and simplifying to desired result**: Correctly finds the magnitude and simplifies to the desired result but with a more complex route than expected. (2/2)

Total for part a: **4/7**

#### Part b) (Total: 6 marks)
- **Expression of the Poynting vector**: Correctly uses the expression for \(\underline{N}\). (1/1)
- **Simplification and real part consideration**: Correctly identifies and simplifies but does not explicitly follow the steps laid out in the mark scheme. (2/3)
- **Averaging over one period**: Correctly averages but does not justify the average of the cosine term explicitly. (1/2)

Total for part b: **4/6**

#### Part c) (Total: 2 marks)
- **Factors of cosine**: Correctly applies both factors of cosine in line with the mark scheme. (2/2)

Total for part c: **2/2**

**Overall Total for Solution 1: 10/15**

#### General Feedback for Solution 1:
The solution demonstrates a good understanding of the electromagnetic wave theory and its implications. However, it lacks explicit statements and derivations as per the mark scheme, particularly in assuming solutions and applying Faraday's law. More direct references to the steps outlined in the mark scheme would improve clarity and adherence to the grading criteria.

### Solution 2 Grading

#### Part a) (Total: 7 marks)
- **Assumption of \(\underline{H}\) solution**: Not explicitly assumed as per the mark scheme. (0/1)
- **Applying Faraday's law and relation between \(\underline{B}\) and \(\underline{H}\)**: Correct application of Faraday's law but misses direct reference to the mark scheme's expectation for explicit mention of \(\underline{B} = \mu_0 \underline{H}\). (1/2)
- **Noting \(\nabla \times \underline{E}\) and its relation to \(\underline{H}\)**: Identifies the relationship but does not directly calculate or show the derivation as expected. (1/2)
- **Taking magnitude and simplifying to desired result**: Correctly finds the magnitude and simplifies, but the route is slightly indirect compared to the mark scheme. (2/2)

Total for part a: **4/7**

#### Part b) (Total: 6 marks)
- **Expression of the Poynting vector**: Correct use of \(\underline{N}\) expression. (1/1)
- **Simplification and real part consideration**: Correctly identifies and simplifies, but the explicit step-by-step process is missing as per the mark scheme. (2/3)
- **Averaging over one period**: Correctly averages but lacks explicit justification for the average of the cosine term. (1/2)

Total for part b: **4/6**

#### Part c) (Total: 2 marks)
- **Factors of cosine**: Correctly applies both factors of cosine, as required. (2/2)

Total for part c: **2/2**

**Overall Total for Solution 2: 10/15**

#### General Feedback for Solution 2:
This solution shows a strong grasp of electromagnetic wave principles but falls short in explicitly detailing steps and assumptions as outlined in the mark scheme. Clarity and adherence to the expected procedural steps could enhance the solution's alignment with the grading criteria.

### Solution 3 Grading

#### Part a) (Total: 7 marks)
- **Assumption of \(\underline{H}\) solution**: Does not clearly assume the solution form for \(\underline{H}\) as required. (0/1)
- **Applying Faraday's law and relation between \(\underline{B}\) and \(\underline{H}\)**: Applies Faraday's law but does not explicitly show the relationship between \(\underline{B}\) and \(\underline{H}\) as expected. (1/2)
- **Noting \(\nabla \times \underline{E}\) and its relation to \(\underline{H}\): Correctly identifies the relationship but falls short in the direct derivation expected. (1/2)
- Taking magnitude and simplifying to desired result: Successfully finds the magnitude and simplifies, albeit not as directly as the mark scheme suggests. (2/2)
Total for part a: 4/7

Part b) (Total: 6 marks)
Expression of the Poynting vector: Correctly uses the Poynting vector expression. (1/1)

Simplification and real part consideration: Identifies and simplifies correctly but doesn't clearly follow the step-by-step expectations of the mark scheme. (2/3)

Averaging over one period: Averages correctly but lacks the explicit justification for the average of the cosine term as expected. (1/2)

Total for part b: 4/6

Part c) (Total: 2 marks)

Factors of cosine: Correctly applies the cosine factors in accordance with the mark scheme. (2/2)

Total for part c: 2/2

Overall Total for Solution 3: 10/15

General Feedback for Solution 3:
While the solution demonstrates a solid understanding of electromagnetic theory, it could be improved by more explicitly following the mark scheme's steps and assumptions. Detailing the derivation processes and directly addressing the mark scheme's criteria would enhance the solution's accuracy and clarity."

\textbf{Response 2:}

"Grading each solution according to the provided mark scheme, we get the following results:

Solution 1

Part a)

Assumes H has solution: 1 mark
Applies Faraday's law correctly, including relation between B and H: 2 marks

Correct use of curl of E and partial derivative of H: 2 marks

Correctly simplifies to the desired result: 2 marks

Total for part a): 7 marks

Part b)

Correctly uses expression of Poynting vector: 1 mark

Real part of E and H used, but incorrect justification for average value and doesn't explicitly justify time average of cosine term as 1/2: 2 marks (since there is an attempt to simplify but lacks correct justification)

Total for part b): 3 marks

Part c)

Correct explanation for the factors of cosine: 2 marks

Total for part c): 2 marks

General Feedback: The solution for part a) is comprehensive and correctly follows the mark scheme. Part b) has the right approach but lacks proper justification for the averaging process. Part c) is correctly justified. Overall, this is a strong solution with minor gaps in the explanation for part b).

Total marks: 12/15

Solution 2

Part a)

Correct assumption about H: 1 mark
Applies Faraday's law, but does not clearly differentiate between B and H in vacuum: 1 mark (missing explicit relation)

Correct approach to curl and partial derivative: 2 marks

Correct final simplification, but the explanation is convoluted and slightly inaccurate in places: 1 mark

Total for part a): 5 marks

Part b)

Correct use of Poynting vector expression: 1 mark

Correct approach to real parts of E and H, but lacks clear simplification and justification for the time average: 2 marks

Correctly arrives at the average Poynting flux but without proper justification: 1 mark

Total for part b): 4 marks

Part c)

Correct explanation for cosine factors: 2 marks

Total for part c): 2 marks

General Feedback: The solution correctly addresses most parts of the problem but lacks clarity and precision in some explanations, especially in part a) and the justification of the average in part b). The steps are generally correct, but the presentation could be improved for better understanding.

Total marks: 11/15

Solution 3

Part a)

Correctly assumes H has a solution: 1 mark

Applies Faraday's law but does not accurately relate B and H: 1 mark
Correct calculation of the curl of E, but the integration step is not part of Faraday's law application as required: 1 mark

Correct conclusion but through an unclear process: 1 mark

Total for part a): 4 marks

Part b)

Correct initial use of Poynting vector: 1 mark

Attempts to use real parts of E and H, but the explanation is muddled and lacks clear derivation for the average value: 1 mark

Correct conclusion, but the derivation process is not fully justified: 1 mark
Total for part b): 3 marks

Part c)

Correctly identifies factors of cosine: 2 marks

Total for part c): 2 marks

General Feedback: This solution makes an attempt at all parts but often lacks clear mathematical justification or follows an incorrect process, particularly in parts a) and b). The explanations are somewhat correct but do not fully adhere to the mark scheme requirements, especially in explaining the relationship between different quantities and in deriving the average Poynting vector.

Total marks: 9/15"

\textbf{Response 3:}

"To grade each solution according to the provided mark scheme, I will evaluate each part of the students' solutions based on the criteria for part a), b), and c).

### Solution 1:

**Part a)**

- Assumes \(\underline{H}(\underline{r}, t)\) has a solution similar to \(\underline{E}(\underline{r}, t)\): **1 mark**
- Applies Faraday's law and notes the relation between \(\underline{B}\) and \(\underline{H}\) in vacuum: **2 marks**
- Correctly notes \(i \underline{k} \times \underline{E} = i \mu_0 \omega \underline{H}\): **2 marks**
- Takes magnitude of equation and simplifies to desired result: **2 marks**

**Total for part a):** 7 marks

**Part b)**

- Uses the expression of the Poynting vector: **1 mark**
- Takes the real part of \(\underline{E}\) and \(\underline{H}\) and simplifies: **3 marks**
- Correctly averages over time and justifies the average: **2 marks**

**Total for part b):** 6 marks

**Part c)**

- Correctly identifies the pressure and accounts for cosine factors: **2 marks**

**Total for part c):** 2 marks

**General Feedback:** Solution 1 provides a comprehensive and accurate approach to solving the problem. All parts are correctly addressed with detailed explanations, showing a strong understanding of electromagnetic wave theory.

**Total Marks:** 15/15

### Solution 2:

**Part a)**

- Assumes \(\underline{H}(\underline{r}, t)\) has a solution similar to \(\underline{E}(\underline{r}, t)\): **1 mark**
- Applies Faraday's law and notes the relation between \(\underline{B}\) and \(\underline{H}\) in vacuum, but with a minor error in expressing \(\underline{B}\) in terms of \(\underline{E}\): **1 mark** (deducted 1 mark for not directly relating to Faraday's law accurately)
- Correctly notes the curl relationship but does not correctly derive the final form for \(\underline{H}\): **1 mark** (deducted 1 mark for incorrect final form)
- Attempts to take magnitude but does not simplify to the exact desired result: **1 mark** (deducted 1 mark for incomplete simplification)

**Total for part a):** 4 marks

**Part b)**

- Uses the expression of the Poynting vector: **1 mark**
- Attempts to take the real part and simplify but does not explicitly use expressions from part a): **2 marks** (deducted 1 mark for lack of clarity)
- Attempts time average without clear justification: **1 mark** (deducted 1 mark for incomplete justification)

**Total for part b):** 4 marks

**Part c)**

- Correctly identifies the pressure and accounts for cosine factors: **2 marks**

**Total for part c):** 2 marks

**General Feedback:** Solution 2 presents a good effort in addressing the problem but lacks precision in some areas, particularly in part a) and b). The explanations could be clearer and more directly linked to the given expressions and mark scheme criteria.

**Total Marks:** 10/15

### Solution 3:

**Part a)**

- Correctly assumes \(\underline{H}(\underline{r}, t)\) solution: **1 mark**
- Applies Faraday's law and notes relation between \(\underline{B}\) and \(\underline{H}\) in vacuum: **2 marks**
- Correctly notes \(i \underline{k} \times \underline{E} = i \mu_0 \omega \underline{H}\), but with minor inaccuracies in derivation steps: **1 mark** (deducted 1 mark for inaccuracies)
- Attempts to take magnitude but does not correctly simplify to desired result: **1 mark** (deducted 1 mark for incorrect simplification)

**Total for part a):** 5 marks

**Part b)**

- Uses expression of the Poynting vector: **1 mark**
- Correctly takes the real part and simplifies: **3 marks**
- Averages over time with justification: **2 marks**

**Total for part b):** 6 marks

**Part c)**

- Correctly identifies the pressure and accounts for cosine factors: **2 marks**

**Total for part c):** 2 marks

**General Feedback:** Solution 3 provides a solid attempt with correct conclusions in most parts, but there are some inaccuracies and assumptions that slightly deviate from the most direct application of the principles. The explanations are generally good, with a few areas requiring more precise application of electromagnetic theory.

Total Marks: 13/15"

\subsection*{Question 10:}

\textbf{Response 1:}

"Based on the problem statement and solutions provided by the students, let's grade each solution according to the mark scheme:

Solution 1:

Part a) The student has attempted to demonstrate the invariance of the scalar product \((\underline{E} \cdot \underline{B})\) by substituting the given transformations. However, the solution lacks explicit expansion and simplification to show the final simplified term \(E_x B_x + E_y B_y + E_z B_z \). There's no clear algebraic demonstration. Mark: 1/4

Part b) The solution outline suggests the use of Lorentz transformations, but there is no actual calculation or demonstration of how the phase transforms, nor is there an explicit derivation of the relationships for \(\omega'\) and \(k'\). Mark: 0/2

Part c) The student mentions examining the limit and touches on the Doppler effect but fails to provide a clear explanation or justification. Mark: 0/1
Total: 1/7

Feedback: The student demonstrates an understanding of the concepts involved but lacks detailed calculations and explicit demonstrations required by the mark scheme. It's crucial to follow through with the algebraic manipulations and provide clear, reasoned arguments to support conclusions.

Solution 2:

Part a) This student correctly identifies the need to calculate the dot product using the transformed components and attempts to simplify the expression. However, the simplification and demonstration of the invariance are not explicitly shown in the solution provided. Mark: 1/4

Part b) The student mentions using Lorentz transformations and substituting \(x'\) and \(t'\) into the phase but does not provide the actual calculation or the explicit derivation of the relationships for \(\omega'\) and \(k'\). Mark: 0/2

Part c) The student correctly identifies the limit \( v/c \rightarrow 0\) and mentions the classical Doppler effect but does not provide detailed reasoning or justification for the physical phenomenon represented by the relationship between \(\omega\) and \(omega'\). Mark: 0/1

Total: 1/7

Feedback: The solution shows an attempt to understand and apply the concepts but lacks the detailed calculations and explicit demonstrations required. More rigorous algebraic manipulation and clear, logical reasoning are necessary to meet the mark scheme's criteria.

Solution 3:
Part a) This student attempts to perform the algebraic manipulation required to demonstrate the invariance of the scalar product \((\underline{E} \cdot \underline{B})\). They mention expanding and simplifying the transformed scalar product but do not provide the final, simplified algebraic form that clearly demonstrates invariance. Mark: 1/4

Part b) There is a mention of using Lorentz transformations and an attempt to discuss the transformation of the phase, but again, there is no explicit calculation showing the derivation of \(\omega'\) and \(k'\) as required. Mark: 0/2

Part c) The explanation about the relationship between \(\omega\) and \(\omega'\) in the limit \( v/c \rightarrow 0\) is somewhat mentioned but lacks the detailed reasoning or justification for the physical phenomenon, which is the Doppler effect, as per the mark scheme. Mark: 0/1

Total: 1/7

Feedback: While there is an attempt to approach the problem systematically, the solution lacks the detailed algebraic demonstration and explicit calculations needed to fully meet the requirements. A deeper understanding and clearer presentation of the concepts and calculations are needed.

For all students, a stronger focus on explicitly following through with the calculations and providing reasoned arguments that align with the given transformations and principles of physics would help in achieving higher marks. Demonstrating a thorough understanding through clear, step-by-step solutions is essential."

\textbf{Response 2:}

"Let's evaluate each solution based on the provided mark scheme:

Solution 1 Evaluation:

Part a) [1/4 marks]

- The solution starts with the correct approach by substituting the given transformations into the scalar product formula. However, it lacks the explicit expansion and simplification required to demonstrate invariance explicitly. It mentions attempting to simplify but does not provide a clear or correct algebraic demonstration. Therefore, it earns 1 mark for attempting substitution but fails to earn the remaining marks due to the lack of explicit expansion and simplification.

Part b) [0/2 marks]

- The solution outlines the steps required for transforming the phase of the electromagnetic wave but does not provide any mathematical derivation or correct application of the Lorentz transformations. It mentions the intention to use Lorentz transformations but fails to show the process or results. Thus, it earns 0 marks for this part.

Part c) [1/1 marks]

- The student correctly identifies the physical phenomenon as the Doppler effect and mentions frequency shift, which is in line with the mark scheme's requirement for recognizing the Doppler effect and providing reasoning involving frequency shift. Hence, it earns the full mark for this part.

\textbf{Total Marks: 2/7}

\textbf{Feedback:} The solution shows an understanding of the problem's requirements but lacks detailed mathematical derivations and explicit calculations. To improve, focus on providing detailed algebraic steps for part a) and b), demonstrating the invariance and the transformation of the wave phase explicitly.

Solution 2 Evaluation:

Part a) [2/4 marks]

- This solution provides a clear attempt to substitute and simplify the scalar product in the transformed frame, but it stops short of fully expanding and simplifying the expression to demonstrate invariance clearly. It earns 2 marks for the correct substitution and partial simplification but does not earn the full marks because it does not arrive at the final simplified term showing explicit invariance.

Part b) [1/2 marks]

- The solution outlines the correct approach by mentioning the use of Lorentz transformations but does not provide a complete or correct derivation of the phase transformation. It earns 1 mark for the correct approach but lacks the detailed derivation needed to earn the full marks.

Part c) [1/1 marks]

- The explanation for the relationship between \( \omega \) and \( \omega' \) as \( v/c \) approaches \( 0 \) is correct, identifying the classical Doppler effect. It earns the full mark for correctly identifying the phenomenon and providing a basic explanation.

\textbf{Total Marks: 4/7}

\textbf{Feedback:} The solution demonstrates a good attempt at addressing the problem but needs to provide more detailed and explicit mathematical derivations, especially in parts a) and b). To improve, ensure that algebraic steps are fully worked through and clearly demonstrate the invariance and transformation required by the question.

Solution 3 Evaluation:

Part a) [4/4 marks]

- This solution correctly substitutes the transformation terms into the scalar product and explicitly expands and simplifies the algebra, which demonstrates the invariance of \( (\underline{E} \cdot \underline{B}) \). It earns the full marks for this part by meeting all the criteria set out in the mark scheme.

Part b) [1/2 marks]

- While the solution mentions the correct approach of using Lorentz transformations and aims to substitute these into the expression for the wave phase, it does not provide a correct or complete derivation of the phase transformation. It earns 1 mark for the correct approach but does not complete the derivation as required.

Part c) [1/1 marks]

- The explanation of the relationship between \( \omega \) and \( \omega' \) as \( v/c \) approaches \( 0 \) is correct and identifies the Doppler effect, providing a reasonable explanation. It earns the full mark for this part.

\textbf{Total Marks: 6/7}

\textbf{Feedback:} The solution provides a comprehensive and mostly correct approach to the problem, with a strong performance in part a) demonstrating the invariance of the scalar product. However, it could be improved by providing a complete and correct derivation for the phase transformation in part b)."

\textbf{Response 3:}

"Grading each solution according to the provided mark scheme:

Solution 1

Part a) [0 marks]: The student begins to address the problem but does not perform any calculations or provide a clear method for showing the invariance of the scalar product \((\underline{E} \cdot \underline{B})\) The solution lacks substitution and algebraic simplification of the expressions as required by the mark scheme.

Part b) [0 marks]: The student outlines an approach to showing the phase transformation but does not perform the necessary steps to derive the expressions for \(\omega'\) and \(k'\) as per the problem statement. There's no clear application of the inverse Lorentz transformations or demonstration of how the phase transforms.

Part c) [0 marks]: While the student mentions examining the relationship between \(\omega\) and \(\omega'\) and suggests that it represents the Doppler effect, there is no explicit reasoning or calculation provided to justify this claim.

General Feedback: The student needs to follow through with calculations and provide explicit steps in their reasoning. The solution lacks the required mathematical detail and fails to demonstrate understanding through calculation and algebraic manipulation. No marks are awarded as the student does not fulfill any specific criteria of the mark scheme.

Solution 2

Part a) [3 marks]: The student performs the substitution of the transformed field components into the scalar product formula and attempts to simplify it. Although the full algebraic expansion and simplification are not explicitly shown in the solution provided, the approach aligns with the required steps up to a certain point, missing a clear final demonstration of invariance.

Part b) [1 mark]: The student mentions using the Lorentz transformations for \(x\) and \(t\) but does not correctly apply these to derive the required relations for \(\omega'\) and \(k'\). Partial credit is given for acknowledging the need to use Lorentz transformations.

Part c) [1 mark]: The explanation touches on the Doppler effect and offers some reasoning involving frequency shift in the limit \( v/c \rightarrow 0\), albeit without detailed justification. The student correctly identifies the phenomenon but lacks a rigorous explanation.

General Feedback: The student shows a good attempt at addressing the problem but needs to provide clearer, more detailed calculations, especially in part a) for full marks. The explanation in part c) captures the essence but could benefit from more detailed reasoning. There's a notable effort to apply relevant concepts, though accuracy and completion are needed for higher marks.

Solution 3

Part a) [2 marks]: This solution correctly identifies the need to substitute and simplify the scalar product in the transformed frame but lacks a clear, detailed algebraic demonstration of invariance. Partial marks are awarded for the substitution attempt.

Part b) [0 marks]: Although the student plans to show the phase transformation, there is no evidence of applying inverse Lorentz transformations or achieving the required simplification to derive \(\omega'\) and \(k'\).

Part c) [1 mark]: The student mentions the Doppler effect and provides a conceptual understanding of the phenomenon in the limit \( v/c \rightarrow 0\), aligning with the mark scheme requirements.

General Feedback: The attempt to tackle the problem is evident, yet the solution lacks the necessary algebraic manipulation and explicit calculations, especially in parts a) and b). The explanation in part c) is on the right track but could be strengthened with more precise justification.

Summary

Solution 1: Total = 0 marks. Lacks detailed calculations and explicit reasoning.

Solution 2: Total = 5 marks. A partial attempt with some correct approaches but incomplete.

Solution 3: Total = 3 marks. Shows understanding but lacks the required detailed calculation and correct application of concepts."

\end{document}