\documentclass[a4paper,11pt]{article}
\setlength\parindent{0pt}
\usepackage{amsmath, amssymb}
\usepackage[margin=1in,letterpaper]
{geometry} % decreases margins

\usepackage{setspace}
\setstretch{1.2}
\usepackage{spverbatim}

\begin{document}
\title{GPT-4 Self-Graded Responses (With Markscheme)}
\author{Group 3A - Ryan Mok, Faraaz Akhtar, Banita Bamrah, \\ Fahim M A, Richang Xie, Kashan Ali }
\date{}
\maketitle

This document compiles the GPT-4 responses when prompted to grade the solutions which have been generated. The markscheme is given to GPT-4 in the prompt. This is the grading for the Classical Mechanics (CM) questions. 

\section{CM}

\subsection*{Question 1:}

\textbf{Response 1:}

"Grading and Feedback for Physics Problem Solutions

Solution 1 Grading:

\begin{itemize}
    \item \textbf{Conservative force definition}: Correct and detailed explanation, covering the independence of path and the zero work done in a closed loop. \textbf{(1 mark)}
    \item \textbf{Potential energy definition}: This solution defines the potential energy in terms of the gradient of potential energy (\( \underline{F} = -\nabla U \)), which is mathematically valid and equivalent to the integral form given in the mark scheme, as it captures the relationship between force and potential energy. \textbf{(2 marks)}
    \item \textbf{Explanation for non-conservative force}: Provides a clear explanation that non-conservative forces are path dependent and dissipate energy, making potential energy undefined. \textbf{(1 mark)}
\end{itemize}

\textbf{Total: 4 marks}

\textbf{Feedback}: Excellent understanding of conservative forces and potential energy. The solution correctly identifies the independence of the work done by conservative forces from the path taken and uses the gradient of potential energy to explain the relationship between force and potential energy. The explanation for why potential energy cannot be defined for non-conservative forces is also accurate and well-articulated.

Solution 2 Grading:

\begin{itemize}
    \item \textbf{Conservative force definition}: Accurately describes the independence of work done by the path taken and mentions examples, meeting the criteria. \textbf{(1 mark)}
    \item \textbf{Potential energy definition}: While this solution discusses the work done and its relationship to potential energy, it does not explicitly mention the integral form as specified in the mark scheme. However, it does correctly identify the relationship between force and potential energy (\( \underline{F} = -\nabla PE \)), which is a correct and equivalent expression. \textbf{(2 marks)}
    \item \textbf{Explanation for non-conservative force}: Correctly explains the path dependency of non-conservative forces and why this makes defining a potential energy impossible, meeting the criteria. \textbf{(1 mark)}
\end{itemize}

\textbf{Total: 4 marks}

\textbf{Feedback}: Solid understanding demonstrated through accurate definitions and explanations. The explanation of conservative forces, their impact on potential energy, and the reason why potential energy cannot be defined for non-conservative forces are all correct. The solution, while not explicitly mentioning the integral form for potential energy, effectively communicates the required concepts.

Solution 3 Grading:

\begin{itemize}
    \item \textbf{Conservative force definition}: Correctly describes the nature of conservative forces, including the path independence and the effect on total mechanical energy. \textbf{(1 mark)}
    \item \textbf{Potential energy definition}: Explicitly uses the integral of the work done against a force to define potential energy, matching the mark scheme's criteria closely and mentioning the condition \( \nabla \times \underline{F} = 0 \) for a force to be conservative. However, it does not explicitly state that the choice of the origin point is arbitrary. \textbf{(2 marks)}
    \item \textbf{Explanation for non-conservative force}: Accurately explains the path dependency of non-conservative forces and the implications for defining potential energy, satisfying the criteria. \textbf{(1 mark)}
\end{itemize}

\textbf{Total: 4 marks}

\textbf{Feedback}: The solution demonstrates a deep understanding of conservative forces and potential energy, including the mathematical conditions required for a force to be conservative. The explanation regarding non-conservative forces accurately reflects why potential energy cannot be defined for them, offering a thorough understanding of the topic."

\textbf{Response 2:}

"Grading and Feedback on Physics Solutions

Solution 1 Grading

\textbf{Conservative force definition}: 1 mark. The definition provided accurately describes a conservative force, highlighting the path independence and the zero work done in a closed loop.

\textbf{Potential energy definition}: 1 mark. Although the solution describes the relationship between the force and potential energy through the gradient \(( \underline{F} = -\nabla U )\), it does not specifically include the integral form or mention that the point of origin is arbitrary, as required by the mark scheme.

\textbf{Non-conservative force explanation}: 1 mark. The explanation is correct but does not explicitly mention the path dependency in terms of the line integral's uniqueness or function form, focusing instead on the general concept of energy not being recoverable in forms like heat.

\textbf{Total}: 3 marks.

\textbf{Feedback}: Your explanation of conservative forces and their implications is solid, capturing the essential features of path independence and energy conservation. However, for the mathematical definition of potential energy, remember to include the integral form and highlight that the origin point's choice is arbitrary for full marks. Additionally, when discussing non-conservative forces, emphasizing the path dependency and its impact on potential energy's definability would strengthen your answer.

Solution 2 Grading

\textbf{Conservative force definition}: 1 mark. Provides a clear definition and examples, aligning well with the required characteristics of a conservative force.

\textbf{Potential energy definition}: 1 mark. The description uses the work-energy principle correctly but does not fully align with the mark scheme's requirement for the integral expression and the arbitrariness of the origin point.

\textbf{Non-conservative force explanation}: 1 mark. Correctly addresses the path dependency of non-conservative forces and the consequent inability to define a unique potential energy but lacks the specific reference to the uniqueness and function form of potential energy as described in the mark scheme.

\textbf{Total}: 3 marks.

\textbf{Feedback}: You've given a solid overview of conservative forces and articulated how potential energy relates to work done. However, to fully meet the criteria, ensure you include the specific integral form of potential energy definition as outlined in the mark scheme, and stress that the point of origin is arbitrary. Additionally, clarify the specific reasons why potential energy cannot be defined for non-conservative forces, focusing on the uniqueness and functional aspects of potential energy.

Solution 3 Grading

\textbf{Conservative force definition}: 1 mark. The definition aligns with the mark scheme, including the path independence and conservation of mechanical energy.

\textbf{Potential energy definition}: 2 marks. This solution includes an integral form related to the work done by the conservative force and discusses the importance of the force being conservative. However, it misses explicitly stating that the origin point is arbitrary but comes very close to the full expression and intent described in the mark scheme.

\textbf{Non-conservative force explanation}: 1 mark. Provides a comprehensive explanation, mentioning path dependence and the inability to define a unique potential energy value, aligning well with the mark scheme's requirements.

\textbf{Total}: 4 marks.

\textbf{Feedback}: Excellent job! Your response captures the essence of conservative forces and clearly explains the mathematical formulation of potential energy, nearly fully aligning with the mark scheme's requirements. You adeptly discuss the relationship between conservative forces and potential energy, and your explanation of why potential energy cannot be defined for non-conservative forces is spot-on, addressing path dependency and the implications for energy conservation."

\textbf{Response 3:}

"Solution 1 Feedback

Marks Awarded: 3/4

Feedback: This solution provides a comprehensive explanation of conservative forces and accurately describes why potential energy cannot be defined for non-conservative forces. The definition of conservative force and the explanation regarding non-conservative forces are correct, earning a full mark for each section (1/1 for the conservative force definition and 1/1 for the explanation on non-conservative forces). However, the mathematical definition of potential energy, while correct in the concept of the force being the negative gradient of potential energy, does not explicitly include the integral form as specified in the mark scheme. Therefore, it receives 1/2 for the mathematical definition of potential energy. Overall, the solution is well-articulated but misses the mark on fully satisfying the integral definition criteria.

Solution 2 Feedback

Marks Awarded: 2/4

Feedback: This solution provides an accurate definition of a conservative force and offers correct examples, earning it 1/1 for the conservative force definition. However, it diverges from the specified integral form for defining potential energy, instead focusing on the work-energy principle and the relation between force and the gradient of potential energy without mentioning the integral form involving the path from a fixed origin point to any point in space, resulting in a score of 0/2 for the mathematical definition of potential energy. The explanation about non-conservative forces is correct but slightly veers off from the concise explanation required for the mark scheme, earning 1/1 for its explanation about the impossibility of defining potential energy for non-conservative forces. The feedback highlights a need for more explicit adherence to the integral form of potential energy definition as per the mark scheme.

Solution 3 Feedback

Marks Awarded: 4/4

Feedback: Solution 3 excellently covers all required aspects according to the mark scheme. It defines a conservative force correctly and in line with the mark scheme (1/1), includes the integral definition of potential energy as a function defined by the negative work done along a path from a fixed origin point, though it uses a slightly different approach to explain this concept, it is considered equivalent and reasonable, thus full marks are given (2/2). Furthermore, it correctly explains why potential energy cannot be defined for a non-conservative force, touching upon the unique values and function definition criteria (1/1). This solution effectively meets all criteria with accurate and sufficient arguments."

\subsection*{Question 2:}

\textbf{Response 1:}

"\textbf{Solution 1:}

Grade: 6/6

Feedback: Solution 1 provides a thorough explanation of all three cases of damping, accurately describing the behavior of the damped harmonic oscillator in each scenario. The conditions for underdamping, critical damping, and overdamping are correctly identified with the respective qualitative behaviors. The descriptions match the mark scheme criteria closely, with clear explanations of how the system behaves in each case. No marks are deducted as the solution meets the criteria for full marks according to the provided mark scheme.

\textbf{Solution 2:}

Grade: 5/6

Feedback: Solution 2 presents a clear and detailed explanation of the three damping cases, including the correct conditions and behaviors. However, there is a minor inconsistency in the use of the discriminant \(\Delta = b^2 - 4mk\), which introduces a notation that wasn't specified in the mark scheme but leads to the same qualitative descriptions. This could potentially confuse the notation consistency, but the argument remains correct and sufficient. Given this slight deviation from the exact terms provided in the mark scheme, a mark is deducted for this slight inconsistency. The feedback otherwise praises the student's understanding and accurate description of the physical phenomena.

\textbf{Solution 3:}

Grade: 6/6

Feedback: Solution 3 excellently outlines the three cases of damping, providing accurate conditions and detailed descriptions of the system's behavior in each case. The explanation is directly in line with the mark scheme, and the student uses a clear and concise approach to describe the motion in each damping scenario. The use of \(b^2 < 4k\), \(b^2 = 4k\), and \(b^2 > 4k\) for underdamped, critically damped, and overdamped cases, respectively, is correctly applied, and the qualitative descriptions are accurate and comprehensive. No deductions are made as the solution meets all the criteria for full marks."

\textbf{Response 2:}

"\textbf{Solution 1:}

Grade: 6/6

Feedback: This solution excellently outlines the three cases of damping for a damped harmonic oscillator, providing clear, qualitative descriptions that align with the mark scheme. The student has correctly identified the conditions for underdamping, critical damping, and overdamping, and has provided a comprehensive explanation for the behavior of the system in each case. The explanation is thorough, using correct terminology and demonstrating a strong understanding of the topic. Well done!

\textbf{Solution 2:}

Grade: 5/6

Feedback: The student has provided a detailed description of the damped harmonic oscillator, correctly identifying and explaining the three cases of damping. However, there is a minor discrepancy: the introduction of the discriminant \(\Delta = b^2 - 4mk\) is not directly relevant to the qualitative behavior descriptions required by the problem statement. While this approach does not detract significantly from the overall correctness, it introduces unnecessary complexity to the explanation. Nonetheless, the student demonstrates a good understanding of the topic. A more direct focus on the qualitative behavior as outlined in the mark scheme would make the answer perfect.

\textbf{Solution 3:}

Grade: 6/6

Feedback: This solution accurately describes the three damping cases of a damped harmonic oscillator, using clear and correct explanations that align well with the mark scheme requirements. The student has effectively detailed the conditions for underdamping, critical damping, and overdamping, providing insightful descriptions of the system's behavior in each scenario. The use of simple terms to explain complex concepts, such as combining exponential decay with sinusoidal functions for the underdamped case, is commendable. This solution demonstrates an excellent grasp of the subject matter, presenting the information in a structured and understandable manner. Great job!"

\textbf{Response 3:}

"Solution 1 Feedback

\textbf{Grade: 6/6}

This solution accurately describes all three cases of damping in the context of a damped harmonic oscillator, closely adhering to the provided mark scheme. The student clearly differentiates between underdamped, critically damped, and overdamped scenarios with precise and correct qualitative descriptions. They have correctly identified the conditions (\(b^2 < 4mk\), \(b^2 = 4mk\), \(b^2 > 4mk\)) for each type of damping and have provided a thorough explanation of the physical behavior in each case. There are no errors in units or precision, and the solution is expressed in an equivalent, acceptable form as per the mark scheme.

\textbf{Feedback:} Excellent work! You have correctly identified and described all three cases of damping with accurate qualitative explanations. Your solution reflects a strong understanding of the damped harmonic oscillator's behavior under different damping conditions.

Solution 2 Feedback

\textbf{Grade: 6/6}

This solution also accurately addresses all three cases of damping, providing detailed descriptions consistent with the mark scheme's requirements. The student uses \(\Delta = b^2 - 4mk\) as a discriminant to classify the damping scenarios, which is a valid approach and falls within the allowance for equivalent expressions. The explanations for underdamping, critical damping, and overdamping are correct, detailed, and align with the expected physical behaviors. The student has not introduced any inappropriate units or precision in their descriptions.

\textbf{Feedback:} Great job! Your use of the discriminant \(\Delta = b^2 - 4mk\) to explain the three cases of damping was effective and accurate. Your detailed descriptions clearly convey the different behaviors of the damped harmonic oscillator under various damping conditions. Your understanding of the subject matter is commendable.

Solution 3 Feedback

\textbf{Grade: 6/6}

Solution 3 offers a comprehensive explanation of the three damping cases, adhering to the mark scheme by accurately defining and describing underdamped, critically damped, and overdamped conditions. The student's choice to simplify with \(m=1\) for the discriminant \(\Delta = b^2 - 4mk\) is acceptable and does not detract from the overall correctness of their explanations. The qualitative behaviors of each damping scenario are correctly and effectively described, with clear distinctions between each case. No issues with units or precision are present.

\textbf{Feedback:} Well done! Your approach to explaining the damping conditions through the discriminant, even with a simplification, was clear and accurate. You've effectively described the qualitative behavior of the damped harmonic oscillator in each damping scenario. Your solution demonstrates a thorough understanding of the topic."

\subsection*{Question 3:}

\textbf{Response 1:}

"Grading and Feedback

\textbf{Solution 1:}

\textbf{Part a)}: Full marks. The student correctly derived the expression for \( \underline{R} \) and identified \( k \) as \( \frac{m_2}{m_1 + m_2} \), following the definition of the center of mass. (2 marks)

\textbf{Part b)}: Full marks. The student accurately calculated the vectors \( \underline{R} - \underline{r}_1 \) and \( \underline{R} - \underline{r}_2 \), and correctly determined the distances from the center of mass to both \( \underline{r_{1}} \) and \( \underline{r_{2}} \), consistent with the mark scheme. (4 marks)

\textit{General Feedback}: This solution comprehensively addresses both parts of the question with clear mathematical manipulation and logical reasoning. The explanations are coherent, and the steps are correctly followed as per the requirements of the mark scheme.

\hrulefill

\textbf{Solution 2:}

\textbf{Part a)}: Full marks. The student's approach to deriving the position vector of the center of mass and identifying \( k \) matches the expected methodology and result. They effectively demonstrated the equivalence of the two expressions for \( \underline{R} \). (2 marks)

\textbf{Part b)}: Full marks. The student showed that the center of mass's distance from both masses is as required by utilizing the derived value of \( k \) and correctly linked these distances to the masses and the total distance \( d \). (4 marks)

\textit{General Feedback}: Excellent work in both parts of the problem. The solution is methodically presented, with each step logically leading to the next. The student's understanding of the concept of the center of mass and its relation to the positions and masses of objects is clearly demonstrated.

\hrulefill

\textbf{Solution 3:}

\textbf{Part a)}: Full marks. The explanation for deriving \( \underline{R} \) and finding \( k \) is clear and correctly follows the definition of the center of mass. The manipulation of the formula to express \( \underline{R} \) in the required format is done accurately. (2 marks)

\textbf{Part b)}: Full marks. The derivation of distances from the center of mass to each mass correctly utilizes the result from part a) and logically deduces the required distances, matching the expectations of the mark scheme. (4 marks)

\textit{General Feedback}: The student presented a thorough and accurate solution, with strong mathematical reasoning and clarity in the explanation. Both the conceptual understanding and the execution are commendable, adhering closely to the guidelines provided in the mark scheme.

Each solution was methodically approached, with correct application of physical principles and mathematical reasoning. All solutions provided accurate and comprehensive responses to the problem, demonstrating a good understanding of the center of mass concept and its implications for the system of two masses."

\textbf{Response 2:}

"Based on the mark scheme provided, here's how each solution would be graded along with concise general feedback:

Solution 1:

\textbf{Part a)}:
\begin{itemize}
    \item Uses definition of center of mass correctly: \textbf{1 mark}.
    \item Manipulates to the form and finds \( k \) correctly: \textbf{1 mark}.
\end{itemize}

\textbf{Part b)}:
\begin{itemize}
    \item Correctly calculates vectors \( \underline{R} - \underline{r}_1 \) and \( \underline{R} - \underline{r}_2 \) using results of a): \textbf{2 marks} (1 mark for each calculation).
    \item Correctly finds the distances from center of mass point to positions \( \underline{r}_1 \) and \( \underline{r}_2 \): \textbf{2 marks} (1 mark for each distance).
\end{itemize}

\textbf{Total: 6 marks}

\textbf{Feedback}: Excellent work demonstrating a thorough understanding of the center of mass concept and its mathematical representation. The calculations were accurate and well-aligned with the problem's requirements. The explanations were clear and logically structured, allowing easy follow-up of your arguments.

Solution 2:

\textbf{Part a)}:
\begin{itemize}
    \item Correct use of the definition of the center of mass: \textbf{1 mark}.
    \item Manipulates to the form and identifies \( k \) correctly: \textbf{1 mark}.
\end{itemize}

\textbf{Part b)}:
\begin{itemize}
    \item Correct calculation of vectors \( \underline{R} - \underline{r}_1 \) and \( \underline{R} - \underline{r}_2 \) using results of a), but not explicitly shown: \textbf{2 marks} (Assuming it is implicitly covered through correct distance calculation).
    \item Correct distance calculations from center of mass to \( \underline{r}_1 \) and \( \underline{r}_2 \): \textbf{2 marks}.
\end{itemize}

\textbf{Total: 6 marks}

\textbf{Feedback}: Well-executed solution with clear demonstrations and logical reasoning. You accurately derived the position vector of the center of mass and demonstrated its location relative to the masses. The mathematical expressions were correctly applied, showing a good grasp of the concepts involved.

Solution 3:

\textbf{Part a)}:
\begin{itemize}
    \item Uses the definition of the center of mass accurately: \textbf{1 mark}.
    \item Correctly manipulates to find \( k \) and matches the given form: \textbf{1 mark}.
\end{itemize}

\textbf{Part b)}:
\begin{itemize}
    \item Correct approach to calculating vectors \( \underline{R} - \underline{r}_1 \) and \( \underline{R} - \underline{r}_2 \), but the explicit vector subtraction step is not shown: \textbf{2 marks} (1 mark for the correct implicit methodology in distance calculation).
    \item Correct calculation of distances from \( \underline{R} \) to \( \underline{r}_1 \) and \( \underline{r}_2 \): \textbf{2 marks}.
\end{itemize}

\textbf{Total: 6 marks}

\textbf{Feedback}: Your solution correctly addressed both parts of the problem, demonstrating a solid understanding of the center of mass. Your derivation and calculation for \( k \) were spot-on, and your logic in linking the distances from the center of mass to the masses was precise. The explanation was clear and well-structured, providing a comprehensive answer to the problem statement."

\textbf{Response 3:}

"Grading and Feedback for Physics Solutions

\textbf{Solution 1 Grading}

Part a) This solution correctly uses the definition of the center of mass and manipulates it to match the given form, successfully identifying \( k = \frac{m_2}{m_1 + m_2} \). Marks awarded: \textbf{2/2}.

Part b) The student has accurately calculated the vectors \( \underline{R} - \underline{r}_1 \) and \( \underline{R} - \underline{r}_2 \), and correctly derived the distances from the center of mass to the masses, matching the expected answers. Marks awarded: \textbf{4/4}.

\textbf{Feedback:} Excellent work! Your methodical approach and clear mathematical manipulation correctly derived both the position vector of the center of mass and the distances from the center of mass to each mass. Your solution is comprehensive and aligns well with the principles of physics regarding the center of mass.

\hrulefill

\textbf{Solution 2 Grading}

Part a) The student has provided a correct use of the center of mass definition and manipulation to find the value of \( k \). This matches the mark scheme perfectly. Marks awarded: \textbf{2/2}.

Part b) This solution correctly demonstrates the calculation of distances from the center of mass to \( \underline{r_{1}} \) and \( \underline{r_{2}} \), using the derived value of \( k \) from part a) effectively. Marks awarded: \textbf{4/4}.

\textbf{Feedback:} Your work is accurate and well-explained, covering both the derivation of the center of mass position vector and the distances from it to the two masses. The logical flow of your arguments and the precision in mathematical manipulation are commendable.

\hrulefill

\textbf{Solution 3 Grading}

Part a) The explanation and mathematical derivation for the center of mass position vector, including finding \( k \), are correct and align with the mark scheme. Marks awarded: \textbf{2/2}.

Part b) The student successfully calculates the distances from the center of mass to each of the masses, clearly showing the center of mass lies on the line connecting the two masses with the correct distances. Marks awarded: \textbf{4/4}.

\textbf{Feedback:} Well done on thoroughly addressing both parts of the question, accurately deriving the position vector of the center of mass and the distances from the center of mass to the masses. Your solution is methodically laid out and clearly demonstrates a solid understanding of the concepts involved."

\subsection*{Question 4:}

\textbf{Response 1:}

"After reviewing the solutions against the provided mark scheme, here are the grades and feedback for each solution:

Solution 1:

\begin{itemize}
    \item \textbf{Part a):} Correct derivation of angular frequency, $\dot{\theta}$, using conservation of angular momentum. (6 marks)
    \item \textbf{Part b):} Incorrect statement that the problem does not require computing $\ddot{\theta}$, and an incorrect conclusion about the angular component of acceleration. The solution does not follow through with the differentiation to find $\ddot{\theta}$ or simplify the expression for the angular component of acceleration accurately. (0 marks)
    \item \textbf{Part c):} Correct derivation of the kinetic energy, K, and its time derivative, $dK/dt$, but fails to show detailed steps for differentiation which are critical to earn full marks. (5 marks)
\end{itemize}

\textbf{Total}: 11/18

\textbf{Feedback}: Your solution for part (a) was excellent, with a correct and clear derivation. However, part (b) lacked the necessary calculation of $\ddot{\theta}$ and incorrect assumptions were made regarding the requirements of the problem. Part (c) had the correct final expressions but lacked detailed calculation steps, particularly for the differentiation process. More rigorous application of the differentiation and a clearer step-by-step approach would improve your solution.

Solution 2:

\begin{itemize}
    \item \textbf{Part a):} Correctly identified the process for conserving angular momentum and derived the expression for $\dot{\theta}$ accurately. (6 marks)
    \item \textbf{Part b):} Completed the differentiation for $\ddot{\theta}$ but incorrectly simplified the angular acceleration component to a value without providing the calculation steps or final expression directly. The description implies a calculation but does not explicitly show the correct final form. (2 marks)
    \item \textbf{Part c):} Correct approach to deriving kinetic energy and its time derivative. However, the sign discrepancy mentioned does not match the mark scheme's expectations, which could indicate a misunderstanding in the differentiation process. (6 marks)
\end{itemize}

\textbf{Total}: 14/18

\textbf{Feedback}: Your approach to the problem demonstrates a good understanding of the concepts, particularly in parts (a) and (c). However, part (b) lacked the precise final expression for the angular acceleration component and did not show all necessary calculation steps. Ensure to follow through with all steps explicitly to maximize marks, and pay close attention to the sign and derivation details to align your solutions with the expected outcomes.

Solution 3:

\begin{itemize}
    \item \textbf{Part a):} Provides a detailed derivation for angular frequency $\dot{\theta}$ that is consistent with the conservation of angular momentum. (6 marks)
    \item \textbf{Part b):} The solution suggests a misunderstanding or incomplete calculation for the angular component of the acceleration, leading to an inaccurate conclusion. (0 marks)
    \item \textbf{Part c):} The narrative suggests correct steps toward finding kinetic energy and its derivative but fails to explicitly calculate or demonstrate understanding of $\ddot{\theta}$ and its impact on the final result. (4 marks)
\end{itemize}

\textbf{Total}: 10/18

\textbf{Feedback}: You have a strong start in part (a) with a correct formula derivation, but parts (b) and (c) show critical gaps in execution and understanding. Part (b) lacks a correct calculation and understanding of the angular acceleration component. For part (c), while the approach towards kinetic energy seems logically structured, the execution lacks precision and misses critical steps in the calculation process. Focus on fully developing each part of your answer, especially where detailed calculations and derivations are required."

\textbf{Response 2:}

"To grade each solution based on the provided mark scheme, let's evaluate them part by part:

### Solution 1

**Part a)** This solution correctly derives the expression for angular frequency \( \dot{\theta} \), following the conservation of angular momentum principle and correctly using initial conditions. The solution is well-aligned with the mark scheme steps. **6 marks**

**Part b)** The student attempts to use the given and derived expressions correctly but does not complete the calculation for \( \ddot{\theta} \) and simplifies the expression incorrectly to a non-zero value, contradicting the mark scheme's expectation of simplification to zero. This part lacks the correct derivation and simplification. **2 marks**

**Part c)** The kinetic energy derivation is correctly approached, and the differentiation leading to the final expression matches the expected outcome in the mark scheme. However, there's a minor inconsistency in the narrative before the correct formula is presented. **6 marks**

**Total for Solution 1:** 14 marks

**General Feedback:** Your approach to the angular frequency in part a) is exemplary. For part b), ensure complete and accurate calculations, including the derivation of \( \ddot{\theta} \) when required, and verify simplifications match expected physical outcomes. In part c), your final answer is correct, but strive for clarity and accuracy in every step of your explanation.

### Solution 2

**Part a)** The student follows a clear, step-by-step process to derive the angular frequency \( \dot{\theta} \) that aligns with the mark scheme. **6 marks**

**Part b)** Although the student correctly identifies the angular component of acceleration and begins to apply the given and derived expressions correctly, the computation of \( \ddot{\theta} \) and its subsequent use in calculating \( a_{\theta} \) are not explicitly shown or incorrectly simplified. **3 marks**

**Part c)** The kinetic energy expression and its differentiation are accurately derived, closely following the mark scheme's guidance and correctly arriving at the given expression. **7 marks**

**Total for Solution 2:** 16 marks

**General Feedback:** Excellent work on parts a) and c), with accurate derivations and clear understanding demonstrated. For part b), ensure you complete all required calculations and clearly show your work, including the derivation of \( \ddot{\theta} \) and its correct application in finding \( a_{\theta} \).

### Solution 3

**Part a)** The solution correctly outlines the process to derive \( \dot{\theta} \) using conservation of angular momentum and the given initial conditions. The integration of \( \dot{r} \) is correctly performed. **6 marks**

**Part b)** The solution suggests a misunderstanding in the calculation of the angular component of the acceleration, mistakenly leading to a simplification to zero, which does not match the expected result. There's an evident confusion or oversight in the application of physical principles. **1 mark**

**Part c)** The narrative suggests an attempt to calculate kinetic energy and its time derivative, but the solution contains several conceptual and mathematical inaccuracies. The correct derivative of kinetic energy is provided, but the process leading to this result is flawed or not clearly justified. **4 marks**

**Total for Solution 3:** 11 marks

**General Feedback:** Your methodical approach in part a) is commendable. However, for part b), ensure your calculations reflect the dynamics of the system correctly, including the derivation and application of \( \ddot{\theta} \). In part c), focus on accurately applying physical principles and mathematical derivations to ensure your results are both correct and justifiably derived."

\textbf{Response 3:}

"Grading and Feedback for Each Solution

Solution 1:

\textbf{Part a)}: Full Marks. The student derived the expression for angular frequency \( \dot{\theta} \) correctly as \(\dot{\theta} = \frac{r_0^2 \delta}{(r_0 - \gamma t)^2}\), utilizing the conservation of angular momentum effectively and applying the given initial conditions correctly. \textbf{(6 marks)}

\textbf{Part b)}: Partial Marks. The student provided the correct form of the angular component of acceleration but did not perform the differentiation of \(\dot{\theta}\) with respect to time to find \(\ddot{\theta}\), which is crucial for finding the angular component of acceleration accurately. This part lacks the complete process to obtain \(a_{\theta}\) correctly. \textbf{(2 marks)}

\textbf{Part c)}: Full Marks. The student derived the kinetic energy correctly and showed its time derivative \(\frac{dK}{dt}\) matches the given expression accurately. The solution correctly applies the definitions and performs the necessary differentiation. \textbf{(7 marks)}

\textbf{General Feedback}: Excellent work on parts a) and c), demonstrating a good understanding of the concepts and application in deriving expressions for angular frequency and kinetic energy. However, part b) requires more thorough calculation, specifically the differentiation needed to find \(\ddot{\theta}\).

Solution 2:

\textbf{Part a)}: Full Marks. Similar to Solution 1, this student also correctly derived the expression for angular frequency using conservation of angular momentum and integrating \(\dot{r}\) to find \(r(t)\). \textbf{(6 marks)}

\textbf{Part b)}: Near Full Marks. The student attempted to calculate \(\ddot{\theta}\) but did not explicitly show the calculation or result for \(\ddot{\theta}\). However, they correctly identified the components necessary for calculating the angular acceleration component. \textbf{(4 marks)}

\textbf{Part c)}: Full Marks. The student accurately derived the expression for kinetic energy and its time derivative, correctly matching the provided expression. The differentiation process and substitution of values were correctly applied. \textbf{(7 marks)}

\textbf{General Feedback}: Solid understanding and application of physical principles across all parts, with detailed and accurate calculations for kinetic energy and its time derivative. However, part b) lacked the explicit calculation of \(\ddot{\theta}\), which was necessary for a full understanding of the angular acceleration component.

Solution 3:

\textbf{Part a)}: Full Marks. The derivation for \(\dot{\theta}\) is correctly done with proper integration of \(\dot{r}\) and application of initial conditions to find \(r(t)\). \textbf{(6 marks)}

\textbf{Part b)}: Insufficient Explanation. The student suggests a non-zero value for \(a_{\theta}\) without providing the complete calculation or correct reasoning behind their conclusion. This part lacks the necessary steps and explanation for full marks. \textbf{(1 mark)}

\textbf{Part c)}: Incorrect Sign. While the student correctly derives the expression for kinetic energy, the sign of the time derivative \(\frac{dK}{dt}\) is incorrectly stated, which significantly impacts the correctness of the physics involved. \textbf{(4 marks)}

\textbf{General Feedback}: The approach and steps in part a) were correctly followed, demonstrating a good grasp of angular momentum conservation and its implications. However, parts b) and c) suffered from significant issues in calculation and interpretation, respectively, impacting the overall accuracy of the solution. More attention to detail in the differentiation process and the physical interpretation of the results is needed."

\subsection*{Question 5:}

\textbf{Response 1:}

"\noindent Solution 1:

\begin{itemize}
    \item[(a)] [2 marks] The student has correctly applied Newton’s second law and Hooke’s law to derive the equations of motion for objects X and Y. They have the correct form and the correct signs in front of the terms. The student gets full marks for this part.
    
    \item[(b)] [2 marks] The student has the right idea by defining new coordinates \( y_{1} \) and \( y_{2} \) and they correctly find the equation of motion for \( y_{1} \), but they incorrectly include the term \( -k(y_{1} - y_{2}) \) in the first equation and \( -k(y_{2} + y_{1}) \) in the second equation which does not arise from the sum and difference of \( x_{1} \) and \( x_{2} \). This error affects the correctness of the equation for \( y_{2} \), so only half of the marks can be awarded for this part.
    
    \item[(c)] [3 marks] The student correctly uses the initial conditions and applies them to the general solution of the simple harmonic motion to find the specific solution for \( y_{1} \). They use the correct process and arrive at the correct solution. Full marks for this part.
\end{itemize}

\noindent Feedback: The student demonstrates a good understanding of the concepts involved in deriving the equations of motion and solving for the specific case of \( y_{1} \). They should be careful with algebra when adding and subtracting equations to avoid introducing errors.

\noindent Total Marks: 7/9

\noindent Solution 2:

\begin{itemize}
    \item[(a)] [2 marks] The student correctly states the force due to the springs and uses this to derive the equations of motion for both objects X and Y. The equations are correct and follow from the application of Hooke’s Law and Newton’s second law.
    
    \item[(b)] [4 marks] The student successfully adds and subtracts the equations of motion to uncouple them and identifies the new coordinates \( y_{1} \) and \( y_{2} \) that obey the equations for simple harmonic motion. They correctly state the equations of motion for \( y_{1} \) and \( y_{2} \) after uncoupling.
    
    \item[(c)] [3 marks] The student has correctly used the initial conditions to find the constants for the general solution of \( y_{1} \), leading to the correct specific solution for \( y_{1} \). 
\end{itemize}

\noindent Feedback: The student shows excellent understanding of the process for decoupling the equations of motion and using initial conditions to find a specific solution. Their work is methodical and accurate.

\noindent Total Marks: 9/9

\noindent Solution 3:

\begin{itemize}
    \item[(a)] [2 marks] This student has also correctly derived the equations of motion for X and Y using Newton's second law and Hooke’s law. They are in the correct form.
    
    \item[(b)] [4 marks] The student accurately uncouples the equations by defining \( y_{1} \) and \( y_{2} \) and then derives the correct equations of simple harmonic motion for both. Their algebra is correct, and the resulting equations are accurate.
    
    \item[(c)] [3 marks] The student again correctly applies the initial conditions to find the constants of the general solution for \( y_{1} \), resulting in the correct specific solution for \( y_{1} \).
\end{itemize}

\noindent Feedback: The student has a solid grasp of the principles involved and is able to apply them correctly to derive both the general and specific solutions for the system. Their algebra and application of initial conditions are both correct.

\noindent Total Marks: 9/9"

\textbf{Response 2:}

"Grades and Feedback for Physics Problem Solutions

Here are the grades and feedback for each solution:

Solution 1:

\textbf{Part a)} - 2 marks
\begin{itemize}
    \item The student correctly derived the equations of motion using Newton’s second law and Hooke’s law, producing the two equations required.
\end{itemize}

\textbf{Part b)} - 2 marks
\begin{itemize}
    \item The student correctly substituted the new coordinates \( y_{1} \) and \( y_{2} \) but made algebraic errors in deriving the final uncoupled equations. They did not arrive at the correct uncoupled equations of motion.
\end{itemize}

\textbf{Part c)} - 2 marks
\begin{itemize}
    \item The student correctly found the general solution for \( y_1 \) and used the initial conditions to find B = 0 and a value for A, but they presented the general solution in terms of \( \cos(\omega t) \) and \( \sin(\omega t) \) instead of directly presenting the sine function as in the mark scheme.
\end{itemize}

\textbf{Total: 6 marks}

\textbf{General Feedback:} The approach to the problem was methodical, and the correct formulas were applied. However, there were algebraic errors in part b) that led to incorrect final equations. Ensure that algebraic manipulation is accurate. In part c), while the result is ultimately correct, the presentation of the solution did not directly match the expected form.

Solution 2:

\textbf{Part a)} - 2 marks
\begin{itemize}
    \item The student derived the correct equations of motion.
\end{itemize}

\textbf{Part b)} - 4 marks
\begin{itemize}
    \item The student successfully uncoupled the coordinates and arrived at the correct equations of motion for \( y_{1} \) and \( y_{2} \), recognizing them as simple harmonic motion.
\end{itemize}

\textbf{Part c)} - 3 marks
\begin{itemize}
    \item The student correctly applied initial conditions to the general solution for \( y_1 \), finding the specific solution as asked for in the problem statement.
\end{itemize}

\textbf{Total: 9 marks}

\textbf{General Feedback:} Excellent work. The student successfully derived the equations of motion, correctly decoupled the system, and properly used initial conditions to find the specific solution for \( y_1 \). The solution was presented clearly and was easy to follow.

Solution 3:

\textbf{Part a)} - 2 marks
\begin{itemize}
    \item The student has derived the correct equations of motion for both objects X and Y.
\end{itemize}

\textbf{Part b)} - 3 marks
\begin{itemize}
    \item The student followed the correct process to uncouple the equations, but the final uncoupled equations for \( y_1 \) and \( y_2 \) included some minor algebraic mistakes. Despite this, they did identify the equations as representing simple harmonic motion.
\end{itemize}

\textbf{Part c)} - 2 marks
\begin{itemize}
    \item The student provided a correct general solution and applied the initial conditions, but they unnecessarily introduced a phase constant \( \phi \) which was not required as per the initial conditions and the given mark scheme. They also provided \( A \) and \( \omega \) correctly.
\end{itemize}

\textbf{Total: 7 marks}

\textbf{General Feedback:} The solution demonstrates a good understanding of the physical concepts and the mathematics involved. However, care should be taken to avoid introducing unnecessary variables, and attention to detail is needed in algebraic manipulation to ensure that the final forms of the equations are accurate."

\textbf{Response 3:}

"Based on the given problem and mark scheme, let's grade each student's solution.

\textbf{Solution 1:}

Part a) The student derived the correct form of the equations of motion for X and Y using Newton’s second law and Hooke's law. \textbf{(2 marks)}

Part b) The student correctly found the equations of motion for the new coordinates \( y_{1} \) and \( y_{2} \) by adding and subtracting the original equations of motion. However, there are some algebraic errors in the working of the student. The correct equations should be \( m \ddot{y}_{1} = -k y_1 \) and \( m \ddot{y}_{2} = -3k y_2 \), not as written in the student's solution, so partial credit is given. \textbf{(2 marks)}

Part c) The student correctly used the initial conditions to find the particular solution for \( y_{1} \). \textbf{(3 marks)}

Total: \textbf{7 marks}

\emph{Feedback for Student 1: You have correctly applied the principles to derive the equations of motion and used the initial conditions effectively. However, be careful with your algebra when combining the equations to get the correct form for \( y_{1} \) and \( y_{2} \).}

\textbf{Solution 2:}

Part a) The equations of motion for X and Y have been correctly derived. \textbf{(2 marks)}

Part b) The student correctly uncoupled the coordinates and demonstrated that \( y_{1} \) and \( y_{2} \) obey the equations of simple harmonic motion. However, they didn’t explicitly mention both equations represent simple harmonic motion, which is required by the mark scheme. \textbf{(3 marks)}

Part c) The student correctly found the particular solution for \( y_{1} \) using the initial conditions. \textbf{(3 marks)}

Total: \textbf{8 marks}

\emph{Feedback for Student 2: Excellent work on deriving the equations and solving for \( y_{1} \). Remember to explicitly state that the equations represent simple harmonic motion to fully meet the requirements of the question.}

\textbf{Solution 3:}

Part a) The equations of motion are correctly derived. \textbf{(2 marks)}

Part b) The student correctly derived the uncoupled equations for \( y_{1} \) and \( y_{2} \). However, they presented the uncoupled equations in their original form instead of in terms of \( y_{1} \) and \( y_{2} \) and did not explicitly mention that these represent simple harmonic motion, leading to a deduction. \textbf{(2 marks)}

Part c) The student derived the correct solution for \( y_{1} \) using the initial conditions. \textbf{(3 marks)}

Total: \textbf{7 marks}

\emph{Feedback for Student 3: You did well in deriving the equations and finding the solution for \( y_{1} \). Ensure that you represent the final uncoupled equations in terms of \( y_{1} \) and \( y_{2} \) and state explicitly that they represent simple harmonic motion.}

These marks are based on the information given in the mark scheme. The feedback aims to encourage the students to focus on the areas that require attention for improvement."

\subsection*{Question 6:}

\textbf{Response 1:}

"Let's grade each solution according to the provided mark scheme.

### Solution 1 Grading:

**Part a):**
- Identification of frictional force and centripetal acceleration: Correct (1 mark)
- Derivation of \(v = \sqrt{\mu_s g R}\): Correct (1 mark)
  
**Total for part a):** 2 marks

**Part b):**
- Normal force providing centripetal force: Incorrect derivation (0 marks)
- Normal force balances weight: Not correctly applied (0 marks)
- Final expression for \(v\): Incorrect (\(v = \sqrt{gR\sin(\phi)}\) instead of \(v = \sqrt{Rg\tan(\phi)}\)) (0 marks)
  
**Total for part b):** 0 marks

**Part c):**
- Largest \(v\) analysis: Incorrect (0 marks)
- Balance vertical forces: Not correctly approached (0 marks)
- Expressions for \(v_{max}\) and \(v_{min}\): Incorrect derivations and final result (0 marks)
- Calculation of \(v_{max}^2 - v_{min}^2\): Incorrect, did not follow through to the required result (0 marks)
  
**Total for part c):** 0 marks

**Overall feedback for Solution 1:** The solution correctly addresses part a) with a clear understanding of the frictional and centripetal forces. However, parts b) and c) have significant inaccuracies, particularly in the application of forces on an inclined plane and the misinterpretation of friction's role, leading to incorrect expressions for \(v\) in different scenarios. A deeper understanding of the physics involved in inclined plane motion and careful attention to the derivation process would improve the solution.

**Total marks for Solution 1:** 2 out of 16

---

### Solution 2 Grading:

**Part a):**
- Identification of frictional force and centripetal acceleration: Correct (1 mark)
- Derivation of \(v = \sqrt{\mu_s g R}\): Correct (1 mark)
  
**Total for part a):** 2 marks

**Part b):**
- Normal force providing centripetal force: Incorrect assumption about gravitational force component (0 marks)
- Normal force balances weight: Incorrect, no direct derivation relating to \(v\) (0 marks)
- Final expression for \(v\): Incorrect (derived \(v = \sqrt{Rg\cos(\phi)}\), which is wrong) (0 marks)
  
**Total for part b):** 0 marks

**Part c):**
- Largest \(v\) analysis: Partially correct approach but with an incorrect final expression (1 mark for attempt)
- Balance vertical forces: Incorrect approach, did not lead to correct derivation (0 marks)
- Expressions for \(v_{max}\) and \(v_{min}\): Incorrect derivations, did not align with mark scheme (0 marks)
- Calculation of \(v_{max}^2 - v_{min}^2\): Incorrect, did not reach the formula as required (0 marks)
  
**Total for part c):** 1 mark

**Overall feedback for Solution 2:** The student shows a good understanding of basic principles but fails to apply them correctly in parts b) and c). Misinterpretation of forces on inclined planes and incorrect expressions for \(v_{max}\) and \(v_{min}\) lead to inaccuracies. The solution could be significantly improved by correctly analyzing the forces involved and accurately deriving the expressions for velocities.

**Total marks for Solution 2:** 3 out of 16

---

### Solution 3 Grading:

**Part a):**
- Identification of frictional force and centripetal acceleration: Correct (1 mark)
- Derivation of \(v = \sqrt{\mu_s g R}\): Correct (1 mark)
  
**Total for part a):** 2 marks

**Part b):**
- Normal force providing centripetal force: Incorrect assumption, no clear derivation (0 marks)
- Normal force balances weight: Not correctly applied to derive \(v\) (0 marks)
- Final expression for \(v\): Incorrect (\(v = \sqrt{g R \sin(\phi)}\) instead of correct formula) (0 marks)
  
**Total for part b):** 0 marks

**Part c):**
- Largest \(v\) analysis: Incorrect approach, did not match required expressions (0 marks)
- Balance vertical forces: Not correctly done, incorrect assumptions (0 marks)
- Expressions for \(v_{max}\) and \(v_{min}\): Incorrect, did not align with provided mark scheme (0 marks)
- Calculation of \(v_{max}^2 - v_{min}^2\): Incorrect, final result did not match the required expression (0 marks)
  
**Total for part c):** 0 marks

**Overall feedback for Solution 3:** The solution demonstrates an understanding of the principles behind circular motion and frictional forces but struggles with the application of these concepts to more complex scenarios involving inclined planes. Incorrect assumptions and derivations in parts b) and c) lead to significant inaccuracies. Focusing on the correct application of physical laws and careful derivation would greatly improve the solution's accuracy.

**Total marks for Solution 3:** 2 out of 16

---

In summary, all solutions demonstrated some understanding of basic physics principles, particularly for part a). However, significant misunderstandings and inaccuracies in applying these principles to scenarios involving inclined planes and friction led to lower scores for parts b) and c). Focused improvements on these areas are recommended for all students."

\textbf{Response 2:}

"\textbf{Solution 1:}

\textbf{Part a)} This solution correctly identifies the relationship between the frictional force and the centripetal force and derives the correct formula for the maximum speed. \textbf{2/2 marks}

\textbf{Part b)} This solution incorrectly identifies the gravitational force component providing the centripetal force. It states that the component perpendicular to the incline (which it incorrectly identifies as \( mg\cos(\phi) \)) does not contribute in a frictionless scenario, which is a misunderstanding. The correct component is \( mg\sin(\phi) \). However, it attempts to derive a formula based on an incorrect understanding. \textbf{0/3 marks}

\textbf{Part c)} The derivation for \( v_{max} \) and \( v_{min} \) starts correctly, but the final expression for the difference \( v_{max}^2 - v_{min}^2 \) does not match the given equation. The student's final expression \( 2Rg\mu_s\sin(2\phi) \) is incorrect, and there is a misunderstanding in the application of forces. \textbf{0/11 marks}

\textbf{Total: 2/16 marks}

\textbf{Feedback:} The student showed understanding in the context of a flat circular track but misunderstood the forces involved in the inclined track scenarios. It's crucial to accurately identify the components of forces acting on the object and correctly apply them to derive the expressions for maximum and minimum speeds.

\textbf{Solution 2:}

\textbf{Part a)} Correctly identifies the frictional force and its relation to the centripetal force, leading to the correct formula for the largest speed. \textbf{2/2 marks}

\textbf{Part b)} The solution makes a mistake in identifying the gravitational force component contributing to centripetal force on a frictionless incline, incorrectly concluding with \( \sqrt{Rg\cos(\phi)} \). \textbf{0/3 marks}

\textbf{Part c)} Attempts to derive expressions for \( v_{max} \) and \( v_{min} \) and their difference but fails to align with the correct formula or understanding of the forces involved, leading to incorrect conclusions. \textbf{0/11 marks}

\textbf{Total: 2/16 marks}

\textbf{Feedback:} The solution starts well with the flat track scenario but fails to accurately handle the physics of the inclined track, especially in the absence and presence of friction. Understanding the correct components of gravitational force and how they contribute to centripetal force is crucial.

\textbf{Solution 3:}

\textbf{Part a)} This solution correctly derives the formula for the largest speed on a flat track, identifying the necessary forces correctly. \textbf{2/2 marks}

\textbf{Part b)} The student correctly identifies the component of gravitational force contributing to centripetal force on an inclined frictionless track but makes a mistake in the final formula, incorrectly substituting \( \cos \) instead of \( \sin \). \textbf{1/3 marks} for recognizing the correct component but erring in the application.

\textbf{Part c)} Although the initial setup for deriving \( v_{max} \) and \( v_{min} \) is correct, the final expression and the attempt to reach the provided equality are incorrect. There's a clear effort to address the problem's requirements, but the final result does not align with the given equation. \textbf{2/11 marks} for the correct approach but incorrect final expression.

\textbf{Total: 5/16 marks}

\textbf{Feedback:} The student demonstrates a good grasp of the problem's fundamentals but needs to ensure the accuracy of their mathematical derivations, especially in complex scenarios involving inclined planes with friction. It's crucial to carefully verify the final expressions against the problem statement."

\textbf{Response 3:}

"Solution 1 Grading:

\begin{itemize}
    \item Part a) [2 marks]
    \begin{itemize}
        \item Correct identification and application of frictional force and centripetal acceleration.
        \item Correct final expression for \(v\).
    \end{itemize}
    \item Part b) [1 mark]
    \begin{itemize}
        \item Incorrect gravitational force component used for centripetal force (used \( \sin \) instead of \( \tan \)).
    \end{itemize}
    \item Part c) [4 marks]
    \begin{itemize}
        \item Incorrect attempt at deriving \( v_{max} \) and \( v_{min} \).
        \item The final equality attempt does not match the mark scheme, showing a fundamental misunderstanding of the forces involved.
    \end{itemize}
\end{itemize}

\textbf{General Feedback for Solution 1:}
The approach in parts a and c demonstrates a basic understanding of the principles involved in circular motion and the effects of friction. However, there's a critical error in part b regarding the gravitational force component contributing to centripetal force, and a significant misunderstanding in part c about how friction and gravity interact on an inclined plane. The error in part b suggests a misinterpretation of how forces contribute to circular motion in different contexts. In part c, the calculation does not lead to the correct formula, indicating a need for a deeper analysis of the forces at play and a more precise application of physics principles to circular motion on inclined planes.

Solution 2 Grading:

\begin{itemize}
    \item Part a) [2 marks]
    \begin{itemize}
        \item Correctly identifies the forces and derives the correct expression for \(v\).
    \end{itemize}
    \item Part b) [0 marks]
    \begin{itemize}
        \item Incorrect application of gravitational force component for centripetal force (used \( \cos \) instead of \( \tan \)).
    \end{itemize}
    \item Part c) [4 marks]
    \begin{itemize}
        \item Attempts to address the forces involved but ultimately arrives at an incorrect final expression for \( v_{max}^2 - v_{min}^2 \) and does not follow the mark scheme correctly.
    \end{itemize}
\end{itemize}

\textbf{General Feedback for Solution 2:}
The solution correctly handles the physics concepts in part a but falters in part b with an incorrect force component for deriving the speed. Part c shows an effort to integrate friction and gravity in determining speeds on an incline, but the solution fails to arrive at the correct formula, demonstrating a misunderstanding of how to correctly apply these forces to the problem of circular motion on an inclined plane. The derivation and application of physics principles need to be more precise and aligned with the given conditions, especially in complex scenarios involving multiple forces.

Solution 3 Grading:

\begin{itemize}
    \item Part a) [2 marks]
    \begin{itemize}
        \item Correct application of forces and derivation of the expression for \(v\).
    \end{itemize}
    \item Part b) [1 mark]
    \begin{itemize}
        \item Incorrect conclusion about the component of gravity contributing to centripetal force on a frictionless inclined plane (mixed up force components).
    \end{itemize}
    \item Part c) [3 marks]
    \begin{itemize}
        \item Incomplete and incorrect derivation towards the final expression. The provided work does not accurately reach the provided formula, indicating a misunderstanding of the problem setup and the physical dynamics involved.
    \end{itemize}
\end{itemize}

\textbf{General Feedback for Solution 3:}
Solution 3 starts strong with a correct analysis in part a but makes significant errors in parts b and c. Part b misinterprets the role of gravity's components in circular motion on an inclined plane. In part c, while there is an attempt to include both friction and gravity's roles, the derivation deviates significantly from the correct process and outcome outlined in the mark scheme. This indicates a need for a clearer understanding of circular motion dynamics, particularly in scenarios involving inclines and friction, and a more careful application of physics principles to ensure accurate results."

\subsection*{Question 7:}

\textbf{Response 1:}

"To evaluate each solution, I will grade them based on the provided mark scheme, assigning marks for parts a), b), and c) respectively, and then provide general feedback.

Solution 1 Evaluation
Part a):

Energy-momentum 4-vector correctly identified, but incomplete specification of components and a minor error in the scalar product calculation. (2 marks)
Scalar product calculation demonstrates invariance but has a sign error. (1 mark)
Part b):

Correct approach to obtaining 4-momenta in lab frame, but the method is slightly confused with a mix of correct and incorrect formulas. (3 marks)
Correct understanding of Lorentz transformation but erroneous application and details. (0 marks)
Part c):

Incorrect wavelengths calculation and misunderstanding of how to transform energies to wavelengths. (0 marks)
Total: 6/13

Feedback: You've demonstrated a grasp of the concepts but need to review the mathematical details, particularly in calculating the scalar product and transforming energies to wavelengths. Ensure accuracy in the application of formulas and clarify your explanations for better understanding.

Solution 2 Evaluation
Part a):

Correct identification and description of the energy-momentum 4-vector. (2 marks)
Accurate calculation of the scalar product and explanation of Lorentz invariance, despite minor errors in interpretation. (1 mark)
Part b):

Correct identification of 4-momenta in the rest frame and understanding of Lorentz transformation. (3 marks)
Slight error in the transformation equations leading to incorrect final expressions. (1 mark)
Part c):

Attempt to calculate wavelengths but incorrect final values and misunderstanding of relativistic effects. (1 mark)
Total: 8/13

Feedback: Your approach shows good comprehension of relativistic physics principles but suffers from inaccuracies in execution. Revisit Lorentz transformations and ensure correct use of relativistic formulas, particularly when calculating photon wavelengths.

Solution 3 Evaluation
Part a):

Correct formulation of energy-momentum 4-vector and Lorentz factor. (2 marks)
Invariant scalar product correctly identified, but computation contains errors. (1 mark)
Part b):

Appropriate approach to deriving 4-momenta in the rest and lab frames, but errors in transformation application. (3 marks)
Correct concept but errors in mathematical details. (1 mark)
Part c):

Incorrect calculation of wavelengths, with a significant misunderstanding of energy to wavelength transformation. (0 marks)
Total: 7/13

Feedback: You've demonstrated a foundational understanding of the topic, but your execution, particularly in part c, needs refinement. Focus on accurately applying the Lorentz transformation and correctly converting energy to wavelength. Also, ensure mathematical calculations are correct and consistent with physical principles.

These evaluations consider the clarity of explanation, mathematical accuracy, and adherence to the mark scheme. Each solution shows an understanding of relativistic physics concepts but varies in mathematical precision and completeness. It's essential for students to review the correct application of Lorentz transformations and the energy-wavelength relationship, especially in relativistic contexts."

\textbf{Response 2:}

Here are the grades and feedback for each solution:

"Solution 1

Part A: 3/4. The definition and relation between energy, momentum, and mass are correctly identified, but the choice of axis for velocity simplification is not explicitly mentioned.

Part B: 4/5. The approach to four-momenta and the transformation to the lab frame is correctly described, although minor details in the explanation could be clearer for full marks.

Part C: 3/4. The method for determining the wavelengths is correct, but there are slight inaccuracies in the final values or concepts.

Total: 10/13.

General Feedback: A well-structured solution with a solid understanding of relativistic physics. Improvements could be made by addressing the minor inaccuracies and providing clearer explanations in certain areas.

Solution 2

Part A: 3/4. The approach is correct, but the explanation of the scalar product calculation is slightly imprecise.

Part B: 4/5. The four-momenta description and transformation to the lab frame are accurate. However, the explicit mention of applying the Lorentz transformation matrix is missing.

Part C: 3/4. The approach to finding the wavelengths is reasonable, but there are inaccuracies in the final calculations or conceptual understanding.

Total: 10/13.

General Feedback: This solution demonstrates a good grasp of the concepts with accurate calculations. Attention to detail in explaining the Lorentz transformation and correcting minor inaccuracies could enhance the solution.

Solution 3

Part A: 2/4. The scalar product calculation is incomplete, and the simplification through the choice of axis is not mentioned.

Part B: 2/5. While the general idea is correct, the solution lacks clarity and precision in the application of Lorentz transformations.

Part C: 1/4. The energy calculations leading to the determination of wavelengths are incorrect.
Total: 5/13.

General Feedback: This solution needs significant improvement in both the clarity of explanations and the accuracy of calculations. A more detailed understanding of Lorentz transformations and correct energy-to-wavelength conversions is necessary.

For all solutions, focus on clear, precise explanations, accurate calculations, and adherence to the problem's requirements will help in achieving full marks."

\textbf{Response 3:}

"Let's grade each solution based on the provided mark scheme:

\textbf{Solution 1:}
\begin{itemize}
    \item \textbf{a) 3 marks:} The explanation of the energy-momentum 4-vector and its Lorentz invariance is accurate but lacks the explicit choice of the axis for \( \underline{v} \), missing out on the 1 mark dedicated to this detail. The scalar product calculation is correct, fulfilling the criteria for 2 marks.
    \item \textbf{b) 4 marks:} This solution accurately describes the four-momenta of decay particles in both the rest and lab frames. However, it slightly deviates from the mark scheme by not explicitly using the exact transformation equations provided in the mark scheme, which could affect clarity.
    \item \textbf{c) 3 marks:} The calculation of wavelengths in the rest and lab frames is done using an implied method, but precise numerical results matching the mark scheme are not provided. The explanation suggests understanding but lacks the clear calculation and final values expected.
\end{itemize}
\textbf{Feedback:} Well-explained with a strong understanding of Lorentz transformations and relativistic effects. Ensure all steps, especially calculations and specific details outlined in the mark scheme, are clearly shown for full marks.

\textbf{Solution 2:}
\begin{itemize}
    \item \textbf{a) 3 marks:} The solution correctly identifies and explains the energy-momentum 4-vector and its Lorentz invariance but does not specifically choose an axis for \( \underline{v} \), missing the specific mark for this requirement. The scalar product and its invariance under Lorentz transformation are correctly shown, fulfilling 2 marks for that section.
    \item \textbf{b) 4 marks:} Accurate explanation of the four-momenta of decay particles and their transformation to the lab frame, though it somewhat rephrases the expected answer and lacks the precise use of the provided inverse Lorentz transformation equations, narrowly missing full marks.
    \item \textbf{c) 3 marks:} While the solution demonstrates an understanding of the problem and attempts to calculate the wavelengths of the photons, it does not provide the exact wavelengths expected per the mark scheme or explain the calculation process in sufficient detail.
\end{itemize}
\textbf{Feedback:} Your understanding of relativistic physics principles is commendable. For improvement, ensure your answers closely follow the expected methodology and provide all required calculations and unit details for full marks.

\textbf{Solution 3:}
\begin{itemize}
    \item \textbf{a) 3 marks:} Similar to the others, it correctly discusses the energy-momentum 4-vector and Lorentz invariance. However, it misses specifying \( \underline{v} \) along an axis, which was needed for full marks. The scalar product calculation is covered well, justifying 2 marks.
    \item \textbf{b) 4 marks:} The explanation covers the essentials of four-momenta transformation but does not follow the mark scheme's instructions verbatim for using and showing the inverse Lorentz transformation calculations as expected.
    \item \textbf{c) 2 marks:} This solution fails to provide accurate wavelengths in alignment with the mark scheme. The values mentioned significantly deviate from those expected, indicating a misunderstanding or calculation error in applying the provided equations and constants.
\end{itemize}
\textbf{Feedback:} You've shown a good grasp of the core concepts but need to ensure that your calculations are accurate and follow the provided instructions more closely. Pay careful attention to detail, especially in applying equations correctly for exact results.

These grades and feedback reflect each solution's adherence to the mark scheme and their clarity, accuracy, and completeness in addressing the problem's parts."

\subsection*{Question 8:}

\textbf{Response 1:}

"Grading each solution according to the provided mark scheme:

\textbf{Solution 1:}

a) Derivation of the equation of motion:
\begin{itemize}
  \item Position of mass: Incorrect form for potential energy [0 marks].
  \item Calculation of kinetic and potential energy to give Lagrangian: Incorrect form for potential energy and incomplete kinetic energy expression [1 mark].
  \item Obtain and simplify Euler-Lagrange equations: They have calculated some parts of the Euler-Lagrange equation but have not completed it. The final form of the equation of motion is not provided [1 mark].
  \item Compute \( \ddot{x} \) and substitute to obtain equation of motion: They computed \( \ddot{x} \) but did not correctly substitute into the equation of motion [1 mark].
\end{itemize}

b) Limit of \( v, A \rightarrow 0 \) and small angles:
\begin{itemize}
  \item Small angle approximation correctly applied, and correct final form for the equation of motion [1 mark].
\end{itemize}

\textbf{Total: 4 marks}

\textbf{Feedback:}
The potential energy formula is incorrect, which affected the derivation of the Lagrangian. The kinetic energy term was also incomplete. You started well with the Euler-Lagrange equation but did not complete it to reach the final form of the equation of motion. The small angle approximation in part b) was correct.

\textbf{Solution 2:}

a) Derivation of the equation of motion:
\begin{itemize}
  \item Position of mass: Correct [1 mark].
  \item Calculation of kinetic and potential energy to give Lagrangian: Only tangential kinetic energy considered and potential energy expression is incorrect [1 mark].
  \item Obtain and simplify Euler-Lagrange equations: Correct steps were taken, but the radial kinetic energy was neglected, which is a significant error [1 mark].
  \item Compute \( \ddot{x} \) and substitute to obtain equation of motion: Correct computation of \( \ddot{x} \) [2 marks].
\end{itemize}

b) Limit of \( v, A \rightarrow 0 \) and small angles:
\begin{itemize}
  \item Correctly identified that the term \( \ddot{x} \) vanishes and applied the small angle approximation [1 mark].
\end{itemize}

\textbf{Total: 6 marks}

\textbf{Feedback:}
You correctly identified the position of the mass and applied the Euler-Lagrange equation correctly. However, the total kinetic energy was incomplete because you did not consider the radial component due to the support movement, which led to an incorrect Lagrangian and equation of motion. You handled the small angle approximation correctly.

\textbf{Solution 3:}

a) Derivation of the equation of motion:
\begin{itemize}
  \item Position of mass: Correct expression for position [1 mark].
  \item Calculation of kinetic and potential energy to give Lagrangian: Correct form for kinetic energy, but incorrect for potential energy [2 marks].
  \item Obtain and simplify Euler-Lagrange equations: Correct derivation of Euler-Lagrange equation but incorrect final equation of motion [2 marks].
  \item Compute \( \ddot{x} \) and substitute to obtain equation of motion: Correct computation of \( \ddot{x} \), but incorrect final form for equation of motion due to potential energy error [1 mark].
\end{itemize}

b) Limit of \( v, A \rightarrow 0 \) and small angles:
\begin{itemize}
  \item Correct application of the small angle approximation and reduction to the classical pendulum problem [1 mark].
\end{itemize}

\textbf{Total: 7 marks}

\textbf{Feedback:}
You have a solid understanding of the kinetic energy and Euler-Lagrange equations but made a critical error in the potential energy expression. The computed \( \ddot{x} \) was correct, but the error in potential energy carried through, affecting the final equation of motion. The small angle approximation in part b) was handled well."

\textbf{Response 2:}

"To assess the provided solutions according to the mark scheme, I will evaluate each step mentioned in the mark scheme and assign marks accordingly. Here is the evaluation for each solution:

Solution 1:

\textbf{a)}
\begin{itemize}
    \item The position of mass is mentioned but not in the exact form required by the mark scheme. (0/1)
    \item The calculation of kinetic and potential energy is attempted but has errors. The potential energy term used is incorrect; it should be \( -mgL \cos(\phi) \), not \( mgL(1 - \cos(\phi)) \). The kinetic energy term is also incorrectly calculated, and there is no clear Lagrangian expression. (0/3)
    \item The Euler-Lagrange equations are attempted, but the computation is incomplete and partially incorrect. The term involving \( \dot{x}L\dot{\phi}\cos(\phi) \) is missing in their final expression. (0/3)
    \item The computation of \( \ddot{x} \) is not shown, and the substitution to obtain the equation of motion is incorrect. (0/2)
\end{itemize}

\textbf{b)}
\begin{itemize}
    \item The second term in the equation of motion is correctly identified as vanishing, and the small-angle approximation is correctly applied. (1/1)
\end{itemize}

Total marks for Solution 1: 1/10

Solution 2:

\textbf{a)}
\begin{itemize}
    \item The position of mass is not explicitly written down. (0/1)
    \item The calculation of kinetic and potential energy is partially correct, but the potential energy is not formulated correctly, and the kinetic energy does not include the cross-term involving \( \dot{x}L\dot{\phi}\cos(\phi) \). (1/3)
    \item The Euler-Lagrange equations are correctly obtained but not simplified, and the equation of motion is derived incorrectly as it does not include the cross-term or the correct expression for potential energy. (1/3)
    \item \( \ddot{x} \) is not correctly computed, and the substitution into the equation of motion is not shown. (0/2)
\end{itemize}

\textbf{b)}
\begin{itemize}
    \item The limit of \( v, A \rightarrow 0 \) is correctly considered, and the small-angle approximation is correctly applied, resulting in the correct classic pendulum equation. (1/1)
\end{itemize}

Total marks for Solution 2: 3/10

Solution 3:

\textbf{a)}
\begin{itemize}
    \item The position of mass is again, not explicitly given in the required form. (0/1)
    \item The kinetic energy calculation includes an incorrect term for the potential energy, and the kinetic energy is calculated without the cross-term that should be present. (1/3)
    \item The Euler-Lagrange equation is not correctly simplified, and the equation of motion is incorrectly derived. The term \( m\dot{x}L\dot{\phi}\sin(\phi) \) is not present in their final expression. (1/3)
    \item The computation of \( \ddot{x} \) and its substitution are not shown, resulting in an incomplete equation of motion. (0/2)
\end{itemize}

\textbf{b)}
\begin{itemize}
    \item The simplification for small angles and vanishing \( v \) and \( A \) is correctly applied, and the resulting equation is correct for the classic pendulum problem. (1/1)
\end{itemize}

Total marks for Solution 3: 3/10

General feedback for all solutions:

All three solutions struggled with the correct expressions for the kinetic and potential energies in the context of the Lagrangian formulation. This led to incorrect or incomplete Lagrangians and consequently incorrect equations of motion. Attention to detail in applying the Euler-Lagrange equations and in simplifying to the classic pendulum problem is crucial. It is important to ensure that all terms are accounted for in the kinetic and potential energies, and that the simplification to the classical problem is carefully justified. Moreover, none of the solutions explicitly used the given expressions for the Lagrangian and the Euler-Lagrange equation, which were expected to be used directly according to the mark scheme."

\textbf{Response 3:}

"Here are the grades and feedback for each solution based on the provided mark scheme:

Solution 1:

\textbf{a) Deriving the equation of motion for \( \phi \):}
\begin{itemize}
    \item Correct position of mass and its differentiation are given, but the kinetic energy expression is incorrect due to algebraic mistakes. The potential energy is also incorrectly written with a factor \(1 - \cos(\phi)\) instead of \(-\cos(\phi)\). Incorrect or missing partial derivatives lead to an incorrect Lagrangian and, consequently, an incorrect equation of motion. Therefore, only partial credit can be given for recognizing the correct position and attempting the differentiation.
    \item[2 marks]
\end{itemize}

\textbf{b) In the limit of small oscillations and approaching 0 for \( v \) and \( A \):}
\begin{itemize}
    \item The student correctly applies the small angle approximation and arrives at the correct classical equation of motion for the pendulum.
    \item[1 mark]
\end{itemize}

\textbf{Total for Solution 1: 3 marks}

\textbf{General Feedback for Solution 1:}
The student started well by recognizing the correct position of the mass but made algebraic mistakes in the kinetic energy expression. The potential energy expression also contained errors. Additionally, the partial derivatives were incorrect, leading to a wrong Lagrangian and an incorrect equation of motion. The student should review the steps to ensure correct algebraic manipulation and application of the Euler-Lagrange equation. The small-angle approximation part was handled correctly.

Solution 2:

\textbf{a) Deriving the equation of motion for \( \phi \):}
\begin{itemize}
    \item The student correctly writes the kinetic energy components and the potential energy. However, the radial kinetic energy should include the mixed term involving \( \dot{x} \) and \( \dot{\phi} \). The equation derived is close to correct but omits the cross-term in the kinetic energy. For the Lagrangian and the Euler-Lagrange equation, the student does not fully calculate the derivatives required to obtain the equation of motion.
    \item[5 marks]
\end{itemize}

\textbf{b) In the limit of small oscillations and approaching 0 for \( v \) and \( A \):}
\begin{itemize}
    \item Correct application of the small-angle approximation and limits for \( v \) and \( A \) to simplify the equation of motion to the classic pendulum problem.
    \item[1 mark]
\end{itemize}

\textbf{Total for Solution 2: 6 marks}

\textbf{General Feedback for Solution 2:}
The student has the right approach and correctly identifies the kinetic and potential energy forms. However, they fail to include the cross-term in the kinetic energy which is essential for a complete Lagrangian. The student also does not calculate the full derivatives needed for the Euler-Lagrange equation, which is necessary for the full equation of motion. In part b, the student successfully simplifies the equation to the classic pendulum equation. More attention to detail in part a would improve the solution.

Solution 3:

\textbf{a) Deriving the equation of motion for \( \phi \):}
\begin{itemize}
    \item The student identifies the correct form of the kinetic and potential energy and writes the correct Lagrangian. However, they do not provide the full calculations for the partial derivatives and their time derivatives needed to apply the Euler-Lagrange equation correctly. Thus, while the approach is correct, the lack of explicit calculation leads to incomplete solutions.
    \item[5 marks]
\end{itemize}

\textbf{b) In the limit of small oscillations and approaching 0 for \( v \) and \( A \):}
\begin{itemize}
    \item Correctly applies the small angle approximation and the limits of \( v \) and \( A \) to obtain the classic equation of motion for the pendulum.
    \item[1 mark]
\end{itemize}

\textbf{Total for Solution 3: 6 marks}

\textbf{General Feedback for Solution 3:}
The student has a good grasp of the Lagrangian mechanics and correctly writes down the Lagrangian. However, the student fails to provide detailed calculations for the Euler-Lagrange equation, which leads to an incomplete solution for the equation of motion. In part b, they correctly simplify the equation to the classic pendulum problem. For future problems, the student should focus on fully carrying out the necessary derivatives for the Euler-Lagrange equation."

\subsection*{Question 9:}

\textbf{Response 1:}

"Evaluation of Student Solutions

Let's evaluate each student's solution according to the mark scheme provided.

\textbf{Solution 1:}

\textit{Part a)}
\begin{itemize}
    \item Correctly applies the torque equation and Newton's second law for rotation [2 marks].
    \item Properly states the relationship between the linear and angular acceleration due to the no-slip condition [1 mark].
    \item Attempts to solve the simultaneous equations, but makes an error in relating \( a \) and \( \alpha \). The final expressions for \( T \), \( a \), and \( \alpha \) are incorrect because they have not combined the equations correctly [0 marks].
\end{itemize}

Total for part a): 3 marks.

\textit{Part b)}
\begin{itemize}
    \item Correct method for substituting values, but since the expressions from part a) are incorrect, the final numerical answers for \( T \), \( a \), and \( \alpha \) will also be incorrect [0 marks].
\end{itemize}

Total for part b): 0 marks.

\textbf{Overall feedback:} The student has the right approach but made a mistake in the algebraic manipulation of the equations. They should review how to solve the simultaneous equations properly to ensure they obtain the correct expressions for \( T \), \( a \), and \( \alpha \).

\textbf{Total marks for Solution 1:} 3 marks.

\textbf{Solution 2:}

\textit{Part a)}
\begin{itemize}
    \item Correctly applies Newton's second law for both linear and rotational motion [2 marks].
    \item Correctly identifies the no-slip condition between linear and angular acceleration [1 mark].
    \item Correctly solves the simultaneous equations and provides the correct expressions for \( T \), \( a \), and \( \alpha \) [3 marks].
\end{itemize}

Total for part a): 6 marks.

\textit{Part b)}
\begin{itemize}
    \item Correctly substitutes the values for \( m \), \( M \), \( R \), and \( g \) into their correct expressions, resulting in the correct numerical answers [1 mark].
\end{itemize}

Total for part b): 1 mark.

\textbf{Overall feedback:} The student has correctly applied the principles of dynamics to solve for the expressions and then accurately calculated the specific values. The solution is complete and follows the mark scheme appropriately.

\textbf{Total marks for Solution 2:} 7 marks.

\textbf{Solution 3:}

\textit{Part a)}
\begin{itemize}
    \item The initial setup for the torque and force equations is correct [2 marks].
    \item Properly identifies the relationship between \( a \) and \( \alpha \) [1 mark].
    \item Makes an algebraic error in the simultaneous equations, incorrectly solving for \( T \) and \( a \). The expressions for \( T \), \( a \), and \( \alpha \) are incorrect because of this algebraic mistake [0 marks].
\end{itemize}

Total for part a): 3 marks.

\textit{Part b)}
\begin{itemize}
    \item Since the expressions derived in part a) are incorrect, any numerical results based on these would also be incorrect. However, the method of substituting the values is correct [0 marks].
\end{itemize}

Total for part b): 0 marks.

\textbf{Overall feedback:} The student has a good understanding of the concepts but needs to focus on accurate algebraic manipulation to derive the correct expressions. Encourage the student to carefully check their algebraic work to avoid such errors.

\textbf{Total marks for Solution 3:} 3 marks."

\textbf{Response 2:}

"Grading Solutions

Grading Solution 1:

a) (4/6)
\begin{itemize}
    \item[1/2] Apply laws of motion for acceleration of block of mass $m$: The student applies Newton's second law correctly but does not consider the rope's constraint directly in the equations.
    \item[1/2] Torque on the circular disk: Correct approach for the torque due to tension.
    \item[0/1] Constraint that acceleration of block equals acceleration of disk: The constraint was used but not explicitly mentioned as part of the derivation.
    \item[2/3] Solve simultaneous equations to give expressions for $T$, $a$, and $ \alpha $: The student found expressions for $T$, $a$, and $ \alpha $, but there is an algebraic error. The correct form of $T$ should be $T = \frac{mgM}{M + 2m}$, but the student simplified incorrectly to $T = \frac{mgM}{M + 2m} \cdot R$ which is dimensionally inconsistent.
\end{itemize}

b) (0/1)
\begin{itemize}
    \item[0/1] Substitute values: The student did not substitute the correct derived expressions into the equations to find the actual values for $a$, $T$, and $ \alpha $.
\end{itemize}

\textbf{Total: 4/7}

Feedback: You applied the laws of motion and the rotational equivalent correctly in part a, but you did not explicitly state the non-slip condition of the rope as a constraint in your equations. Additionally, you made an algebraic error when simplifying your expression for the tension in the rope. For part b, you should ensure to substitute the correct derived expressions into the equations to find the actual values for $a$, $T$, and $ \alpha $.

Grading Solution 2:

a) (6/6)
\begin{itemize}
    \item[2/2] Apply laws of motion for acceleration of block of mass $m$, torque on the circular disk: The student correctly applies Newton's second law for both rotation and linear motion.
    \item[1/1] Constraint that acceleration of block equals acceleration of disk: The non-slip condition of the rope is correctly used to relate linear and angular acceleration.
    \item[3/3] Solve simultaneous equations to give expressions for $T$, $a$, and $ \alpha $: The student successfully derives the correct expressions for $T$, $a$, and $ \alpha $.
\end{itemize}

b) (1/1)
\begin{itemize}
    \item[1/1] Substitute values: The student correctly substitutes the values and calculates the correct numerical values for $a$, $T$, and $ \alpha $.
\end{itemize}

\textbf{Total: 7/7}

Feedback: Excellent work. Your methodical application of Newton's second laws for both linear and rotational motion is commendable. The explicit mention of the non-slip condition and the correct derivation of the simultaneous equations led to accurate expressions for $T$, $a$, and $ \alpha $. Well done on correctly substituting the values in part b to find the actual values.

Grading Solution 3:

a) (5/6)
\begin{itemize}
    \item[2/2] Apply laws of motion for acceleration of block of mass $m$, torque on the circular disk: The student correctly applies Newton's second law for rotation.
    \item[1/1] Constraint that acceleration of block equals acceleration of disk: The non-slip condition of the rope is used appropriately to relate linear and angular accelerations.
    \item[2/3] Solve simultaneous equations to give expressions for $T$, $a$, and $ \alpha $: The expressions for $T$ and $a$ are correctly derived, but the student made a slight mistake in the final expression for $ \alpha $, where they did not cancel out $R$ correctly.
\end{itemize}

b) (1/1)
\begin{itemize}
    \item[1/1] Substitute values: The student correctly substitutes the values into the derived expressions and calculates the correct values for $a$, $T$, and $ \alpha $.
\end{itemize}

\textbf{Total: 6/7}

Feedback: You have done a good job in part a, correctly applying Newton's laws and acknowledging the non-slip condition between the rope and the disk. However, there is a minor error in the final expression for \(\alpha\) that should be addressed. In part b, you correctly used the derived expressions to calculate the numerical values, which is commendable."

\textbf{Response 3:}

"Let's grade each solution based on the mark scheme provided.

Solution 1:

a)

Correct application of laws of motion for block and torque for the disk: 2 marks

Correct relation between linear and angular acceleration: 0 marks (this was incorrectly derived)

Solving simultaneous equations: 0 marks (the final expressions are incorrect)
b)

Correct substitution of values: 0 marks (the values used in calculations are incorrect due to the wrong expressions derived in part a)

Total for Solution 1: 2 marks

General Feedback: The student correctly applied the laws of motion and torque equations, but the relationship between linear and angular acceleration was incorrectly derived, which led to incorrect final expressions. Additionally, the final calculations are incorrect due to the use of these wrong expressions.

Solution 2:

a)

Correct application of laws of motion for block and torque for the disk: 2 marks

Correct relation between linear and angular acceleration: 1 mark

Solving simultaneous equations: 3 marks (the final expressions are correct)
b)

Correct substitution of values: 1 mark (values substituted correctly and final numerical answers are reasonable)

Total for Solution 2: 7 marks

General Feedback: The student correctly applied the laws of motion and torque equations. They also correctly related the linear and angular acceleration and derived the correct final expressions. The numerical values were calculated correctly.

Solution 3:

a)

Correct application of laws of motion for block and torque for the disk: 2 marks

Correct relation between linear and angular acceleration: 1 mark

Solving simultaneous equations: 3 marks (the final expressions are correct)
b)

Correct substitution of values: 1 mark (values substituted correctly and final numerical answers are reasonable)

Total for Solution 3: 7 marks

General Feedback: The student correctly applied the laws of motion and torque equations, correctly related linear and angular acceleration, and derived the correct final expressions. The numerical values were also calculated correctly."

\subsection*{Question 10:}

\textbf{Response 1:}

"Solution 1 Evaluation
Part a)

Correctly applies Newton's second law and identifies the forces acting on the ball: 1 mark
Solves the differential equation with a general solution and identifies the steady state correctly: 3 marks
Correctly identifies the form of the solution, matching the equation given in the problem statement: 2 marks
Total for part a): 6 marks

Part b)

Correctly uses the initial condition to find the expression for \(\underline{A}\): 2 marks

Total for part b): 2 marks

Overall Grade: 8 marks

Feedback: The student has demonstrated a clear understanding of Newton's laws of motion and has effectively solved the differential equation. They have accurately applied initial conditions to find the constant vector \(\underline{A}\), aligning perfectly with the mark scheme.

Solution 2 Evaluation

Part a)

Correctly applies Newton's second law and identifies the forces: 1 mark

Approach to solving the differential equation is correct, but it lacks a detailed explanation of the solution process matching the mark scheme's requirement (does not explicitly show the general solution or the process of getting there): 2 marks (given for correct end result but incomplete process)

Correctly identifies the solution's form but does not explicitly demonstrate the balance of forces at terminal velocity as required for full marks: 1 mark

Total for part a): 4 marks

Part b)

Uses the initial condition correctly to find \(\underline{A}\): 2 marks

Total for part b): 2 marks

Overall Grade: 6 marks

Feedback: The student correctly identified and applied the principles required to solve the problem but lacked detailed explanation in part a), specifically in solving the differential equation and showing how terminal velocity is considered. The answer to part b) was correct.

Solution 3 Evaluation

Part a)

Correct application of Newton's second law, identifying all forces on the ball: 1 mark

Correctly formulates and solves the differential equation, but the explanation of verifying the solution is not exactly what's required by the mark scheme (the step of deriving the general solution form explicitly before applying conditions is assumed but not shown): 2 marks (for arriving at the correct form but not following the exact solving process outlined in the mark scheme) 

Identifies terminal velocity correctly but the explanation is slightly indirect compared to the mark scheme’s expectations: 1 mark

Total for part a): 4 marks

Part b)

Correct use of the initial condition to solve for \(\underline{A}\): 2 marks

Total for part b): 2 marks

Overall Grade: 6 marks

Feedback: The student has shown a good understanding of the problem and arrived at the correct solution. However, the process in part a) was not fully aligned with the expected steps, particularly in the explicit demonstration of solving the differential equation and applying terminal velocity conditions directly."

\textbf{Response 2:}

"Grading each solution according to the provided mark scheme:

\textbf{Solution 1:}

\textit{Part a)}

\begin{itemize}
    \item Correct application of Newton's second law and identification of forces: 1 mark.
    \item Solution of the differential equation with the correct general form: 3 marks.
    \item Correct use of the boundary condition to find the constant term: 2 marks.
\end{itemize}

\textbf{Total for part a):} 6 marks

\textit{Part b)}

\begin{itemize}
    \item Correct use of the initial boundary condition: 1 mark.
    \item Correct substitution to find \( \underline{A} \): 1 mark.
\end{itemize}

\textbf{Total for part b):} 2 marks

\textbf{Overall total:} 8 marks

\textbf{Feedback:} Solution 1 accurately applies Newton's second law, correctly solves the differential equation, and appropriately uses the boundary condition to find \( \underline{A} \). The solution is comprehensive and well-structured, clearly showing each step of the problem-solving process.

\textbf{Solution 2:}

\textit{Part a)}

\begin{itemize}
    \item Correctly applies Newton's second law and identifies the forces: 1 mark.
    \item Attempts to solve the differential equation but does not explicitly show the process of solving the differential equation in detail as required (e.g., separation of variables or integrating factor): 2 marks.
    \item Correct use of the terminal velocity condition but lacks clarity in the explanation of deriving \( \underline{B} \): 1 mark.
\end{itemize}

\textbf{Total for part a):} 4 marks

\textit{Part b)}

\begin{itemize}
    \item Correct use of initial conditions: 1 mark.
    \item Correctly finds \( \underline{A} \) but the explanation is somewhat implicit: 1 mark.
\end{itemize}

\textbf{Total for part b):} 2 marks

\textbf{Overall total:} 6 marks

\textbf{Feedback:} Solution 2 outlines the correct approach but lacks some detail in the explanation of solving the differential equation and applying the terminal velocity condition. A more explicit derivation and clearer reasoning would improve this solution.

\textbf{Solution 3:}

\textit{Part a)}

\begin{itemize}
    \item Correct application of Newton's second law and identification of forces: 1 mark.
    \item Accurately solves the differential equation showing an understanding of the method but without explicitly showing a step-by-step solution or the method used (e.g., separation of variables, integrating factor): 3 marks.
    \item Correct use of boundary conditions to find the constant term, but does not fully explain the logic behind choosing these conditions: 1 mark.
\end{itemize}

\textbf{Total for part a):} 5 marks

\textit{Part b)}

\begin{itemize}
    \item Uses initial boundary condition appropriately: 1 mark.
    \item Correct substitution to find \( \underline{A} \), but lacks detailed explanation: 1 mark.
\end{itemize}

\textbf{Total for part b):} 2 marks

\textbf{Overall total:} 7 marks

\textbf{Feedback:} Solution 3 provides a correct approach with some omissions in the detailed explanation of solving the differential equation and applying boundary conditions. More explicit steps in the derivation and a clearer explanation of the logic behind each step would make this solution more comprehensive."

\textbf{Response 3:}

"Solution 1 Evaluation

\textbf{Part a)}
\begin{itemize}
    \item Correct identification and application of Newton's second law: \textbf{1 mark}
    \item Solving the differential equation and noting the general solution form, but missing explicit step-by-step solving process (skips directly to general solution form): \textbf{2 marks} (out of 3 for solving DE)
    \item Correct use of boundary conditions to find the terminal velocity and determining \( \underline{B} = - \frac{mg}{b} \hat{\underline{k}} \): \textbf{2 marks}
\end{itemize}

\textbf{Total for part a):} 5 marks

\textbf{Part b)}
\begin{itemize}
    \item Correct use of initial conditions: \textbf{1 mark}
    \item Correctly solving for \( \underline{A} \): \textbf{1 mark}
\end{itemize}

\textbf{Total for part b):} 2 marks

\textbf{Overall Total:} 7 marks

\textbf{Feedback:} The solution correctly identifies the forces acting on the ball and applies Newton's second law effectively. The method for solving the differential equation is mentioned, but a step-by-step derivation could enhance the explanation. The application of initial conditions to find \( \underline{A} \) is well executed.

Solution 2 Evaluation

\textbf{Part a)}
\begin{itemize}
    \item Correct application of Newton's second law: \textbf{1 mark}
    \item Correctly solving the differential equation with a bit more explanation than Solution 1, but still lacks detailed steps of integration: \textbf{2 marks} (out of 3 for solving DE)
    \item Correct use of boundary conditions to identify terminal velocity and thus \( \underline{B} \): \textbf{2 marks}
\end{itemize}

\textbf{Total for part a):} 5 marks

\textbf{Part b)}
\begin{itemize}
    \item Correct use of initial conditions: \textbf{1 mark}
    \item Correctly solving for \( \underline{A} \): \textbf{1 mark}
\end{itemize}

\textbf{Total for part b):} 2 marks

\textbf{Overall Total:} 7 marks

\textbf{Feedback:} The solution adequately uses Newton's second law and arrives at the correct differential equation. The explanation for solving the DE could be improved with more detailed integration steps. The handling of initial conditions to solve for \( \underline{A} \) is correct and well-explained.

Solution 3 Evaluation

\textbf{Part a)}
\begin{itemize}
    \item Correct application of Newton's second law: \textbf{1 mark}
    \item Detailed explanation and correct solving of the differential equation: \textbf{3 marks}
    \item Accurate use of boundary conditions and identification of \( \underline{B} \): \textbf{2 marks}
\end{itemize}

\textbf{Total for part a):} 6 marks

\textbf{Part b)}
\begin{itemize}
    \item Correct application of initial conditions: \textbf{1 mark}
    \item Accurately solving for \( \underline{A} \): \textbf{1 mark}
\end{itemize}

\textbf{Total for part b):} 2 marks

\textbf{Overall Total:} 8 marks

\textbf{Feedback:} This solution provides a comprehensive explanation, correctly applying Newton's second law and solving the differential equation with an adequate level of detail. The method for identifying \( \underline{A} \) using initial conditions is correctly executed and clearly explained."

\end{document}

