\documentclass[a4paper,11pt]{article}
\setlength\parindent{0pt}
\usepackage{amsmath, amssymb}
\usepackage[margin=1in,letterpaper]
{geometry} % decreases margins

\usepackage{setspace}
\setstretch{1.2}
\usepackage{spverbatim}

\begin{document}
\title{GPT-4 Self-Graded Responses (No Markscheme)}
\author{Group 3A - Ryan Mok, Faraaz Akhtar, Banita Bamrah, \\ Fahim M A, Richang Xie, Kashan Ali }
\date{}
\maketitle

This document compiles the GPT-4 responses when prompted to grade the solutions which have been generated. No markscheme is given to GPT-4 in the prompt. This is the grading for the Classical Mechanics (CM) questions. 

\section{CM}

\subsection*{Question 1:}

\textbf{Response 1:}

"Solution 1

\textbf{Grade: 4}

Solution 1 provides a comprehensive and accurate description of conservative forces in classical mechanics, clearly explaining the independence of work done by a conservative force on the path taken and its reliance solely on the initial and final positions. The mathematical definition of potential energy as the negative gradient of potential energy with respect to a conservative force is correctly stated. Furthermore, the explanation of why potential energy cannot be defined for non-conservative forces---due to their path-dependent nature and the dissipation of energy in non-recoverable forms---is both precise and insightful. This solution covers all essential aspects of the problem statement with clarity and accuracy.

\textbf{Feedback:} Excellent work! You've provided a clear, thorough, and accurate explanation of conservative forces, their mathematical representation through potential energy, and the rationale behind the impossibility of defining potential energy for non-conservative forces. Your explanation is well-structured and effectively communicates the key concepts.

Solution 2

\textbf{Grade: 3}

Solution 2 effectively describes what a conservative force is and provides examples, which enhances understanding. The mathematical definition involving the work done and its relationship to potential energy change is accurately presented, including the essential equation \( \underline{F} = -\nabla PE \). However, the explanation of why potential energy cannot be defined for non-conservative forces, while correct, could benefit from a more detailed discussion on the implications of path dependency and energy dissipation in practical terms. This solution is accurate but slightly lacks the depth found in Solution 1 regarding the implications of non-conservative forces.

\textbf{Feedback:} Good job! Your explanation of conservative forces and the mathematical representation of potential energy is clear and accurate. To further improve, consider expanding on the practical implications of non-conservative forces' path dependency and how it affects the system's energy, offering a more comprehensive understanding of why potential energy cannot be defined for such forces.

Solution 3

\textbf{Grade: 4}

Solution 3 provides a detailed and accurate explanation of conservative forces, including the importance of the total mechanical energy remaining constant in the absence of non-conservative forces. The mathematical formulation, particularly the introduction of the work done concept and its relation to potential energy through differential forms, is both precise and insightful. The explanation of why a force must be conservative to define a unique potential energy function, including the condition that the curl of \( \underline{F} \) must be zero, is exceptionally well articulated. This solution excellently addresses the mathematical and conceptual aspects of conservative and non-conservative forces.

\textbf{Feedback:} Excellent work! Your detailed explanation of the criteria for conservative forces, including the mathematical derivation of potential energy and the condition for a force to be conservative, is outstanding. Your discussion on the impossibility of defining potential energy for non-conservative forces due to path dependency and energy conversion is comprehensive and enhances understanding of the core concepts."

\textbf{Response 2:}

"\textbf{Solution 1}

\textbf{Grade:} 4

\textbf{Feedback:} This solution provides a comprehensive explanation of conservative forces, correctly defines the mathematical relationship between a conservative force and potential energy, and succinctly explains why potential energy cannot be defined for non-conservative forces. The explanation is clear, accurate, and directly addresses all parts of the question. Excellent work.

\textbf{Solution 2}

\textbf{Grade:} 3

\textbf{Feedback:} This solution offers a good understanding of conservative forces, including examples and a correct mathematical definition of potential energy in relation to conservative forces. However, the explanation of why potential energy cannot be defined for non-conservative forces, while correct, could be improved by emphasizing the irreversible nature of energy dissipation in non-conservative forces, such as friction. Overall, a solid response with room for a bit more detail on the implications of non-conservative forces.

\textbf{Solution 3}

\textbf{Grade:} 4

\textbf{Feedback:} This response excellently outlines the concept of conservative forces, including a precise mathematical formulation for potential energy and the conditions that make a force conservative. The explanation of why potential energy cannot be defined for non-conservative forces is particularly well articulated, with an emphasis on the mathematical condition that a conservative force must meet \(\nabla \times \underline{F} = 0\). This solution thoroughly addresses all aspects of the question, demonstrating a deep understanding of the topic."

\textbf{Response 3:}

"Grading each solution based on the clarity, completeness, accuracy, and understanding demonstrated in their explanations:

Solution 1:

Grade: 4/4

Feedback: This solution provides a comprehensive and accurate definition of conservative forces, clearly explains the mathematical relationship between conservative forces and potential energy, and succinctly addresses why potential energy cannot be defined for non-conservative forces. The explanation is thorough, well-structured, and demonstrates a strong understanding of the principles involved.

Solution 2:

Grade: 3/4

Feedback: This solution accurately defines conservative forces and correctly explains the mathematical relationship between these forces and potential energy. However, it slightly misrepresents the definition of potential energy by emphasizing the work done between two points without explicitly stating the relationship between force and the negative gradient of potential energy as clearly as Solution 1. The explanation of why potential energy cannot be defined for non-conservative forces is correct but could benefit from a clearer link to the fundamental principles that differentiate conservative from non-conservative forces.

Solution 3:

Grade: 4/4

Feedback: This solution effectively combines a clear definition of conservative forces with a detailed mathematical explanation of potential energy in the context of conservative forces. It excellently highlights the importance of the curl of the force being zero as a condition for conservativeness, which adds depth to the explanation. The discussion about non-conservative forces and the impossibility of defining a unique potential energy for them is thorough and insightful, demonstrating a deep understanding of the subject matter."

\subsection*{Question 2:}

\textbf{Response 1:}

"Grading and Feedback for Each Solution

\textbf{Solution 1}

\textbf{Grade}: 5/6

\textbf{Feedback}: This solution provides a comprehensive description of the three cases of damping for a damped harmonic oscillator, clearly explaining the physical behavior associated with underdamping, critical damping, and overdamping. The explanation includes the mathematical conditions for each case and their physical implications, making it easy to understand. However, it could be improved by explicitly mentioning the discriminant (\(b^2 - 4mk\)) which is a key concept in distinguishing between the different damping scenarios. Additionally, incorporating the impact of the damping coefficient (\(b\)) and spring constant (\(k\)) on the system's behavior in more detail could provide a deeper understanding.

\textbf{Solution 2}

\textbf{Grade}: 6/6

\textbf{Feedback}: This solution excellently outlines the three damping cases of a damped harmonic oscillator, providing a detailed explanation of underdamping, critical damping, and overdamping. It accurately describes the mathematical condition (\(\Delta = b^2 - 4mk\)) that differentiates these cases and explains the physical behavior of the oscillator in each scenario. The explanations are clear, concise, and incorporate both the mathematical and physical aspects of the problem effectively. This response does well to connect the theoretical aspects with practical applications, such as the use of critical damping in automotive shock absorbers, which adds value to the explanation.

\textbf{Solution 3}

\textbf{Grade}: 4/6

\textbf{Feedback}: This solution correctly identifies the three cases of damping and provides a general description of their characteristics. However, it incorrectly states the discriminant as \(\Delta = b^2 - 4mk\) while considering \(m=1\) for simplicity, leading to a slight confusion in the explanation of the conditions for underdamping, critical damping, and overdamping. The condition should ideally be represented without implying that \(m=1\), as the discriminant's correct form is \(b^2 - 4mk\). Despite this, the qualitative descriptions of each damping case are accurately presented, though the explanation could benefit from further clarity and detail, especially in connecting the mathematical conditions to the physical behaviors of the system."

\textbf{Response 2:}

"Solution 1:

Grade: 5/6

Feedback: This solution provides a clear and thorough explanation of the three damping cases, accurately defining the conditions for underdamping, critical damping, and overdamping, along with their qualitative behavior. It effectively uses the relationship between \(b\), \(m\) and \(k\) to describe the system's response. However, it slightly misleads by stating the system does not return to equilibrium as quickly as possible in the underdamped case, which might imply inefficiency, rather than highlighting the oscillatory nature which is inherent to underdamping. Additionally, incorporating more specific effects of damping on system dynamics or examples could further enrich the explanation.

Solution 2:

Grade: 4/6

Feedback: This response correctly identifies the three damping scenarios and provides a decent qualitative description of each. The introduction of the discriminant \(\Delta\) to classify damping types is correct but less commonly used in this context without further mathematical development. The explanation is generally accurate but could be enhanced by clarifying the physical implications of each damping case more distinctly. For example, emphasizing the practical applications or implications of each damping scenario could make the explanations more relatable. Moreover, the solution could benefit from a more explicit discussion on how the damping coefficient influences the system's behavior in each case.

Solution 3:

Grade: 6/6

Feedback: This solution excellently outlines the three cases of damping, providing a clear, concise, and accurate description of the behavior of a damped harmonic oscillator in each scenario. It correctly applies the concept of the discriminant in the characteristic equation to distinguish between the cases, which is a standard approach in solving such differential equations. The descriptions are technically accurate, incorporating both the mathematical basis and the physical understanding of the system's response in each damping case. The explanation of how the damping coefficient \(b\) and spring constant \(k\) affect the system's return to equilibrium in different damping scenarios is particularly well articulated. This solution demonstrates a comprehensive understanding of the damped harmonic oscillator's dynamics."

\textbf{Response 3:}

"Grading each solution on a scale of 0 to 6, along with concise feedback:

\textbf{Solution 1:}

Grade: 5

Feedback: This solution provides a clear and detailed explanation of the three cases of damping, correctly identifying the conditions for underdamping, critical damping, and overdamping with accurate descriptions of the physical behavior in each case. However, it lacks a precise mathematical context in explaining the discriminant (\(b^2 - 4mk\)) which is crucial for a complete understanding of the damping cases. A more explicit connection between the mathematical conditions and the physical behaviors could enhance the explanation.

\textbf{Solution 2:}

Grade: 6

Feedback: This response excellently outlines the three damping scenarios, incorporating both the physical descriptions and the mathematical criteria (\(\Delta = b^2 - 4mk\)) that define each case. By discussing the discriminant and its impact on the system's motion, the solution provides a comprehensive understanding of the damped harmonic oscillator. The inclusion of practical examples, like automotive shock absorbers for critical damping, helps contextualize the theoretical concepts. This solution balances mathematical rigor with physical insight effectively.

\textbf{Solution 3:}

Grade: 4

Feedback: The explanation correctly identifies and describes the three damping conditions, offering a basic understanding of underdamped, critically damped, and overdamped scenarios. However, it simplifies the discussion by assuming \(m=1\), which could limit the understanding of the general case. While it accurately presents the conditions for each damping case, the explanation could be improved by elaborating on the physical implications of each scenario more vividly and providing a clearer mathematical foundation for the discriminant (\(\Delta = b^2 - 4mk\)) in a general context."

\subsection*{Question 3:}

\textbf{Response 1:}

"Solution 1 Evaluation

Part a)
\begin{itemize}
    \item Correctly identified the formula for the center of mass and manipulated it to match the given form, finding the correct value of \( k \). \textbf{[2 marks]}
\end{itemize}

Part b)
\begin{itemize}
    \item Accurately showed that the center of mass lies on the line connecting the two masses and calculated the distances from both masses to the center of mass correctly. \textbf{[4 marks]}
\end{itemize}

\textbf{Total: 6/6}

\textbf{Feedback:} Solution 1 provides a thorough and accurate explanation for both parts of the problem. The student demonstrated a clear understanding of the concept of the center of mass and its calculation, as well as how to manipulate algebraic expressions to find the value of \( k \). The distances from the masses to the center of mass were correctly derived, confirming the position of the center of mass relative to the two masses.

Solution 2 Evaluation

Part a)
\begin{itemize}
    \item Successfully showed how the center of mass's position vector is derived and found the correct value of \( k \) by equating the given expression with the standard formula. \textbf{[2 marks]}
\end{itemize}

Part b)
\begin{itemize}
    \item Correctly demonstrated the center of mass's position relative to the two masses and provided the precise calculations for the distances, aligning perfectly with the problem statement. \textbf{[4 marks]}
\end{itemize}

\textbf{Total: 6/6}

\textbf{Feedback:} Solution 2 is well-structured and precisely addresses the requirements of both parts of the question. The explanation is clear, and the student has correctly derived the value of \( k \) and the distances of the center of mass from the two masses. This solution shows a good grasp of the principles involved in calculating the center of mass and its implications.

Solution 3 Evaluation

Part a)
\begin{itemize}
    \item Correctly derived the position vector of the center of mass and identified the value of \( k \) by comparing the mass-weighted average formula with the given expression. \textbf{[2 marks]}
\end{itemize}

Part b)
\begin{itemize}
    \item Accurately showed the center of mass's position on the line connecting the two masses and calculated the distances from the masses to the center of mass, matching the problem's requirements. \textbf{[4 marks]}
\end{itemize}

\textbf{Total: 6/6}

\textbf{Feedback:} Solution 3 effectively addresses both parts of the question, with clear and correct derivations for the position vector of the center of mass and the distances from the two masses. The student demonstrated an excellent understanding of the concept and applied it effectively to solve the problem. The calculations and explanations are precise and align with the principles of physics related to the center of mass."

\textbf{Response 2:}

"Solution 1 Grading

\textbf{Part a):} 2 marks. The student correctly derives the position vector of the center of mass, \( \underline{R} \), and successfully shows how \( k \) can be found by comparing coefficients. The explanation is clear, and the algebraic manipulation is accurately performed to find \( k = \frac{m_2}{m_1 + m_2} \).

\textbf{Part b):} 4 marks. The student accurately shows that the center of mass lies on the line connecting the two masses and correctly derives the distances from the masses to the center of mass as \( \frac{d m_2}{m_1 + m_2} \) and \( \frac{d m_1}{m_1 + m_2} \), respectively. The explanation is thorough and logically follows from the derivation in part a).

\textbf{General Feedback:} Your solution is well-structured and clearly demonstrates a strong understanding of the concepts involved in finding the position vector of the center of mass and its implications. You've correctly derived and interpreted the value of \( k \) and effectively linked it to the distances from the masses to the center of mass. Excellent work!

Solution 2 Grading

\textbf{Part a):} 2 marks. This student also correctly approaches the derivation of the center of mass's position vector and identifies the appropriate value of \( k \) by equating the given formula with the mass-weighted average of position vectors. The steps taken are logical and lead to the correct conclusion.

\textbf{Part b):} 4 marks. The distances from each mass to the center of mass are correctly found using the value of \( k \) obtained from part a). The explanation clearly shows the center of mass's position relative to the masses and confirms it lies on the line connecting the two masses, with distances determined by the ratio of their masses.

\textbf{General Feedback:} Your approach to both parts of the problem is methodical and accurate, showcasing a good understanding of the principles of physics related to the center of mass. Your explanation is clear, and the mathematical reasoning is sound, leading to the correct outcomes. Well done!

Solution 3 Grading

\textbf{Part a):} 2 marks. Like the others, this student successfully derives the position vector of the center of mass and correctly identifies \( k \) by equating the given representation with the definition of the center of mass. The explanation is clear, and the derivation is correctly performed.

\textbf{Part b):} 4 marks. The student correctly demonstrates that the center of mass lies on the line connecting the two masses and calculates the distances from the masses to the center of mass accurately, using the value of \( k \). The logical flow from part a) to part b) is maintained, and the conclusions are correctly drawn based on the derivations.

\textbf{General Feedback:} Your solution demonstrates a comprehensive understanding of the concept of the center of mass and its practical implications. You've accurately derived the necessary expressions and applied them correctly to deduce the position of the center of mass in relation to the masses. Your work is exemplary!"

\textbf{Response 3:}

"\textbf{Solution 1:}

\begin{itemize}
    \item \textbf{Part a):} 2 marks. The student correctly derived the position vector of the center of mass and found the appropriate value of \( k \) by comparing coefficients. They showed a clear understanding of how to manipulate the equations to find \( k \).
    \item \textbf{Part b):} 4 marks. This solution effectively demonstrates that the center of mass lies on the line connecting the two masses and calculates the distances from the masses to the center of mass accurately, using the derived value of \( k \).
\end{itemize}

\textbf{General Feedback:} Solution 1 is well-structured and demonstrates a comprehensive understanding of the concept of center of mass. The mathematical manipulations are correct, and the explanations are clear, leading to the correct results for both parts of the question.

\textbf{Solution 2:}

\begin{itemize}
    \item \textbf{Part a):} 2 marks. Similar to Solution 1, this student correctly identifies the position vector of the center of mass and the value of \( k \). They clearly outline the steps taken to equate the mass-weighted average with the given expression and find \( k \).
    \item \textbf{Part b):} 4 marks. The student correctly shows that the center of mass is located on the line connecting the two masses and accurately determines the distances from the center of mass to each mass, using logical reasoning based on the value of \( k \) found in part a).
\end{itemize}

\textbf{General Feedback:} Solution 2 provides a clear and correct approach to solving the problem. The explanations are thorough, and the mathematical reasoning is sound. The student effectively communicates the relationship between the masses, their positions, and the center of mass.

\textbf{Solution 3:}

\begin{itemize}
    \item \textbf{Part a):} 2 marks. This student, like the others, accurately calculates the position vector of the center of mass and identifies the correct value for \( k \). Their approach to equating the given expression with the definition of the center of mass is correct and well-explained.
    \item \textbf{Part b):} 4 marks. Solution 3 also successfully demonstrates that the center of mass lies on the line connecting the two masses and calculates the distances from the masses to the center of mass with precision, making use of the derived \( k \) value.
\end{itemize}

\textbf{General Feedback:} Solution 3 is another example of a well-executed response to the problem. The student demonstrates a solid understanding of the concepts involved and provides clear, step-by-step explanations for their calculations. The results are correct and well-justified.

Overall Comments

All three solutions accurately addressed both parts of the question, demonstrating a thorough understanding of the center of mass concept. Each solution correctly derived the position vector of the center of mass and the value of \( k \), and successfully showed how the center of mass relates to the positions of the two masses. The calculations were correct, and the explanations were clear and logical in all three cases."

\subsection*{Question 4:}

\textbf{Response 1:}

"Solution 1 Evaluation

\textbf{Part a)} This solution correctly identifies the conservation of angular momentum and derives the expression for angular frequency, \( \dot{\theta} \), in terms of \( r_0 \), \( \gamma \), and \( \delta \). The steps are logically presented and correctly applied. \textbf{6/6}

\textbf{Part b)} The response correctly identifies the angular component of the acceleration but makes a significant error by not correctly calculating or considering the value of \( \ddot{\theta} \), which is crucial for accurately describing the angular component of the acceleration. The discussion around \( \ddot{\theta} \) being constant or not explicitly needed is misleading, as its calculation is essential for determining the angular acceleration. \textbf{2/5}

\textbf{Part c)} The kinetic energy derivation is correctly approached, but the explanation regarding the negative sign in the derivative of kinetic energy is confusing. The correct expression for \( \frac{dK}{dt} \) is derived, but the reasoning around the negative sign and its relevance to the direction of \(\gamma\) could have been clearer and more directly connected to the principles of work and energy. \textbf{5/7}

\textbf{General Feedback:} The solution demonstrates a good understanding of angular momentum conservation and kinetic energy calculation in polar coordinates. However, it lacks clarity in explaining the significance of the negative sign in the rate of change of kinetic energy and inaccurately addresses the necessity of calculating \( \ddot{\theta} \) for the angular acceleration component. To improve, focus on a more thorough explanation of each physical quantity's role and ensure all necessary calculations are accurately performed and justified.

Solution 2 Evaluation

\textbf{Part a)} The derivation of angular frequency, \( \dot{\theta} \), is correctly executed, with clear steps that logically follow from the conservation of angular momentum. \textbf{6/6}

\textbf{Part b)} This solution attempts to calculate \( \ddot{\theta} \) and substitute it into the expression for the angular component of the acceleration. However, the final expression for \( a_{\theta} \) is not provided, and the calculation of \( \ddot{\theta} \) is mentioned without showing the complete process or final form, which is critical for evaluating the correctness of the solution. \textbf{3/5}

\textbf{Part c)} The kinetic energy and its time derivative are correctly derived, including the necessary steps to match the given expression for \( \frac{dK}{dt} \). The correct application of physical principles and mathematical procedures is demonstrated. \textbf{7/7}

\textbf{General Feedback:} This solution shows a strong grasp of the concepts of angular momentum conservation and kinetic energy in polar coordinates. However, it falls short in fully detailing the calculation of \( \ddot{\theta} \) and its impact on the angular acceleration, which is a key aspect of part b). To improve, ensure that all steps, especially those involving differentiation, are clearly shown and accurately calculated to provide a comprehensive understanding of the problem's solution.

Solution 3 Evaluation

\textbf{Part a)} The solution correctly derives the expression for \( \dot{\theta} \) based on angular momentum conservation. The steps are clear and accurately follow from the given conditions. \textbf{6/6}

\textbf{Part b)} The solution incorrectly concludes that the angular component of the acceleration simplifies to zero, which is a significant misunderstanding. This error suggests a fundamental misconception or calculation error regarding angular acceleration in this context. \textbf{1/5}

\textbf{Part c)} The discussion around the kinetic energy and its time derivative is problematic. It suggests a misunderstanding in the calculation or interpretation of the rate of change of kinetic energy. While it attempts to address the discrepancy in signs, the confusion and incorrect conclusion regarding the angular acceleration component significantly undermine the solution's accuracy. \textbf{2/7}

\textbf{General Feedback:} This solution demonstrates an understanding of angular momentum conservation but significantly falters in its treatment of angular acceleration and kinetic energy's rate of change. The misunderstanding of \( a_{\theta} \) and the misinterpretation of the sign in \( \frac{dK}{dt} \) reflect a critical need for a more thorough and accurate application of the principles of dynamics and calculus. To improve, focus on correctly applying differentiation and understanding its physical implications, especially in the context of polar coordinates."

\textbf{Response 2:}

"\textbf{Solution 1:}

\begin{itemize}
    \item \textbf{Part a):} 6/6. The student correctly derives the expression for angular frequency \(\dot{\theta}\) using the conservation of angular momentum. The solution is thorough and accurately reflects the relationship between \(\dot{\theta}\), \(r_0\), \(\gamma\), and \(\delta\).
    
    \item \textbf{Part b):} 2/5. The student provides an expression for the angular component of acceleration but incorrectly concludes that \(\ddot{\theta}\) is not needed or its calculation is not pursued further. The lack of explicit derivation for \(\ddot{\theta}\) and incorrect assumption about its necessity impacts the accuracy of the solution.
    
    \item \textbf{Part c):} 5/7. The student correctly derives an expression for the kinetic energy and its time derivative. However, the explanation around the negative sign discrepancy and the incorrect justification for omitting the calculation of \(\ddot{\theta}\) in part b) affects the coherence of the solution. The correct approach to deriving \(\frac{dK}{dt}\) is shown, but the explanation regarding the negative sign is somewhat confusing.
\end{itemize}

\textbf{General Feedback for Solution 1:} The solution provides a detailed and mostly accurate approach to solving the problem, with a strong understanding of angular momentum conservation and kinetic energy in polar coordinates. However, it lacks in accurately addressing the angular acceleration component, which impacts the overall coherence of the solution. It's crucial to consider all aspects of the problem, including the calculation of \(\ddot{\theta}\), for a complete understanding.

\hrulefill

\textbf{Solution 2:}

\begin{itemize}
    \item \textbf{Part a):} 6/6. This solution correctly identifies and solves for \(\dot{\theta}\) using the principle of angular momentum conservation. The steps are clear, and the final expression is accurate.
    
    \item \textbf{Part b):} 4/5. The student attempts to derive \(\ddot{\theta}\) but does not fully complete the process or incorporate it correctly into the angular acceleration expression. The effort to address \(\ddot{\theta}\) is noted, but the execution is incomplete.
    
    \item \textbf{Part c):} 6/7. The derivation of kinetic energy and its time derivative is largely correct, with a minor error in the calculation of \(\ddot{\theta}\) affecting the angular acceleration component. The student shows a good understanding of the principles but has a small oversight in calculation.
\end{itemize}

\textbf{General Feedback for Solution 2:} The student demonstrates a good grasp of the principles behind angular momentum conservation and kinetic energy calculations. However, the solution could be improved by correctly completing the calculation of \(\ddot{\theta}\) and more accurately integrating it into the angular acceleration component. Attention to detail in calculations would enhance the accuracy of the solution.

\hrulefill

\textbf{Solution 3:}

\begin{itemize}
    \item \textbf{Part a):} 6/6. The student successfully derives the expression for \(\dot{\theta}\) using conservation of angular momentum, correctly integrating \(\dot{r}\) to find \(r(t)\) and then solving for \(\dot{\theta}\).
    
    \item \textbf{Part b):} 1/5. The solution to part b) is significantly flawed. The student incorrectly simplifies \(a_{\theta}\) to \(0\), which demonstrates a misunderstanding of the problem's dynamics or a miscalculation.
    
    \item \textbf{Part c):} 3/7. The approach to deriving kinetic energy and its time derivative acknowledges the necessary components but is marred by inaccuracies in previous parts, particularly the angular acceleration component. The attempt to reconcile the sign discrepancy in \(\frac{dK}{dt}\) without correctly calculating \(\ddot{\theta}\) or accurately addressing its impact results in a partially correct solution.
\end{itemize}

\textbf{General Feedback for Solution 3:} The solution starts strong with correct application of angular momentum conservation but falters in parts b) and c) due to significant misunderstandings and inaccuracies in calculations, especially regarding angular acceleration. A more careful review of the principles and calculations involved in angular motion and energy would greatly improve the solution's accuracy and coherence."

\textbf{Response 3:}

"Let's evaluate each solution based on the criteria for parts (a), (b), and (c) of the problem, with up to 6 marks for part (a), up to 5 marks for part (b), and up to 7 marks for part (c).

Solution 1 Evaluation

\begin{itemize}
    \item \textbf{Part (a):} The student correctly derives the expression for angular frequency \(\dot{\theta}\) using the conservation of angular momentum and correctly substitutes the expression for \(r(t)\). This part is well-executed and shows a good understanding of the concept. \textbf{Marks: 6/6}
    \item \textbf{Part (b):} The student identifies the angular component of the acceleration and substitutes the given values for \(\dot{r}\) and derived values for \(\dot{\theta}\). However, they did not calculate \(\ddot{\theta}\), which is crucial for a complete answer. Partial marks for identifying the correct formula and partial substitution. \textbf{Marks: 3/5}
    \item \textbf{Part (c):} The kinetic energy derivation is correctly approached, and the differentiation to find \(\frac{dK}{dt}\) is attempted. The explanation suggests a correct pathway but lacks clarity in the final derivation steps. Given the effort and partially correct approach, partial marks are awarded. \textbf{Marks: 5/7}
\end{itemize}

\textbf{Total Marks for Solution 1:} 14/18

\textbf{Feedback:} The student shows a strong understanding of angular momentum conservation and kinetic energy in polar coordinates. For improvement, ensure all components of equations are fully calculated, such as \(\ddot{\theta}\) for the angular acceleration part, and clarify the final steps of derivations.

Solution 2 Evaluation

\begin{itemize}
    \item \textbf{Part (a):} The student correctly derives the expression for \(\dot{\theta}\) using angular momentum conservation. The steps are clear, and the final expression is correct. \textbf{Marks: 6/6}
    \item \textbf{Part (b):} The student attempts to calculate \(\ddot{\theta}\) and substitutes it into the expression for angular acceleration. The effort to include \(\ddot{\theta}\) is commendable, but there's a misunderstanding in the calculation of \(\ddot{\theta}\). Partial marks for the correct approach and identification of components. \textbf{Marks: 3/5}
    \item \textbf{Part (c):} The kinetic energy derivation is correctly approached, and the differentiation to match the given expression for \(\frac{dK}{dt}\) is mostly correct, despite a minor sign issue discussed. The effort and understanding demonstrated warrant partial marks. \textbf{Marks: 6/7}
\end{itemize}

\textbf{Total Marks for Solution 2:} 15/18

\textbf{Feedback:} The student demonstrates good comprehension of the problem's physics, especially in conserving angular momentum and deriving kinetic energy. To improve, ensure accurate calculation of derivatives and consider the implications of sign conventions in physical expressions.

Solution 3 Evaluation

\begin{itemize}
    \item \textbf{Part (a):} The student correctly derives the expression for \(\dot{\theta}\), showing a solid understanding of angular momentum conservation. \textbf{Marks: 6/6}
    \item \textbf{Part (b):} The explanation suggests a misunderstanding in calculating the angular component of acceleration, leading to an incorrect simplification. This part lacks the correct application of the given and derived variables. \textbf{Marks: 2/5}
    \item \textbf{Part (c):} The student makes an attempt to derive the kinetic energy and its time derivative, but there are significant errors in the approach and misunderstanding in the application of the physics concepts, leading to an incorrect conclusion about the sign discrepancy. \textbf{Marks: 3/7}
\end{itemize}

\textbf{Total Marks for Solution 3:} 11/18

\textbf{Feedback:} The student has a partial understanding of the physics concepts involved but needs to improve on the application of these concepts, especially in calculating acceleration components and correctly interpreting the results of derivations. Focus on the detailed application of derivatives and review the physical significance of your results.

General Feedback

Each solution demonstrates varying levels of understanding and application of physics principles to solve the problem. Key areas for improvement include ensuring the accuracy of derivative calculations, the correct application of physical laws (such as conservation of angular momentum), and careful interpretation of the results, especially with regard to signs and physical significance. Attention to detail in each part of the problem will strengthen your physics problem-solving skills."

\subsection*{Question 5:}

\textbf{Response 1:}

"Grading each solution:

\textbf{Solution 1:}

\begin{itemize}
    \item Part a) Equations of Motion (2 marks) \\
    Correctly derived the equations of motion for both objects X and Y.
    \item Part b) Uncoupling the Equations (2 marks) \\
    The student made an algebraic error in the process of uncoupling the equations. The terms involving \(y_1\) and \(y_2\) were not correctly simplified, leading to incorrect equations of motion for \(y_1\) and \(y_2\).
    \item Part c) Solution for \(y_1\) (3 marks) \\
    The student correctly applied the initial conditions and derived the correct solution for \(y_1\).
\end{itemize}

\textbf{Total: 7 marks}

Feedback: The student displayed a solid understanding of the physics involved and correctly applied initial conditions to find the solution for \(y_1\). However, there was an error in the algebra while attempting to uncouple the equations. It is important to carefully handle the terms when adding and subtracting equations.

\textbf{Solution 2:}

\begin{itemize}
    \item Part a) Equations of Motion (2 marks) \\
    The student correctly wrote the equations of motion for objects X and Y.
    \item Part b) Uncoupling the Equations (4 marks) \\
    The student correctly decoupled the equations by defining new coordinates and derived the correct equations of motion for \(y_1\) and \(y_2\).
    \item Part c) Solution for \(y_1\) (3 marks) \\
    The student correctly solved for \(y_1\) with the proper initial conditions.
\end{itemize}

\textbf{Total: 9 marks}

Feedback: The student showed a thorough understanding of the problem and executed the solution flawlessly. The algebra and calculus were handled correctly, and the physical interpretation of the initial conditions was accurate.

\textbf{Solution 3:}

\begin{itemize}
    \item Part a) Equations of Motion (2 marks) \\
    The student correctly derived the equations of motion for objects X and Y.
    \item Part b) Uncoupling the Equations (4 marks) \\
    The student successfully uncoupled the equations and found the correct differential equations for \(y_1\) and \(y_2\).
    \item Part c) Solution for \(y_1\) (3 marks) \\
    The solution for \(y_1\) was correctly determined using the initial conditions provided.
\end{itemize}

\textbf{Total: 9 marks}

Feedback: This student demonstrated a clear understanding of the problem and the physics involved. The mathematical manipulations were correct, and the solutions for the uncoupled equations and \(y_1\) were accurate. The application of initial conditions was correctly done, leading to the correct final solution."

\textbf{Response 2:}

"The image provided seems to be a diagram relevant to the physics problem given, which depicts a mechanical system with two masses and springs. Let's evaluate each solution:

Solution 1:

\textbf{Part a) (2 marks)}\\
Correctly applies Newton's second law to both objects X and Y, derives the equations of motion, and identifies that they are coupled. 

\textbf{Part b) (3 marks)}\\
Correctly defines the new coordinates and derives the equations for \( y_1 \) and \( y_2 \), showing that they represent simple harmonic motion. However, there is a mistake in the final form of the equations of motion for \( y_1 \) and \( y_2 \) (the terms with \( y_2 \) are not canceled properly).

\textbf{Part c) (2 marks)}\\
The initial conditions are applied correctly to the general solution of simple harmonic motion, and the correct solution for \( y_1 \) is found. The process is correct but the explanation could be clearer in specifying why \( A = 0 \).

\textbf{General Feedback:}\\
The solution correctly identifies the equations of motion and proceeds to uncouple them using a standard method. However, there is a slight error in part b when uncoupling the equations, leading to a mistake in the equations of motion for \( y_1 \) and \( y_2 \). Part c is correct but could benefit from a clearer explanation.

Solution 2:

\textbf{Part a) (2 marks)}\\
The student provides the correct equations of motion for objects X and Y using Hooke's law and Newton's second law.

\textbf{Part b) (4 marks)}\\
The student successfully shows how to uncouple the coordinates using new variables \( y_1 \) and \( y_2 \), leading to correct equations of simple harmonic motion.

\textbf{Part c) (3 marks)}\\
The student correctly applies the initial conditions to the general solution for simple harmonic motion, arriving at the correct expression for \( y_1 \).

\textbf{General Feedback:}\\
This is a complete and correct solution to the problem. The equations of motion are derived correctly, and the uncoupling is done properly. The initial conditions are applied correctly to find the solution for \( y_1 \).

Solution 3:

\textbf{Part a) (2 marks)}\\
The student derives the correct equations of motion for both objects X and Y.

\textbf{Part b) (4 marks)}\\
The student correctly uncouples the equations of motion for \( y_1 \) and \( y_2 \) and finds the correct form of the equations for simple harmonic motion.

\textbf{Part c) (3 marks)}\\
The student correctly applies the initial conditions and finds the correct solution for \( y_1 \) using the appropriate sinusoidal function.

\textbf{General Feedback:}\\
Solution 3 correctly solves the problem in all parts. The uncoupling method is applied correctly, and the solution for \( y_1 \) is derived correctly with the initial conditions applied properly.

Summary of Grades:

\begin{itemize}
  \item \textbf{Solution 1:} 2 + 3 + 2 = 7/9
  \item \textbf{Solution 2:} 2 + 4 + 3 = 9/9
  \item \textbf{Solution 3:} 2 + 4 + 3 = 9/9
\end{itemize}

\textbf{Overall Feedback:}\\
Students 2 and 3 have provided a complete and accurate solution to the problem, while Student 1 made a minor error in part b that should be reviewed for a complete understanding of uncoupling the equations. All students show a good grasp of the concepts involved in the problem."

\textbf{Response 3:}

"Based on the provided solutions, here are the grades and feedback for each:

Solution 1:

\begin{itemize}
    \item a) Correct derivation of the equations of motion for objects X and Y. (+2 marks)
    \item b) Correct approach to uncouple the equations, but there is an algebraic error in the final equations for \( y_1 \) and \( y_2 \). The final terms should be \( -ky_2 \) for the \( y_1 \) equation and \( +ky_1 \) for the \( y_2 \) equation. (+2 marks)
    \item c) Correct identification of the initial conditions and correct final solution for \( y_1 \). (+3 marks)
\end{itemize}

Total: 7/9 marks

Feedback: Good work on parts a) and c). Be careful with algebra in part b); small mistakes can lead to incorrect conclusions.

Solution 2:

\begin{itemize}
    \item a) Correctly identified the forces and derived the equations of motion. (+2 marks)
    \item b) Correct method for uncoupling the coordinates, but there is a sign error in the final equations after substituting \( y_1 \) and \( y_2 \). The correct equations should not have negative signs in front of the \( ky_1 \) and \( 3ky_2 \) terms. (+2 marks)
    \item c) Correct use of initial conditions to find the solution for \( y_1 \). (+3 marks)
\end{itemize}

Total: 7/9 marks

Feedback: The approach and methodology are sound. Pay attention to the signs when substituting and simplifying terms to avoid mistakes.

Solution 3:

\begin{itemize}
    \item a) Correct application of Newton's second law and Hooke's law, leading to the correct equations of motion for X and Y. (+2 marks)
    \item b) Correct derivation of uncoupled equations for \( y_1 \) and \( y_2 \). (+4 marks)
    \item c) Appropriate use of initial conditions and correct identification of the phase angle, leading to the correct solution for \( y_1 \). (+3 marks)
\end{itemize}

Total: 9/9 marks

Feedback: Excellent work across all parts. The equations were derived and uncoupled correctly, and the solution for \( y_1 \) was found correctly using the initial conditions."

\textbf{Response 3:}

"Based on the provided solutions, here are the grades and feedback for each:

Solution 1:

\begin{itemize}
    \item a) Correct derivation of the equations of motion for objects X and Y. (+2 marks)
    \item b) Correct approach to uncouple the equations, but there is an algebraic error in the final equations for \( y_1 \) and \( y_2 \). The final terms should be \( -ky_2 \) for the \( y_1 \) equation and \( +ky_1 \) for the \( y_2 \) equation. (+2 marks)
    \item c) Correct identification of the initial conditions and correct final solution for \( y_1 \). (+3 marks)
\end{itemize}

Total: 7/9 marks

Feedback: Good work on parts a) and c). Be careful with algebra in part b); small mistakes can lead to incorrect conclusions.

Solution 2:

\begin{itemize}
    \item a) Correctly identified the forces and derived the equations of motion. (+2 marks)
    \item b) Correct method for uncoupling the coordinates, but there is a sign error in the final equations after substituting \( y_1 \) and \( y_2 \). The correct equations should not have negative signs in front of the \( ky_1 \) and \( 3ky_2 \) terms. (+2 marks)
    \item c) Correct use of initial conditions to find the solution for \( y_1 \). (+3 marks)
\end{itemize}

Total: 7/9 marks

Feedback: The approach and methodology are sound. Pay attention to the signs when substituting and simplifying terms to avoid mistakes.

Solution 3:

\begin{itemize}
    \item a) Correct application of Newton's second law and Hooke's law, leading to the correct equations of motion for X and Y. (+2 marks)
    \item b) Correct derivation of uncoupled equations for \( y_1 \) and \( y_2 \). (+4 marks)
    \item c) Appropriate use of initial conditions and correct identification of the phase angle, leading to the correct solution for \( y_1 \). (+3 marks)
\end{itemize}

Total: 9/9 marks

Feedback: Excellent work across all parts. The equations were derived and uncoupled correctly, and the solution for \( y_1 \) was found correctly using the initial conditions."

\subsection*{Question 6:}

\textbf{Response 1:}

"Grading and Feedback for Each Solution

\textbf{Solution 1:}
\begin{itemize}
    \item Part a) \textbf{2 marks}: Correct derivation of the maximum speed \(v = \sqrt{\mu_s g R}\) for a Go-kart on a flat circular track using the balance between centripetal force and maximum frictional force.
    \item Part b) \textbf{2 marks}: The explanation correctly identifies the component of gravitational force acting towards the center as providing the necessary centripetal force, but it inaccurately states that \(mg\sin(\phi)\) provides this force without considering the normal force's direction on an inclined plane. The correct component for centripetal force should be based on the orientation of the gravitational force relative to the incline.
    \item Part c) \textbf{6 marks}: The student attempted to derive expressions for \(v_{max}\) and \(v_{min}\) and to show the relationship between \(v_{max}^2 - v_{min}^2\) and the given equation. However, the final expression derived does not match the expected outcome, and there's a clear misunderstanding in the simplification process. The effort to address the problem is noted, but the error in the final result and the lack of correct derivation to match the given equation significantly impacts the score.
\end{itemize}

\textbf{Total: 10 marks}

\textit{General Feedback:} Your approach to solving the problem demonstrates a good understanding of the principles of circular motion and friction. However, your solution for part c) contains errors in the derivation and simplification process, leading to an incorrect final result. Review the application of frictional forces on inclined planes and ensure a careful analysis of force components when solving complex problems.

\textbf{Solution 2:}
\begin{itemize}
    \item Part a) \textbf{2 marks}: Correctly identified and derived the expression for the maximum speed on a flat circular track using frictional force.
    \item Part b) \textbf{1 mark}: Incorrectly identifies the gravitational component contributing to the centripetal force; the explanation incorrectly uses \(mg\cos(\phi)\) instead of the correct \(mg\sin(\phi)\) component for a frictionless inclined plane. This represents a fundamental misunderstanding of the force components on an incline.
    \item Part c) \textbf{4 marks}: Attempts to address the derivation for \(v_{max}\) and \(v_{min}\) and their squared difference but concludes with an incorrect simplification. The attempt to relate the forces and solve the problem is noted, but the final expression and understanding of the physics behind the scenario are incorrect.
\end{itemize}

\textbf{Total: 7 marks}

\textit{General Feedback:} Your work on part a) was accurate and well-explained. However, parts b) and c) contain significant conceptual errors, especially in understanding the role of gravitational components in circular motion on an incline. Ensure a correct analysis of forces when dealing with inclined planes and friction to accurately solve problems of this nature.

\textbf{Solution 3:}
\begin{itemize}
    \item Part a) \textbf{2 marks}: Accurate derivation of the maximum speed formula for circular motion on a flat track with friction.
    \item Part b) \textbf{3 marks}: Correctly identifies the gravitational force component (\(mg\sin(\phi)\)) responsible for providing centripetal force on an inclined frictionless track, leading to an accurate expression for maximum speed.
    \item Part c) \textbf{8 marks}: Provides a comprehensive attempt to derive minimum and maximum speeds on an inclined track with friction. However, the final step to show the exact relationship between \(v_{max}^2 - v_{min}^2\) and the given equation is incorrect. The effort to solve and the partial correctness of the approach are acknowledged, but the inability to correctly prove the given relationship impacts the score.
\end{itemize}

\textbf{Total: 13 marks}

\textit{General Feedback:} Your solutions for parts a) and b) are accurate and well-founded. In part c), your approach and the initial steps towards deriving \(v_{max}\) and \(v_{min}\) were on the right path, but the final result did not align with the expected equation. This suggests a need for a more careful review of the physics principles involved and a more precise handling of the mathematical relationships between forces, friction, and circular motion on inclined planes."

\textbf{Response 2:}

"Solution 1 Grading:

\begin{itemize}
    \item \textbf{Part a)} Correct derivation of the largest speed formula \(v = \sqrt{\mu_s g R}\) for a flat circular track. \textbf{2/2 marks}
    \item \textbf{Part b)} Correct derivation of the largest speed formula \(v = \sqrt{gR\sin(\phi)}\) for an inclined frictionless track. \textbf{3/3 marks}
    \item \textbf{Part c)} Incorrect simplification and final expression for the difference \(v_{max}^2 - v_{min}^2\). The student attempted to show the derivation but ended up with a different result, \(2Rg\mu_s\sin(2\phi)\), which does not match the target equation. \textbf{5/11 marks} due to a significant conceptual mistake in deriving the final expression.
\end{itemize}

\textbf{Total: 10/16}

\textbf{Feedback:} Your derivations for parts a and b were correct and well-explained. However, in part c, there was a misunderstanding or miscalculation that led to an incorrect final expression. It's important to carefully analyze the forces at play and correctly apply the physics principles of circular motion, especially when dealing with inclined planes and friction. Revisiting the problem with a focus on correctly balancing the forces and understanding their contributions to the centripetal force would be beneficial.

Solution 2 Grading:

\begin{itemize}
    \item \textbf{Part a)} Accurately derived the largest speed formula for a flat circular track. \textbf{2/2 marks}
    \item \textbf{Part b)} Incorrect component of gravitational force used for the centripetal force calculation, leading to a wrong formula \(v = \sqrt{Rg\cos(\phi)}\). \textbf{1/3 marks} due to a fundamental misunderstanding of the force direction.
    \item \textbf{Part c)} Incorrect assumption about the gravitational force component and a mistake in the final expression, simplifying to \(2g\mu_s\cos(\phi)\) which is incorrect. \textbf{4/11 marks} for correctly identifying the role of friction and gravity but failing to accurately derive the final expression.
\end{itemize}

\textbf{Total: 7/16}

\textbf{Feedback:} Your solution to part a was correct. However, in part b, there was a misunderstanding regarding the direction of the gravitational force component that contributes to centripetal force on an inclined plane. Part c saw an attempt to address the forces involved but ended with an incorrect simplification. Focus on the correct analysis of forces, especially the components contributing to centripetal force on inclined planes, and ensure mathematical accuracy in your derivations.

Solution 3 Grading:

\begin{itemize}
    \item \textbf{Part a)} Correct derivation of the maximum speed formula for a flat circular track. \textbf{2/2 marks}
    \item \textbf{Part b)} Correctly identified the gravitational force component but incorrectly stated the component for the frictionless track; the explanation seems confused but arrives at the correct formula for \(v_{max}\). \textbf{2/3 marks} due to slight confusion in explanation but correct formula.
    \item \textbf{Part c)} Attempts to derive minimum and maximum speeds with friction but ends with a mistaken simplification that does not match the given equality. \textbf{6/11 marks} for the attempt to incorporate friction and gravity but failing to accurately derive the final required expression.
\end{itemize}

\textbf{Total: 10/16}

\textbf{Feedback:} Your approach to parts a and b was generally correct, despite some confusion in part b. For part c, you recognized the complexity introduced by friction but did not manage to correctly derive the final expression as requested. It's crucial to carefully consider the effects of friction in both directions and ensure that your mathematical derivations accurately reflect the physical scenario. Revisiting the principles of circular motion on inclined planes and the role of friction could help clarify these aspects."

\textbf{Response 3:}

"Solution 1 Evaluation

\textbf{Part a)} The student correctly derived the expression for the largest speed the Go-kart can have whilst staying on the same circular path, \(v = \sqrt{\mu_s g R}\). This is an accurate derivation based on the relationship between centripetal force and frictional force. \textbf{[2 marks]}

\textbf{Part b)} The explanation for deriving the largest speed on an inclined circular track without friction, \(v = \sqrt{gR\sin(\phi)}\), is correct but not entirely precise. The gravitational component acting towards the center should indeed be \(mg\sin(\phi)\), but the explanation could have been clearer regarding the direction of forces. \textbf{[2 marks]}

\textbf{Part c)} The student's attempt to derive the minimum and maximum speed on an inclined circular track with friction shows a good understanding of how friction affects the centripetal force. However, their final expression and attempt to match the provided equality were incorrect. The student acknowledges the mistake but does not correct it. The correct approach involves a nuanced understanding of the forces along the inclined plane. While the effort and steps towards the solution are commendable, the failure to accurately derive the final expression as requested impacts the score. \textbf{[5 marks]}

\textbf{General Feedback:} The student shows a strong grasp of the physics concepts involved and attempts to address each part of the problem methodically. However, the error in the final part indicates a misunderstanding of the complex interplay of forces on an inclined plane. A more careful analysis and review of the force components and their contributions to circular motion would enhance the accuracy of the solution.

Solution 2 Evaluation

\textbf{Part a)} Accurately derived the expression for the largest speed on a flat circular track, \(v = \sqrt{\mu_s g R}\). The explanation was clear and correctly applied the relationship between frictional and centripetal forces. \textbf{[2 marks]}

\textbf{Part b)} The student mistakenly identified the component of gravitational force providing centripetal force as \(mg\cos(\phi)\) when it should be \(mg\sin(\phi)\) for an inclined plane. This fundamental error affects the accuracy of the derived expression for speed, \(v = \sqrt{Rg\cos(\phi)}\), which is incorrect. \textbf{[1 mark]}

\textbf{Part c)} The explanation for deriving the minimum and maximum speed on an inclined track with friction starts well, but the final simplification and the attempt to match the provided equality were incorrect. Despite acknowledging the mistake, the student does not provide a correct derivation or match the provided equality. The initial approach and understanding of how friction affects centripetal force were correct, but the error in the final expression significantly impacts the score. \textbf{[4 marks]}

\textbf{General Feedback:} The student demonstrates an understanding of how friction and gravitational forces contribute to circular motion. However, critical errors in parts b and c, especially misidentifying the gravitational force component and failing to accurately derive the final expression, indicate a need for a deeper review of physics principles related to motion on inclined planes.

Solution 3 Evaluation

\textbf{Part a)} Correctly derived the expression for the largest speed on a flat circular track, \(v = \sqrt{\mu_s g R}\). The explanation clearly and correctly applies the principles of frictional and centripetal forces. \textbf{[2 marks]}

\textbf{Part b)} The explanation initially mentions the wrong component of gravitational force as contributing to centripetal force but then correctly identifies \(mg \sin(\phi)\) as the contributing component for a frictionless incline. However, the mention of \(mg \cos(\phi)\) as the contributing force introduces confusion. Despite the error, the correct expression for speed, \(v = \sqrt{g R \sin(\phi)}\), is provided. \textbf{[2 marks]}

\textbf{Part c)} The student correctly identifies the forces contributing to the centripetal force for both the minimum and maximum speeds but fails to accurately derive or simplify the expression to match the provided equality. The error in the final expression and the inability to match the given equality, despite an otherwise correct approach to identifying the forces at play, impacts the score. \textbf{[6 marks]}

\textbf{General Feedback:} The student demonstrates a good understanding of the physics principles underlying circular motion and the effects of friction. However, confusion in part b and an incorrect simplification in part c highlight the need for more careful analysis and verification of derived expressions. A more rigorous approach to applying physics concepts and verifying calculations would improve the accuracy of the solutions."

\subsection*{Question 7:}

\textbf{Response 1:}

"Grading and Feedback for Each Solution

Solution 1

\begin{itemize}
    \item \textbf{Part a) Energy-Momentum 4-Vector and Lorentz Invariance}: 4 marks \\
    The solution correctly defines the energy-momentum 4-vector and explains the relationship between energy, momentum, and mass. It also correctly demonstrates the invariance of the scalar product under Lorentz transformation.
    
    \item \textbf{Part b) Four-Momenta of Decay Particles in Lab Frame}: 4 marks \\
    The explanation of the transformation of the four-momenta to the lab frame is accurate and clearly demonstrates understanding of the Lorentz transformation. However, it slightly lacks the explicit demonstration of how the energies \(E_0\) and momenta \(p_0\) are related in the rest frame, assuming knowledge of \(E = cp\) without derivation.
    
    \item \textbf{Part c) Wavelengths in Rest and Lab Frames}: 4 marks \\
    The calculation of wavelengths in both frames is conceptually correct but lacks the explicit numerical calculations that would provide the exact wavelengths. The answer gives an approximation without showing the detailed calculation process.
\end{itemize}

\textbf{General Feedback}: This solution provides a thorough and mostly accurate explanation of the concepts involved in the problem. It would benefit from more explicit numerical examples in part c) to fully demonstrate the application of the concepts.

Solution 2

\begin{itemize}
    \item \textbf{Part a) Energy-Momentum 4-Vector and Lorentz Invariance}: 3 marks \\
    The solution provides a correct definition and derivation for the energy-momentum 4-vector and its scalar product. However, it could have been clearer in demonstrating the Lorentz invariance of the scalar product by explicitly stating its independence from the frame of reference.
    
    \item \textbf{Part b) Four-Momenta in the Lab Frame}: 4 marks \\
    The approach to transforming the four-momenta to the lab frame is correct. The solution effectively uses Lorentz transformation but could improve by more explicitly relating the transformations to the physical interpretation of the results.
    
    \item \textbf{Part c) Wavelengths of the Photons}: 2 marks \\
    The solution attempts to calculate the wavelengths but makes significant errors in the application of physical formulas and constants. The energy values used for the calculations in the lab frame do not appear to be correctly derived, leading to incorrect wavelengths.
\end{itemize}

\textbf{General Feedback}: The solution shows a good understanding of relativistic energy and momentum but falls short in accurately applying these concepts to calculate the wavelengths of photons. More careful attention to detail in calculations and clearer explanations would improve this response.

Solution 3

\begin{itemize}
    \item \textbf{Part a) Energy-momentum 4-vector and Lorentz Invariance}: 4 marks \\
    This solution correctly identifies and explains the energy-momentum 4-vector and its scalar product, demonstrating clear understanding of Lorentz invariance.
    
    \item \textbf{Part b) Four-momenta of Decay Products in the Lab Frame}: 3 marks \\
    The solution outlines the process for obtaining the four-momenta in the lab frame but lacks detailed explanation of the Lorentz transformation applied to the decay products, making it less thorough than it could be.
    
    \item \textbf{Part c) Wavelengths of the Photons}: 1 mark \\
    The solution provides an attempt to calculate the wavelengths, but the values and methodology appear significantly flawed. The energy values and resulting wavelengths are not consistent with the given information or physical principles.
\end{itemize}

\textbf{General Feedback}: This response demonstrates a good foundational understanding of the topics but struggles with accurate application and calculation, particularly in part c). Improving the mathematical rigor and accuracy in calculations would greatly enhance the quality of this solution."

\textbf{Response 2:}

"Grading and Feedback for Each Solution
Solution 1
Part a): Correct definition and explanation of the energy-momentum 4-vector and its Lorentz invariance. 4/4 marks.

Part b): Accurate derivation of the four-momenta of the decay particles in the lab frame, with clear explanation and correct use of Lorentz transformation. 5/5 marks.

Part c): Correct approach to finding the wavelengths in both frames with appropriate formulas and relativistic effects considered. The calculations are not shown, but the explanation indicates a correct understanding. 4/4 marks.

General Feedback: This solution provides a thorough and accurate explanation for each part of the problem. The student demonstrates a strong understanding of relativistic physics, including Lorentz transformations and the properties of energy and momentum in different frames. The calculation details for part c could be more explicit, but the final results and understanding of relativistic Doppler effects are correctly identified.

Solution 2
Part a): Correct explanation of the energy-momentum 4-vector and its invariant nature under Lorentz transformations. 4/4 marks.

Part b): Correct approach to transforming the four-momenta to the lab frame, but the explanation lacks clarity in how the Lorentz transformation applies to energy and momentum components. 4/5 marks.

Part c): The approach to calculating the wavelengths in both frames is conceptually correct, but the final wavelengths provided do not match the expected values from the given energy levels, indicating a potential calculation error. 3/4 marks.

General Feedback: The student shows a good understanding of the concepts involved in relativistic physics and the effects of Lorentz transformations. However, there are minor issues in the application of these concepts in part b and a significant error in the calculations in part c, which affects the accuracy of the final results.

Solution 3
Part a): Correctly identifies the energy-momentum 4-vector and its invariant nature. However, the explanation of the Lorentz invariance is somewhat convoluted and could be clearer. 3/4 marks.

Part b): The solution correctly identifies the need to use the Lorentz transformation for calculating the four-momenta in the lab frame but fails to provide the explicit transformed expressions and final forms for the four-momenta. 3/5 marks.

Part c): The conceptual approach to calculating wavelengths is correct, but the provided wavelengths and the interpretation of the relativistic Doppler effect seem to be significantly off from expected values, indicating major errors in calculation or conceptual understanding. 2/4 marks.

General Feedback: The student demonstrates an understanding of the key concepts but struggles with clarity and accuracy in their application. The explanations are somewhat convoluted, making it difficult to follow the logic in parts b and c. Additionally, significant calculation errors in part c suggest a misunderstanding of the relationship between energy, wavelength, and the relativistic effects on these quantities. Improvements in clarity, accuracy, and calculation are needed for a better score."

\textbf{Response 3:}

"To evaluate and grade each solution, I will consider the accuracy, completeness, and clarity of explanation for each part of the problem, adhering to the grading criteria provided: up to 4 marks for part a), up to 5 marks for part b), and up to 4 marks for part c), with only whole numbers for marks. Here's the evaluation for each solution:

Solution 1

Part a) [3 marks]: The student correctly identifies the form of the energy-momentum 4-vector and demonstrates its Lorentz invariance, including the correct relationship between energy, momentum, and mass. However, there's a slight inconsistency in the explanation of the scalar product's derivation, which could be clearer, especially in transitioning between steps.

Part b) [4 marks]: The explanation for obtaining the four-momenta of the decay particles in the lab frame is correct and well-detailed, showing a good understanding of the Lorentz transformation and its implications for the energy and momentum of massless particles.

Part c) [3 marks]: The calculation of wavelengths in both frames shows a good understanding of the relationship between energy and wavelength for photons. However, the calculation process is not explicitly shown, and the final answers are stated without showing the intermediate steps, which could help in understanding how these values were derived.

General Feedback: The solution provides a solid understanding of relativistic physics principles, particularly Lorentz invariance and transformations. It would benefit from more detailed calculations in part c) and clearer step-by-step reasoning in part a).

Solution 2

Part a) [4 marks]: This solution clearly and accurately describes the energy-momentum 4-vector, its components, and the Lorentz invariance of its scalar product. The explanation is thorough and demonstrates a strong understanding of the concept.

Part b) [5 marks]: The student excellently handles the transition of the four-momenta from the rest frame to the lab frame, correctly applying Lorentz transformation equations and clearly explaining the process. This part is detailed and accurately reflects the physics involved.

Part c) [3 marks]: While the student mentions the calculation of wavelengths for the photons in both frames, the actual calculations and the use of the given values (mass, velocity) to derive these wavelengths are not shown. The explanation lacks the detail and clarity needed to fully understand how the results were obtained.

General Feedback: Overall, this solution demonstrates a thorough understanding of relativistic physics, with particularly strong explanations in parts a) and b). To improve, the student should include detailed calculations in part c) to support their conclusions.

Solution 3

Part a) [4 marks]: The explanation of the energy-momentum 4-vector and its Lorentz invariance is comprehensive and clear, correctly showing the relationship between energy, momentum, and mass, and explaining why the scalar product is invariant under Lorentz transformations.

Part b) [4 marks]: The student provides a correct approach to deriving the four-momenta of the decay products in the lab frame but lacks some clarity in the application of the Lorentz transformation matrix. While correct, the explanation could be more detailed in demonstrating the step-by-step transformation.

Part c) [2 marks]: This part is less well executed. The student mentions calculating wavelengths in both frames but does not provide a clear method or accurate calculations. The wavelengths given are unrealistic and do not seem to be derived from correctly applied physical principles, indicating a significant misunderstanding or error in calculation.

General Feedback: Solution 3 shows a good understanding of the foundational concepts of relativistic physics but falls short in accurately applying these concepts to calculate the wavelengths of photons in part c). Improvements could be made by providing more detailed and accurate calculations in this part.

In summary, the grades for each solution are as follows:

Solution 1: 10/13

Solution 2: 12/13

Solution 3: 10/13

Each solution demonstrates strengths in understanding relativistic physics concepts, with Solution 2 standing out for its thoroughness and clarity across all parts. Solutions 1 and 3 could be improved by providing more detailed calculations and clearer explanations in certain areas, particularly in the calculation of photon wavelengths in part c)."

\subsection*{Question 8:}

\textbf{Response 1:}

"\textbf{Solution 1:}

Grading:

a) The student's approach to deriving the equation of motion for the pendulum is methodical and shows a good understanding of the Lagrangian mechanics. They correctly identify the expressions for kinetic and potential energy, include the contributions from the motion of the support, and apply the Euler-Lagrange equation to find the equation of motion. The use of Python is mentioned for simplifying the algebraic manipulations, which is appropriate for the complexity of the problem. However, the solution does not show the complete steps after taking the derivatives, which is critical for full marks.

b) The student successfully shows that in the limit of small \( v \), \( A \), and \( \phi \), the equation reduces to the classic pendulum problem. They correctly use the small-angle approximation.

Marks: a) 7/9, b) 1/1. Total: 8/10

Feedback: The solution provided a solid derivation using the Lagrangian approach and correctly identified the transition to the classic pendulum problem under the specified limits. To improve, ensure that all intermediate steps are shown, especially when taking derivatives, as these are crucial for the complete understanding and verification of the solution.

\textbf{Solution 2:}

Grading:

a) This student also has a good grasp of the Lagrangian formalism, and they lay out the kinetic and potential energy terms clearly. However, the solution has a significant error in the kinetic energy expression. The radial kinetic energy is not correctly expressed, as it misses the component due to the motion of the support which should also include a term accounting for the changing angle \( \phi \). This mistake will affect the subsequent derivation of the equation of motion.

b) The student identifies the correct limits for small angles and constants approaching zero, and correctly simplifies the equation of motion to that of a classic pendulum.

Marks: a) 5/9, b) 1/1. Total: 6/10

Feedback: You have demonstrated an understanding of Lagrangian mechanics but made a critical error in the kinetic energy term by not accounting for the changing angle \( \phi \). Ensure that all components of motion are included in both kinetic and potential energies to accurately derive the equations of motion.

\textbf{Solution 3:}

Grading:

a) This solution correctly outlines the procedure for deriving the Lagrangian and the equation of motion for the pendulum. The kinetic and potential energies are appropriately defined. The student also takes into account the moving support in their expressions for \( T \) and \( U \), and correctly applies the Euler-Lagrange equation to derive the equation of motion. The solution is detailed and shows a clear understanding of the concepts involved.

b) The limit for small angles and constants approaching zero is correctly taken, and the student appropriately simplifies the equation to that of the classical pendulum.

Marks: a) 9/9, b) 1/1. Total: 10/10

Feedback: Excellent work in deriving the equation of motion for the pendulum using Lagrangian mechanics. Your step-by-step approach and attention to detail in accounting for all terms in the kinetic and potential energy expressions led to a correct final equation of motion. You also correctly applied the small-angle approximation to reduce the equation to the classical case."

\textbf{Response 2:}

"Upon reviewing the students' solutions, here's the feedback and grading:

\textbf{Solution 1:}

Grade:
\begin{itemize}
    \item Part a: 8/9
    \item Part b: 1/1
\end{itemize}

Feedback:
The student did a good job deriving the Lagrangian and applying the Euler-Lagrange equation to get the equation of motion. The solution is well-structured and shows a clear understanding of the dynamics involved. However, there are some minor errors in the algebraic manipulation which need correction, specifically in the expression for the kinetic energy which does not correctly translate to the Lagrangian's terms. The result obtained for the equation of motion matches the expected form derived from the symbolic computation, confirming the correctness of the approach. The student also correctly simplified the equation in the limit of small angles and constants approaching zero, reverting to the classic simple pendulum equation.

\textbf{Solution 2:}

Grade:
\begin{itemize}
    \item Part a: 7/9
    \item Part b: 1/1
\end{itemize}

Feedback:
The student has correctly identified the components of the kinetic and potential energy and presented a clear approach to finding the Lagrangian and using the Euler-Lagrange equation. However, the radial kinetic energy due to the support moving was not accounted for properly in the Lagrangian, and thus the student missed the cross-terms in the kinetic energy. This omission affects the equation of motion. The simplification for the small angle approximation in part b) was correctly executed, resulting in the correct classic equation of motion for a simple pendulum.

\textbf{Solution 3:}

Grade:
\begin{itemize}
    \item Part a: 9/9
    \item Part b: 1/1
\end{itemize}

Feedback:
This solution correctly derives the Lagrangian, including both the tangential and radial kinetic energy components. The student accurately applies the Euler-Lagrange equation, resulting in the correct equation of motion. The algebra is well-handled, and the final form of the equation of motion corresponds to the expected result from the symbolic computation. The simplification under the small-angle approximation and the limit of \(v, A \rightarrow 0\) in part b) is correct, yielding the classic pendulum problem's equation of motion. Excellent work on the comprehensive and correct solution."

\textbf{Response 3:}

"Please note that I am not able to execute Python code or any other scripts to verify mathematical expressions. However, I can provide a qualitative assessment based on the information provided in each solution.

Solution 1:

Part a): The student correctly identifies the Lagrangian as the difference between kinetic and potential energies. The expressions for the kinetic and potential energies account for both the swinging motion and the motion of the support. The student then applies the Euler-Lagrange equation to these expressions to derive the equation of motion for the pendulum. However, there is a mistake in the kinetic energy term where they have not correctly accounted for the derivative of \(x(t)\) which should include both the \(vt^3\) and \(A\sin(\omega t)\) terms. The derivation is somewhat comprehensive but may include calculation errors due to not handling the derivatives correctly.

Part b): The student correctly applies the small-angle approximation and recognizes that the terms involving \(v\) and \(A\) vanish, leading to the classic pendulum equation. This shows a good understanding of the small oscillations limit.

Marks: 7/9 for part a), 1/1 for part b). There is a significant deduction in part a) for the errors in handling the kinetic energy term.

Feedback: The approach is correct, but be careful with the derivatives, especially when combining terms involving \(x(t)\). Revisit the kinetic energy expression and make sure to include all contributions from the support's motion.

Solution 2:

Part a): This student provides a clear and concise derivation of the Lagrangian and the equation of motion. They correctly separate the tangential and radial components of the kinetic energy and write down the potential energy. The application of the Euler-Lagrange equation is correct, and they simplify to the correct equation of motion for \(\phi\). However, they have not included the effect of the moving support in the kinetic energy, which would affect the radial part and hence the equation of motion.

Part b): The student makes the appropriate small-angle approximations and correctly identifies the equation of motion for a classic pendulum.

Marks: 6/9 for part a), 1/1 for part 
b). The deduction in part a) is for omitting the effect of the moving support in the kinetic energy.

Feedback: Good work on the small-angle approximation. However, you neglected the effect of the support's movement on the radial kinetic energy. This is a significant part of the problem and affects the equation of motion. Make sure to include all forces and energies in the system.

Solution 3:

Part a): The student starts well by defining the kinetic energy with correct consideration of both horizontal and vertical components and writes down the potential energy correctly. They also correctly apply the Euler-Lagrange equation and reach an equation of motion that includes the effects of the moving support. However, there is a potential error in not explicitly showing the derivatives and the final equation of motion contains terms that should not appear under the correct application of the Euler-Lagrange equation.

Part b): The student correctly uses the small-angle approximation and derives the classic pendulum equation.

Marks: 6/9 for part a), 1/1 for part b). The deduction in part a) is for potential errors in the application of the Euler-Lagrange equation and the final equation of motion.

Feedback: You have the right idea for the kinetic and potential energies, but there may be errors in your application of the Euler-Lagrange equation. Ensure that the derivatives are taken correctly and that only relevant terms are included in the final equation of motion."

\subsection*{Question 9:}

\textbf{Response 1:}

"Grading and Feedback for Physics Problem Solutions

Solution 1:

\textbf{Part a) Derivation: 5/6}
\begin{itemize}
  \item Correctly identifies the relationship between torque, tension, and radius.
  \item Uses Newton's second law for rotation accurately to relate torque to angular acceleration.
  \item Establishes the correct relationship between angular acceleration and linear acceleration.
  \item Minor mistake: the final equation for the acceleration of the block \( a \) is missing the radius \( R \) in the denominator.
\end{itemize}

\textbf{Part b) Calculation: 1/1}
\begin{itemize}
  \item Correctly substitutes the given values into the derived equations.
  \item Correctly calculates the numerical values for tension, angular acceleration, and block acceleration.
\end{itemize}

\textbf{Feedback:}
Overall, the solution is mostly accurate. However, there's a minor error in the final equation for the block's acceleration where the radius \( R \) should appear in the denominator. This is due to a missed substitution step in the algebra. Please double-check the final steps in your derivations to ensure all variables are correctly accounted for.

Solution 2:

\textbf{Part a) Derivation: 6/6}
\begin{itemize}
  \item Accurately uses Newton's second law for both linear and rotational motion.
  \item Correctly relates the linear acceleration of the block to the angular acceleration of the disk.
  \item Successfully eliminates the tension to solve for the block's acceleration and angular acceleration.
  \item The final expressions for \( a \) and \( \alpha \) are correct and simplified properly.
\end{itemize}

\textbf{Part b) Calculation: 1/1}
\begin{itemize}
  \item Correctly applies the derived formulas with given values to find the required quantities.
  \item The calculations for tension, angular acceleration, and block acceleration are correct.
\end{itemize}

\textbf{Feedback:}
The student provided a very clear and accurate solution, correctly applying physical laws and mathematical reasoning. The final expressions are correct, and the student successfully calculated the required values. This demonstrates a good understanding of the problem and the physics involved.

Solution 3:

\textbf{Part a) Derivation: 5/6}
\begin{itemize}
  \item Correctly identifies the expressions for torque and uses Newton's second law for rotation.
  \item Establishes the correct relationship between linear and angular accelerations.
  \item Minor mistake: There's a minor algebraic error in the final expressions for tension and acceleration. The moment of inertia is given as \( I = \frac{1}{2} M R^2 \), which means the correct factor should be \( \frac{1}{2} M \) instead of \( M \) when solving for \( T \) and \( a \).
\end{itemize}

\textbf{Part b) Calculation: 1/1}
\begin{itemize}
  \item Correctly calculates the numerical values for tension, angular acceleration, and block acceleration using the given values.
\end{itemize}

\textbf{Feedback:}
The solution is coherent and shows a good understanding of the problem. However, there's an algebraic slip when solving for \( T \) and \( a \). The moment of inertia's influence on the equations was not handled correctly. It's important to be careful with algebraic manipulations and ensure that each step logically follows from the previous."

\textbf{Response 2:}

"Feedback on Physics Problem Solutions

Solution 1 Feedback:

\textbf{Part a)}
\begin{itemize}
    \item The student correctly identified the relationship between torque, tension, and angular acceleration.
    \item They used the correct moment of inertia for the disk.
    \item The relationship between the linear acceleration of the block and the angular acceleration of the disk was correctly applied.
    \item The student derived a correct formula for tension using the equations of motion for both the disk and the block.
\end{itemize}

\textbf{Part b)}
\begin{itemize}
    \item The student plugged in the numbers correctly to find the values of tension, angular acceleration, and linear acceleration.
\end{itemize}

\textbf{Marks:}
\begin{itemize}
    \item Part a): 6 marks for correct derivations and formulas.
    \item Part b): 1 mark for correctly calculating the values with the given numbers.
\end{itemize}

\textbf{Total: 7 marks}

\textbf{General Feedback:} The student demonstrated a clear understanding of the problem and applied the concepts correctly to derive the expressions and calculate the numerical values.

Solution 2 Feedback:

\textbf{Part a)}
\begin{itemize}
    \item The student correctly applied Newton's second law for the block and the rotational equivalent for the disk.
    \item The relationship between the torque and the angular acceleration was accurately identified.
    \item They correctly related the linear acceleration of the block to the angular acceleration of the disk.
    \item The student derived correct expressions for \( a \) and \( \alpha \).
\end{itemize}

\textbf{Part b)}
\begin{itemize}
    \item The calculations for the given values are correct.
\end{itemize}

\textbf{Marks:}
\begin{itemize}
    \item Part a): 6 marks for correct derivations and formulas.
    \item Part b): 1 mark for correctly calculating the values with the given numbers.
\end{itemize}

\textbf{Total: 7 marks}

\textbf{General Feedback:} The student's approach was methodical, and they efficiently derived the equations needed to solve for the unknowns. The calculations were executed correctly.

Solution 3 Feedback:

\textbf{Part a)}
\begin{itemize}
    \item The torque due to the tension was correctly calculated.
    \item The student applied Newton's second law for rotation appropriately.
    \item They established the correct relationship between the block's acceleration and the disk's angular acceleration.
    \item The student provided correct expressions for \( T \), \( a \), and \( \alpha \) after solving the system of equations.
\end{itemize}

\textbf{Part b)}
\begin{itemize}
    \item The substitution of the given values into the derived expressions was performed correctly.
\end{itemize}

\textbf{Marks:}
\begin{itemize}
    \item Part a): 6 marks for correct derivations and formulas.
    \item Part b): 1 mark for correctly calculating the values with the given numbers.
\end{itemize}

\textbf{Total: 7 marks}

\textbf{General Feedback:} The student showed a comprehensive understanding of the physical principles involved and derived the correct expressions. The numerical solutions provided are accurate."

\textbf{Response 3:}

"Solution 1:

Part a) - 4 marks:
- The student correctly identifies the relationship between torque and angular acceleration, and correctly derives the expression for angular acceleration \( \alpha \).
- The expression for tension \( T \) is also correct, as is the relationship between \( a \) and \( \alpha \).
- However, there is a mathematical error in the final expression for \( a \). The student incorrectly simplifies the expression and does not include \( R \) in the denominator when substituting \( T \) into the expression for \( a \).

Part b) - 0 marks:
- The student provides a correct method to calculate the values but does not actually perform the calculations. Therefore, no marks are awarded for this part.

General feedback: Good work on the derivation, but be sure to check your final expressions for algebraic accuracy. Also, remember to perform the actual calculations when required in part b).

Solution 2:

Part a) - 3 marks:
- The student sets up the correct equations for the block and the disk using Newton's second law.
- However, there is a mistake in the manipulation of equations to eliminate \( T \) and solve for \( a \) and \( \alpha \). The student incorrectly removes \( R \) when substituting \( a = R \cdot \alpha \) into the equation for the disk.

Part b) - 0 marks:
- The student claims to have calculated the values but does not provide the actual calculations or final numeric results. No marks are awarded for this part.

General feedback: Ensure that when substituting expressions into equations, the units and variables are consistent. In the calculations, always include your work and the final numeric answers.

Solution 3:

Part a) - 6 marks:
- The student correctly derives the expressions for angular acceleration \( \alpha \), tension \( T \), and acceleration \( a \). The equations are set up properly, and the student demonstrates good algebraic manipulation skills.

Part b) - 1 mark:
- The student provides the correct calculations with the correct values substituted into the derived expressions, leading to correct final answers for \( T \), \( a \), and \( \alpha \).

General feedback: Excellent work on both the derivation and the calculation. Your methodical approach and clear presentation of your work are commendable."

\subsection*{Question 10:}

\textbf{Response 1:}

"To evaluate each solution, let's break down the assessment based on the criteria outlined:

Solution 1:

\textbf{Part a)}

\begin{itemize}
    \item Correctly applies Newton's second law: 2 marks
    \item Identifies forces acting on the ball correctly: 2 marks
    \item Derives the differential equation accurately: 1 mark
    \item Solves the differential equation correctly, matching the form required in the problem statement: 1 mark
\end{itemize}

\textbf{Total for part a): 6 marks}

\textbf{Part b)}

\begin{itemize}
    \item Correctly solves for \( \underline{A} \) using the initial conditions: 2 marks
\end{itemize}

\textbf{Total for part b): 2 marks}

\textbf{Overall feedback:} The solution accurately follows Newton's laws of motion and correctly derives and solves the differential equation, providing a clear and logical progression of thought. The explanation is thorough, showing a good understanding of the physics involved.

\textbf{Total marks: 8/8}

Solution 2:

\textbf{Part a)}

\begin{itemize}
    \item Correctly applies Newton's second law: 2 marks
    \item Accurately identifies forces and forms the differential equation: 2 marks
    \item Attempts to separate components but does not explicitly solve the differential equation as clearly as Solution 1: 1 mark
    \item Correctly identifies the general solution but with slight confusion in explanation: 1 mark
\end{itemize}

\textbf{Total for part a): 6 marks}

\textbf{Part b)}

\begin{itemize}
    \item Correctly identifies \( \underline{A} \) using initial conditions: 2 marks
\end{itemize}

\textbf{Total for part b): 2 marks}

\textbf{Overall feedback:} The solution demonstrates a correct approach to solving the problem, with a good attempt to break down the equation into components. However, the explanation could be clearer in showing the direct path to the solution. The initial condition is correctly used to find \( \underline{A} \).

\textbf{Total marks: 8/8}

Solution 3:

\textbf{Part a)}

\begin{itemize}
    \item Correct application of Newton's second law: 2 marks
    \item Correct identification of forces and differential equation formation: 2 marks
    \item Correct approach to solving the differential equation but with a slightly more complex explanation than necessary: 1 mark
    \item Correctly matches the given solution form, but with an additional step that adds complexity without enhancing understanding: 1 mark
\end{itemize}

\textbf{Total for part a): 6 marks}

\textbf{Part b)}

\begin{itemize}
    \item Correctly solves for \( \underline{A} \) with clear logical steps: 2 marks
\end{itemize}

\textbf{Total for part b): 2 marks}

\textbf{Overall feedback:} Solution 3 accurately addresses both parts of the question, effectively applying Newton's laws and solving the differential equation. The approach is technically correct, though it could be presented more succinctly to improve clarity. The use of initial conditions to find \( \underline{A} \) is correctly executed.

\textbf{Total marks: 8/8}

\textbf{General Feedback:} All three solutions demonstrate a strong understanding of the physics concepts and mathematical techniques required to solve the problem. The main difference lies in the clarity and simplicity of the explanations. Solutions that provide clear, direct paths to the answer, explaining each step logically and succinctly, are especially commendable."

\textbf{Response 2:}

"To grade each solution out of a total of 8 marks, with up to 6 marks for part a) and up to 2 marks for part b), we'll evaluate the completeness and accuracy of their explanations and mathematical derivations.

Solution 1

\begin{itemize}
    \item \textbf{Part a):} This solution correctly applies Newton's second law of motion, identifies the forces acting on the ball, and arrives at the correct differential equation. It then proceeds to solve the differential equation, correctly identifying the form of the solution and matching it to the form given in the problem statement. The explanation is clear and logically structured. \textbf{(6 marks)}
    \item \textbf{Part b):} The initial condition is used correctly to solve for the constant vector \( \underline{A} \). The explanation is straightforward and accurately follows from the setup in part a). \textbf{(2 marks)}
\end{itemize}

\textbf{General Feedback}: Solution 1 is comprehensive and correctly solves the problem with a clear and logical progression of ideas. The mathematical derivations are accurate, and the explanations are well-articulated.

Solution 2

\begin{itemize}
    \item \textbf{Part a):} This solution also accurately applies Newton's second law and solves the differential equation correctly. However, it provides a slightly more detailed explanation for separating the components and discusses the physical interpretation of each term in the solution, which wasn't strictly necessary for full marks but demonstrates a good understanding. \textbf{(6 marks)}
    \item \textbf{Part b):} Identical in correctness to Solution 1, it correctly finds \( \underline{A} \) using the initial condition. The explanation is equally valid and clearly presented. \textbf{(2 marks)}
\end{itemize}

\textbf{General Feedback}: Solution 2 demonstrates a strong understanding of the problem and provides a thorough explanation. The detailed discussion of the equation's components is commendable but not necessary for full marks. Overall, the solution is accurate and well-explained.

Solution 3

\begin{itemize}
    \item \textbf{Part a):} Solution 3 provides a correct application of Newton's second law and arrives at the correct differential equation. It then suggests a solution form and verifies it by differentiation, which is a more rigorous approach than the other solutions. This method demonstrates a deep understanding of differential equations and their solutions. \textbf{(6 marks)}
    \item \textbf{Part b):} Correctly uses the initial condition to find \( \underline{A} \), just like the previous solutions. The method is accurately applied and well-explained. \textbf{(2 marks)}
\end{itemize}

\textbf{General Feedback}: Solution 3 is particularly strong in its rigorous approach to verifying the solution to the differential equation. This demonstrates not only a correct solution but also a deeper understanding of the solution's validity. The explanations are clear, and the mathematical reasoning is solid.

\textbf{Overall}, all three solutions effectively address the problem and provide correct solutions for both parts a) and b). Each solution earns the full 8 marks, differentiated slightly in their approach and depth of explanation. The feedback for each highlights the strengths of their respective approaches and the correctness of their solutions."

\textbf{Response 3:}

"Let's evaluate each solution based on the criteria provided, grading each part separately and then providing general feedback.

Solution 1

\textbf{Part a)} This solution correctly applies Newton's second law of motion, identifies the forces acting on the ball, derives the differential equation, and solves it to find the general form of velocity as a function of time. The explanation is clear, and the steps are logically connected. The solution correctly reaches the form of the velocity vector given in the problem statement.
\begin{itemize}
    \item \textbf{Grade for part a): 6/6}
\end{itemize}

\textbf{Part b)} The student correctly uses the initial condition to solve for the constant vector \( \underline{A} \). The explanation is straightforward and accurately follows from the given initial condition.
\begin{itemize}
    \item \textbf{Grade for part b): 2/2}
\end{itemize}

\textbf{General Feedback:} Solution 1 is comprehensive, demonstrating a good understanding of Newton's laws, differential equation solving, and applying initial conditions. The explanation is clear and directly addresses the problem statement.

Solution 2

\textbf{Part a)} This solution also correctly applies Newton's second law, separates the forces, and solves the differential equation to find the velocity's general form. However, the explanation regarding separating the components and directly addressing the horizontal and vertical components could potentially confuse readers since it suggests a separate treatment but does not follow through with a distinct solution for horizontal motion.
\begin{itemize}
    \item \textbf{Grade for part a): 5/6} due to slightly confusing explanations about component separation.
\end{itemize}

\textbf{Part b)} Identical to Solution 1, it accurately uses the initial condition to find \( \underline{A} \). The explanation is clear and correct.
\begin{itemize}
    \item \textbf{Grade for part b): 2/2}
\end{itemize}

\textbf{General Feedback:} Solution 2 is thorough and correct but introduces unnecessary complexity by mentioning component separation without needing to detail this aspect for solving the problem. It maintains accuracy in applying initial conditions.

Solution 3

\textbf{Part a)} The approach to applying Newton's second law and solving the differential equation is correct and similar to the previous solutions. This solution goes a step further by verifying the solution through differentiation, which adds to the completeness of the answer, although this step is not strictly necessary for the problem's requirements.
\begin{itemize}
    \item \textbf{Grade for part a): 6/6}
\end{itemize}

\textbf{Part b)} This solution mirrors the previous ones in correctly determining \( \underline{A} \) using the initial conditions. The method is accurately described and follows logically from the setup.
\begin{itemize}
    \item \textbf{Grade for part b): 2/2}
\end{itemize}

\textbf{General Feedback:} Solution 3 provides a detailed and accurate explanation, including verification of the solution, which, while not required, demonstrates a thorough understanding. The solution is well-explained and correctly addresses both parts of the problem.

In summary, all three solutions demonstrate a strong understanding of the problem's requirements and physics principles. Solutions 1 and 3 are slightly stronger in their clarity and thoroughness, while Solution 2, despite being correct, introduces a bit of unnecessary complexity in its explanation."

\end{document}