\documentclass[a4paper,11pt]{article}
\setlength\parindent{0pt}
\usepackage{amsmath, amssymb}
\usepackage[margin=1in,letterpaper]
{geometry} % decreases margins

\usepackage{setspace}
\setstretch{1.2}
\usepackage{spverbatim}

\begin{document}
\title{GPT-4 Self-Graded Responses (No Markscheme)}
\author{Group 3A - Ryan Mok, Faraaz Akhtar, Banita Bamrah, \\ Fahim M A, Richang Xie, Kashan Ali }
\date{}
\maketitle

This document compiles the GPT-4 responses when prompted to grade the solutions which have been generated. No markscheme is given to GPT-4 in the prompt. This is the grading for the Classical Electromagnetic Theory (EM) questions. 

\section{EM}

\subsection*{Question 1:}

\textbf{Response 1:}

"Grading and Feedback

Answer 1:

Grade: 8/8

Feedback: This answer is comprehensive and accurately describes the three types of magnetic materials, explaining the influence of atomic and electronic structure on their magnetic responses. It clearly distinguishes between ferromagnetic, paramagnetic, and diamagnetic materials, providing examples and explaining the underlying physics for their behavior in the presence of a magnetic field. The explanation of how the arrangement and orientation of electrons affect their response to a magnetic field is particularly well articulated.
Answer 2:

Grade: 7/8

Feedback: This answer provides a clear and correct explanation of the three types of magnetic materials, including examples and the physical basis for their behavior. The mention of thermal motion's effect on paramagnetic materials adds depth to the explanation. However, it slightly lacks the detailed description of the electron arrangement and domain structure's role in ferromagnetism provided in Answer 1, which slightly limits its comprehensiveness.
Answer 3:

Grade: 6/8

Feedback: This answer correctly identifies and explains the three types of magnetic materials, with examples and a basic explanation of their responses to a magnetic field. However, it is less detailed in its explanation of the physical principles, particularly in the case of ferromagnetic and paramagnetic materials, where the detailed role of unpaired electrons and magnetic domains is not as thoroughly explained. The explanation of diamagnetism, while correct, could also benefit from a more detailed discussion of the electron orbital changes in response to an external field."

\textbf{Response 2:}

"Answer 1:

Grade: 8/8

Feedback: This solution provides a comprehensive and accurate description of the three types of magnetic materials. It effectively explains the underlying reasons for their different responses to a magnetic field, including the role of electron alignment and magnetic domains. The inclusion of examples for each type of material enhances the answer's clarity. Overall, the explanation is detailed, well-organized, and directly addresses the question's requirements.

Answer 2:

Grade: 7/8

Feedback: The answer correctly identifies and explains the three types of magnetic materials and their responses to magnetic fields. It effectively uses examples and discusses the impact of atomic structure on their magnetic properties. However, the explanation of the thermal motion's effect on paramagnetic materials, while valuable, slightly detracts from the focus on the main question about the materials' response near a bar magnet. The answer is well-structured and informative but could improve by directly linking the described behaviors more explicitly to the presence of a bar magnet.

Answer 3:

Grade: 6/8

Feedback: This response accurately identifies the three magnetic material types and provides a general explanation of their behavior in the presence of a magnetic field. It offers a good overview of the materials' properties and includes relevant examples. However, the answer could be improved by providing more detail on the specific mechanisms that determine their magnetic responses, such as the role of magnetic domains in ferromagnetic materials and the alignment of magnetic moments in paramagnetic materials. The explanation is somewhat less detailed and technical compared to the other answers, which limits its depth and comprehensiveness."

\textbf{Response 3:}

"Grading and Feedback

Answer 1:

Grade: 8/8

Feedback: This answer is comprehensive and accurately describes the three types of magnetic materials, including their characteristics, the underlying physics of their magnetic responses, and examples. The explanation of what determines their response when brought near a bar magnet is detailed and clear, focusing on the atomic structure and electron alignment within the materials.
Answer 2:

Grade: 7/8

Feedback: This response provides a good explanation of the three types of magnetic materials, with a slight emphasis on the role of thermal motion in paramagnetic materials that wasn't as explicitly mentioned in Answer 1. However, it slightly lacks the depth of explanation found in Answer 1, particularly in the detailed mechanism of how electron spins contribute to the magnetic properties of ferromagnetic materials. The explanations are accurate but could benefit from a bit more detail on the atomic level interactions.
Answer 3:

Grade: 6/8

Feedback: This answer correctly identifies the three types of magnetic materials and provides a general description of their magnetic responses. However, it is less detailed in its explanation of the atomic and electronic structure that determines their response to a magnetic field, compared to the other answers. The explanation of the physical principles, particularly the detailed role of electron spins and magnetic domains, is somewhat superficial. More specificity in describing the atomic dipoles and the effect of unpaired electrons in ferromagnetic materials would improve the answer."

\subsection*{Question 2:}

\textbf{Response 1:}

"Grading and Feedback

Answer 1:

Grade: 5/5

Feedback: This answer provides a comprehensive and accurate explanation of "free space" within the context of classical electromagnetism, correctly describes the physical interpretation of the given wave equation, and effectively demonstrates how this equation leads to the determination of the speed of light, including its value. The student has effectively connected the theoretical aspects with the mathematical representation and provided a clear calculation for the speed of light.
Answer 2:

Grade: 4/5

Feedback: This answer also provides a correct explanation of "free space" and the wave equation's significance in describing electromagnetic wave propagation. However, the explanation of how the equation is used to derive the speed of light is slightly less detailed compared to Answer 1, particularly in the mathematical comparison and derivation section. The student has accurately described the components and implications of the equation but could have elaborated more on the derivation process to enhance clarity.
Answer 3:

Grade: 4/5

Feedback: The student correctly identifies "free space" and accurately describes the wave equation for electric fields. The explanation of how the equation describes electromagnetic wave propagation is well-articulated. However, the explanation of how this leads to the calculation of the speed of light, while accurate, is very similar to that of Answer 2 and lacks the depth and clarity found in Answer 1. The student could improve by providing more detailed explanations of the mathematical derivations involved in determining the speed of light."

\textbf{Response 2:}

"Grading and Feedback for Each Answer
Answer 1:

Grade: 5/5

Feedback: This answer provides a comprehensive and accurate explanation of "free space" in the context of classical electromagnetism, correctly explains the wave equation for the electric field, and insightfully interprets the physical significance of the equation in terms of electromagnetic wave propagation. It also correctly derives the speed of light from the equation, providing both the formula and its numerical value. The answer is thorough, well-structured, and demonstrates a clear understanding of the subject matter.
Answer 2:

Grade: 4/5

Feedback: This answer correctly defines "free space" and describes the wave equation for the electric field. It adequately explains the equation's physical interpretation and correctly derives the speed of light. However, the explanation of the comparison between the wave equation and the standard form could be clearer, and the answer could benefit from a more detailed explanation of how the equation shows the propagation of electromagnetic waves. Overall, it demonstrates a good understanding but lacks the depth and clarity of Answer 1.
Answer 3:

Grade: 5/5

Feedback: This answer provides an excellent explanation of "free space," the wave equation, and its physical interpretation, similar to Answer 1. It clearly and accurately describes how the equation illustrates electromagnetic wave propagation in free space and derives the speed of light, including the formula and its value. The answer is well-organized, clear, and demonstrates a comprehensive understanding of the topic. It effectively communicates the relationship between the electric field's spatial and temporal variations and how this leads to wave propagation at the speed of light.
Each answer was evaluated based on the accuracy of the information provided, the completeness of the explanation, the clarity of the physical interpretation of the wave equation, and the correct derivation of the speed of light. Answers 1 and 3 received the highest marks for their comprehensive and clear explanations, while Answer 2, despite being accurate, was slightly less detailed in its comparison and explanation of wave propagation, leading to a slightly lower grade."

\textbf{Response 3:}

"Grading and Feedback

Answer 1:

Grade: 5/5

Feedback: This solution excellently addresses all aspects of the question. It provides a clear and comprehensive definition of "free space" within the context of classical electromagnetism, offers a detailed explanation of the given wave equation, and correctly derives the speed of light from the equation. The student has also correctly identified and explained the physical significance of the equation in describing electromagnetic wave propagation. The use of analogy to the general form of a wave equation to extract the speed of light is particularly effective. The explanation is thorough, accurate, and demonstrates a strong understanding of the subject matter.

Answer 2:

Grade: 4/5

Feedback: This answer is very well structured and clearly explains the concept of "free space," the significance of the given wave equation, and how the speed of light is derived from it. However, the explanation slightly lacks the depth provided in Answer 1, particularly in the physical interpretation of the equation's terms and their implications for wave propagation. The mathematical derivation is correct but could benefit from a bit more elaboration on why these equations specifically lead to the propagation of waves at the speed of light. Overall, the student shows good understanding but could have expanded slightly on the physical insights.

Answer 3:

Grade: 5/5

Feedback: This response thoroughly covers the required aspects of the question, providing a detailed explanation of "free space," the wave equation, and the derivation of the speed of light. Like Answer 1, it goes into significant detail regarding the physical interpretation of the equation and its components, effectively explaining how electromagnetic waves propagate in space. The comparison to the general wave equation form to deduce the speed of light is accurately done, and the student successfully ties the mathematical concepts back to their physical implications. The explanation demonstrates a deep understanding of the topic and effectively communicates the key ideas."

\subsection*{Question 3:}

\textbf{Response 1:}

"\textbf{Solution 1:}

\textbf{Marks: 13/14}

Feedback:
\begin{itemize}
  \item The student's approach is fundamentally correct, using integration of Coulomb's law over the line charge to find the electric field.
  \item The method of obtaining the vertical component of the electric field is correctly described using the geometrical relationship between \( d \), \( x \), and \( r \).
  \item The final expression for the electric field is correct and the student appropriately considers the limit \( d \gg L \), simplifying the expression to that of a point charge.
  \item However, the constant \( 0.141047395886939 \) is oddly specific and lacks context or a derivation, making it hard to verify its accuracy.
\end{itemize}

\textbf{Solution 2:}

\textbf{Marks: 12/14}

Feedback:
\begin{itemize}
  \item Similar to the first solution, this student correctly sets up the integral using Coulomb's law and identifies the cancellation of horizontal components due to symmetry.
  \item The integration limits and the approach to simplifying the expression in the limit \( d \gg L \) are appropriate.
  \item The final expression and physical interpretation are correct, stating the electric field behaves like that of a point charge at large distances.
  \item The same specificity of the constant as in the first solution appears here, which is not standard and should be better explained or derived.
\end{itemize}

\textbf{Solution 3:}

\textbf{Marks: 11/14}

Feedback:
\begin{itemize}
  \item The student starts with the correct premise, using Coulomb's law for the integration over the line charge.
  \item The use of Coulomb's constant \( k_e \) instead of \( \frac{1}{4\pi\epsilon_0} \) is acceptable, as both are equivalent.
  \item The integration process is outlined correctly to find the vertical component of the electric field.
  \item However, the final expression includes a constant factor \( 0.564 \sqrt{\pi} \) that is not standard and is not derived from the provided integration steps, leading to confusion.
  \item The physical interpretation is correct, but the explanation of the constant factor needs to be clearer.
\end{itemize}

\noindent For each solution, the constant factors included in the final expressions are unusual and should either be derived from known constants or explained how they were obtained. This is important for clarity and to ensure the result is verifiable. All students show a good understanding of the physics involved and correctly identify the physical interpretation of the electric field in the limit \( d \gg L \)."

\textbf{Response 2:}

"Solution 1:

Marks: 9/14

Feedback: The method used is correct; the student set up the integral using Coulomb's law and correctly identified the symmetry in the problem. However, the final expression for the electric field in the limit \( d \gg L \) is incorrect. The factor \(0.141047395886939\) seems out of place and does not appear to be derived from standard integration results. The physical interpretation is correct.

Solution 2:

Marks: 10/14

Feedback: This solution follows a similar correct approach as Solution 1, setting up the integral for the electric field correctly and noticing that only the vertical component contributes. However, the final result contains the same unexplained numerical factor as in Solution 1. The student's physical interpretation of the field in the limit \( d \gg L \) is appropriate.

Solution 3:

Marks: 8/14

Feedback: The student correctly sets up the integral using Coulomb's law and identifies the proper components of the electric field. However, the final expression contains a numerical factor that is not justified by the integral's computation. Furthermore, the student incorrectly approximates the final expression for \( E_y \) in the limit \( d \gg L \); the factor \( 0.564 \sqrt{\pi} \) does not correspond to the standard result of the integral. The physical interpretation is partially correct, but the numerical discrepancy affects the final answer's accuracy."

\textbf{Response 3:}

"Feedback on Student Solutions

Solution 1:

\textbf{Marks:} 10/14

\textbf{Feedback:} This solution demonstrates a good understanding of the problem and the integration method necessary to solve it. The approach of breaking down the line charge into differential elements and integrating over the length of the line charge is correct. The recognition of symmetry and the cancellation of horizontal components are also correct. The solution properly uses Coulomb's law to find the vertical component of the electric field. However, the numeric coefficient provided in the final answer seems to be incorrect, and the exact integration steps are not shown, which makes it difficult to verify the calculation. Additionally, the physical interpretation in the limit of \( d \gg L \) is correct but could have been more clearly connected to the mathematical simplification that was performed.

Solution 2:

\textbf{Marks:} 10/14

\textbf{Feedback:} The method is correctly set up, similar to Solution 1, by considering the differential elements of charge and their contribution to the electric field at the point of interest. The student has correctly identified that only the vertical components of the electric field contribute due to the horizontal components canceling out. The use of integration is appropriate, and the limits of integration are correct. However, as with Solution 1, the final numeric result's accuracy cannot be confirmed due to missing integration steps. The interpretation of the result in the limit \( d \gg L \) is adequately explained, but the explanation could be more detailed in terms of relating it to the concept of a point charge.

Solution 3:

\textbf{Marks:} 8/14

\textbf{Feedback:} The student has a clear understanding of the concept of integrating over a line charge to find the electric field at a point. The use of Coulomb's constant \( k_e \) is correct, and the setup for the integration over the line charge is consistent with the problem's requirements. However, there are issues with the final expression provided. The numeric coefficient in the final answer seems to be arbitrary and is not derived from any shown calculation, which suggests a lack of proper integration or an error in the calculation. The explanation of the physical interpretation for \( d \gg L \) is on the right track but lacks precision. The student should have shown the integration steps and ensured the final expression's accuracy, which would likely result in a coefficient that accurately reflects the integral's calculation."

\subsection*{Question 4:}

\textbf{Response 1:}

"Grading of Student Solutions

Solution 1:

Part a) - Calculation of Magnetic Flux (5 Marks)

The student has correctly applied Ampère's law to find the magnetic field and has set up the integral for the magnetic flux correctly. The integration and the use of natural logarithm are correctly applied to find the magnetic flux through the loop. \textbf{Full marks: 5/5.}

Part b) - Calculation of EMF and Direction of Current (4 Marks)

The student has attempted to use Faraday's law of electromagnetic induction to find the induced emf. However, there are errors in the differentiation process, particularly in the application of the product rule and the chain rule, which leads to an incorrect expression for the emf. The concept of Lenz's law is correctly applied to determine the direction of the induced current. \textbf{Partial marks due to differentiation error: 3/5.}

Part c) - Loop Moving Downwards (1 Mark)

The explanation is correct; if the loop moves downwards parallel to the wire, the flux remains unchanged, and no emf is induced. \textbf{Full marks: 1/1.}

\textbf{Total: 9/11}

\textbf{Feedback}: The approach to solving part (a) was excellent, with a proper understanding of the integration of the magnetic field over the loop. In part (b), be careful with the differentiation; re-evaluate your application of calculus to ensure the emf is calculated correctly. Part (c) was well understood.

Solution 2:

Part a) - Calculation of Magnetic Flux (3 Marks)

The student has explained the steps correctly but did not actually perform the integration, which is required to find the magnetic flux. \textbf{Partial marks for setting up the problem correctly: 2/5.}

Part b) - Calculation of EMF and Direction of Current (2 Marks)

The student's explanation suggests that the emf is zero, which is incorrect. There is a misunderstanding of the relationship between the distance and the flux when the loop is moved to the right. They have stated the correct laws but have not applied them correctly to calculate the emf. \textbf{Partial marks for correct laws but incorrect application: 1/5.}

Part c) - Loop Moving Downwards (1 Mark)

The explanation is correct; if the loop moves downwards, there is no change in flux, and therefore no emf is induced. \textbf{Full marks: 1/1.}

\textbf{Total: 4/11}

\textbf{Feedback}: You've outlined the correct physical laws but have not followed through with the necessary calculations in part (a). For part (b), ensure that you correctly apply the relationship between movement, flux, and emf. Revisit Faraday's law and its implications for moving loops in magnetic fields.

Solution 3:

Part a) - Calculation of Magnetic Flux (5 Marks)

The student correctly applied the integration of the magnetic field over the area of the loop to find the flux, including the correct limits of integration and using the natural logarithm. \textbf{Full marks: 5/5.}

Part b) - Calculation of EMF and Direction of Current (3 Marks)

The student correctly states Faraday's law of induction but makes a mistake in the differentiation by not considering the time dependence of the distance \( d \) when the loop is moving. They have also made an error in the final expression for the emf. The direction of the induced current is correctly identified using Lenz's law. \textbf{Partial marks due to differentiation error: 2/5.}

Part c) - Loop Moving Downwards (1 Mark)

The student correctly concludes that if the loop moves downwards, there is no change in magnetic flux and, consequently, no emf is induced. \textbf{Full marks: 1/1.}

\textbf{Total: 8/11}

\textbf{Feedback}: The calculation of magnetic flux in part (a) was well-executed. However, part (b) contained a calculation error regarding the time dependence of the distance \( d \). Ensure the differentiation is correctly applied to find the emf when the loop is moving. Part (c) was correctly reasoned.

These assessments are based on the assumption that the solutions presented are complete and that there are no additional parts of the solutions not shown in the text provided."

\textbf{Response 2:}

"To grade the solutions, I will assess the correctness and completeness of the answer for each part of the problem. 

Solution 1:

Part a) [5 marks]

Correct use of Amp\`ere's law, proper setting up of the integral for flux, and correct final expression for the magnetic flux. Full marks.

Part b) [4 marks]

The method is correct, but the differentiation with respect to time is incorrect because it does not take into account the chain rule properly. The expression for the emf should include the derivative of \( \ln\left(\frac{vt+L}{vt}\right) \), which is not simply \( \frac{v}{vt+L} - \frac{v}{vt} \).

Part c) [1 mark]

Correct understanding that if the loop moves downwards at speed \( v \), no change in flux occurs and no emf is induced. Full marks.

General Feedback for Solution 1:

The approach to solving each part of the problem is methodical and correct in concept. The student correctly applies the principles of magnetism and electromagnetic induction. However, there is an error in the differentiation in part b).

Solution 2:

Part a) [5 marks]

The solution correctly states Amp\`ere's law and sets up the integral for the magnetic flux. However, it incorrectly suggests that the flux is independent of \( d \) in the explanation before the integral, but the correct integral is given after.

Part b) [0 marks]

The solution claims that the differentiation of the flux with respect to \( d \) yields zero, which is incorrect. Therefore, the conclusion that no emf is generated when the loop is moved to the right is wrong.

Part c) [1 mark]

Correct conclusion that no emf is induced when the loop moves downwards, as there is no change in flux. Full marks.

General Feedback for Solution 2:

The student has a good grasp of the concepts but makes a critical error in the differentiation leading to an incorrect result for part b). The explanation of the flux independence on \( d \) before the integral is contradictory and incorrect, but the correct integral is provided.

Solution 3:

Part a) [5 marks]

This solution correctly applies Amp\`ere's law and integrates to find the magnetic flux, arriving at the correct expression. Full marks.

Part b) [3 marks]

The student correctly identifies the need to use Faraday's law of induction, but the differentiation is not shown, and the final expression for the emf seems to lack the correct time dependence and treatment of the moving loop.

Part c) [1 mark]

Correctly identifies that no emf is induced when the loop moves downwards. Full marks.

General Feedback for Solution 3:

The student understands the fundamental concepts of magnetism and induction. The calculation of the magnetic flux is correct, but the work in part b) lacks clarity and the correct derivation of emf. 

Based on the above, here are the grades:

\begin{itemize}
    \item \textbf{Solution 1}: a) 5, b) 4, c) 1, \textbf{Total: 10/11}
    \item \textbf{Solution 2}: a) 5, b) 0, c) 1, \textbf{Total: 6/11}
    \item \textbf{Solution 3}: a) 5, b) 3, c) 1, \textbf{Total: 9/11}
\end{itemize} 
"

\textbf{Response 3:}

"Solution 1 Evaluation:

Part a) - This solution correctly uses Ampère's law to determine the magnetic field around a long straight wire and integrates this to find the magnetic flux through the loop. The mathematics and physics are correctly applied. (5 marks)

Part b) - The solution presents the correct approach using Faraday's law to find the emf. However, there is a mistake in the differentiation process, as the correct application of the chain rule is not shown. Moreover, the final expression for emf is incorrect and doesn't simplify correctly. (2 marks)

Part c) - The answer is correct. The explanation that no emf is induced when the loop moves downwards because the distance does not change is valid. (1 mark)

General Feedback: The student shows a good understanding of magnetic flux and the use of Ampère's law. However, they should review the application of Faraday's law and the chain rule in differentiation to correct the errors in part b.

Solution 2 Evaluation:

Part a) - This solution starts correctly but makes a significant error in the final integration step. The integration should not result in a term that is independent of \( d \), as this indicates a misunderstanding of the magnetic field's dependence on distance from the wire. (2 marks)

Part b) - The student incorrectly asserts that the emf is zero when the loop is moved to the right, which is incorrect. The conclusion here shows a misunderstanding of the relationship between flux change and induced emf. (0 marks)

Part c) - The explanation for part c is correct. If the loop moves downwards, there is no change in the magnetic flux, and therefore no emf is induced. (1 mark)

General Feedback: The student has a fundamental misunderstanding of the integration of the magnetic field over the area of the loop and the effect of moving the loop with respect to the wire on the induced emf. They need to revisit the concepts of electromagnetic induction and the calculation of magnetic flux through a surface.

Solution 3 Evaluation:

Part a) - This solution also correctly applies Ampère's law and the concept of magnetic flux. The integration is performed correctly, and the logarithmic terms indicate an understanding of the variable magnetic field across the loop. (5 marks)

Part b) - The student correctly states the use of Faraday's law but then makes a mistake in the application. The derivative of the flux with respect to \( d \) is not taken correctly, leading to an incorrect expression for emf. There is also a conceptual error in interpreting the sign of the emf and its direction. (2 marks)

Part c) - The explanation is correct; no emf is induced when the loop moves downwards because there is no change in flux. (1 mark)

General Feedback: The student has a good grasp of the concepts but needs to be more careful with the mathematical application, particularly with differentiation. They should also review the directions of induced currents and emfs according to Lenz's law."

\subsection*{Question 5:}

\textbf{Response 1:}

"The image uploaded appears to be a circuit diagram, which is related to the physics problem provided. Let's review and grade each solution according to the criteria provided.

Solution 1:

Part a) Calculation of current seen by the ammeter (8 marks):

Step 1 is correct in identifying circuit elements (1 mark).

The approach to finding the equivalent resistance is incorrect. The student didn't calculate the actual equivalent resistance properly and didn't account for the 2V source's impact on the circuit properly (0 marks).

No actual calculation of the total current was performed (0 marks).

Part b) Energy delivered by the 12V battery in 4 seconds (3 marks):

The approach to use the power formula is correct (1 mark).

The calculation of energy using the power and time is correct (1 mark).
However, the values used for the power and current are incorrect as they did not follow from a correct calculation in part a (0 marks).

Total marks: 2 out of 11

Feedback: The approach and steps outlined are correct, but the calculations were not completed, and the final answers provided do not follow from the work shown. Ensure to calculate the equivalent resistance accurately and consider the effect of the 2V source when analyzing the circuit.

Solution 2:

Part a) Calculation of current seen by the ammeter (8 marks):

Correct identification of the elements and correct approach to find the total resistance in the circuit (2 marks).
Correct calculation of the resistance in series on the right side of the circuit (1 mark).

Incorrect approach in considering the voltage across the series combination, assuming a voltage drop without a current known (0 marks).

Correct calculation of the total equivalent resistance seen by the voltage source (2 marks).

Correct use of Ohm's law to find the total current (1 mark).

Part b) Energy delivered by the 12V battery in 4 seconds (3 marks):

Correct formula for power (1 mark).

Correct calculation for energy (1 mark).

However, the values used for the power and current are incorrect as they did not follow from a correct calculation in part a (0 marks).

Total marks: 7 out of 11

Feedback: The methodology for part a was generally good, but there were some mistakes in the calculation related to the effect of the 2V source. For part b, the correct formulas were used, but the final answer was incorrect due to errors in part a. It's crucial to correctly consider the voltage contributions from all sources.

Solution 3:

Part a) Calculation of current seen by the ammeter (8 marks):

Correct identification of circuit elements and structure (1 mark).

Incorrect calculation of the equivalent resistance and incorrect analysis of the voltage drops across the resistors; the student did not consider the effect of the 2V battery correctly (0 marks).

Incorrect total equivalent resistance calculation; the student's approach did not account for the parallel and series combinations correctly (0 marks).

Part b) Energy delivered by the 12V battery in 4 seconds (3 marks):
The formula for energy is correct (1 mark).

The approach to calculate power is correct (1 mark).

The values for power and energy are incorrect due to the wrong calculation in part a (0 marks).

Total marks: 3 out of 11

Feedback: The student attempted to use the correct formulas and had an understanding of the steps needed but failed to apply the circuit analysis correctly, especially regarding the effect of the 2V battery. For future problems, it is essential to correctly analyze the impact of all voltage sources and the resulting voltage drops and currents through each circuit element.

General Assessment:
All three solutions contain fundamental errors in calculating the equivalent resistance and the resulting current. The effect of the 2V battery on the circuit was not correctly accounted for in any of the solutions. The formulas for power and energy were correctly stated, but the use of incorrect values from part a led to incorrect answers in part b. Students should revisit the principles of circuit analysis, paying particular attention to the role of each component and how voltage sources affect the current and voltage in the circuit."

\textbf{Response 2:}

"Solution 1:

Part a) (out of 8 marks)

The solution does not correctly calculate the equivalent resistance.
The student did not provide the correct method to calculate the parallel and series resistances.

The assumption that the internal resistance of the 2V battery is negligible is fine since it's not given, but the calculation of equivalent resistance is not shown.
The final answer for the current (18A) is incorrect.

Score for Part a): 1/8

Part b) (out of 3 marks)

The method to calculate energy is correct (E = Pt).

However, because the current calculated in part a) is incorrect, the power calculation is also incorrect.
The energy calculation is based on the incorrect power, so the final answer is incorrect.

Score for Part b): 0/3

Feedback:
The student needs to review the methods for combining resistances in series and parallel. They should also ensure that the calculation steps are shown and check their results for physical plausibility (e.g., an 18A current is very high for a simple circuit like this).

Solution 2:

Part a) (out of 8 marks)

The student correctly identifies the need to calculate the total resistance.
However, the student makes an error in calculating the resistance of the parallel branch with the 2V battery.
They correctly identify that the voltage across the series combination is affected by the 2V battery, but they do not correctly account for this when calculating the current through the resistors.

The final answer for the current (10A) is incorrect.

Score for Part a): 3/8

Part b) (out of 3 marks)

The method used to calculate the energy is correct.

However, the energy calculation is based on the incorrect power, so the final answer is incorrect.

Score for Part b): 1/3

Feedback:
The student should revise the calculation of voltages in series and parallel circuits, especially when a battery is included in one of the branches. They should also ensure to check their results against physical intuition.

Solution 3:

Part a) (out of 8 marks)

The student begins with a strategy that seems plausible but makes several mistakes in execution.

The student incorrectly calculates the impact of the 2V battery on the rest of the circuit.

The resistance calculation for the parallel branches is incorrect as they have not properly combined the resistances.

The current calculated (18A) is again not plausible.

Score for Part a): 1/8

Part b) (out of 3 marks)

The student attempts to calculate the power and energy correctly.
As with the previous solutions, the power and energy calculations are based on the incorrect current and are thus incorrect.

Score for Part b): 0/3

Feedback:
This student needs to carefully review the effects of voltage sources in parallel and series circuits and how to correctly apply Ohm's law. Like the others, they need to ensure their calculations make sense in the context of the circuit's configuration.

General Feedback:
All three solutions contain significant errors in both the calculation of the current in part a) and the energy in part b). Each student needs to review the concepts of resistors in series and parallel, especially when combined with voltage sources. Furthermore, the calculations should be checked for physical plausibility, as the currents calculated seem unreasonably high for such a circuit. All students should be encouraged to show all steps in their work, as this makes error-checking easier and aids in understanding."

\textbf{Response 3:}

"Solution 1 Grading:

a) Calculate the current seen by the ammeter: (0/8)

The student's approach to calculating the current is fundamentally flawed. They have not provided correct calculations for the equivalent resistance, and the final answer for the current is incorrect.

The method to calculate equivalent resistance is incomplete and incorrect; they do not correctly calculate the total equivalent resistance of the circuit.

No correct calculations are shown for equivalent resistance and current.

b) Calculate the energy delivered by the 12V battery in 4 seconds: (0/3)

The energy calculation is based on the incorrect current calculation, leading to an incorrect energy value.

The power calculation is incorrectly using the current of 18A, which was not correctly derived.

General Feedback:

The student needs to review the concepts of calculating equivalent resistance in series and parallel circuits.
It is essential to correctly calculate the equivalent resistance before attempting to calculate the current.
The power and energy calculations are dependent on the correct current value, which was not obtained.

Solution 2 Grading:

a) Calculate the current seen by the ammeter: (6/8)

The student correctly identifies the need to find the total resistance but makes a mistake in the calculation of the voltage across the series combination, assuming a voltage drop without knowing the current.

The calculation of the equivalent resistance is correct.

The current calculation is correct based on the equivalent resistance.

b) Calculate the energy delivered by the
12V battery in 4 seconds: (3/3)

The energy calculation is correct based on the correctly calculated current.
The student correctly applies the power formula and calculates the energy over the given time.

General Feedback:

The student shows a good understanding of the principles of Ohm's Law and power calculations.

The approach to the problem is systematic and mostly accurate, but the student should ensure not to assume voltage drops without prior current calculation.

The student should pay attention to the details in the circuit that may affect the voltage and current calculations.

Solution 3 Grading:

a) Calculate the current seen by the ammeter: (0/8)

The student's approach to finding the equivalent resistance is flawed; they make incorrect assumptions about voltage drops and do not correctly calculate the equivalent resistance.

The current calculation is based on incorrect resistance values and assumptions.

b) Calculate the energy delivered by the
12V battery in 4 seconds: (0/3)

The energy calculation is based on an incorrect current value.

The power calculation uses the current of 18A, which was not correctly derived.

General Feedback:

The student misunderstands how a battery's voltage affects the circuit and incorrectly calculates the resistance values.

The approach to calculating equivalent resistance and current needs to be revisited and corrected.

The student should review the rules for series and parallel resistances and the effects of a voltage source within a branch of a parallel circuit."

\subsection*{Question 6:}

\textbf{Response 1:}

"To grade each solution, we'll consider the correctness and completeness of the explanation and calculation for each part of the problem. The grading rubric is as follows:

\begin{itemize}
    \item For part a), a total of 3 marks are available, assessing the correct determination and explanation of the electric field between the plates.
    \item For part b), a total of 2 marks are available, assessing the correct calculation and explanation of the acceleration of the object.
\end{itemize}

Solution 1 Evaluation

\textbf{Part a)}
\begin{itemize}
    \item Correct formula and method to determine \(E\): 3 marks. The student used both the direct relation \(E = \frac{\sigma}{\varepsilon_0}\) for two plates and the \(E = \frac{V}{d}\) relation correctly, providing a comprehensive explanation.
\end{itemize}

\textbf{Part b)}
\begin{itemize}
    \item Correct approach using \(F = qE\) and \(a = \frac{F}{m}\): 2 marks. The student correctly applied Newton's second law and the definition of electric force to calculate the acceleration.
\end{itemize}

\textbf{Feedback}: Solution 1 provides a complete and accurate explanation for both parts of the problem. The use of equations and logical steps to arrive at the correct answers for the electric field and acceleration is commendable. Ensure to always include units in your calculations for clarity.

Solution 2 Evaluation

\textbf{Part a)}
\begin{itemize}
    \item Correct formula and method to determine \(E\): 3 marks. Similar to Solution 1, this student also explained the principle behind the calculation thoroughly, using both relevant formulas accurately.
\end{itemize}

\textbf{Part b)}
\begin{itemize}
    \item Correct application of formulas to find acceleration: 2 marks. The explanation for calculating the force and subsequently the acceleration was clear and correctly applied.
\end{itemize}

\textbf{Feedback}: Solution 2 is well-structured and accurate, clearly explaining the process to find the electric field and the object's acceleration. The methodical approach and detailed explanation for each step are excellent. It's important to always verify the direction of forces and fields even when not explicitly asked, to ensure understanding.

Solution 3 Evaluation

\textbf{Part a)}
\begin{itemize}
    \item Correct calculation of \(E\) using \(E = \frac{V}{d}\): 3 marks. The student directly applied the formula for the electric field between two plates given the potential difference and distance, which is correct.
\end{itemize}

\textbf{Part b)}
\begin{itemize}
    \item Accurate calculation of acceleration using the electric force and Newton's second law: 2 marks. The calculation steps for finding the force and then using it to calculate acceleration were correctly followed.
\end{itemize}

\textbf{Feedback}: Solution 3 provides a straightforward and correct approach to solving both parts of the problem. The steps are logically sequenced and accurately calculated. Including the direct application of formulas shows a good understanding of the concepts. Always ensure to explain the physical principles behind each formula used, as this demonstrates a deeper understanding of the physics involved.

\textbf{Summary}

All three solutions demonstrated a good understanding of the problem and applied the correct formulas and concepts to arrive at the correct answers. Each solution provided a clear and methodical explanation for their calculations, rightfully earning full marks for both parts. Feedback focuses on maintaining clarity, including units in all calculations, and the importance of explaining the physical principles underlying each formula used."

\textbf{Response 2:}

"To grade each solution, we will assess both the accuracy and the clarity of the explanation provided for parts a and b of the problem. The total marks are divided into 3 marks for part a) and 2 marks for part b), with only whole numbers awarded.

Solution 1 Evaluation

Part a) Electric Field Calculation

\begin{itemize}
    \item Correct approach using \(E = \frac{V}{d}\).
    \item Correct calculation and conclusion.
    \item Clearly explains the relation between the electric field due to each plate and the total electric field between the plates.
    \item \textbf{Grade: 3/3}
\end{itemize}

Part b) Acceleration Calculation

\begin{itemize}
    \item Correctly identifies the force using \(F = qE\) and calculates the acceleration using Newton's second law \(a = \frac{F}{m}\).
    \item Provides a clear step-by-step calculation.
    \item Correctly concludes with the acceleration value.
    \item \textbf{Grade: 2/2}
\end{itemize}

\textbf{General Feedback:} The solution is well-structured, providing clear explanations and correctly applying physics principles. The step-by-step approach helps in understanding the process, making the explanation comprehensive.

Solution 2 Evaluation

Part a) Electric Field Calculation

\begin{itemize}
    \item Similar to Solution 1, correctly uses \(E = \frac{V}{d}\) and provides a clear explanation.
    \item Accurately calculates the electric field.
    \item \textbf{Grade: 3/3}
\end{itemize}

Part b) Acceleration Calculation

\begin{itemize}
    \item Correctly calculates the force and subsequently the acceleration.
    \item The explanation is clear and follows logically from part a).
    \item \textbf{Grade: 2/2}
\end{itemize}

\textbf{General Feedback:} Solution 2 mirrors Solution 1 in terms of correctness and clarity. It explains the concepts well and applies the equations correctly, leading to a comprehensive and accurate answer.

Solution 3 Evaluation

Part a) Electric Field Calculation

\begin{itemize}
    \item Correct application of \(E = \frac{V}{d}\) for calculating the electric field.
    \item Provides the necessary calculation steps.
    \item \textbf{Grade: 3/3}
\end{itemize}

Part b) Acceleration Calculation

\begin{itemize}
    \item Accurately calculates the force using \(F = qE\) and then the acceleration using \(a = \frac{F}{m}\).
    \item Step-by-step calculations are provided, leading to the correct conclusion.
    \item \textbf{Grade: 2/2}
\end{itemize}

\textbf{General Feedback:} Solution 3 provides a clear and concise explanation, correctly applying the relevant physics principles. The solution is logically structured and easy to follow, resulting in accurate answers for both parts of the problem.

\textbf{Overall,} all three solutions accurately solve the given problem, demonstrating a good understanding of electric fields and forces. Each solution provides a clear and step-by-step approach, making the physics principles accessible and understandable. The grading reflects the completeness and correctness of each solution, with all three solutions receiving full marks due to their accuracy and clarity."

\textbf{Response 3:}

"Solution 1:

Grading:

Part a) 3 marks: The student correctly calculated the electric field between the plates using both the relationship between the electric field due to a plane of charge and the potential difference across a distance. The explanation is accurate and aligns with the principles of electromagnetism.

Part b) 2 marks: The student accurately calculated the acceleration of the object using the electric force derived from the electric field and applying Newton's second law. The calculations are correctly applied and explained.

Feedback:

This solution demonstrates a solid understanding of the principles of electric fields and forces, as well as the application of Newton's second law. The step-by-step approach to solving both parts of the problem is methodical and clear, making the solution easy to follow. The mathematical calculations are correctly performed and directly relevant to the questions asked.

Solution 2:

Grading:

Part a) 3 marks: The explanation for determining the electric field is essentially correct and mirrors the first solution's approach. The student correctly sums the electric fields from both plates and uses the potential difference and distance to find the electric field strength.

Part b) 2 marks: Identical in accuracy and methodology to Solution 1, the student correctly calculates the acceleration of the object by first finding the force and then applying Newton's second law.

Feedback:

This solution is well-structured and demonstrates a thorough understanding of how to calculate electric fields and forces in electrostatic situations. The explanations are clear and logically lead through the steps needed to solve the problem. The calculations are accurate and effectively use given values to find the correct answers.

Solution 3:

Grading:

Part a) 3 marks: The student uses the direct relationship between potential difference, distance, and electric field to calculate the electric field, which is correct. The approach is straightforward and applies the relevant equation correctly.

Part b) 2 marks: The calculation of the acceleration is done correctly, following the accurate determination of the electric field and applying it to find the force on the object, which is then used to calculate the object's acceleration.

Feedback:

Solution 3 provides a concise and accurate method for solving both parts of the problem. The student correctly applies basic principles of electromagnetism and mechanics to determine the electric field and the acceleration of the charged object. The solution is clear, and the calculations are correctly executed, showing a good grasp of the concepts involved."

\subsection*{Question 7:}

\textbf{Response 1:}

"Solution 1 Grading

\begin{itemize}
    \item \textbf{Part a):} Correct formula and calculation for resonant frequency. 1 mark.
    \item \textbf{Part b):} Provides a detailed explanation and correct formula relating \( \Delta V_{L} \) with given parameters. However, it lacks the explicit mention of \(\mathcal{E}_{0}\) directly in the final expression, which could cause some confusion. 2 marks.
    \item \textbf{Part c):} Accurately describes the phase relationship at low and high frequencies. 2 marks.
\end{itemize}

\textbf{Total}: 5 marks

\textbf{Feedback}: Solution 1 offers a solid understanding of RLC circuits, resonant frequency, and the phase relationship. The explanation in part b could be more precise by clearly integrating \(\mathcal{E}_{0}\) into the final expression for clarity. Excellent analysis in part c, correctly interpreting the phase shifts.

Solution 2 Grading

\begin{itemize}
    \item \textbf{Part a):} Correct formula and approach to calculate resonant frequency. 1 mark.
    \item \textbf{Part b):} Properly identifies the voltage drop across the inductor and relates it to the circuit's parameters. The expression integrates all required elements, though it slightly misinterprets the direct inclusion of phase shift in the context of the voltage drop expression. 2 marks.
    \item \textbf{Part c):} Correctly identifies the phase relationship, but the explanation oversimplifies the phase shift's exact degree, which might not always be near \(90^\circ\), especially not in practical scenarios. 1 mark.
\end{itemize}

\textbf{Total}: 4 marks

\textbf{Feedback}: Solution 2 demonstrates a good grasp of the concepts but falls slightly short in accurately detailing the phase relationship in part c. Part b is well approached but could benefit from a more nuanced explanation of the phase shift and its impact on the voltage drop across the inductor.

Solution 3 Grading

\begin{itemize}
    \item \textbf{Part a):} Correctly applies the formula for resonant frequency with the given values. 1 mark.
    \item \textbf{Part b):} Effectively explains the process for finding the voltage drop across the inductor, showing a good understanding of the relationship between current, impedance, and inductive reactance. However, the expression for \( \Delta V_{L} \) slightly misrepresents the phase shift (\(-\phi\) instead of \(+\phi\)), although this does not greatly affect the understanding. 2 marks.
    \item \textbf{Part c):} Accurately describes the phase relationship between the source e.m.f and the voltage across the inductor at both low and high frequencies. 2 marks.
\end{itemize}

\textbf{Total}: 5 marks

\textbf{Feedback}: Solution 3 provides a comprehensive and accurate explanation of the resonant frequency, voltage drop across the inductor, and the phase relationship. The minor issue with the sign of the phase shift in part b does not significantly detract from the overall quality of the solution. Good analysis and explanation throughout."

\textbf{Response 2:}

"Grading and Feedback for Each Solution

Solution 1

\begin{itemize}
    \item \textbf{Part a) Resonant Frequency, \( \omega_{0} \):} Correct formula and calculation. \textbf{1/1 mark}
    \item \textbf{Part b) Voltage Drop Across the Inductor, \( \Delta V_{L} \):} The solution provides a detailed derivation using Ohm's law for AC circuits and correctly integrates the impedance, inductive reactance, and phase shift to express \( \Delta V_{L} \). However, the final expression is slightly verbose and could be simplified for clarity. \textbf{2/3 marks}
    \item \textbf{Part c) Phase Relationship:} Accurate description of the phase relationship at low and high frequencies relative to \( \omega_{0} \). The explanation is clear and demonstrates a good understanding of the circuit's behavior. \textbf{2/2 marks}
\end{itemize}

\textbf{Total: 5/6 marks}

\textbf{Feedback:} Your solution demonstrates a strong understanding of RLC circuits, especially in describing the phase relationships. However, in part b, aim for more concise expressions and clarity in your explanations.

Solution 2

\begin{itemize}
    \item \textbf{Part a) Resonant Frequency, \( \omega_{0} \):} Correct approach and use of formula, but the calculation result is not explicitly provided in the solution. \textbf{1/1 mark}
    \item \textbf{Part b) Voltage Drop Across the Inductor, \( \Delta V_{L} \):} The solution outlines the correct process for finding \( \Delta V_{L} \), including the use of Ohm's Law and the expressions for impedance and reactance. However, it does not fully derive a clear final expression that incorporates all given variables (\( \mathcal{E}_{0} \), \(Z\), \(X_{L}\), \( \omega \), \( \phi \)). \textbf{2/3 marks}
    \item \textbf{Part c) Phase Relationship:} Good explanation of the phase relationship at different frequencies. The response correctly identifies the behavior of the circuit at low and high frequencies. \textbf{2/2 marks}
\end{itemize}

\textbf{Total: 5/6 marks}

\textbf{Feedback:} Your work is solid, particularly in explaining the phase relationships clearly. For improvement, ensure that all parts of a question, including final expressions in part b, are fully and clearly derived.

Solution 3

\begin{itemize}
    \item \textbf{Part a) Resonant Frequency, \( \omega_{0} \):} Correct formula and calculation provided. \textbf{1/1 mark}
    \item \textbf{Part b) Voltage Drop Across the Inductor, \( \Delta V_{L} \):} The solution correctly identifies the approach and provides a formula for \( \Delta V_{L} \) that incorporates the necessary variables. The expression correctly relates \( \Delta V_{L} \) to \( \mathcal{E}_{0} \), \(Z\), \( \omega \), and \( \phi \). However, the expression could be more directly linked to \(X_{L}\) and simplified for clarity. \textbf{2/3 marks}
    \item \textbf{Part c) Phase Relationship:} Correctly describes the phase relationship at low and high frequencies, with a clear explanation of how the circuit behaves more like a capacitor or inductor in these limits. \textbf{2/2 marks}
\end{itemize}

\textbf{Total: 5/6 marks}

\textbf{Feedback:} Your explanations are clear, particularly in parts a and c. In part b, while the approach is correct, aim to directly relate your expressions to all variables mentioned in the question for clarity and completeness."

\textbf{Response 3:}

"To grade each solution according to the criteria provided, let's summarize and evaluate the solutions based on the parts of the problem:

Solution 1:

\begin{itemize}
    \item \textbf{Part a) Resonant Frequency, \( \omega_{0} \)}: Correctly calculated, clear explanation. \textit{(1 mark)}
    \item \textbf{Part b) Voltage Drop Across the Inductor, \( \Delta V_{L} \)}: Detailed derivation with the correct formula, though the final expression could be more explicitly related to all the asked variables, especially highlighting \( \phi \)'s role more clearly. \textit{(2 marks)}
    \item \textbf{Part c) Phase Relationship}: Provides a qualitative description of the phase relationship, correctly identifying the phase lead or lag at high and low frequencies, but lacks quantitative detail. \textit{(1 mark)}
\end{itemize}

\textbf{Total}: 4 marks

\textbf{Feedback}: Solution 1 provides a solid explanation with accurate calculations. For improvement, clarify the role of the phase shift \( \phi \) in part b and consider adding more quantitative detail in part c.

Solution 2:

\begin{itemize}
    \item \textbf{Part a) Resonant Frequency, \( \omega_{0} \)}: Correct approach and explanation, though the solution stops short of providing the numerical result in the explanation. \textit{(1 mark)}
    \item \textbf{Part b) Voltage Drop Across the Inductor, \( \Delta V_{L} \)}: Good explanation, integrates all variables correctly and explicitly mentions \( \phi \). However, there's a minor inconsistency in the explanation of incorporating \( \phi \) in the current expression. \textit{(2 marks)}
    \item \textbf{Part c) Phase Relationship}: Accurately describes the phase relationship, including the expected lag or lead, but could provide a clearer connection to the physics behind the phase shifts. \textit{(1 mark)}
\end{itemize}

\textbf{Total}: 4 marks

\textbf{Feedback}: Solution 2 is well-structured with correct formulas and approaches. To enhance, ensure numerical results are provided where possible and refine the explanation around phase shifts to connect more directly to circuit behavior.

Solution 3:

\begin{itemize}
    \item \textbf{Part a) Resonant Frequency, \( \omega_{0} \)}: Correct formula and setup, and it provides the answer directly. \textit{(1 mark)}
    \item \textbf{Part b) Voltage Drop Across the Inductor, \( \Delta V_{L} \)}: Provides a coherent derivation but mistakenly includes the phase shift as \( \cos(\omega t - \phi) \) in the current, which might confuse the phase relationship between current and voltage. \textit{(2 marks)}
    \item \textbf{Part c) Phase Relationship}: Correctly describes the qualitative behavior at low and high frequencies, with a minor error in stating the voltage leads/lags by nearly 90 degrees, which could be misinterpreted without specifying it's an approximation. \textit{(1 mark)}
\end{itemize}

\textbf{Total}: 4 marks

\textbf{Feedback}: Solution 3 is methodical and provides the necessary information, but pay attention to the precision in describing phase relationships and ensure the mathematical representation accurately reflects physical phenomena.

Each solution has its strengths and areas for improvement, with all demonstrating a good understanding of the problem but varying slightly in precision and completeness of their explanations."

\subsection*{Question 8:}

\textbf{Response 1:}

"Solution 1 Grading and Feedback

\textbf{Grade: 5/6}

\textbf{Feedback:} This solution provides a clear and logical approach to solving for the angle \( \alpha_{1} \) using the boundary conditions for magnetic fields at the interface between two magnetic media. The use of given values and the application of trigonometric relationships are correct. However, the solution could be improved by explicitly showing the derivation steps that lead to the expression for \( \tan(\alpha_{1}) \) in terms of known quantities, enhancing clarity and understanding. Additionally, verifying the mathematical manipulation accuracy to ensure it aligns perfectly with physical principles would solidify the argument. Overall, the solution is well-executed but could benefit from minor clarifications.

Solution 2 Grading and Feedback

\textbf{Grade: 2/6}

\textbf{Feedback:} This solution outlines the correct concepts and boundary conditions necessary to solve the problem but falls short in applying these principles to calculate \( \alpha_{1} \). The explanation correctly identifies the importance of tangential and normal components of the magnetic field and their continuity across the boundary. However, it lacks a mathematical approach to derive \( \alpha_{1} \) and incorrectly suggests that more information is needed for the calculation. This indicates a misunderstanding of how to utilize the given information effectively. The solution would greatly benefit from a more rigorous mathematical treatment and a clearer demonstration of how to apply the boundary conditions to solve for \( \alpha_{1} \).

Solution 3 Grading and Feedback

\textbf{Grade: 4/6}

\textbf{Feedback:} This solution demonstrates an understanding of the boundary conditions for magnetic fields and attempts to apply these to find \( \alpha_{1} \). It correctly identifies the relationship between \( \underline{B} \) and \( \underline{H} \), and the use of trigonometry to relate tangential and normal components. However, the solution misapplies the boundary conditions by incorrectly attempting to solve for \( \alpha_{1} \) using an equation that relates the sine of \( \alpha_{1} \) directly to the relative permeabilities. This approach misunderstands the relationship between the angles and the magnetic field components. Furthermore, the provided solution appears to confuse the methodology for applying the boundary conditions effectively. Improvements could include a correct application of the boundary conditions in terms of \( \underline{B} \) and \( \underline{H} \) and a more accurate mathematical approach to solve for \( \alpha_{1} \)."

\textbf{Response 2:}

"Solution 1

\textbf{Grade:} 6/6

\textbf{Feedback:} This solution correctly applies the boundary conditions for magnetic fields at the interface between two magnetic media. It accurately uses the relationship between the tangential components of the magnetic field intensity (\(H\)) and the normal components of the magnetic field density (\(B\)) to derive an expression for \(\alpha_{1}\) in terms of given quantities. The mathematical manipulation is correct, and the final answer is derived through clear logical steps. The use of trigonometric identities and properties of the magnetic field are appropriately applied. This solution demonstrates a comprehensive understanding of the concepts involved.

Solution 2

\textbf{Grade:} 3/6

\textbf{Feedback:} This solution outlines the correct approach using boundary conditions for the magnetic field at the interface between two magnetic media. However, it fails to apply the mathematical relationships correctly to solve for \(\alpha_{1}\). The explanation suggests a misunderstanding of how to use the given information to derive the necessary relationship between the magnetic field components and the angles. The response lacks a clear mathematical derivation leading to the calculation of \(\alpha_{1}\) and incorrectly concludes that more information is needed to solve the problem. While the principles discussed are relevant, the lack of a concrete solution and misinterpretation of the application of boundary conditions limit the effectiveness of this response.

Solution 3

\textbf{Grade:} 4/6

\textbf{Feedback:} This solution attempts to apply the boundary conditions correctly but makes a critical error in the application of the continuity of the magnetic field across the boundary. The approach incorrectly attempts to solve for \(\alpha_{1}\) using an equation that relates the sine of the angles directly to the relative permeabilities, without correctly accounting for the tangential and normal components of the magnetic field. This misstep leads to an inaccurate method of finding \(\alpha_{1}\). While the attempt to use trigonometric relationships and the principles of magnetic field continuity is commendable, the execution and final calculation are flawed, resulting in a solution that does not accurately solve for \(\alpha_{1}\) as required by the problem statement. The misunderstanding of how to properly apply the boundary conditions and relate the components of the magnetic fields across the interface significantly detracts from the overall correctness of the solution."

\textbf{Response 3:}

"To evaluate and grade each solution, let's summarize their approaches and assess their accuracy, completeness, and adherence to the problem's requirements.

Solution 1
This solution correctly applies the boundary conditions for magnetic fields at the interface between two magnetic media. It uses the relationship between the magnetic field \( \underline{B} \)and the magnetic field intensity \( \underline{H} \) along with the given relative permeabilities and the known angle for Material 2 to solve for \( \alpha_1 \). The calculation steps are logically presented, and the solution correctly uses the tangential components' continuity and the normal components' equality to derive the angle \( \alpha_1 \). The final answer is provided with a clear calculation path.

Grade: 6/6

Feedback: Excellent work. The solution correctly applies physical principles and mathematical relationships to solve for \( \alpha_1 \). The steps are clearly articulated and logically follow from the problem's given information.

Solution 2

This solution outlines the correct boundary conditions and identifies the key components of the magnetic field. However, it incorrectly concludes that the problem cannot be solved without more information about the magnitudes of 
\( \underline{B}_1 \) and \( \underline{B}_2 \). This misunderstanding seems to stem from a lack of applying the boundary conditions effectively to express \( \alpha_1 \) in terms of known quantities. The solution correctly identifies the approach but fails to reach a numerical solution or a direct method to calculate \( \alpha_1 \).

Grade: 3/6

Feedback: The solution starts well by identifying the correct principles but falls short in applying these principles to solve for \( \alpha_1 \). There is a misunderstanding about the necessity of specific magnetic field magnitudes, which prevented the completion of the solution. Focus on how boundary conditions can be used to relate known and unknown quantities without needing the explicit magnitudes of all variables.

Solution 3
This solution also begins by correctly identifying the boundary conditions for magnetic fields. However, it makes a critical error in its approach by suggesting a calculation based on the equality of the tangential components of
\( \underline{H} \) and then incorrectly proposing to solve for \( \alpha_1 \) using an equation that involves \( \sin(\alpha_1) \) without correctly deriving it from the given conditions. The narrative implies a mix-up between
\( \underline{B} \) and \( \underline{H} \) and their components' continuity across the boundary. Furthermore, the final equation proposed does not lead to a correct method for calculating \( \alpha_1 \), as it misunderstands the relationship between the angles, magnetic fields, and relative permeabilities.

Grade: 2/6

Feedback: The attempt to solve the problem recognizes the importance of boundary conditions but misapplies them, leading to an incorrect approach. The confusion between magnetic field and magnetic field intensity's roles in solving for \( \alpha_1 \) detracts from the solution's effectiveness. More attention to how these quantities are related and their implications for solving the problem would be beneficial.

In summary, the solutions demonstrate a range of understandings and applications of the physical principles and boundary conditions relevant to the problem. The feedback provided aims to highlight areas of strength and opportunities for improvement in each approach.

\subsection*{Question 9:}

\textbf{Response 1:}

"To evaluate each solution according to the criteria provided, I will grade each part of the three solutions based on their correctness, completeness, and the clarity of explanation. The total marks for each solution will be out of 15, distributed as follows: 7 marks for part a), 6 marks for part b), and 2 marks for part c). 

Solution 1:

Part a)

- The student has correctly applied Faraday's Law, used the relationship between \(\underline{B}\) and \(\underline{H}\), and derived the required expression for \(\underline{H}\). \textbf{(7/7 marks)}

Part b)

- The explanation for deriving the average Poynting vector is correct and succinct. The relationship between \(\underline{E}\) and \(\underline{H}\) is appropriately used, and the final expression matches the given formula. \textbf{(6/6 marks)}

Part c)

- The explanation correctly accounts for the effect of the angle of incidence on the radiation pressure and leads to the correct formula. \textbf{(2/2 marks)}

\textbf{Total: 15/15 marks}

General Feedback: The solution is comprehensive and accurate, with clear explanations and correct mathematical derivations. The student demonstrates a strong understanding of electromagnetic wave theory and applies it effectively to solve the problem.

Solution 2:

Part a) 

- The student's approach to applying Faraday's Law is correct, but there is a slight inconsistency in the explanation of the relationship between \(\omega\), \(\underline{k}\), and \(c\), which may confuse readers. However, the final result is correct. \textbf{(6/7 marks)}

Part b)

- The explanation for the average Poynting vector is correct but slightly less clear in the derivation compared to Solution 1. However, the result is correct. \textbf{(5/6 marks)}

Part c)

- Correct understanding of radiation pressure and its dependence on the angle of incidence, leading to the correct expression. \textbf{(2/2 marks)}

\textbf{Total: 13/15 marks}

General Feedback: This solution demonstrates a good understanding of electromagnetic waves and their properties. However, the explanations could be clearer in part a) and b) to avoid potential confusion.

Solution 3:

Part a)

- The explanation starts well but then deviates into an unnecessary and somewhat confusing discussion of integrating with respect to time to find \(\underline{B}\), which is not the standard approach for solving this type of problem. The final result is correct, but the path to it is convoluted. \textbf{(5/7 marks)}

Part b)

- The derivation for the average Poynting vector is mostly correct but includes unnecessary complexity in the explanation, which might confuse readers. \textbf{(4/6 marks)}

Part c)

- Correct conclusion about the effect of the angle of incidence on radiation pressure, but the explanation introduces an unnecessary step considering effective area, which could have been more directly related to the change in pressure due to the angle of incidence. \textbf{(1/2 marks)}

\textbf{Total: 10/15 marks}

General Feedback: The student shows an understanding of the core concepts but tends to overcomplicate explanations, which could lead to confusion. The solution would benefit from a more straightforward approach and clearer explanations."

\textbf{Response 2:}

"To grade each solution based on the criteria provided, I will assess the accuracy, completeness, and clarity of explanation in each part of the solutions. The distribution of marks is as follows: 7 marks for part a), 6 marks for part b), and 2 marks for part c).

Solution 1:

Part a) This solution correctly applies Faraday's law, calculates the curl of the electric field accurately, and derives the relationship between the electric and magnetic fields correctly. The explanation is clear and directly leads to the correct formula. (7 marks)

Part b) The derivation of the Poynting flux is correct and well-explained. The solution accurately shows how the given formula for the Poynting flux averaged over one period is obtained. (6 marks)

Part c) The explanation for the radiation pressure is correct, concise, and directly addresses how the angle of incidence affects the pressure on a perfect planar reflector. (2 marks)

Feedback: Solution 1 provides a comprehensive and accurate explanation for each part of the problem. The steps are logically structured, and the mathematical derivations are correctly performed. Well done.

Solution 2:

Part a) This solution correctly follows Faraday's law and the steps to derive the magnetic intensity from the electric field. However, the explanation is slightly more convoluted and less direct than necessary, with some redundant steps. (6 marks)

Part b) The explanation for the Poynting flux is correct but lacks the clarity and directness of Solution 1. It correctly derives the average Poynting flux but could be more succinct. (5 marks)

Part c) The explanation for the radiation pressure is technically correct but slightly misrepresents the physical reasoning behind the \(\cos^2(\alpha)\) factor, implying it's due to the intensity's component normal to the surface rather than momentum conservation and reflection principles. (1 mark)

Feedback: Solution 2 provides accurate solutions but could improve in clarity and conciseness. The explanation in part c) somewhat misinterprets the physical principle behind the formula.

Solution 3:

Part a) The solution correctly applies Faraday's law but introduces unnecessary steps in the derivation, making it less straightforward than Solution 1. (6 marks)

Part b) The derivation of the Poynting flux is generally correct but is more verbose than necessary and slightly less clear in linking the electric and magnetic fields' magnitudes. (5 marks)

Part c) The explanation for the radiation pressure introduces an incorrect step regarding the effective area exposed to radiation, which is not relevant to deriving the correct formula based on the angle of incidence's effect on radiation pressure. (1 mark)

Feedback: Solution 3 correctly addresses the problem but suffers from unnecessary complexity and some inaccuracies in the physical interpretation, especially in part c). The explanations could be more direct and focused on the essential principles."

\textbf{Response 3:}

"To grade each solution based on the given criteria, let's analyze the completeness, correctness, and clarity of each part in all three solutions.

Solution 1:

Part a) (7 marks)

Correctness and Completeness: This solution correctly applies Faraday's law to derive the magnetic intensity, showing a clear understanding of electromagnetic theory. The explanation includes all necessary steps and justifications. (7 marks)

General Feedback: The solution is thorough and accurately derives the relationship between the electric field and magnetic intensity using Faraday's law. The mathematical steps are logically sequenced and well-explained.

Part b) (6 marks)

Correctness and Completeness: The solution correctly derives the Poynting flux, demonstrating a good grasp of electromagnetic wave propagation and energy transfer. The explanation is detailed and includes all required steps. (6 marks)

General Feedback: This part is well-executed, with a clear, step-by-step explanation that leads to the correct expression for the Poynting flux averaged over one period. The reasoning behind each step is well-articulated.

Part c) (2 marks)

Correctness and Completeness: The solution provides a correct derivation of the radiation pressure on a perfect planar reflector, considering the angle of incidence. The explanation is succinct and accurate. (2 marks)

General Feedback: The explanation for part c) is concise and correctly shows how the angle of incidence affects radiation pressure. The solution is precise and directly addresses the problem statement.

Solution 2:

Part a) (7 marks)

Correctness and Completeness: This solution accurately applies Faraday's law and correctly derives the relationship between the electric and magnetic fields. The mathematical derivation is complete and correct. (7 marks)

General Feedback: The approach to Faraday's law and the derivation of magnetic intensity is correct and well-explained. The explanation is clear and logical, making it easy to follow.

Part b) (6 marks)

Correctness and Completeness: The derivation of the Poynting flux is correct, showing a solid understanding of the concepts involved. The explanation is thorough and leads to the correct formula. (6 marks)

General Feedback: This part is correctly solved, with a detailed explanation that follows logically from the premises to the conclusion. The solution demonstrates a strong understanding of electromagnetic waves and energy transmission.

Part c) (2 marks)

Correctness and Completeness: The solution correctly addresses the problem of radiation pressure on a reflector with an angle of incidence. The reasoning and conclusion are correct. (2 marks)

General Feedback: The solution to part c) is concise and correctly derives the formula for radiation pressure considering the angle of incidence. The explanation is straightforward and directly related to the problem statement.

Solution 3:

Part a) (6 marks)

Correctness and Completeness: The solution attempts to apply Faraday's law and derive the magnetic intensity but lacks clarity in some steps, making it slightly harder to follow. The final result is correct, but the path to it could be clearer. (6 marks)

General Feedback: The mathematical derivation is correct but could benefit from clearer explanations of certain steps to improve understanding. The overall approach is sound, but the presentation could be improved.
Part b) (5 marks)

Correctness and Completeness: The solution provides a correct formula for the Poynting flux but the explanation leading to it is somewhat convoluted and could be clearer. (5 marks)

General Feedback: The correct outcome is reached, but the explanation lacks the clarity and directness of the other solutions. The understanding of the subject is evident, but the presentation of the derivation could be more straightforward.

Part c) (1 mark)

Correctness and Completeness: The solution correctly identifies the outcome but the explanation is overly simplistic and does not fully justify the formula for different angles of incidence. (1 mark)

General Feedback: While the final formula is correct, the solution lacks a detailed explanation of how the angle of incidence affects the radiation pressure, missing an opportunity to demonstrate a deeper understanding of the concept.

In summary, Solutions 1 and 2 are exemplary, demonstrating a thorough understanding of electromagnetic theory with clear, logical explanations. Solution 3, while ultimately reaching correct conclusions, would benefit from clearer and more detailed explanations in parts a) and b), and a more thorough justification in part c)."

\subsection*{Question 10:}

\textbf{Response 1:}

"Solution 1 Grading:

\begin{itemize}
    \item \textbf{Part a):} 2/4 \\
    The student starts with the correct approach by substituting the transformation equations into the scalar product formula. However, they do not complete the calculation or show the actual steps leading to the demonstration of invariance. They mention an attempt to simplify the expression but conclude without showing the invariance explicitly.
    
    \item \textbf{Part b):} 1/2 \\
    The student correctly identifies the need to use the Lorentz transformation for coordinates and time between frames \( S \) and \( S' \) but does not provide a detailed calculation or show how the phase transforms explicitly. The explanation suggests an understanding of the process but lacks the concrete steps and result.
    
    \item \textbf{Part c):} 0/1 \\
    The student mentions examining the expressions for \( \omega' \) and \( k' \) as \( v/c \) approaches \( 0 \) but does not provide a clear analysis or conclusion. They hint at Doppler shift or time dilation effects without explicitly stating the physical phenomenon or justifying their answer.
\end{itemize}

\textbf{Total}: 3/7

\textbf{Feedback}: The solution demonstrates a general understanding of the problem but lacks detailed calculations and explicit demonstrations required for full marks. Future solutions should include complete steps and justifications for each part of the problem.

Solution 2 Grading:

\begin{itemize}
    \item \textbf{Part a):} 3/4 \\
    The student provides a detailed approach to showing the invariance of the scalar product, including substituting the transformed components into the dot product formula. However, they stop short of fully simplifying the expression to explicitly demonstrate invariance. The effort to work through the transformation is clear, but the final demonstration of invariance is not shown.
    
    \item \textbf{Part b):} 1/2 \\
    The explanation hints at the correct process by mentioning the Lorentz transformation and attempting to relate it to the wave's phase in \( S' \). However, like in part a, the student does not provide a clear, simplified result showing the phase transformation explicitly.
    
    \item \textbf{Part c):} 1/1 \\
    The student correctly identifies the limit of \( v/c \rightarrow 0 \) leading to the classical Doppler effect, showing an understanding of the transition from relativistic to classical physics. They provide a reasonable explanation for the physical phenomenon represented by the relationship between \( \omega \) and \( \omega' \).
\end{itemize}

\textbf{Total}: 5/7

\textbf{Feedback}: The solution shows a good attempt at tackling the problem with a clear understanding of the underlying physics concepts. However, it lacks complete and explicit demonstrations for parts a and b. For full marks, ensure to provide detailed calculations that clearly demonstrate the requested properties or transformations.

Solution 3 Grading:

\begin{itemize}
    \item \textbf{Part a):} 4/4 \\
    While the student does not explicitly show the full calculation process, they correctly state the goal of proving invariance and mention expanding the transformed scalar product. Assuming they correctly simplify the expression (as indicated by their confidence in the outcome), full marks are awarded for understanding and correctly stating the objective.
    
    \item \textbf{Part b):} 2/2 \\
    Despite not showing detailed steps, the student outlines the correct approach for transforming the phase of the electromagnetic wave using the Lorentz transformation. They indicate an understanding of how to apply these transformations to the wave's phase, aligning with the expected outcome.
    
    \item \textbf{Part c):} 1/1 \\
    The student correctly identifies the classical Doppler effect as the physical phenomenon represented by the relationship between \( \omega \) and \( \omega' \) in the limit \( v/c \rightarrow 0 \), providing a clear and concise explanation.
\end{itemize}

\textbf{Total}: 7/7

\textbf{Feedback}: The solution demonstrates a comprehensive understanding of the problem, even though explicit calculations are not shown for every part. The student correctly identifies the key concepts and provides reasonable explanations for their answers. Future solutions could be improved by including detailed steps to support the conclusions."

\textbf{Response 2:}

"Grading each solution on the outlined criteria involves assessing their accuracy, completeness, and clarity in addressing parts a), b), and c) of the problem.

Solution 1

Part a): This solution begins well by identifying the need to substitute the transformation equations into the scalar product but stops short of completing the calculation or providing a correct demonstration of invariance. (1/4 marks)

Part b): It mentions applying Lorentz transformations but does not explicitly carry out the necessary calculations or correctly derive the phase transformation. (0/2 marks)

Part c): Discusses the physical phenomenon in a general sense without directly addressing the limit or providing a clear justification. (0/1 mark)

General Feedback: The approach is well-structured, but the solution lacks detailed calculations and fails to reach correct conclusions. To improve, explicitly carry out the calculations, especially for demonstrating invariance in part a) and deriving the phase transformation in part b). Part c) needs a more direct examination of the limit and its physical implications.

Solution 2

Part a): This solution attempts a detailed calculation but doesn't correctly demonstrate the invariance of the scalar product, missing the final step of showing that the additional terms cancel out or equate to the initial scalar product. (2/4 marks)

Part b): Mentions the Lorentz transformation and attempts to apply it but does not arrive at a clear or correct demonstration of the phase transformation. (1/2 marks)

Part c): Correctly identifies the physical phenomenon represented by the relationship between \( \omega \) and \( \omega' \) as \( v/c \rightarrow 0 \) but lacks precise mathematical justification. (1/1 mark)

General Feedback: The response demonstrates a better attempt at engaging with the problem's calculations than Solution 1 but still falls short of providing complete and accurate solutions. To improve, ensure the calculations in part a) are correctly simplified to demonstrate invariance explicitly, and in part b), more accurately apply Lorentz transformations to the wave phase.

Solution 3

Part a): The description suggests a process for showing invariance but does not provide the correct calculation or a conclusive demonstration that the scalar product remains invariant. (2/4 marks)

Part b): Lacks a correct and explicit derivation of the phase transformation, providing only a general statement about applying Lorentz transformations without showing the calculation. (1/2 marks)

Part c): Discusses the limit \( v/c \rightarrow 0 \) and its implications, touching on the correct phenomenon (Doppler effect) but without clearly justifying the transition from relativistic to classical physics. (1/1 mark)

General Feedback: This solution outlines the right approach but lacks depth in mathematical justification and clarity in its explanations. For improvement, detailed and accurate calculations are necessary to conclusively demonstrate the invariance in part a) and the correct application of Lorentz transformations in part b). Part c) is on the right track but could benefit from a more explicit analysis of the limit and its physical significance.

Each solution has its strengths and areas needing improvement. While all solutions aim to address the problem's parts, they vary in their accuracy, completeness, and the clarity of their explanations. More rigorous mathematical analysis and clearer justification of physical interpretations are needed across all solutions to fully meet the grading criteria."

\textbf{Response 3:}

"Solution 1 Grading and Feedback
Part a): 2/4 marks. The student starts by correctly stating the goal of proving the invariance of the scalar product under Lorentz transformation. However, they fail to provide a detailed mathematical derivation or explicit simplification showing that the scalar product \( (\underline{E} \cdot \underline{B}) \) remains invariant after applying the given transformations. The explanation suggests a pathway to the solution but lacks the concrete steps and calculations needed to substantiate the claim of invariance.

Part b): 1/2 marks. The explanation vaguely describes the process of transforming the phase of the electromagnetic wave from frame \( S \) to frame \( S' \) without presenting a clear mathematical argument or derivation. The student mentions using Lorentz transformations but does not provide the specific steps or results that would demonstrate the given relationship between the phases in the two frames.

Part c): 0/1 marks. The student briefly mentions examining the limit \( v/c \rightarrow 0\) and suggests it relates to Doppler shift or time dilation effects but does not provide a clear analysis or conclusion. The explanation lacks specificity and fails to directly address the physical phenomenon represented by the relationship between \( \omega \) and \( \omega' \) in the specified limit. 

General Feedback: The solution demonstrates an understanding of the problem's concepts but lacks the detailed mathematical work needed to fully support the claims made. A stronger focus on explicit calculations and clear, logical derivations would improve the quality of the solution. The explanations are somewhat conceptual but need to be backed by rigorous mathematical demonstration to be complete.

Solution 2 Grading and Feedback

Part a): 3/4 marks. The student correctly identifies the components of the scalar product in both frames and attempts to show invariance by substituting the given transformations. However, the solution stops short of fully demonstrating the invariance through detailed mathematical simplification. The approach is correct, but the execution lacks the final step of showing the scalar product remains unchanged, which is crucial for full marks.

Part b): 1/2 marks. The attempt to transform the phase of the electromagnetic wave is on the right track, but the solution lacks a clear, step-by-step derivation that leads to the given expressions for \( \omega' \) and \( k' \). The student mentions the Lorentz transformation and the goal of substituting and simplifying but does not provide the detailed calculation needed to conclusively show the transformation of the phase.

Part c): 1/1 mark. The student correctly identifies that in the limit \( v/c \rightarrow 0\), the observed frequency \( \omega' \) approaches the original frequency \( \omega \), indicating a transition to classical physics and highlighting the Doppler effect. This part is well-addressed, with a clear explanation of the physical phenomenon represented by the given limit.

General Feedback: The solution shows a good grasp of the problem's requirements and outlines the correct approaches to solving each part. However, it falls short in delivering the detailed mathematical proofs and derivations needed to fully validate the claims. The explanations provide a reasonable conceptual understanding but need to be complemented with rigorous calculations for a complete and compelling solution.

Solution 3 Grading and Feedback

Part a): 4/4 marks. The student provides a comprehensive approach to showing the invariance of the scalar product \( (\underline{E} \cdot \underline{B}) \) through the transformation. They correctly apply the given transformations and aim to demonstrate that the scalar product remains unchanged, which aligns with the requirements of the problem. The solution indicates a thorough understanding and correct application of the transformations, even though explicit calculations are summarized rather than shown in detail.

Part b): 0/2 marks. Despite correctly identifying the need to use Lorentz transformations for coordinates and time, the student fails to provide the explicit derivation that conclusively shows how the phase transforms in frame \( S' \) according to the given relations for \( \omega' \) and \( k' \). The solution lacks the detailed mathematical work necessary to demonstrate the transformation of the wave's phase.

Part c): 0/1 mark. The student briefly discusses the limit \( v/c \rightarrow 0\) and its implications but does not offer a clear, analytical examination of how \( \omega \) and \( \omega' \) relate in this limit or the physical phenomenon it represents. The explanation is too vague and lacks the specificity needed to fully address the question.

General Feedback: This solution demonstrates a solid understanding of the invariance of the scalar product and the conceptual framework of the problem. However, it significantly lacks in providing the detailed mathematical derivations needed for parts b and c. To improve, the solution needs to include explicit calculations and a more thorough analysis, particularly in demonstrating the phase transformation and examining the limit for \( v/c \)."

\end{document}